\exo{}

\begin{multicols}{2}
\begin{enumerate}[itemsep=2em]
	\item $ \dfrac{5}{4} = 1+\dfrac{1}{4} $
	\item $ \dfrac{12}{5} = 2+\dfrac{2}{5} $
	\item $ \dfrac{3}{2} = 1+\dfrac{1}{2} $
	\item $ \dfrac{13}{10} = 1+\dfrac{3}{10} $
	\item $ \dfrac{33}{8} = 4+\dfrac{1}{8} $
	\item $ \dfrac{19}{4} = 4+\dfrac{3}{4} $
	\item $ \dfrac{7}{2} = 3+\dfrac{1}{2} $
	\item $ \dfrac{17}{8} = 2+\dfrac{1}{8} $
\end{enumerate}
\end{multicols}

\exo{Compléter les égalités suivantes.}

\begin{multicols}{3}
\begin{description}[itemsep=2em]
	\item $4= \dfrac{8}{2}$
	\item $2= \dfrac{6}{3}$
	\item $1= \dfrac{5}{5}$
	\item $3= \dfrac{12}{4}$
	\item $4= \dfrac{16}{4}$
	\item $10= \dfrac{30}{3}$
\end{description}
\end{multicols}

\exo{Encadrer les fractions entre deux nombres entiers consécutifs.}

\begin{tabularx}{\linewidth}{*{4}{X}}
	 $2<\dfrac{5}{2}<3$
	&$3<\dfrac{10}{3}<4$
	&$2<\dfrac{11}{4}<3$	
	&$4<\dfrac{48}{10}<5$\\		
\end{tabularx}