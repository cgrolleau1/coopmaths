\exo{Écrire sous la forme de la somme d'un nombre entier et d'une fraction inférieure à 1.}

\begin{multicols}{2}
\begin{description}[itemsep=2em]
	\item $ \dfrac{5}{4} = \phantom{0000} + \dfrac{\phantom{00000000}}{} $
	\item $ \dfrac{12}{5} = \phantom{0000} + \dfrac{\phantom{00000000}}{} $
	\item $ \dfrac{3}{2} = \phantom{0000} + \dfrac{\phantom{00000000}}{} $
	\item $ \dfrac{13}{10} = \phantom{0000} + \dfrac{\phantom{00000000}}{} $
	\item $ \dfrac{33}{8} = \phantom{0000} + \dfrac{\phantom{00000000}}{} $
	\item $ \dfrac{19}{4} = \phantom{0000} + \dfrac{\phantom{00000000}}{} $
	\item $ \dfrac{7}{2} = \phantom{0000} + \dfrac{\phantom{00000000}}{} $
	\item $ \dfrac{17}{8} = \phantom{0000} + \dfrac{\phantom{00000000}}{} $
\end{description}
\end{multicols}



%\exo{}
%
%\begin{multicols}{2}
%\begin{enumerate}[itemsep=2em]
%	\item $ \dfrac{5}{4} = 1+\dfrac{1}{4} $
%	\item $ \dfrac{12}{5} = 2+\dfrac{2}{5} $
%	\item $ \dfrac{3}{2} = 1+\dfrac{1}{2} $
%	\item $ \dfrac{13}{10} = 1+\dfrac{3}{10} $
%	\item $ \dfrac{33}{8} = 4+\dfrac{1}{8} $
%	\item $ \dfrac{19}{4} = 4+\dfrac{3}{4} $
%	\item $ \dfrac{7}{2} = 3+\dfrac{1}{2} $
%	\item $ \dfrac{17}{8} = 2+\dfrac{1}{8} $
%\end{enumerate}
%\end{multicols}

\exo{Compléter les égalités suivantes.}

\begin{multicols}{3}
\begin{description}[itemsep=2em]
	\item $4= \dfrac{\phantom{0000}}{2}$
	\item $2= \dfrac{\phantom{0000}}{3}$
	\item $1= \dfrac{\phantom{0000}}{5}$
	\item $3= \dfrac{\phantom{0000}}{4}$
	\item $4= \dfrac{\phantom{0000}}{4}$
	\item $10= \dfrac{\phantom{0000}}{3}$
\end{description}
\end{multicols}

\exo{Encadrer les fractions entre deux nombres entiers consécutifs.}

\begin{tabularx}{\linewidth}{*{4}{X}}
	 $\ldots\ldots<\dfrac{5}{2}<\ldots\ldots$
	&$\ldots\ldots<\dfrac{10}{3}<\ldots\ldots$
	&$\ldots\ldots<\dfrac{11}{4}<\ldots\ldots$	
	&$\ldots\ldots<\dfrac{48}{10}<\ldots\ldots$\\		
\end{tabularx}