\documentclass[a4paper,12pt,fleqn]{article}
%%%%%%%%%%%%%%%%%%%%%%%%%%%%%%%%%%%%%%%%%%%%
%%% Préambule pour le fichier Coop-Maths %%%
%%%%%%%%%%%%%%%%%%%%%%%%%%%%%%%%%%%%%%%%%%%%



% Remarques : 
%		
%		* il faut penser à  doubler l'interligne le plus souvent possible avec \begin{spacing}{2} ou \begin{enumerate}[itemsep=1em]
%
%

% Prérequis : 
%		* Compiler avec XeTeX
%		* Avoir le répertoire images avec les entêtes, pieds de pages et icones
%
%
% Utilisation : 
%		* La fiche commence avec \theme{nombres|gestion|grandeurs|geo|algo}{Texte (entrainement, évaluation, mise en route...}{numéro de version ou vide}{titre du thême et niveau}
%
%		* On peut préférer \themelight{couleur_theme}{couleur_numerotation} notamment pour faire des corrigés à  décalquer
%
%
%
%		* Chaque exercice commence par \exo{titre ou vide}. Le numéro de l'exercice utilise le compteur de section
%		
%		* \begin{correction}...\end{correction} remet le compteur d'exercice à  0, met le cadre adéquat et change la couleur du thême en vert 
%
%		* Plusieurs environnements sont définis : objectif, methode (dans lequel il faut faire précéder le titre d'un \iconemethode
%
%		* Une macro \cfbox permet de faire la même chose que \fbox mais avec la couleur du thême
%
%		* Une macro \reference pour donner la référence d'un objectif dans la couleur du thême. Par exemple \reference{G01}
%
%       * Une macro \motimportant qui met en gras dans la couleur du thême.
%
%		* Une macro \mathimportant qui met en gras dans la couleur du thême à  l'intérieur d'un mode math.
%
%		* Une macro \mathimportantcorr qui met en gras dans la couleur de la correction à  l'intérieur d'un mode math.
%

%		* Une macro \pointillés qui met des pointillés sur la longueur choisie
%
%		* Une macro \point{x}{y}{nom} qui trace un point dans le repêre au couleur du thême
%		
%		* Une macro \pointpos{x}{y}{nom}{position} qui trace un point dans le repêre au couleur du thême
%
%		* Une macro \pointcorr{x}{y}{nom} qui fait la même chose dans la couleur de la correction
%
%		* Une macro \pointcorrpos{x}{y}{nom}{position} qui trace un point dans le repêre au couleur du thême
%
%		* Une macro \tampon(nom) qui insêre le graphique tampon-nom en haut à  droite de la feuille
%
%	* Une macro \version(v) qui insêre le numéro de version en haut à  droite de la feuille (à  ne pas utiliser en même temps que tampon)
%
%		* Une macro \mathunderline pour souligner en couleur correction une partie d'une expression mathématique
%
%	* Une macro \deg pour le signe des degrés
%
%	* Une macro \exalea{id} qui met l'image du dé et le lien en gras vers l'exercice 
%

\usepackage[left=1.5cm,right=1.5cm,top=3.5cm,bottom=2cm]{geometry}
%\usepackage[utf8]{inputenc}		        % Accents, encodage utf8
% Inutile avec XeTeX ?
%\usepackage[T1]{fontenc}		        		% Encodage des caractêres
\usepackage{lmodern}			        		% Choix de la fonte (Latin Modern de D. Knuth)
\usepackage[french]{babel}	        		% Les rêgles de typo. franà§aises
\usepackage{multicol} 					% Multi-colonnes
\usepackage{calc} 						% Calculs 
\usepackage{enumerate}					% Pour modifier les numérotations
\usepackage{enumitem}
\usepackage{graphicx}					% Pour insérer des images
\usepackage{tabularx}					% Pour faire des tableaux
\usepackage{pgf,tikz}					% Pour les images et figures géométriques
\usetikzlibrary{arrows,calc,fit,patterns,plotmarks,shapes.geometric,shapes.misc,shapes.symbols,shapes.arrows,
shapes.callouts, shapes.multipart, shapes.gates.logic.US,shapes.gates.logic.IEC, er, automata,backgrounds,chains,topaths,trees,petri,mindmap,matrix, calendar, folding,fadings,through,positioning,scopes,decorations.fractals,decorations.shapes,decorations.text,decorations.pathmorphing,decorations.pathreplacing,decorations.footprints,decorations.markings,shadows}
\usepackage{tkz-euclide}				% Géométrie euclidienne avec TikZ
\usepackage{pgfplots}					% Représentations graphiques de fonctions
\usepackage{amsmath,amsfonts,amssymb,mathrsfs}  % Spécial math
\usepackage[squaren]{SIunits}			% Pour les unités (gêre le conflits avec  \square de l'extension amssymb)
\usepackage{pifont}						% Pour les symboles "ding"
\usepackage{bbding}						% Pour les symboles
\usepackage[misc]{ifsym}						% Pour les symboles
\usepackage{cancel}						% Pour pouvoir barrer les nombres
\usepackage{url} 			        		% Pour afficher correctement les url
 
\usepackage{eurosym}						% Pour utiliser la commande \euro
\usepackage{fancyhdr,lastpage}          	% En-têtes et pieds
 \pagestyle{fancy}                      	% de pages personnalisés
\usepackage{fancybox}					% Pour les encadrés
\usepackage{xlop}						% Pour les calculs posés
\usepackage{standalone}					% Pour avoir un apercu d'un fichier qui sera utilisé avec un input
\usepackage{multido}						% Pour faire des boucles
\usepackage[hidelinks]{hyperref}					% Pour gérer les liens hyper-texte
\renewcommand{\url}[1]{\textcolor{couleur_theme}{\href{http://#1}{#1}}} 	% Pour ajouter automatiquement le http
\usepackage{fourier}
\usepackage{colortbl} 					% Pour des tableaux en couleur
\usepackage{setspace}					% Pour \begin{spacing}{2.0} \end{spacing}
\usepackage{multirow}					% Pour des cellules multilignes dans un tableau
\usepackage{import}						% Equivalent de input mais en spécifiant le répertoire de travail
\usepackage[]{qrcode}					% Pour la commande \qrcode
\usepackage{pdflscape}
\usepackage[framemethod=tikz]{mdframed} % Pour les cadres
\usepackage{tikzsymbols}
\usepackage{tasks}						% Pour les listes horizontales


\graphicspath{{./images/}}				% Le répertoire


%%% Choix de la police


\usepackage{mathspec}
\setmainfont{Myriad Pro}
\setmathrm{Myriad Pro}
\setmathsf{Myriad Pro}
\setmathtt{Myriad Pro}
%\setboldmathrm[BoldFont={Optima ExtraBlack}]{Optima Bold}
\setmathfont(Digits){Myriad Pro}
\setmathfont(Latin){Palatino}

\spaceskip=2\fontdimen2\font plus 3\fontdimen3\font minus3\fontdimen4\font\relax %Pour doubler l'espace entre les mots


%%% Mise en forme

\setlength{\parindent}{0mm}				% Pas de retrait en début de paragraphe
\setlength\multicolsep{0pt} 				% Pour que l'environnement multicols ne commence pas par un saut de ligne
\renewcommand{\arraystretch}{1.5}		% Interligne dans les tableaux
\renewcommand{\labelenumi}{\textbf{\theenumi{}.}}		% Numérotation en gras
\renewcommand{\labelenumii}{\textbf{\theenumii{}.}}		% Numérotation de niveau 2 en gras
\renewcommand{\thesection}{\Roman{section}.}				% Numérotation des sections en chiffres romains
\renewcommand{\thesubsection}{\alph{subsection})}		% Numérotation des sous-sections en lettres
\setlength{\columnsep}{1cm}								% Séparation des colonnes
\setlength{\columnseprule}{1.5pt}
\renewcommand{\columnseprulecolor}{\color{couleur_theme}} % Trait de séparation des colones en couleur


\setlength\arrayrulewidth{1.5pt} 	% Epaisseur des les filets des tableaux
\arrayrulecolor{couleur_theme}		% Couleur des filets des tableaux


\renewcommand{\headrulewidth}{0pt} % Pour enlever les traits en en-tête et en pied de page
\renewcommand{\footrulewidth}{0pt}
\fancyhead[L]{}
\fancyhead[R]{}



%%% Couleurs

\definecolor{nombres}{cmyk}{0,.8,.95,0}
\definecolor{gestion}{cmyk}{.75,1,.11,.12}
\definecolor{gestionbis}{cmyk}{.75,1,.11,.12}
\definecolor{grandeurs}{cmyk}{.02,.44,1,0}
\definecolor{geo}{cmyk}{.62,.1,0,0}
\definecolor{algo}{cmyk}{.69,.02,.36,0}
\definecolor{correction}{cmyk}{.63,.23,.93,.06}



%%% Environnements


\newmdenv[linecolor=couleur_theme, linewidth=3pt,topline=true,rightline=false,bottomline=false,frametitlerule=false,frametitlefont={\color{couleur_theme}\bfseries},frametitlerulewidth=1pt]{methode}


\newmdenv[startcode={\setlength{\multicolsep}{0cm}\setlength{\columnsep}{.2cm}\setlength{\columnseprule}{0pt}\vspace{0cm}},linecolor=white, linewidth=3pt,innerbottommargin=10pt,innertopmargin=5pt,innerrightmargin=20pt,splittopskip=20pt,splitbottomskip=10pt,everyline=true,tikzsetting={draw=couleur_theme,line width=4pt,dashed,dash pattern= on 10pt off 10pt},frametitleaboveskip=-.6cm,frametitle={\tikz\node[anchor= east,rectangle,fill=white]{\textcolor{couleur_theme}{\raisebox{-.3\height}{\includegraphics[width=.8cm]{\iconeobjectif}}\; \bfseries \Large Objectifs}};}]{objectif}

\newmdenv[startcode={\colorlet{couleur_numerotation}{correction}\renewcommand{\columnseprulecolor}{\color{correction}}
\setcounter{section}{0}\arrayrulecolor{correction}},linecolor=white, linewidth=4pt,innerbottommargin=10pt,innertopmargin=5pt,splittopskip=20pt,splitbottomskip=10pt,everyline=true,frametitle=correction,tikzsetting={draw=correction,line width=3pt,dashed,dash pattern= on 15pt off 10pt},frametitleaboveskip=-.4cm,frametitle={\tikz\node[anchor= east,rectangle,fill=white]{\; \textcolor{correction}{\raisebox{-.3\height}{\includegraphics[width=.6cm]{icone-correction}}\; \bfseries \Large Corrections}};}]{correction}

\newmdenv[roundcorner=0,linewidth=0pt,frametitlerule=false, backgroundcolor=gray!40,leftmargin=8cm]{remarque}



%%% Macros



\newcommand{\theme}[4]
{
	%\theme{nombres|gestion|grandeurs|geo|algo}{Texte (entrainement, évaluation, mise en route...}{numéro de version ou vide}{titre du thême et niveau}
	\fancyhead[C]{
	\begin{tikzpicture}[remember picture,overlay]
	  \node[anchor=north east,inner sep=0pt] at ($(current page.north east)+(0,-.8cm)$) {\includegraphics{header-#1}};
	  \node[anchor=east, fill=white] at ($(current page.north east)+(-2,-1.4cm)$) {\Huge \textcolor{couleur_theme}{\bfseries \#} #2 \textcolor{couleur_theme}{\bfseries \MakeUppercase{#3}}};
	  \node[anchor=center, color=white] at ($(current page.north)+(0,-2.65cm)$) {\Large \bfseries \MakeUppercase{#4}};
	\end{tikzpicture}
	}
	\fancyfoot[C]{
	\begin{tikzpicture}[remember picture,overlay]
	  \node[anchor=south west,inner sep=0pt] at ($(current page.south west)+(0,0)$) {\includegraphics{footer-#1}};
	\end{tikzpicture} 
	}
	\colorlet{couleur_theme}{#1}
	\colorlet{couleur_numerotation}{couleur_theme}
	\def\iconeobjectif{icone-objectif-#1}
	\def\urliconeomethode{icone-methode-#1}
}

\newcommand{\themelight}[2]
{
\pagestyle{empty}
\colorlet{couleur_numerotation}{#2}
\colorlet{couleur_theme}{#1}
\geometry{top=1cm,bottom=1cm}

}




\newcommand{\numb}[1]{ % Dessin autour du numéro d'exercice
\begin{tikzpicture}[overlay,yshift=-.3cm,scale=.8]
	\draw[fill=couleur_numerotation,couleur_numerotation](-.3,0)rectangle(.5,.8);
	\draw[line width=.05cm,couleur_numerotation,fill=white] (0,0)--(.5,.5)--(1,0)--(.5,-.5)--cycle;
	\node[couleur_numerotation]  at (.5,0) { \large \bfseries #1};
		\draw (-.4,.8) node[white,anchor=north west]{\bfseries EX}; 
\end{tikzpicture}
}


\usepackage{titlesec} % Le titre de section est en fait un numéro d'exercice avec sa consigne alignée à  gauche.


\titleformat{\section}{}{\numb{\arabic{section}}}{1cm}{\hspace{0em}}{}


\newcommand{\exo}[1]{ % Un titre d'exercice est en f	ait une nouvelle section avec la consigne écrite en caractêres normaux
\section{\textmd{#1}}
\medskip
}

\newcommand{\iconemethode}[0]{ % Pour l'icone des méthodes dans la bonne couleur
\raisebox{-.3\height}{\includegraphics[width=1cm]{\urliconeomethode}}\quad
}


\newcommand{\cfbox}[1]{% Cadre dans la couleur du thême
	\setlength{\fboxrule}{2pt}
    \colorlet{currentcolor}{.}%
    {\color{couleur_theme}%
    \fbox{\color{currentcolor}#1}}%
}

\newcommand{\corrfbox}[1]{% Cadre dans la couleur de la correction
	\setlength{\fboxrule}{2pt}
    \colorlet{currentcolor}{.}%
    {\color{correction}%
    \fbox{\color{currentcolor}#1}}%
}


\newcommand{\reference}[1]{\ignorespaces
\textcolor{couleur_theme}{\textbf{#1 -}}\xspace
}

\newcommand{\refexo}[1]{\ignorespaces
\textcolor{couleur_theme}{\textbf{\;Ex #1}}\xspace
}

\newcommand{\motimportant}[1]{\ignorespaces
\textcolor{couleur_theme}{\textbf{#1}}\xspace
}

\newcommand{\mathimportant}[1]{
\mathbin{\color{couleur_theme}\pmb{#1}}
}

\newcommand{\mathimportantcorr}[1]{
{\color{correction}\pmb{#1}}
}

\newcommand{\motimportantcorr}[1]{\ignorespaces
\textcolor{correction}{\textbf{#1}}\xspace
}

\newcommand{\pointilles}[1]{
\makebox[#1]{\dotfill}
}

\newcommand{\point}[3]{
\draw[couleur_theme,ultra thick] (#1,#2)--++(.3,.3)--++(-.6,-.6)--++(.3,.3)--++(-.3,.3)--++(.6,-.6);
\draw[couleur_theme] (#1,#2)  node[above right=.1,black]{$#3$};
}

\newcommand{\pointpos}[4]{
\draw[couleur_theme,ultra thick] (#1,#2)--++(.3,.3)--++(-.6,-.6)--++(.3,.3)--++(-.3,.3)--++(.6,-.6);
\draw[couleur_theme] (#1,#2)  node[#4=.1,black]{$#3$};
}


\newcommand{\pointcorr}[3]{
\draw[correction,ultra thick] (#1,#2)--++(.3,.3)--++(-.6,-.6)--++(.3,.3)--++(-.3,.3)--++(.6,-.6);
\draw[correction] (#1,#2)  node[above right=.1,black]{$#3$};
}

\newcommand{\pointcorrpos}[4]{
\draw[correction,ultra thick] (#1,#2)--++(.3,.3)--++(-.6,-.6)--++(.3,.3)--++(-.3,.3)--++(.6,-.6);
\draw[correction] (#1,#2)  node[#4=.1,black]{$#3$};
}



\newcommand{\tampon}[1]{
	\fancyhead[R]{
	\begin{tikzpicture}[remember picture,overlay]
		\node[anchor=north east,inner sep=0pt] at ($(current page.north east)+(-.3,-.3cm)$) {\includegraphics[width=1.6cm]{tampon-#1}};
	\end{tikzpicture}
	}
}

\newcommand{\version}[1]{
	\fancyhead[R]{
	\begin{tikzpicture}[remember picture,overlay]
		\node[anchor=north east,inner sep=0pt] at ($(current page.north east)+(-.5,-.5cm)$) {\large \textcolor{couleur_theme}{\bfseries V#1}};
	\end{tikzpicture}
	}
}

\newcommand{\exalea}[1]{
	\raisebox{-.5\height}{\includegraphics[height=1cm]{exalea}} \textcolor{couleur_theme}{\bfseries \url{coopmaths.fr/ex#1}}
}

\newcommand{\youtube}[1]{
	\raisebox{-.3\height}{\includegraphics[height=.6cm]{youtube}} \textcolor{couleur_theme}{\bfseries \url{coopmaths.fr/video#1}}
}


\def\mathunderline#1{\color{correction}\underline{{\color{black}#1}}\color{black}}

\renewcommand{\deg}{\ensuremath{^\circ}}


\usepackage{pdfpages}
\begin{document}

\titleformat*{\subsection}{\color{couleur_theme}\bfseries}
\renewcommand{\labelitemi}{}

\theme{nombres}{Référentiel}{3\up{e}}{Nombres et Calculs}


\subsection*{Calcul littéral}

\begin{itemize}
	\item \reference{3L10}Déterminer l’opposé d’une expression littérale.
	\item \reference{3L11}Développer (par simple et double distributivités), factoriser, réduire des expressions algébriques simples.
	\item \reference{3L12}Factoriser une expression du type $a^2-b^2$ et développer des expression du type $(a+b)(a-b)$.
	\item \reference{3L13}Résoudre algébriquement une équation du premier degré.
	\item \reference{3L14}Résoudre algébriquement une équation produit.
	\item \reference{3L15}Résoudre algébriquement une équations de la forme $x^2=a$ sur des exemples simples.
	\item \reference{3L16}Résoudre des problèmes se ramenant à une équation, qui peuvent être internes aux mathématiques ou en lien avec d’autres disciplines.

\end{itemize}

\subsection*{Arithmétique}

\begin{itemize}
	\item \reference{3A10}Décomposer un nombre entier en produit de facteurs premiers (à la main, à l’aide d’un tableur ou d’un logiciel de programmation).
	\item \reference{3A11}Simplifier une fraction pour la rendre irréductible.
	\item \reference{3A12}Modéliser et résoudre des problèmes mettant en jeu la divisibilité (engrenages, conjonction de phénomènes...).
\end{itemize}

\subsection*{Nombres et calculs}

\begin{itemize}
	\item \reference{3N10}Utiliser les puissances d’exposants positifs ou négatifs pour simplifier l’écriture des produits.
	\item \reference{3N11}Calculer avec les nombres rationnels, notamment dans le cadre de résolution de problèmes.
	\item \reference{3N12}Résoudre des problèmes mettant en jeu des racines carrées.
	\item \reference{3N13}Résoudre des problèmes avec des puissances, notamment en utilisant la notation scientifique.
\end{itemize}


\newpage
%%%%%%%%%%%%%%
%%% Mesure %%%
%%%%%%%%%%%%%%

\theme{gestion}{Référentiel}{3\up{e}}{Organisation et gestion de données, fonctions}

\subsection*{Généralités sur les fonctions}

\begin{itemize}
	\item \reference{3F10}Utiliser les notations et le vocabulaire fonctionnels.
	\item \reference{3F11}Passer d’un mode de représentation d’une fonction à un autre.
	\item \reference{3F12}Déterminer, à partir de tous les modes de représentation, l’image d’un nombre.
	\item \reference{3F13}Déterminer un antécédent à partir d‘une représentation graphique ou d’un tableau de valeurs d’une fonction.
	\item \reference{3F14}Modéliser un phénomène continu par une fonction.
	\item \reference{3F15}Résoudre des problèmes modélisés par des fonctions en utilisant un ou plusieurs modes de représentation.
\end{itemize}

\subsection*{Fonctions affines et linéaires}

\begin{itemize}
	\item \reference{3F20}Représenter graphiquement une fonction linéaire, une fonction affine.
	\item \reference{3F21}Interpréter les paramètres d’une fonction affine suivant l’allure de sa courbe représentative.
	\item \reference{3F22}Modéliser une situation de proportionnalité à l’aide d’une fonction linéaire.
	\item \reference{3F23}Déterminer de manière algébrique l’antécédent par une fonction, dans des cas se ramenant à la résolution d’une équation du premier degré.
\end{itemize}

\subsection*{Proportionnalité}

\begin{itemize}
	\item \reference{3P10}Utiliser le lien entre pourcentage d’évolution et coefficient multiplicateur.
	\item \reference{3P11}Mener des calculs sur des grandeurs mesurables, notamment des grandeurs composées, et exprimer les résultats dans les unités adaptées.
	\item \reference{3P12}Résoudre des problèmes utilisant les conversions d’unités sur des grandeurs composées.
	\item \reference{3P13}Vérifier la cohérence des résultats du point de vue des unités pour les calculs de grandeurs simples ou composées.
	\item \reference{3P14}Résoudre des problèmes en utilisant la proportionnalité dans le cadre de la géométrie.
\end{itemize}

\subsection*{Statistiques}

\begin{itemize}
	\item \reference{3S10}Lire, interpréter et représenter des données sous forme d’histogrammes pour des classes de même amplitude.
	\item \reference{3S11}Calculer et interpréter l’étendue d’une série présentée sous forme de données brutes, d’un tableau, d’un diagramme en bâtons, d’un diagramme circulaire ou d’un histogramme.
	\item \reference{3S12}Calculer des effectifs et des fréquences.
\end{itemize}

\subsection*{Probabilités}

\begin{itemize}
	\item \reference{3S20}À partir de dénombrements, calculer des probabilités pour des expériences aléatoires simples à une ou deux épreuves.
	\item \reference{3S21}Faire le lien entre stabilisation des fréquences et probabilités.
\end{itemize}

%%%%%%%%%%%%%%%%%
%%% Géométrie %%%
%%%%%%%%%%%%%%%%%

\theme{geo}{Référentiel}{3\up{e}}{Géométrie}

\subsection*{Homothétie et rotation}

\begin{itemize}
	\item \reference{3G10}Transformer une figure par rotation et comprendre l’effet d’une rotation.
	\item \reference{3G11}Transformer une figure par homothétie et comprendre l’effet d’une homothétie.
	\item \reference{3G12}Identifier des rotations et des homothéties dans des frises, des pavages et des rosaces.
	\item \reference{3G13}Mobiliser les connaissances des figures, des configurations, de la rotation et de l’homothétie pour déterminer des grandeurs géométriques.
	\item \reference{3G14}Calculer des grandeurs géométriques (longueurs, aires et volumes) en utilisant les transformations (symétries, rotations, translations, homothétie).
	

\end{itemize}

\subsection*{Théorème de Thalès}

\begin{itemize}
	\item \reference{3G20}Calculer une longueur avec le théorème de Thalès.
	\item \reference{3G21}Démontrer que des droites sont parallèles avec le théorème de Thalès.
	\item \reference{3G22}Connaître et utiliser une définition et une propriété caractéristique des triangles semblables.
\end{itemize}


\subsection*{Trigonométrie}

\begin{itemize}
	\item \reference{3G30}Calculer une longueur dans un triangle rectangle.
	\item \reference{3G31}Calculer la mesure d'un angle dans un triangle rectangle.
	\item \reference{3G32}Résoudre un problème géométrique.
\end{itemize}

\subsection*{Espace}

\begin{itemize}
	\item \reference{3G40}Se repérer sur une sphère (latitude, longitude).
	\item \reference{3G41}Construire et mettre en relation différentes représentations des solides étudiés au cours du cycle (représentations en perspective cavalière, vues de face, de dessus, en coupe, patrons) et leurs sections planes.
	\item \reference{3G42}Calculer le volume d’une boule.
	\item \reference{3G43}Calculer les volumes d’assemblages de solides étudiés au cours du cycle.
\end{itemize}

\end{document}