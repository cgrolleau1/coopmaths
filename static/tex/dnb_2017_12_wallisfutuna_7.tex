
\medskip

Pour soutenir la lutte contre l'obésité, un collège décide d'organiser une course.

\parbox{0.48\linewidth}{Un plan est remis aux élèves participant à l'épreuve.

Les élèves doivent partir du point A et se rendre
au point E en passant par les points B, C et D.

C est le point d'intersection des droites (AE) et
(BD)

La figure ci-contre résume le plan, elle n'est pas à
l'échelle.}\hfill
\parbox{0.47\linewidth}{\psset{unit=0.65cm,arrowsize=2pt 4}
\begin{pspicture}(10.5,5)
%\psgrid
\psline[ArrowInside=->](1.7,4)(0.5,2)(10.2,3.8)(8.5,0.5)
\psline[linestyle=dashed](1.7,4)(8.5,0.5)
\rput{-118}(1.7,4){\psframe(0.25,0.25)}
\rput{62}(8.5,0.5){\psframe(0.25,0.25)}
\uput[u](1.7,4){A (départ)}\uput[dl](0.7,2){B}\uput[u](4.3,2.8){C}\uput[ur](10.2,3.8){D}
\uput[r](8.5,0.5){E (arrivée)}
\rput(0.5,3.3){300~m}\rput(3.5,3.6){400~m}\rput(7.3,1.7){\np{1000}~m}
\end{pspicture}
}

\smallskip

On donne AC $= 400$~m, EC $= \np{1000}$~m et AB $= 300$~m.

\medskip

\begin{enumerate}
\item Calculer BC.
\item Montrer que ED $= 750$m.
\item Déterminer la longueur réelle du parcours ABCDE.
\end{enumerate}



