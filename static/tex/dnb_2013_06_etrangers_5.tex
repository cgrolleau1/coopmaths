\textbf{\textsc{Exercice 5} \hfill 4 points}

\medskip

\begin{tabularx}{\linewidth}{|c|c|X|}\hline
Année &SMIC&\multirow{12}{9cm}{On considère la série statistique donnant le SMIC\\
\textbf{1.} Quelle est l'étendue de cette série ? Interpréter ce résultat.
\textbf{2.} Quelle est la médiane ?\\
\textbf{3.} Paul remarque qu'entre 2001 et 2002, l'augmentation du SMIC horaire brut est de 16 centimes alors qu'entre 2007 et 2008, elle est de 19 centimes.\\
Il affirme que \og le pourcentage d'augmentation entre  2007 et 2008 est supérieur à celui pratiqué entre 2001 et 2002 \fg.\\
A-t-il raison ?}\\ \cline{1-2} 	 
2011& 	9,40& \\ \cline{1-2}	 
2010& 	9,00& \\ \cline{1-2}	 
2009& 	8,82& \\ \cline{1-2}	 
2008& 	8,63& \\ \cline{1-2}	 
2007& 	8,44& \\ \cline{1-2}	 
2006& 	8,27& \\ \cline{1-2}	 
2005& 	8,03& \\ \cline{1-2}	 
2004& 	7,61& \\ \cline{1-2}	 
2003& 	7,19& \\ \cline{1-2}
2002& 	6,83& \\ \cline{1-2}	 
2001& 	6,67&\\ \hline
\multicolumn{3}{r}{SMIC : salaire minimum interprofessionnel de croissance horaire brut en euros }\\
\multicolumn{3}{r}{de 2001 à 2011 (source : INSEE)}
\end{tabularx}

\medskip

