\textbf{Exercice 4 : Vitesse du navire}

\medskip

\begin{enumerate}
\item En 40 secondes, le bateau a parcouru sa propre longueur, soit 246 m. 
\item $v=\dfrac{d}{t}$, soit $v=\dfrac{246}{40}=6,15$ m/s. \\[2mm]
Naviguer à  $1$ nœud signifie parcourir $0,5$ mètre en $1$ seconde, donc :

\quad $\bullet$ Naviguer à  $20$ nœud signifie parcourir $20\times0,5$ mètres en $1$ seconde, soit une vitesse de 10 m/s.

\quad $\bullet$ Naviguer à  $10$ nœud signifie parcourir $10\times0,5$ mètres en $1$ seconde, soit une vitesse de 5 m/s.

 Eva est donc la plus proche de la vérité. 
\end{enumerate}


\vspace{0,5cm}

