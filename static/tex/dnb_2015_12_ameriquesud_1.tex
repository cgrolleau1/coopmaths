\textbf{\textsc{Exercice 1} \hfill 4 points}

\medskip

\emph{Dans ce questionnaire à choix multiple, pour chaque question, une seule proposition
est exacte. Pour chacune des questions, écrire le numéro de la question et recopier
la bonne réponse. Aucune justification n'est attendue. Une réponse correcte rapporte
$1$ point. Une réponse fausse ou l'absence de réponse ne retire aucun point.}

\medskip
\begin{tabularx}{\linewidth}{|X|X|}\hline
\multicolumn{1}{|c|}{Questions}&\multicolumn{1}{|c|}{Propositions}\\ \hline
\textbf{Question 1}

$\left(4\sqrt{2}\right)^2$ est&
\begin{enumerate}
\item égal à 16
\item  le PGCD de $128$ et de $96$
\item égal à $8\sqrt{2}$
\end{enumerate}\\ \hline
\textbf{Question 2}

La médiane de la série de valeurs :

7~;~ 8~;~8~;~12~;~12~;~14~;~15~;~15~;~41&
\begin{enumerate}
\item est supérieure à la moyenne de cette série.
\item est inférieure à la moyenne de cette série.
\item est égale à la moyenne de cette série.
\end{enumerate}\\ \hline
\textbf{Question 3}

Dans une classe de 30 élèves, les $\dfrac{2}{3}$ des élèves
viennent en bus. Combien d'élèves ne viennent pas en bus ?&
\begin{enumerate}
\item $\dfrac{2}{3} \times  30$
\item $1 - \dfrac{2}{3} \times 30$
\item $\left(1 - \dfrac{2}{3}\right) \times 30$
\end{enumerate}\\ \hline
\textbf{Question 4}

Le système $\left\{\begin{array}{l c r}
2x+ \phantom{-}y	&=&11\\
\phantom{2}x - 3y 	&=&- 12
\end{array}\right.$ a pour solution :&
\begin{enumerate}
\item le couple (3,5~;~4)
\item le couple $(- 12~;~0)$
\item le couple (3~;~5)
\end{enumerate}\\ \hline
\end{tabularx}
\medskip

\vspace{0.25cm}

