\textbf{\textsc{Exercice 5 \hfill 7 points}}

\medskip

%\parbox{0.6\linewidth}{Une pyramide régulière de sommet S a pour base le carré ABCD telle que son volume V est égal à 108 cm$^3$.
% 
%Sa hauteur [SH] mesure 9~cm.
% 
%Le volume d'une pyramide est donné par la relation :
% 
%\small{$\text{Volume d'une pyramide} = \dfrac{\text{aire de la base} \times \text{hauteur}}{3}.$} 

\begin{enumerate}
\item %Vérifier que l'aire de ABCD est bien 36 cm$^2$.
On a donc $108 = \dfrac{\mathcal{A} \times 9}{3} $ soit $\mathcal{A} = \dfrac{108}{3} = 36$~cm$^2$.
%En déduire la valeur de AB.

36 est le carré de 6, donc AB = 6~cm. 
%Montrer que le périmètre du triangle ABC est égal à $12 + 6\sqrt{2}$ cm. 
ABC est un triangle rectangle en B ; le théorème de Pythagore montre que :

AC$^2 = 6^2 + 6^2 = 2\times 6^2$, donc AC $ = 6\sqrt{2}$.

Le périmètre de ABC est donc égal à : $6 + 6 + 6\sqrt{2} = 12 + 6\sqrt{2}$~(cm).
\item  %SMNOP est une réduction de la pyramide SABCD.
 
%On obtient alors la pyramide SMNOP telle que  l'aire du carré MNOP soit égale à 4 cm$^2$. 
	\begin{enumerate}
		\item %Calculer le volume de la pyramide SMNOP.
On a $\dfrac{\text{aire(ABCD)}}{\text{aire(MNOP)}} = \dfrac{36}{4} = 9 = 3^2$. Donc le rapport de réduction est $\dfrac{1}{3}$. La hauteur de la pyramide SMNOP mesure donc $\dfrac{9}{3} = 3$~cm.

Son volume est égal à $\dfrac{4 \times 3}{3} = 4 $~cm$^3$.  
		\item %\textbf{Pour cette question toute trace de recherche, même incomplète, sera prise en compte dans l'évaluation.}

%Elise pense que pour obtenir le périmètre du triangle MNO, il suffit de diviser le périmètre du triangle ABC par 3.
 
%Êtes-vous d'accord avec elle ?
Oui le coefficient de réduction étant de $\dfrac{1}{3}$ il faut diviser les dimensions par 3.
Le périmètre de MNO est donc égal à $\dfrac{12 + 6\sqrt{2}}{3} = 4 + 2\sqrt{2}$.
	\end{enumerate} 
\end{enumerate}
%}
%\hfill
%\parbox{0.32\linewidth}{\psset{unit=0.67cm}\begin{center}
%\begin{pspicture}(6,12)
%\pspolygon(0.4,6.8)(3.8,6.8)(5.6,8.1)(3,11.5)(3.8,6.8)(0.4,6.8)(3,11.5)
%\psline[linestyle=dashed](0.4,6.8)(5.6,8.1)(2.3,8.1)(3.8,6.8)
%\psline[linestyle=dashed](0.4,6.8)(2.3,8.1)(3,11.5)(3,7.45)
%\uput[dl](0.4,6.8){A} \uput[dr](3.8,6.8){B} \uput[r](5.6,8.1){C} \uput[ul](2.3,8.1){D} \uput[u](3,11.5){S}
%\psline(3,7.8)(3.3,7.6)(3.3,7.2)
%\psline(3,7.8)(3.4,7.9)(3.4,7.52)
%\uput[d](3,7.45){H}
%%%%%%%%%%%%%%
%\pspolygon(0.4,0.6)(3.8,0.6)(5.6,1.9)(3,5.35)(3.8,0.6)(0.4,0.6)(3,5.3)
%\psline[linestyle=dashed](0.4,0.6)(5.6,1.9)(2.3,1.9)(3.8,0.6)
%\psline[linestyle=dashed](0.4,0.6)(2.3,1.9)(3,5.3)(3,1.25)
%\uput[dl](0.4,0.6){A} \uput[dr](3.8,0.6){B} \uput[r](5.6,1.9){C} \uput[ul](2.3,1.9){D} \uput[u](3,5.3){S}
%\psline(3,1.6)(3.3,1.4)(3.3,1)
%\psline(3,1.6)(3.4,1.7)(3.4,1.32)
%%%%%%%%%%%
%\psline(1.9,3.1)(3.5,3.1)(4.35,3.7)
%\psline[linestyle=dashed](4.35,3.7)(2.7,3.7)(1.9,3.1)
%\uput[ul](1.9,3.1){M}\uput[dr](3.5,3.1){N}\uput[ur](4.35,3.7){O}
%\uput[ul](2.7,3.7){P}\uput[d](3,1.25){H}  
%\end{pspicture}
%\end{center}
%}

\bigskip

