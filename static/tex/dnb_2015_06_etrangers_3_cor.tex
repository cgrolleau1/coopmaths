\textbf{\textsc{Exercice 3} \hfill 6 points}

\medskip

%Soit un cercle de diamètre [KM] avec KM = 6~cm.
%
%Soit un point L sur le cercle tel que ML = 3~cm.
%
%\medskip

\begin{enumerate}
\item %Faire une figure.
On dessine un cercle de diamètre 6~cm donc de rayon 3~cm. Le cercle de centre M et de même rayon 3~cm coupe le cercle en deux points L répondant au problème ; on en choisit un.
\item %Déterminer l'aire en cm$^2$ du triangle KLM. Donner la valeur exacte puis un arrondi au cm$^2$ près.
Le triangle KLM est inscrit dans un cercle admettant pour diamètre l'un de ses côtés ; ce triangle est donc rectangle en L, d'hypoténuse [KM].

L'aire de ce triangle est égale au demi-produit des mesures des deux côtés de l'angle droit en L :

$\mathcal{A}_{\text{KLM}} = \dfrac{\text{KL} \times \text{LM} }{2}$.

Le triangle KLM étant rectangle en L, le théorème de Pythagore permet d'élire :

$\text{KM}^2 = \text{KL}^2 + \text{LM}^2$, soit 

$6^2 = \text{KL}^2 + 3^2$ ou $\text{KL}^2 = 6^2 - 3^2 = (6 + 3)(6 - 3) = 9 \times 3$, donc 

$\text{KL} = \sqrt{9 \times 3} = \sqet{9} \times \sqrt{3} = 3\sqrt{3}$.

Donc $\mathcal{A}_{\text{KLM}} = \dfrac{3\sqrt{3} \times 3 }{2} = \dfrac{9\sqrt{3}}{2} \approx 7,79$ soit 8~cm$^2$ à 1 cm$^2$ près.
\end{enumerate}

\vspace{0,5cm}

