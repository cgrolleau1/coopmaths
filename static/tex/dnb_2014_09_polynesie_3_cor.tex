\textbf{Exercice 3 \hfill 3 points}

\bigskip 

\begin{enumerate}
\item 

%Une bouteille opaque contient 20~billes dont les couleurs peuvent être différentes. Chaque bille a une seule couleur. En retournant la bouteille, on fait apparaà®tre au goulot une seule bille à  la fois. La bille ne peut pas sortir de la bouteille.
% 
%Des élèves de troisième cherchent à  déterminer les couleurs des billes contenues dans la bouteille et leur effectif. Ils retournent la bouteille 40 fois et obtiennent le tableau suivant :
%
%\begin{center}
%\begin{tabularx}{0.8\linewidth}{|m{3cm}|*{3}{>{\centering \arraybackslash}X|}}\hline
%Couleur apparue& rouge& bleue& verte\\ \hline 
%Nombre d'apparitions de la couleur&18 &8 &14\\ \hline
%\end{tabularx}
%\end{center} 
% 
%Ces résultats permettent-ils d'affirmer que la bouteille contient exactement 9 billes rouges, 4 billes bleues et 7 billes vertes ? 
La fréquence d'apparition des couleurs rouge, bleue et verte sont respectivement : $\dfrac{18}{40} = \dfrac{9}{20}, \quad \dfrac{8}{20} = \dfrac{4}{10}$ et $\dfrac{14}{40} = \dfrac{7}{20}$ ; ces fréquences ne permettent pas de conclure au nombre de billes de chaque couleur.
\item %Une seconde bouteille opaque contient 24 billes qui sont soit bleues, soit rouges, soit vertes. 

%On sait que la probabilité de faire apparaà®tre une bille verte en retournant la bouteille est égale à  $\dfrac{3}{8}$ et la probabilité de faire apparaitre une bille bleue est égale à  $\dfrac{1}{2}$. Combien de billes rouges contient la  bouteille ?
La probabilité de faire apparaà®tre une bille rouge est égale à  :

$1 - \dfrac{3}{8} - \dfrac{1}{2} = \dfrac{8}{8} - \dfrac{3}{8} - \dfrac{4}{8} = \dfrac{1}{8}$.

Comme il y a 24 billes le nombre de rouges est $24 \times \dfrac{1}{8} = 3$. 
\end{enumerate}

\bigskip
 
