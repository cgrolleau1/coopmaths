\documentclass[10pt]{article}
\usepackage[T1]{fontenc}
\usepackage[utf8]{inputenc}%ATTENTION codage UTF8
\usepackage{fourier}
\usepackage[scaled=0.875]{helvet}
\renewcommand{\ttdefault}{lmtt}
\usepackage{amsmath,amssymb,makeidx}
\usepackage[normalem]{ulem}
\usepackage{diagbox}
\usepackage{fancybox}
\usepackage{tabularx,booktabs}
\usepackage{colortbl}
\usepackage{pifont}
\usepackage{multirow}
\usepackage{dcolumn}
\usepackage{enumitem}
\usepackage{textcomp}
\usepackage{lscape}
\newcommand{\euro}{\eurologo{}}
\usepackage{graphics,graphicx}
\usepackage{pstricks,pst-plot,pst-tree,pstricks-add}
\usepackage[left=3.5cm, right=3.5cm, top=3cm, bottom=3cm]{geometry}
\newcommand{\R}{\mathbb{R}}
\newcommand{\N}{\mathbb{N}}
\newcommand{\D}{\mathbb{D}}
\newcommand{\Z}{\mathbb{Z}}
\newcommand{\Q}{\mathbb{Q}}
\newcommand{\C}{\mathbb{C}}
\usepackage{scratch}
\renewcommand{\theenumi}{\textbf{\arabic{enumi}}}
\renewcommand{\labelenumi}{\textbf{\theenumi.}}
\renewcommand{\theenumii}{\textbf{\alph{enumii}}}
\renewcommand{\labelenumii}{\textbf{\theenumii.}}
\newcommand{\vect}[1]{\overrightarrow{\,\mathstrut#1\,}}
\def\Oij{$\left(\text{O}~;~\vect{\imath},~\vect{\jmath}\right)$}
\def\Oijk{$\left(\text{O}~;~\vect{\imath},~\vect{\jmath},~\vect{k}\right)$}
\def\Ouv{$\left(\text{O}~;~\vect{u},~\vect{v}\right)$}
\usepackage{fancyhdr}
\usepackage[french]{babel}
\usepackage[dvips]{hyperref}
\usepackage[np]{numprint}
%Tapuscrit : Denis Vergès
%\frenchbsetup{StandardLists=true}

\begin{document}
\setlength\parindent{0mm}
% \rhead{\textbf{A. P{}. M. E. P{}.}}
% \lhead{\small Brevet des collèges}
% \lfoot{\small{Polynésie}}
% \rfoot{\small{7 septembre 2020}}
\pagestyle{fancy}
\thispagestyle{empty}
% \begin{center}
    
% {\Large \textbf{\decofourleft~Brevet des collèges Polynésie 7 septembre 2020~\decofourright}}
    
% \bigskip
    
% \textbf{Durée : 2 heures} \end{center}

% \bigskip

% \textbf{\begin{tabularx}{\linewidth}{|X|}\hline
%  L'évaluation prend en compte la clarté et la précision des raisonnements ainsi que, plus largement, la qualité de la rédaction. Elle prend en compte les essais et les démarches engagées même non abouties. Toutes les réponses doivent être justifiées, sauf mention contraire.\\ \hline
% \end{tabularx}}

% \vspace{0.5cm}\textbf{Exercice 7 \hfill 8 points}

\medskip 

Pour préparer un séjour d'une semaine à Naples, un couple habitant Nantes a constaté que le tarif des billets d'avion aller-retour Nantes-Naples était beaucoup plus élevé que celui des billets Paris-Naples. Il étudie donc quel serait le coût d'un trajet aller-retour Nantes-Paris pour savoir s'il doit effectuer son voyage en 
avion à partir de Nantes ou à partir de Paris.
 
Voici les informations que ce couple a relevées :

\medskip
 
\textbf{Information 1 :}  Prix et horaires des billets d'avion.

\medskip

%\parbox{0.4\linewidth}{\begin{tabular}{|>{\small }l l|}\hline 
%\multicolumn{2}{|l|}{\emph{Vol aller-retour au départ de Nantes}}\\\hline
%Départ de Nantes le 23/11/2014 :&06 h 35\\ 
%Arrivée à Naples le 23/11/2014 :&09 h 50\\
%~& ~\\
%Départ de Naples le 30/1112014 :&12 h 50\\ 
%Arrivée à Nantes le 30/1112014 :& 16 h 25\\
%~&~\\ 
%\multicolumn{2}{|p{6cm}|}{Prix par personne du vol aller-retour: 530~\euro}\\ \hline
%\end{tabular}} \hspace{0.3cm}
%\parbox{0.4\linewidth}{\begin{tabular}{|>{\small }l l|}\hline  
%\multicolumn{2}{|l|}{\emph{Vol aller-retour au départ de Paris}}\\ \hline
%Départ de Paris le 23/11/2014 :&11 h 55\\ 
%Arrivée à Naples le 23/11/2014 :&14 h 10 \\
%~&~\\ 
%Départ de Naples le 30/11/2014 :&13 h 10\\ 
%Arrivée à Paris le 30/11/2014 : & 15 h 30\\ 
%~&~\\
%\multicolumn{2}{|p{6cm}|}{Prix par personne du vol aller-retour: 350~\euro}\\ \hline
%\end{tabular}} 

\begin{tabularx}{\linewidth}{|>{\footnotesize}X r|>{\footnotesize}X r|}\hline 
\multicolumn{2}{|l|}{\emph{Vol aller-retour au départ de Nantes}}&\multicolumn{2}{|l|}{\emph{Vol aller-retour au départ de Paris}}\\\hline
Départ de Nantes le 23/11/2014 :&06 h 35 &Départ de Paris le 23/11/2014 :&11 h 55\\ 
Arrivée à Naples le 23/11/2014 :&09 h 50&Arrivée à Naples le 23/11/2014 :&14 h 10\\
~&&&\\
Départ de Naples le 30/1112014 :&12 h 50&Départ de Naples le 30/11/2014 :&13 h 10\\ 
Arrivée à Nantes le 30/1112014 :& 16 h 25&Arrivée à Paris le 30/11/2014 : & 15 h 30\\
~&&&\\ 
\multicolumn{2}{|p{6cm}|}{\footnotesize Prix par personne du vol aller-retour: 530~\euro}&\multicolumn{2}{|p{6cm}|}{\footnotesize Prix par personne du vol aller-retour: 350~\euro}\\ \hline
\end{tabularx}

\medskip

\emph{Les passagers doivent être présents 2 heures avant le décollage pour procéder à l'embarquement.}

\rule{\linewidth}{.5pt}

\medskip
 
\textbf{Information 2 : Prix et horaires des trains pour un passager}

\begin{center}
\begin{tabularx}{\linewidth}{X >{\centering \arraybackslash}X X>{\centering \arraybackslash}X}
\multicolumn{2}{l}{\emph{Trajet Nantes - Paris (Aéroport)}}&\multicolumn{2}{l}{\emph{Trajet Paris (Aéroport) - Nantes}}\\ 
				&23 novembre 	&				&30 novembre\\ 
Départ			&06 h 22		&Départ			&18 h 20\\
Prix			&51,00 \euro	&Prix 			&42,00 \euro\\
Durée 			& 03 h 16 direct&Durée			&03 h 19 direct\\ 
Voyagez avec 	& TGV 			&Voyagez avec 	&TGV\\
\end{tabularx}
\end{center} 
\rule{\linewidth}{.5pt}
\begin{center}
\begin{tabularx}{\linewidth}{XX}
\textbf{Information 3 : Trajet en voiture}	&\textbf{Information 4 : Parking de l'aéroport de Paris} \\
Consommation moyenne : 6 litres aux 100 km	&Tarif: 58~\euro{} pour une semaine\\
Péage Nantes-Paris: 35,90~\euro				&\\ 
Distance domicile-aéroport de Paris : 409 km&\\
Carburant : 1,30~\euro{} par litre			&\\ 
Temps estimé : 4 h 24 min					&\\
\end{tabularx}
\end{center} 

\begin{enumerate}
\item Expliquer pourquoi la différence entre les prix des 2 billets d'avion s'élève à 360~\euro{} pour ce couple. 
\item Si le couple prend la voiture pour aller à l'aéroport de Paris: 
	\begin{enumerate}
		\item Déterminer l'heure avant laquelle il doit partir de Nantes. 
		\item Montrer que le coût du carburant pour cet aller est de 31,90~\euro.
	\end{enumerate} 
\item Quelle est l'organisation de voyage la plus économique? 
\end{enumerate}
\end{document}\end{document}