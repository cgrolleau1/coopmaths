\textbf{Exercice 2 \hfill 4 points}

\medskip

%Pour construire un mur vertical, il faut parfois utiliser un coffrage et un étayage qui maintiendra la structure verticale le temps que le béton sèche. Cet étayage peut se représenter par le schéma suivant. Les poutres de fer sont coupées et fixées de façon que :
%
%\parbox{0.6\linewidth}{
%\setlength\parindent{5mm}
%\begin{itemize}
%\item[$\bullet~~$] Les segments [AB] et [AE] sont perpendiculaires ; 
%\item[$\bullet~~$] C est situé sur la barre [AB] ; 
%\item[$\bullet~~$] D est situé sur la barre [BE] ; 
%\item[$\bullet~~$] AB = 3,5 m ; AE = 2,625 m et CD = 1,5 m.
%\end{itemize}\setlength\parindent{0mm}}\hfill
%\parbox{0.35\linewidth}{\psset{unit=0.8cm}
%\begin{pspicture}(4.5,6)
%%\psgrid
%\pspolygon(0.5,0.5)(4.5,0.5)(4.5,5.5)
%\psline(2,2.4)(4.5,2.4)
%\uput[l](0.5,0.5){E} \uput[r](4.5,0.5){A} \uput[ur](4.5,5.5){B} 
%\uput[l](2,2.4){D} \uput[r](4.5,2.4){C} 
%\end{pspicture}
%}  
%
%\medskip

\begin{enumerate}
\item %Calculer BE.
Le triangle BAE est rectangle en A, on peut donc écrire d'après le théorème de Pythagore :

$\text{AB}^2 + \text{AE}^2  = \text{BE}^2$, soit $3,5^2 + 2,625^2 = \text{BE}^2 = \np{19,40625} = 4,375^2$.

Donc BE $ = 4,375$.
\item %Les barres [CD] et [AE] doivent être parallèles. 

%À quelle distance de B faut-il placer le point C ?
Si les droites sont parallèle, on a une situation où l'on peut utiliser le théorème de Thalès, soit :

$\dfrac{\text{BC}}{\text{BA}} =  \dfrac{\text{CD}}{\text{AE}}$ c'est-dire :

$\dfrac{\text{BC}}{3,5} =  \dfrac{1,5}{2,625}$ soit en multipliant chaque membre par 3,5  :

$\text{BC} = \dfrac{3,5 \times 1,5}{2,625}  = 2$.

Il faut placer C à 2 de B sur le segment [BA].
\end{enumerate}

\bigskip

