\textbf{Exercice 1 \hfill 4 points}

\medskip

On place des boules toutes indiscernables au toucher dans un sac. Sur chaque boule colorée est inscrite une lettre. Le tableau suivant présente la répartition des boules :

\begin{center}
\begin{tabularx}{0.8\linewidth}{|c|*{3}{>{\centering \arraybackslash}X|}}\hline 
\diagbox{Lettre}{Couleur}&Rouge&Vert&Bleu\\ \hline
A& 3&5& 2\\ \hline 
B& 2&2& 6\\ \hline
\end{tabularx}
\end{center} 
 
\begin{enumerate}
\item Combien y a-t-il de boules dans le sac ? 
\item On tire une boule au hasard, on note sa couleur et sa lettre.
	\begin{enumerate}
		\item Vérifier qu'il y a une chance sur dix de tirer une boule bleue portant la lettre A. 
		\item Quelle est la probabilité de tirer une boule rouge ? 
		\item A-t-on autant de chance de tirer une boule portant la lettre A que de tirer une boule portant la lettre B ?
	\end{enumerate}
\end{enumerate}
 
\bigskip

