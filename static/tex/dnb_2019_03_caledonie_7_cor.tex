
\medskip

\begin{enumerate}
	\item L'affirmation est \textbf{Vraie}.
	
	\textit{Justification :} Dans le triangle ABC, le côté le plus long est le côté [AB]. 
	
On a d'une part : $\text{AB}^2 = 7,5^2 = 56,25$ \qquad et, d'autre part : $\text{AC}^2  + \text{BC}^2 = 4,5^2 + 6^2 = 20,25 + 36 = 56,25$.
	
On a donc $\text{AB}^2 = \text{AC}^2 + \text{BC}^2$, donc, d'après la réciproque du théorème de Pythagore, on en déduit que le triangle ABC est rectangle (en C).
	
	\item L'affirmation est \textbf{Fausse}.
	
Un contre-exemple permet de l'établir : $(-1) \times (-2) \times 3 \times 4 \times 5 = 120$.
	
Le produit 120 est strictement positif, alors que deux des facteurs étaient négatifs (en fait, il faut et il suffit que le nombre de facteurs négatifs soit pair pour que le produit soit positif).
	
	\item L'affirmation est \textbf{Fausse}.
	
En effet, $\np[m]{56} \times \dfrac{1}{28}= \np[m]{2} = \np[cm]{200}$.
	
Le rapport de réduction correct est : $\dfrac{1}{280}$
\end{enumerate}

\vspace{0,5cm}

