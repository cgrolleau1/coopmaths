\textbf{\textsc{Exercice 3} \hfill 5 points}

\medskip

Trois figures codées sont données ci-dessous. Elles ne sont pas dessinées en vraie grandeur. 

Pour chacune d'elles, déterminer la longueur AB au millimètre près. 

\textbf{Dans cet exercice, on n'attend pas de démonstration rédigée. Il suffit d'expliquer brièvement le raisonnement suivi et de présenter clairement les calculs.}

\begin{center}
\begin{tabularx}{\linewidth}{|*{2}{>{\centering \arraybackslash}X|}}\hline 
\textbf{Figure 1} &\textbf{Figure 2}\\
\psset{unit=1cm}
\begin{pspicture}(5.75,3.)
%\psgrid
\pspolygon(0.5,0.5)(4,0.75)(1.6,2.3)%CAB
\uput[l](0.5,0.5){C} \uput[r](4,0.75){A} \uput[u](1.6,2.3){B} \uput[d](2.25,0.625){J}
\rput{-120}(1.6,2.3){\psframe(.3,.3)}
%\pscircle(2.25,0.625){1.78}
\psdots(2.25,0.625)
\psdots[dotstyle=+,dotangle=45](1.05,1.4)(1.375,0.575)(3.125,0.675) 
\rput(4.75,2.5){BC = 6~cm}
\end{pspicture}& \psset{unit=1cm}
\begin{pspicture}(5.75,3.)
\pspolygon(0.5,0.5)(4,0.5)(0.5,3)%ABC
\uput[ul](0.5,3){C} \uput[l](0.5,0.5){A} \uput[dr](4,0.5){B}
\uput[ur](2.25,1.75){36~cm}
\psarc(0.5,3){4mm}{-90}{-40}
\psframe(0.5,0.5)(0.8,0.8)
\rput(0.8,2.3){53\degres}
\end{pspicture}\\ \hline
\multicolumn{1}{|c}{\textbf{Figure 3}}&\multicolumn{1}{c|}{~}\\
\multicolumn{2}{|c|}{\psset{unit=1cm}\begin{pspicture}(-3,-1.5)(8,1.5)
%\psgrid
\pscircle(0,0){1.5}\psline(1.5;40)(1.5;220)
\uput[l](1.5;220){A} \uput[ur](1.5;40){B}
\uput[ul](0,0){O}
\psdots[dotstyle=+,dotangle=45](0.75;220)(0.75;40)
\psdots(0,0)
\uput[r](2.,0.5){[AB] est un diamètre du cercle de centre O.}
\uput[r](2.,-0.5){ La longueur du cercle est 154 cm.}
\end{pspicture}}\\ \hline
\end{tabularx}
\end{center}

\bigskip

