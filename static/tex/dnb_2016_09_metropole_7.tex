\textbf{\textsc{Exercice 7} \hfill 5 points}

\medskip

Une pizzeria fabrique des pizzas rondes  de 34~cm  de diamètre et des pizzas carrées de 34~cm de côté.

\begin{center}
\begin{tabularx}{\linewidth}{*{2}{\centering \arraybackslash}X}
\psset{unit=1cm}
\begin{pspicture}(-2.5,-2.5)(2.5,2.5)
\pscircle(0,0){2.25}
\multido{\n=0+45}{8}{\psline(2.25;\n)}
\end{pspicture}&\psset{unit=1cm}
\begin{pspicture}(-2.5,-2.5)(2.5,2.5)
\psframe(-2.25,-2.25)(2.25,2.25)
\multido{\n=-2.25+1.50}{3}{\psline(-2.25,\n)(2.25,\n)}
\multido{\n=-2.25+1.50}{3}{\psline(\n,-2.25)(\n,2.25)}
\end{pspicture}
\end{tabularx}
\end{center}

Toutes les pizzas

\begin{itemize}
\item[$\bullet~~$]ont la même épaisseur ;
\item[$\bullet~~$]sont livrées dans des boîtes identiques.
\end{itemize}
 
\medskip
 
Les  pizzas carrées coûtent 1~\euro{} de plus que les pizzas rondes.
 
 \medskip
 
\begin{enumerate}
\item Pierre achète deux pizzas : une ronde et une carrée. Il paye 14,20~\euro. Quel est le prix de chaque pizza ?
\item Les pizzas rondes sont découpées en huit parts de même taille et les pizzas carrées en neuf parts de même taille.

Dans quelle pizza trouve-t-on les parts les plus grandes ? 
\end{enumerate}

\newpage
\begin{center}
{\large \textbf{Annexe}}

{\large \textbf{À rendre avec la copie à la fin de l'épreuve}}

{\large \textbf{(À placer à l'intérieur de la copie)}}
\end{center}

\bigskip

\textbf{Fiche technique trouvée sur le blog}

\begin{center}
\begin{tabular}{|m{8cm} m{3.2cm}|}\hline
\multicolumn{2}{|c|}{\textbf{TRACER UNE ÉTOILE A CINQ BRANCHES}}\\
\textbf{1.~~}Tracer un cercle de centre O, puis tracer deux diamètres
perpendiculaires [AB] et [CD].

\textbf{2.~~} Placer le milieu du segment [OC]. Le nommer J.

\textbf{3.~~} Placer la pointe du compas sur J, placer le crayon sur C et tourner.

\textbf{4.~~} Représenter la demi-droite [JA]. Elle coupe ce cercle en M.

\textbf{5.~~} Placer la pointe du compas sur A, placer le crayon sur M et tourner.

\textbf{6.~~} Le cercle obtenu coupe le cercle de centre O et de rayon [OC] en E et F{}.

\textbf{7.~~} À partir du point F{}, reporter trois fois la longueur EF sur le cercle pour obtenir dans cet ordre les points G, H et I.

\textbf{8.~~} Tracer les segments [EG], [GI], [IF], [FH] et [HE].&\psset{unit=0.5cm}
\begin{pspicture}(-3,-3)(3,5)
\pscircle(0,0){3}
\pspolygon[fillstyle=solid,fillcolor=lightgray](3;45)(3;189)(3;333)(3;477)(3;621)
\uput[ur](3;45){F} \uput[ul](3;189){I} \uput[dr](3;333){G} 
\uput[ul](3;477){E} \uput[dl](3;621){H} 
\end{pspicture}\\ \hline
\end{tabular}
\end{center}

\medskip

\textbf{Construction débutée par Archibald}

\begin{center}
\begin{pspicture}(-3,-3)(3,5)
%\psgrid
\pscircle(0,0){3}
\psdots(0,0)(-3,0)(3,0)(0,-3)(0,3)(-1.5,0)(-0.8,1.36)(1.74,2.44)
\pscircle[linestyle=dashed](-1.5,0){1.5}
\psline(-3,0)(3,0)\psline(0,-3)(0,3)\uput[ul](0,3){A}
\psline[linestyle=dashed](-1.5,0)(1,5)
\pscircle[linestyle=dashed](0,3){1.82}
\uput[l](-3,0){C}\uput[ur](0,0){O}
%\pspolygon[fillstyle=solid,fillcolor=lightgray](3;45)(3;189)(3;333)(3;477)(3;621)
\end{pspicture}\\ \hline
\end{tabularx}
\end{center}

