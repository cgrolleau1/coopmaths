\textbf{Exercice 8 : Sphères de stockage \hfill 4 points}

\medskip

Le dépit de carburant de Koumourou, à  Ducos, dispose de trois sphères de stockage de butane.

\medskip

\begin{enumerate}
\item La plus grande sphère du dépôt a un diamètre de 19,7~m. Montrer que son volume de stockage est d'environ \np{4000}~m$^{3}$.

\emph{On rappelle que le volume d'une boule est donné par : $V = \dfrac{4}{3} \times \pi \times R^3$, où $R$ est le rayon de la boule.} 
\item Tous les deux mois, \np{1200} tonnes de butane sont importées sur le territoire. 

1 m$^3$ de butane pèse 580 kg. Quel est le volume, en m$^3$, correspondant aux \np{1200}~tonnes ? 

Arrondir le résultat à l'unité. 
\item  Les deux plus petites sphères ont des volumes de \np{1000} m$^3$ et 600 m$3$. Seront-elles suffisantes pour stocker les \np{1200}~tonnes de butane, ou bien aura-t-on besoin de la grande sphère ?

Justifier la réponse. 
\end{enumerate}
\newpage
\begin{center}

\textbf{ANNEXE 1 - Exercice 6}

\vspace{1cm}

\psset{unit=1cm}
\begin{pspicture}(10,14)
\psline(5,2)(5,10)
\psdots(5,2)(5,10)
\uput[l](5,2){B}\uput[l](5,10){A}
\end{pspicture} 

\newpage

\textbf{ANNEXE 2 - Exercice 7}

\vspace{3cm} 

\psset{unit=0.9cm}
\begin{pspicture*}(-0.75,-0.75)(13,11.5)
\psgrid[gridlabels=0,subgriddiv=1,gridcolor=cyan](13,12)
\psaxes[linewidth=1pt](0,0)(-0.75,-0.75)(13,11.5)
\psaxes[linewidth=1.5pt]{->}(0,0)(1,1)
\psplot[plotpoints=3000,linewidth=1.25pt,linecolor=blue]{-1.5}{6}{2 x mul}
\uput[dl](0,0){O}
\end{pspicture*}
\end{center}
