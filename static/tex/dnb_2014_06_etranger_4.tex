\documentclass[10pt]{article}
\usepackage[T1]{fontenc}
\usepackage[utf8]{inputenc}%ATTENTION codage UTF8
\usepackage{fourier}
\usepackage[scaled=0.875]{helvet}
\renewcommand{\ttdefault}{lmtt}
\usepackage{amsmath,amssymb,makeidx}
\usepackage[normalem]{ulem}
\usepackage{diagbox}
\usepackage{fancybox}
\usepackage{tabularx,booktabs}
\usepackage{colortbl}
\usepackage{pifont}
\usepackage{multirow}
\usepackage{dcolumn}
\usepackage{enumitem}
\usepackage{textcomp}
\usepackage{lscape}
\newcommand{\euro}{\eurologo{}}
\usepackage{graphics,graphicx}
\usepackage{pstricks,pst-plot,pst-tree,pstricks-add}
\usepackage[left=3.5cm, right=3.5cm, top=3cm, bottom=3cm]{geometry}
\newcommand{\R}{\mathbb{R}}
\newcommand{\N}{\mathbb{N}}
\newcommand{\D}{\mathbb{D}}
\newcommand{\Z}{\mathbb{Z}}
\newcommand{\Q}{\mathbb{Q}}
\newcommand{\C}{\mathbb{C}}
\usepackage{scratch}
\renewcommand{\theenumi}{\textbf{\arabic{enumi}}}
\renewcommand{\labelenumi}{\textbf{\theenumi.}}
\renewcommand{\theenumii}{\textbf{\alph{enumii}}}
\renewcommand{\labelenumii}{\textbf{\theenumii.}}
\newcommand{\vect}[1]{\overrightarrow{\,\mathstrut#1\,}}
\def\Oij{$\left(\text{O}~;~\vect{\imath},~\vect{\jmath}\right)$}
\def\Oijk{$\left(\text{O}~;~\vect{\imath},~\vect{\jmath},~\vect{k}\right)$}
\def\Ouv{$\left(\text{O}~;~\vect{u},~\vect{v}\right)$}
\usepackage{fancyhdr}
\usepackage[french]{babel}
\usepackage[dvips]{hyperref}
\usepackage[np]{numprint}
%Tapuscrit : Denis Vergès
%\frenchbsetup{StandardLists=true}

\begin{document}
\setlength\parindent{0mm}
% \rhead{\textbf{A. P{}. M. E. P{}.}}
% \lhead{\small Brevet des collèges}
% \lfoot{\small{Polynésie}}
% \rfoot{\small{7 septembre 2020}}
\pagestyle{fancy}
\thispagestyle{empty}
% \begin{center}
    
% {\Large \textbf{\decofourleft~Brevet des collèges Polynésie 7 septembre 2020~\decofourright}}
    
% \bigskip
    
% \textbf{Durée : 2 heures} \end{center}

% \bigskip

% \textbf{\begin{tabularx}{\linewidth}{|X|}\hline
%  L'évaluation prend en compte la clarté et la précision des raisonnements ainsi que, plus largement, la qualité de la rédaction. Elle prend en compte les essais et les démarches engagées même non abouties. Toutes les réponses doivent être justifiées, sauf mention contraire.\\ \hline
% \end{tabularx}}

% \vspace{0.5cm}\textbf{\textsc{Exercice 4} \hfill 5 points}

\medskip

\parbox{0.65\linewidth}{Paul en visite à Paris admire la Pyramide, réalisée en verre feuilleté au centre de la cour intérieure du Louvre. 
Cette pyramide régulière a :

\setlength\parindent{6mm} 
\begin{itemize}
\item[$\bullet~~$] pour base un carré ABCD de côté 35 mètres ; 
\item[$\bullet~~$] pour hauteur le segment [SO] de longueur 22 mètres.
\end{itemize}
\setlength\parindent{0mm}}
\hfill 	\parbox{0.3\linewidth}{\psset{unit=1cm}
\begin{pspicture}(3.3,4.4)
\psline(0.3,0.3)(2.3,0.3)(3,1.5)%ABC
\psline[linestyle=dashed](0.3,0.3)(1,1.5)(3,1.5)%ADC
\psline[linestyle=dashed](0.3,0.3)(3,1.5)%AC
\psline[linestyle=dashed](2.3,0.3)(1,1.5)%BD
\psline[linestyle=dashed](1.65,0.9)(1.65,3.9)(1,1.5)%OSD
\psline(0.3,0.3)(1.65,3.9)(2.3,0.3)%ASB
\psline(1.65,3.9)(3,1.5)%SC
\uput[dl](0.3,0.3){A} \uput[dr](2.3,0.3){B} \uput[r](3,1.5){C} \uput[ul](1,1.5){D} \uput[u](1.65,3.9){S} 
\end{pspicture}}

\medskip
	 
Paul a tellement apprécié cette pyramide qu'il achète comme souvenir de sa visite une lampe à huile dont le réservoir en verre est une réduction à l'échelle $\dfrac{1}{500}$ de la  vraie pyramide.
 
Le mode d'emploi de la lampe précise que, une fois allumée, elle brûle 4 cm$^3$ d'huile par heure.
 
Au bout de combien de temps ne restera-t-il plus d'huile dans le réservoir ? Arrondir à l'unité d'heures.
 
\textbf{Rappel :} \emph{Volume d'une pyramide = un tiers du produit de l'aire de la base par la hauteur}

\medskip
 
\textbf{Faire apparaitre sur la copie la démarche utilisée. Toute trace de recherche sera prise en compte lors de l'évaluation même si le travail n'est pas complètement abouti.}
 
\vspace{0,5cm}

\end{document}