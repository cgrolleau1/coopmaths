\textbf{Exercice 2 :\hfill 5 points}

\medskip

Heiata et Hiro ont choisi comme gâteau de mariage une pièce montée composée de 3 gâteaux cylindriques superposés, tous centrés sur l'axe (d) comme l'indique la figure ci-dessous :

\medskip
 
\parbox{0.4\linewidth}{
\psset{unit=0.8cm}
\begin{pspicture}(-3.3,0)(3.3,7.5)
\rput(0,2){\pstEllipse[linewidth=0.8pt]{3.3}{0.8}{-5}{-175}}
\rput(0,3.6){\pstEllipse[linewidth=0.8pt]{3.3}{0.8}{44}{-224}}
\rput(0,3.8){\pstEllipse[linewidth=0.8pt]{2.4}{0.6}{-5}{-175}}
\rput(0,5.1){\pstEllipse[linewidth=0.8pt]{2.4}{0.6}{56}{-236}}
\rput(0,5.4){\pstEllipse[linewidth=0.8pt]{1.3}{0.3}{-5}{-175}}
\psellipse(0,7)(1.3,0.3)
\psline(-1.3,5.4)(-1.3,7)\psline(1.3,5.4)(1.3,7)
\psline(-2.4,3.7)(-2.4,5.1)\psline(2.4,3.7)(2.4,5.1)
\psline(-3.3,1.95)(-3.3,3.57)\psline(3.3,1.95)(3.3,3.57)
\uput[r](0,7.5){(d)}
\psline[linewidth=0.2pt](0,1)(0,7.5)
\rput(0,0.5){La figure n'est pas à l'échelle}
\rput(-2.5,2.2){\no 1} \rput(-1.8,4){\no 2} \rput(-0.8,5.8){\no 3} 
\end{pspicture}}\hfill
\parbox{0.55\linewidth}{
\begin{itemize}
\item[$\bullet~~$] Les trois gâteaux cylindriques sont de même hauteur : 10 cm. 
\item[$\bullet~~$] Le plus grand gâteau cylindrique, le \no 1, a pour rayon 30 cm. 
\item[$\bullet~~$] Le rayon du gâteau \no 2 est égal au $\frac{2}{3}$ de  
celui du gâteau \no 1. 
\item[$\bullet~~$] Le rayon du gâteau \no 3 est égal au $\frac{3}{4}$ de  celui du gâteau \no 2.
\end{itemize}}

\medskip 
	 	 
\begin{enumerate}
\item Montrer que le rayon du gâteau \no 2 est de $20$~cm. 
\item Calculer le rayon du gâteau \no 3. 
\item Montrer que le volume total \textbf{exact} de la pièce montée est égal à \np{15250}$\pi$ cm$^3$.
 
Rappel :  le volume $V$ d'un cylindre de rayon $R$ et de hauteur $h$ est donné par la formule  $V = \pi \times R^2 \times h$. 
\item Quelle fraction du volume total représente le volume du gâteau \no 2 ? Donner le résultat sous forme de fraction irréductible. 
\end{enumerate}

\bigskip

