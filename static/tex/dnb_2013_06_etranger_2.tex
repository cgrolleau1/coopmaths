\documentclass[10pt]{article}
\usepackage[T1]{fontenc}
\usepackage[utf8]{inputenc}%ATTENTION codage UTF8
\usepackage{fourier}
\usepackage[scaled=0.875]{helvet}
\renewcommand{\ttdefault}{lmtt}
\usepackage{amsmath,amssymb,makeidx}
\usepackage[normalem]{ulem}
\usepackage{diagbox}
\usepackage{fancybox}
\usepackage{tabularx,booktabs}
\usepackage{colortbl}
\usepackage{pifont}
\usepackage{multirow}
\usepackage{dcolumn}
\usepackage{enumitem}
\usepackage{textcomp}
\usepackage{lscape}
\newcommand{\euro}{\eurologo{}}
\usepackage{graphics,graphicx}
\usepackage{pstricks,pst-plot,pst-tree,pstricks-add}
\usepackage[left=3.5cm, right=3.5cm, top=3cm, bottom=3cm]{geometry}
\newcommand{\R}{\mathbb{R}}
\newcommand{\N}{\mathbb{N}}
\newcommand{\D}{\mathbb{D}}
\newcommand{\Z}{\mathbb{Z}}
\newcommand{\Q}{\mathbb{Q}}
\newcommand{\C}{\mathbb{C}}
\usepackage{scratch}
\renewcommand{\theenumi}{\textbf{\arabic{enumi}}}
\renewcommand{\labelenumi}{\textbf{\theenumi.}}
\renewcommand{\theenumii}{\textbf{\alph{enumii}}}
\renewcommand{\labelenumii}{\textbf{\theenumii.}}
\newcommand{\vect}[1]{\overrightarrow{\,\mathstrut#1\,}}
\def\Oij{$\left(\text{O}~;~\vect{\imath},~\vect{\jmath}\right)$}
\def\Oijk{$\left(\text{O}~;~\vect{\imath},~\vect{\jmath},~\vect{k}\right)$}
\def\Ouv{$\left(\text{O}~;~\vect{u},~\vect{v}\right)$}
\usepackage{fancyhdr}
\usepackage[french]{babel}
\usepackage[dvips]{hyperref}
\usepackage[np]{numprint}
%Tapuscrit : Denis Vergès
%\frenchbsetup{StandardLists=true}

\begin{document}
\setlength\parindent{0mm}
% \rhead{\textbf{A. P{}. M. E. P{}.}}
% \lhead{\small Brevet des collèges}
% \lfoot{\small{Polynésie}}
% \rfoot{\small{7 septembre 2020}}
\pagestyle{fancy}
\thispagestyle{empty}
% \begin{center}
    
% {\Large \textbf{\decofourleft~Brevet des collèges Polynésie 7 septembre 2020~\decofourright}}
    
% \bigskip
    
% \textbf{Durée : 2 heures} \end{center}

% \bigskip

% \textbf{\begin{tabularx}{\linewidth}{|X|}\hline
%  L'évaluation prend en compte la clarté et la précision des raisonnements ainsi que, plus largement, la qualité de la rédaction. Elle prend en compte les essais et les démarches engagées même non abouties. Toutes les réponses doivent être justifiées, sauf mention contraire.\\ \hline
% \end{tabularx}}

% \vspace{0.5cm}\textbf{\textsc{Exercice 2} \hfill 4 points}

\medskip

\emph{On considère l'expérience aléatoire suivante: on tire au hasard une carte dans un jeu bien mélangé de $32$ cartes (il y a $4$ \og familles \fg{} cœur, trèfle, carreau et pique et on a $8$ cœurs, $8$ trèfles, $8$ carreaux et $8$ piques).\\
On relève pour la carte tirée la \og famille \fg{} (trèfle, carreau, cœur ou pique) puis on remet la carte dans le jeu et on mélange.}
 
On note $A$ l'évènement : \og la carte tirée est un trèfle \fg.

\medskip

\begin{enumerate}
\item Quelle est la probabilité de l'évènement A ? 
\item On répète 24 fois l'expérience aléatoire ci-dessus. La représentation graphique ci-dessous donne la répartition des couleurs obtenues lors des vingt-quatre premiers tirages: 

\begin{center}
\psset{xunit=1.2cm,yunit=0.4cm}
\begin{pspicture}(-3,-2)(5,11)
\psframe(-3,-2)(5,11)
\psline(4,0)(0,0)(0,10)
\multido{\n=0+2}{6}{\psline[linewidth=0.2pt](0,\n)(4,\n) \uput[l](0,\n){\n}}
\psframe[fillstyle=solid,fillcolor=lightgray](0.333,0)(0.666,6)
\psframe[fillstyle=solid,fillcolor=lightgray](1.333,0)(1.666,8)
\psframe[fillstyle=solid,fillcolor=lightgray](2.333,0)(2.666,3)
\psframe[fillstyle=solid,fillcolor=lightgray](3.333,0)(3.666,7)
\uput[d](0.5,0){cœur}\uput[d](1.5,0){trèfle} 
\uput[d](2.5,0){carreau}\uput[d](3.5,0){pique}
\rput(-1.5,8){nombre}
\rput(-1.5,6){de fois où}
\rput(-1.5,4){la carte}
\rput(-1.5,2){est tirée}
\end{pspicture}
\end{center} 
 
Calculer la fréquence d'une carte de la \og famille \fg{} cœur et d'une carte de la \og famille \fg{} trèfle. 
\item On reproduit la même expérience qu'à la question 2. Arthur mise sur une carte de la \og famille \fg{} cœur et Julie mise sur d'une carte de la \og famille \fg{} trèfle. 

Est-ce que l'un d'entre deux a plus de chance que l'autre de gagner ? 
\end{enumerate}
 
\bigskip

\end{document}