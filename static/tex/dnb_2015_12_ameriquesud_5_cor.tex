\textbf{\textsc{Exercice 5} \hfill 4 points}

\medskip
 
%Pour chacune des affirmations suivantes, dire si elle est vraie ou fausse.
%
%On rappelle que les réponses doivent être justifiées.
%
%\medskip

\textbf{Affirmation 1 :} %$n$ désigne un nombre entier naturel.

%L'expression $n^2 - 6n + 9$ est toujours différente de $0$.
$n^2 - 6n + 9 = (n - 3)^2$ : cette expression est nulle si $n = 3$. Affirmation fausse.
\smallskip

\textbf{Affirmation 2 :} %Un faucon pèlerin vole vers sa proie à une vitesse de 180~km/h. Il est plus rapide qu'un ballon de football tiré à la vitesse de 51~m/s.
Le ballon fait 51 m en 1 seconde donc  $51 \times 60 \times 60 = \np{183600}$~m en une heure, soit 183,6~km/h : il est plus rapide que le faucon. Affirmation fausse.
%%%%%%%%%%%%%%
\vspace{0.25cm}

