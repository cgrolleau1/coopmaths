
\medskip

%On considère le programme de calcul suivant :
%
%\begin{center}
%\begin{tabularx}{0.4\linewidth}{|lX|}\hline
%$\bullet~~$& Choisir un nombre ;\\
%$\bullet~~$& Le multiplier par - 4 ;\\
%$\bullet~~$& Ajouter 5 au résultat.\\ \hline
%\end{tabularx}
%\end{center}

\begin{enumerate}
\item %Vérifier que lorsque l'on choisit $- 2$ avec ce programme, on obtient 13.
On a $(- 2) \times (- 4) = 8$ et $8 + 5 = 13$.
\item %Quel nombre faut-il choisir au départ pour obtenir $- 3$ ?
On peut revenir au nombre de départ :

$- 3 - 5 = - 8$ puis $\dfrac{- 8}{- 4} = 2$.
\item %Salomé fait exécuter le script suivant:

%\begin{center}
%\textbf{Script}
%
%\begin{scratch}
%\blockinit{Quand \greenflag cliqué}
%\blocksensing{demander \txtbox{Choisir un nombre} et attendre}
%\blockifelse{si \boolsensing{- 4 * réponse + 5 < 0} alors}
%{\blocklook{dire \txtbox{Bravo}}
%}
%{\blocklook{dire \txtbox{Essaie encore}}}
%\end{scratch}
%\end{center}

	\begin{enumerate}
		\item %Quelle sera la réponse du lutin si elle choisit le nombre $12$ ?
		On a $- 4 \times 12 = - 48$ et $- 48 + 5 = - 43 < 0$. Le lutin dira Bravo. 
		\item %Quelle sera la réponse du lutin si elle choisit le nombre $- 5$ ?
		On a $ - 4 \times - 5 = 20$ et $20 + 5 = 25 >0$. La lutin dira Essaie encore.
 	\end{enumerate}
\item  %Le programme de calcul ci-dessus peut se traduire par l'expression littérale
$- 4x + 5$ avec $x$ représentant le nombre choisi.

%Résoudre l'inéquation suivante : $- 4x + 5 < 0$
Si $- 4x + 5 < 0$, alors $5 < 4x$ puis $\dfrac{5}{4} < x$ ou $x > \dfrac{5}{4}$.

Les nombres solutions sont les supérieurs à 1,25.
\item  %À quelle condition, portant sur le nombre choisi, est-on certain que la réponse
%du lutin sera \og Bravo \fg{} ?
Le lutin dira Bravo dès que lon choisira un nombre supérieur à 1,25.
\end{enumerate}

\bigskip

