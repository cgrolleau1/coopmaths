\documentclass[10pt]{article}
\usepackage[T1]{fontenc}
\usepackage[utf8]{inputenc}%ATTENTION codage UTF8
\usepackage{fourier}
\usepackage[scaled=0.875]{helvet}
\renewcommand{\ttdefault}{lmtt}
\usepackage{amsmath,amssymb,makeidx}
\usepackage[normalem]{ulem}
\usepackage{diagbox}
\usepackage{fancybox}
\usepackage{tabularx,booktabs}
\usepackage{colortbl}
\usepackage{pifont}
\usepackage{multirow}
\usepackage{dcolumn}
\usepackage{enumitem}
\usepackage{textcomp}
\usepackage{lscape}
\newcommand{\euro}{\eurologo{}}
\usepackage{graphics,graphicx}
\usepackage{pstricks,pst-plot,pst-tree,pstricks-add}
\usepackage[left=3.5cm, right=3.5cm, top=3cm, bottom=3cm]{geometry}
\newcommand{\R}{\mathbb{R}}
\newcommand{\N}{\mathbb{N}}
\newcommand{\D}{\mathbb{D}}
\newcommand{\Z}{\mathbb{Z}}
\newcommand{\Q}{\mathbb{Q}}
\newcommand{\C}{\mathbb{C}}
\usepackage{scratch}
\renewcommand{\theenumi}{\textbf{\arabic{enumi}}}
\renewcommand{\labelenumi}{\textbf{\theenumi.}}
\renewcommand{\theenumii}{\textbf{\alph{enumii}}}
\renewcommand{\labelenumii}{\textbf{\theenumii.}}
\newcommand{\vect}[1]{\overrightarrow{\,\mathstrut#1\,}}
\def\Oij{$\left(\text{O}~;~\vect{\imath},~\vect{\jmath}\right)$}
\def\Oijk{$\left(\text{O}~;~\vect{\imath},~\vect{\jmath},~\vect{k}\right)$}
\def\Ouv{$\left(\text{O}~;~\vect{u},~\vect{v}\right)$}
\usepackage{fancyhdr}
\usepackage[french]{babel}
\usepackage[dvips]{hyperref}
\usepackage[np]{numprint}
%Tapuscrit : Denis Vergès
%\frenchbsetup{StandardLists=true}

\begin{document}
\setlength\parindent{0mm}
% \rhead{\textbf{A. P{}. M. E. P{}.}}
% \lhead{\small Brevet des collèges}
% \lfoot{\small{Polynésie}}
% \rfoot{\small{7 septembre 2020}}
\pagestyle{fancy}
\thispagestyle{empty}
% \begin{center}
    
% {\Large \textbf{\decofourleft~Brevet des collèges Polynésie 7 septembre 2020~\decofourright}}
    
% \bigskip
    
% \textbf{Durée : 2 heures} \end{center}

% \bigskip

% \textbf{\begin{tabularx}{\linewidth}{|X|}\hline
%  L'évaluation prend en compte la clarté et la précision des raisonnements ainsi que, plus largement, la qualité de la rédaction. Elle prend en compte les essais et les démarches engagées même non abouties. Toutes les réponses doivent être justifiées, sauf mention contraire.\\ \hline
% \end{tabularx}}

% \vspace{0.5cm}\textbf{\textsc{Exercice 8} \hfill 4 points}

\medskip

%Sophie habite Toulouse et sa meilleure amie vient de déménager à Bordeaux. Elles
%décident de continuer à se voir. Sophie consulte les tarifs de train entre les deux
%villes :
%
%\setlength\parindent{8mm}
%\begin{itemize}
%\item un aller-retour coûte 40~\euro
%\item si elle achète un abonnement pour une année à 442~\euro, un aller-retour coûte
%alors moitié prix.
%\end{itemize}
%\setlength\parindent{0mm}
%
%Aider Sophie à choisir la formule la plus avantageuse en fonction du nombre de
%voyages.
%
%\emph{Dans cet exercice, toute trace de recherche, même non aboutie, sera prise en
%compte dans l'évaluation.}
Soit $x$ le nombre d'aller(s)-retour(s)

Sans abonnement Sophie paiera : $40x$ dans l'année.

Avec l'abonnement Sophie paiera : $442 + 20x$.

$\bullet~~$$40x < 442 + 20x$ ou $20x < 442$ ou $10x < 221$ et enfin $x < 22,1$ : jusqu'à 22 allers-retours il vaut mieux ne pas prendre l'abonnement.

$\bullet~~$$40x > 442 + 20x$ ou $20x > 442$ ou $10x > 221$ et enfin $x > 22,1$ : à partir de 23 allers-retours il est plus intéressant pour Sophie de prendre l'abonnement.

\emph{Remarque} : on peut aussi faire la représentation graphique de la fonction linéaire et de la fonction affine et lire pour quelles valeurs de $x$ l'une est en dessous de l'autre.
\end{document}\end{document}