
\medskip

\begin{enumerate}
\item 
	\begin{enumerate}
		\item Le rectangle \textcircled{3} est l'image du rectangle \textcircled{4} par la translation qui transforme C en E. 
		\item Le rectangle \textcircled{3} est l'image du rectangle \textcircled{1} par la rotation de centre F et d'angle $90$\degres{} dans le sens des aiguilles d'une montre. 
		\item Le rectangle ABCD est l'image du rectangle \textcircled{2} par l'homothétie de centre D et de rapport 3,
		
ou bien, le rectangle ABCD est l'image du rectangle \textcircled{3} par l'homothétie de centre B et de rapport 3,

ou bien,  le rectangle ABCD est l'image du rectangle \textcircled{4} par l'homothétie de centre C et de rapport 3.	
	\end{enumerate}
\item Un petit rectangle est donc une réduction du grand rectangle de rapport $\dfrac{1}{3}$. 

Son aire est : aire du grand $\times \left(\dfrac{1}{3}\right)^2 = 1,215  \times \dfrac{1}{9}  = 0,135$~m$^2$.

Dans une réduction de rapport $k$, les aires sont multipliées par $k^2$.
\item Soit $\ell$ la largeur et $L$ la longueur du rectangle ABCD. 

Le ratio longueur : largeur étant égal à $3 : 2$, on a $2L = 3\ell$, soit $L = 1,5\ell$.

On veut $\ell \times L = 1,215$, soit successivement :

$\ell \times 1,5  \ell = 1,215$ ; $1,5\ell^2 = 1,215$ ;
$\ell^2 = \dfrac{1,215}{1,5} = 0,81$ ; d'où $\ell = 0,9$.

On a alors $L = 1,5 \times  0,9 = 1,35$.

Le rectangle ABCD mesure 0,9 m sur 1,35 m.
\end{enumerate}

\bigskip

