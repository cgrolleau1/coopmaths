\textbf{Exercice 7 : Concours Australien \hfill 5,5 points}

\bigskip
 
L'épreuve du concours australien de mathématiques est divisée en trois catégories :

\setlength\parindent{8mm} 
\begin{itemize}
\item[$\bullet~~$] \og Junior \fg{} qui regroupe les classes de 5\up{e} et 4\up{e} 
\item[$\bullet~~$] \og Intermédiaire \fg{} pour les classes de 3\up{e} et 2\up{nde} 
\item[$\bullet~~$] \og Senior \fg{} avec les classes de 1\up{re} et de terminale.
\end{itemize}
\setlength\parindent{0mm} 
 
Cette année 25 établissements se sont inscrits. Plus de \np{3000}~élèves, répartis comme l'indique le tableau de l'annexe 1, ont participé à ce concours.

\medskip
 
\begin{enumerate}
\item Compléter le tableau de l'annexe 1 en page 5. Les cases barrées ne sont pas à remplir. 
\item Quel est le niveau où il y a le plus d'inscrits ? 
\item Quelle est la catégorie ayant le moins d'inscrits ? 
\item En moyenne, combien d'élèves par établissement ont participé? Arrondir à l'unité. 
\item Le tableau de l'annexe est une copie d'écran d'un tableur. 

Quelle formule faut-il écrire dans la case G5 pour obtenir l'effectif total ?
\end{enumerate}

\bigskip 

