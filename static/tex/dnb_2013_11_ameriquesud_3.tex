\textbf{Exercice 3 \hfill 7 points}

\medskip 

Un pâtissier a préparé 840 financiers* et \np{1176} macarons*. Il souhaite faire des lots, tous identiques, en mélangeant financiers et macarons. Il veut utiliser tous les financiers et tous les macarons. 

\medskip

\begin{enumerate}
\item 
	\begin{enumerate}
		\item Sans faire de calcul, expliquer pourquoi les nombres $840$ et \np{1176} ne sont pas premiers entre eux. 
		\item Le pâtissier peut-il faire $21$ lots? Si oui, calculer le nombre de financiers et le nombre de macarons dans chaque lot. 
		\item Quel est le nombre maximum de lots qu'il peut faire? Quelle sera alors la composition de chacun des lots? 
	\end{enumerate}
\item Cette année, chaque lot de $5$ financiers et $7$ macarons est vendu $22,40$~\euro. 

L'année dernière, les lots, composés de $8$ financiers et de $14$ macarons étaient vendus $42$~\euro. 

Sachant qu'aucun prix n'a changé entre les deux années, calculer le prix d'un financier et d'un macaron. 
\medskip
\end{enumerate}
* Les financiers et les macarons sont des pâtisseries. 


\bigskip

