\textbf{Exercice 2 \hfill 5 points}

\medskip

Dans chaque cas, dire si l'affirmation est vraie ou fausse (\emph{on rappelle que toutes les réponses doivent être justifiées}).

\medskip

\parbox{0.53\linewidth}{\textbf{Affirmation 1}

L'angle $\widehat{\text{ABC}}$ mesure au dixième de degré près 36,9\degres.}
\hfill
\parbox{0.45\linewidth}{\psset{unit=1cm}
\begin{pspicture}(-0.75,0)(5,3.5)
\pscurve(0.5,0.25)(2.5,0.22)(4.5,0.25)
\pscurve(4.5,0.25)(2.5,2.2)(0.5,3.5)
\pscurve(0.5,3.5)(0.45,2)(0.5,0.25)
\psframe(0.5,0.25)(0.8,0.55)
\uput[dl](0.5,0.25){A}\uput[r](4.5,0.25){B}\uput[ul](0.5,3.5){C}
\uput[l](0.5,1.875){3 cm}\uput[d](2.5,0.25){4 cm}\uput[ur](2.5,2.1){5 cm}
\end{pspicture}
}
\medskip

\textbf{Affirmation 2}

Le nombre 3 est une solution de l'équation $x^2 + 2x - 15 = 0$

\newpage

\textbf{Affirmation 3}

\parbox{0.5\linewidth}{Le prix avant la remise est de 63,70~\euro.}
\hfill
\parbox{0.5\linewidth}{\psset{unit=1cm}
\begin{pspicture}(5,3.5)
\psframe(5,3.5)
\rput(1.75,3){Prix avant remise : \ldots \euro}
\rput(2.5,2){\Large Soldes $ - 30$\,\%}
\rput(4,1){Nouveau prix}
\rput(4,0.5){49~\euro}
\end{pspicture}}

\medskip

\textbf{Affirmation 4}

On a plus de chance de gagner en choisissant l'urne 2.

\textbf{Règle du jeu :}

Deux urnes contiennent des boules indiscernables au toucher. On choisit une des deux urnes
et on en extrait une boule au hasard. On gagne si la boule obtenue est rouge.

\medskip
\begin{tabularx}{\linewidth}{|*{2}{>{\centering \arraybackslash}X|}}\hline
\psset{unit=1cm}
\begin{pspicture}(4,3)
\rput(2,2.5){\textbf{Urne 1}}
\rput(2,1.5){35 boules rouges}
\rput(2,1){et}
\rput(2,0.5){65 boules blanches}
\end{pspicture}&\psset{unit=1cm}
\begin{pspicture}(4,3)
\rput(2,2.5){\textbf{Urne 2}}
\rput(2,1.5){19 boules rouges}
\rput(2,1){et}
\rput(2,0.5){31 boules blanches}
\end{pspicture}\\ \hline
\end{tabularx}
\medskip

\vspace{0,5cm}

