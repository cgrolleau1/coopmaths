\documentclass[10pt]{article}
\usepackage[T1]{fontenc}
\usepackage[utf8]{inputenc}%ATTENTION codage UTF8
\usepackage{fourier}
\usepackage[scaled=0.875]{helvet}
\renewcommand{\ttdefault}{lmtt}
\usepackage{amsmath,amssymb,makeidx}
\usepackage[normalem]{ulem}
\usepackage{diagbox}
\usepackage{fancybox}
\usepackage{tabularx,booktabs}
\usepackage{colortbl}
\usepackage{pifont}
\usepackage{multirow}
\usepackage{dcolumn}
\usepackage{enumitem}
\usepackage{textcomp}
\usepackage{lscape}
\newcommand{\euro}{\eurologo{}}
\usepackage{graphics,graphicx}
\usepackage{pstricks,pst-plot,pst-tree,pstricks-add}
\usepackage[left=3.5cm, right=3.5cm, top=3cm, bottom=3cm]{geometry}
\newcommand{\R}{\mathbb{R}}
\newcommand{\N}{\mathbb{N}}
\newcommand{\D}{\mathbb{D}}
\newcommand{\Z}{\mathbb{Z}}
\newcommand{\Q}{\mathbb{Q}}
\newcommand{\C}{\mathbb{C}}
\usepackage{scratch}
\renewcommand{\theenumi}{\textbf{\arabic{enumi}}}
\renewcommand{\labelenumi}{\textbf{\theenumi.}}
\renewcommand{\theenumii}{\textbf{\alph{enumii}}}
\renewcommand{\labelenumii}{\textbf{\theenumii.}}
\newcommand{\vect}[1]{\overrightarrow{\,\mathstrut#1\,}}
\def\Oij{$\left(\text{O}~;~\vect{\imath},~\vect{\jmath}\right)$}
\def\Oijk{$\left(\text{O}~;~\vect{\imath},~\vect{\jmath},~\vect{k}\right)$}
\def\Ouv{$\left(\text{O}~;~\vect{u},~\vect{v}\right)$}
\usepackage{fancyhdr}
\usepackage[french]{babel}
\usepackage[dvips]{hyperref}
\usepackage[np]{numprint}
%Tapuscrit : Denis Vergès
%\frenchbsetup{StandardLists=true}

\begin{document}
\setlength\parindent{0mm}
% \rhead{\textbf{A. P{}. M. E. P{}.}}
% \lhead{\small Brevet des collèges}
% \lfoot{\small{Polynésie}}
% \rfoot{\small{7 septembre 2020}}
\pagestyle{fancy}
\thispagestyle{empty}
% \begin{center}
    
% {\Large \textbf{\decofourleft~Brevet des collèges Polynésie 7 septembre 2020~\decofourright}}
    
% \bigskip
    
% \textbf{Durée : 2 heures} \end{center}

% \bigskip

% \textbf{\begin{tabularx}{\linewidth}{|X|}\hline
%  L'évaluation prend en compte la clarté et la précision des raisonnements ainsi que, plus largement, la qualité de la rédaction. Elle prend en compte les essais et les démarches engagées même non abouties. Toutes les réponses doivent être justifiées, sauf mention contraire.\\ \hline
% \end{tabularx}}

% \vspace{0.5cm}\textbf{\textsc{Exercice 7} \hfill 5 points}

\medskip

On peut lire au sujet d'un médicament :
 
\og Chez les enfants (12 mois à 17 ans), la posologie doit être établie en fonction de la surface corporelle du patient [voir formule de Mosteller]. \fg
 
\og Une dose de charge unique de 70 mg par mètre carré (sans dépasser 70 mg par jour) devra être administrée \fg

\medskip
 
Pour calculer la surface corporelle en m$^2$ on utilise la formule suivante : 

Formule de Mosteller : Surface corporelle en m$^2 = \sqrt{\dfrac{\text{taille (en cm)} \times \text{masse (en kg}}{\np{3600}}}$.

\bigskip

On considère les informations ci-dessous :

\medskip
\begin{tabularx}{\linewidth}{|*{5}{>{\centering \arraybackslash}X|}}\hline 
Patient &Âge &Taille (m) &Masse (kg) &Dose administrée\\ \hline 
Lou &5 ans 	&1,05 &17,5 &50 mg\\ \hline  
Joé &15 ans &1,50 &50 	&100 mg\\ \hline 
\end{tabularx}
\medskip 

\begin{enumerate}
\item La posologie a-t-elle été respectée pour Joé ? Justifier la réponse. 
\item Vérifier que la surface corporelle de Lou est environ de $0,71$ m$^2$. 

\medskip

\textbf{Dans cette question, toute trace de recherche, même incomplète, sera prise en compte dans l'évaluation.}
 
\item La posologie a-t-elle été respectée pour Lou ? Justifier la réponse 
\end{enumerate}
\end{document}\end{document}