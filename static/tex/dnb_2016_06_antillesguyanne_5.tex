\textbf{\textsc{Exercice 5} \hfill 5,5 points}

\medskip 

\parbox{0.55\linewidth}{La figure PRC ci-contre représente un terrain appartenant à une commune. 

Les points P{}, A et R sont alignés. 

Les points P{}, S et C sont alignés. 

Il est prévu d'aménager sur ce terrain: 

\begin{itemize}
\item[$\bullet~~$] une \og zone de jeux pour enfants\fg{} sur la partie PAS ; 
\item[$\bullet~~$] un \og skatepark \fg{} sur la partie RASC. 
\end{itemize}

On connaît les dimensions suivantes : 

PA = 30~m ; AR = 10~m ; AS = 18~m.}\hfill \parbox{0.45\linewidth}{\psset{unit=0.9cm}
\begin{pspicture}(6,8)
%\psgrid
\pspolygon(1.5,0.5)(5,1.2)(0.5,7)%RCP
\psline(1.2,2.2)(3.9,2.7)
\rput{10}(1.5,0.5){\psframe(0.3,0.3)}
\rput{10}(1.22,2.2){\psframe(0.3,0.3)}
\uput[u](3.5,6){zone de jeux pour enfants}
\psline{->}(3.5,6)(2,4)
\uput[u](5.1,2.3){skatepark}\psline{->}(5,2.4)(3.8,1.8)
\uput[l](1.5,0.5){R} \uput[dr](5,1.2){C} \uput[u](0.5,7){P} \uput[l](1.2,2.2){A} \uput[ur](3.9,2.7){S}
\pspolygon[fillstyle=hlines](1.22,2.2)(3.86,2.7)(5,1.2)(1.5,0.5) 
\end{pspicture}
}

\begin{enumerate}
\item La commune souhaite semer du gazon sur la \og zone de jeux pour enfants\fg. Elle décide d'acheter des sacs de $5$~kg de mélange de graines pour gazon à 13,90~\euro{} l'unité. Chaque sac permet de couvrir une surface d'environ 140~m$^2$. 

Quel budget doit prévoir cette commune pour pouvoir semer du gazon sur la totalité de la \og zone de jeux pour enfants\fg{} ? 
\item Calculer l'aire du \og skatepark \fg. 
\end{enumerate}

\bigskip

