
\medskip

L'entraîneur d'un club d'athlétisme a relevé les performances de ses lanceuses de poids
sur cinq lancers. Voici une partie des relevés qu'il a effectués (il manque trois
performances pour une des lanceuses) :

\begin{center}
\begin{tabularx}{\linewidth}{|m{2cm}|*{6}{>{\centering \arraybackslash}X|}}\cline{3-7}
\multicolumn{2}{c|}{~}						&\multicolumn{5}{c}{Lancers}\\ \cline{3-7}
\multicolumn{2}{c|}{~}						&\no 1 		&\no 2 	&\no 3 	&\no 4 	&\no 5\\\hline
\multirow{3}{2cm}{Performances (en mètre)}	&Solenne 	&17,8 	&17,9 	&18 	&19,9 	&17,4\\ \cline{2-7}
											&Rachida 	&17,9 	&17,6 	&18,5 	&18 	&19\\ \cline{2-7}
											&Sarah 		&18 	&$?$ 	&19,5 	&$?$ 	&$?$\\ \hline
\end{tabularx}
\end{center}

\medskip

On connaît des caractéristiques de la série d'une des lanceuses :

\begin{center}
\begin{tabular}{|c|}\hline
\textbf{Caractéristiques des cinq lancers :}\\\hline
Étendue : 2,5 m\\ \hline
Moyenne : 18,2 m\\ \hline
Médiane : 18 m\\ \hline
\end{tabular}
\end{center}

\begin{enumerate}
\item Expliquer pourquoi ces caractéristiques ne concernent ni les résultats de Solenne, ni
ceux de Rachida.
\item Les caractéristiques données sont donc celles de Sarah. Son meilleur lancer est
de 19,5 m.

Indiquer sur la copie quels peuvent être les trois lancers manquants de Sarah ?
\end{enumerate}

\vspace{0,5cm}

