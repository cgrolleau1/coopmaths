
\medskip

\begin{enumerate}
\item Le script carré trace un carré en traçant 4 fois deux demi-côtés de 5 pixels, donc chaque côté du carré correspond à 10 pixels, donc à 5 cm.
\item Le script 1 dessine 23 fois un carré suivi d'un tiret, donc le dessin B.

Le script 2 dessine 46 fois de manière aléatoire un carré ou un tiret, donc le dessin A.
\item 
	\begin{enumerate}
		\item En exécutant le script 2, le premier élément tracé est un carré si le nombre aléatoire prend l'un des deux valeurs possible. La probabilité est $0,5$. 
		\item Pour les deux premiers éléments dessinés, il y a 4 possibilités équiprobables :
		
carré – carré ; carré – tiret ; tiret – carré ; tiret – tiret.

La probabilité que les deux premiers éléments dessinés soient des carrés est $\dfrac{1}{4}$, soit $0,25$.
	\end{enumerate} 
\item Au niveau de la ligne 7 du script 2, on peut insérer :

si nombre aléatoire entre 1 et 2 = 1 alors mettre la couleur du stylo à rouge
sinon 

mettre la couleur du stylo à noir.
\end{enumerate}
\bigskip

