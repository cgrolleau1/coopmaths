\documentclass[10pt]{article}
\usepackage[T1]{fontenc}
\usepackage[utf8]{inputenc}%ATTENTION codage UTF8
\usepackage{fourier}
\usepackage[scaled=0.875]{helvet}
\renewcommand{\ttdefault}{lmtt}
\usepackage{amsmath,amssymb,makeidx}
\usepackage[normalem]{ulem}
\usepackage{diagbox}
\usepackage{fancybox}
\usepackage{tabularx,booktabs}
\usepackage{colortbl}
\usepackage{pifont}
\usepackage{multirow}
\usepackage{dcolumn}
\usepackage{enumitem}
\usepackage{textcomp}
\usepackage{lscape}
\newcommand{\euro}{\eurologo{}}
\usepackage{graphics,graphicx}
\usepackage{pstricks,pst-plot,pst-tree,pstricks-add}
\usepackage[left=3.5cm, right=3.5cm, top=3cm, bottom=3cm]{geometry}
\newcommand{\R}{\mathbb{R}}
\newcommand{\N}{\mathbb{N}}
\newcommand{\D}{\mathbb{D}}
\newcommand{\Z}{\mathbb{Z}}
\newcommand{\Q}{\mathbb{Q}}
\newcommand{\C}{\mathbb{C}}
\usepackage{scratch}
\renewcommand{\theenumi}{\textbf{\arabic{enumi}}}
\renewcommand{\labelenumi}{\textbf{\theenumi.}}
\renewcommand{\theenumii}{\textbf{\alph{enumii}}}
\renewcommand{\labelenumii}{\textbf{\theenumii.}}
\newcommand{\vect}[1]{\overrightarrow{\,\mathstrut#1\,}}
\def\Oij{$\left(\text{O}~;~\vect{\imath},~\vect{\jmath}\right)$}
\def\Oijk{$\left(\text{O}~;~\vect{\imath},~\vect{\jmath},~\vect{k}\right)$}
\def\Ouv{$\left(\text{O}~;~\vect{u},~\vect{v}\right)$}
\usepackage{fancyhdr}
\usepackage[french]{babel}
\usepackage[dvips]{hyperref}
\usepackage[np]{numprint}
%Tapuscrit : Denis Vergès
%\frenchbsetup{StandardLists=true}

\begin{document}
\setlength\parindent{0mm}
% \rhead{\textbf{A. P{}. M. E. P{}.}}
% \lhead{\small Brevet des collèges}
% \lfoot{\small{Polynésie}}
% \rfoot{\small{7 septembre 2020}}
\pagestyle{fancy}
\thispagestyle{empty}
% \begin{center}
    
% {\Large \textbf{\decofourleft~Brevet des collèges Polynésie 7 septembre 2020~\decofourright}}
    
% \bigskip
    
% \textbf{Durée : 2 heures} \end{center}

% \bigskip

% \textbf{\begin{tabularx}{\linewidth}{|X|}\hline
%  L'évaluation prend en compte la clarté et la précision des raisonnements ainsi que, plus largement, la qualité de la rédaction. Elle prend en compte les essais et les démarches engagées même non abouties. Toutes les réponses doivent être justifiées, sauf mention contraire.\\ \hline
% \end{tabularx}}

% \vspace{0.5cm}\textbf{Exercice 8 : Le faré \hfill 5 points}

\medskip

François aide son papa à reconstruire le faré du jardin.

Le toit a la forme d'une pyramide à base carrée représentée ci-dessous.

François doit acheter du bois de charpente pour refaire les traverses de ce toit à quatre pans.
\begin{center}
\psset{unit=1cm}
\begin{pspicture}(10,7)
%\psgrid
\pspolygon(0.5,0.5)(7,0.5)(9.7,2.2)(5.1,6.6)(7,0.5)%AB?CBA
\psline(0.5,0.5)(5.1,6.6)%AC
\psline[linestyle=dashed](0.5,0.5)(9.7,2.2)
\psline[linestyle=dashed](3.2,2.2)(7,0.5)
\psline[linestyle=dashed](0.5,0.5)(3.2,2.2)(9.7,2.2)
\psline[linestyle=dashed](3.2,2.2)(5.1,6.6)(5.1,1.35)
\psline(2.1,2.65)(6.37,2.65)(8.12,3.7)%GF--
\psline[linestyle=dashed](2.1,2.65)(3.85,3.7)(8.12,3.7)%G--
\psline(3.5,4.5)(5.8,4.5)(6.7,5.1)%DE--
\psline[linestyle=dashed](3.5,4.5)(4.4,5.1)(6.7,5.1)%D--
\uput[dl](0.5,0.5){A} \uput[dr](7,0.5){B} \uput[u](5.1,6.6){C} \uput[ul](3.5,4.5){D} 
\uput[ur](5.7,4.55){E} \uput[ur](6.27,2.65){F} \uput[ul](2.1,2.65){G} \uput[d](5.1,1.38){H}
\rput(8.4,4.8){traverses}
\psline{->}(8.4,4.6)(6.2,4.8)
\psline{->}(8.4,4.6)(7.3,3.2)
\psline{->}(8.4,4.6)(8.6,1.52) 
\end{pspicture}

AC = 3,60 m, \quad AH = 2,88 m,\quad CH = 2,16 m
\end{center}

\begin{enumerate}
\item Montrer que le triangle ACH est rectangle en H.
\item On a représenté ci-dessous le pan ABC.

\parbox{0.5\linewidth}{\psset{unit=0.9cm}
\begin{pspicture}(6.3,5)
%\psgrid
\def\barbar{\psline(-0.15,0.15)(0.15,-0.15)\psline(-0.08,0.15)(0.22,-0.15)}
\def\barbard{\psline(0.15,0.15)(-0.15,-0.15)\psline(0.08,0.15)(-0.23,-0.15)}
\pspolygon(0.5,0.5)(5.8,0.5)(3.15,4.5)%ABC
\psline(1.38,1.8)(4.92,1.8)%GF
\psline(2.2,3.1)(4.1,3.1)%DE
\uput[dl](0.5,0.5){A} \uput[dr](5.8,0.5){B} \uput[u](3.15,4.5){C} 
\uput[ul](2.2,3.1){D} \uput[ur](4.1,3.1){E} \uput[ur](4.92,1.8){F} 
\uput[ul](1.38,1.8){G}
\rput(1,1.3){\barbar}\rput(1.8,2.5){\barbar}\rput(2.7,3.8){\barbar}
\rput(5.3,1.23){\barbard}\rput(4.5,2.5){\barbard}\rput(3.6,3.8){\barbard}
\end{pspicture}
}\hfill\parbox{0.48\linewidth}{ABC est un triangle isocèle en C.

AC = 3,60 m

Les distances AG, GD, DC, CE, EF et FB sont
égales.

Les droites (DE), (GF) et (AB) sont parallèles.}

	\begin{enumerate}
		\item Le pan ABC comprend trois traverses [DE], [GF] et [AB].
François a coupé une traverse [AB] de 4,08 m.
Calculer DE.
		\item On donne de plus GF = 2,72 m. Les quatre pans de la toiture sont identiques.
Calculer la longueur totale des traverses nécessaires pour refaire la toiture.
	\end{enumerate} 
\end{enumerate}
\end{document}\end{document}