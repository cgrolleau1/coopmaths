\documentclass[10pt]{article}
\usepackage[T1]{fontenc}
\usepackage[utf8]{inputenc}%ATTENTION codage UTF8
\usepackage{fourier}
\usepackage[scaled=0.875]{helvet}
\renewcommand{\ttdefault}{lmtt}
\usepackage{amsmath,amssymb,makeidx}
\usepackage[normalem]{ulem}
\usepackage{diagbox}
\usepackage{fancybox}
\usepackage{tabularx,booktabs}
\usepackage{colortbl}
\usepackage{pifont}
\usepackage{multirow}
\usepackage{dcolumn}
\usepackage{enumitem}
\usepackage{textcomp}
\usepackage{lscape}
\newcommand{\euro}{\eurologo{}}
\usepackage{graphics,graphicx}
\usepackage{pstricks,pst-plot,pst-tree,pstricks-add}
\usepackage[left=3.5cm, right=3.5cm, top=3cm, bottom=3cm]{geometry}
\newcommand{\R}{\mathbb{R}}
\newcommand{\N}{\mathbb{N}}
\newcommand{\D}{\mathbb{D}}
\newcommand{\Z}{\mathbb{Z}}
\newcommand{\Q}{\mathbb{Q}}
\newcommand{\C}{\mathbb{C}}
\usepackage{scratch}
\renewcommand{\theenumi}{\textbf{\arabic{enumi}}}
\renewcommand{\labelenumi}{\textbf{\theenumi.}}
\renewcommand{\theenumii}{\textbf{\alph{enumii}}}
\renewcommand{\labelenumii}{\textbf{\theenumii.}}
\newcommand{\vect}[1]{\overrightarrow{\,\mathstrut#1\,}}
\def\Oij{$\left(\text{O}~;~\vect{\imath},~\vect{\jmath}\right)$}
\def\Oijk{$\left(\text{O}~;~\vect{\imath},~\vect{\jmath},~\vect{k}\right)$}
\def\Ouv{$\left(\text{O}~;~\vect{u},~\vect{v}\right)$}
\usepackage{fancyhdr}
\usepackage[french]{babel}
\usepackage[dvips]{hyperref}
\usepackage[np]{numprint}
%Tapuscrit : Denis Vergès
%\frenchbsetup{StandardLists=true}

\begin{document}
\setlength\parindent{0mm}
% \rhead{\textbf{A. P{}. M. E. P{}.}}
% \lhead{\small Brevet des collèges}
% \lfoot{\small{Polynésie}}
% \rfoot{\small{7 septembre 2020}}
\pagestyle{fancy}
\thispagestyle{empty}
% \begin{center}
    
% {\Large \textbf{\decofourleft~Brevet des collèges Polynésie 7 septembre 2020~\decofourright}}
    
% \bigskip
    
% \textbf{Durée : 2 heures} \end{center}

% \bigskip

% \textbf{\begin{tabularx}{\linewidth}{|X|}\hline
%  L'évaluation prend en compte la clarté et la précision des raisonnements ainsi que, plus largement, la qualité de la rédaction. Elle prend en compte les essais et les démarches engagées même non abouties. Toutes les réponses doivent être justifiées, sauf mention contraire.\\ \hline
% \end{tabularx}}

% \vspace{0.5cm}\textbf{Exercice 6 \hfill 12 points}

\medskip

Une personne pratique le vélo de piscine depuis plusieurs années dans un centre
aquatique à raison de deux séances par semaine. Possédant une piscine depuis peu,
elle envisage d'acheter un vélo de piscine pour pouvoir l'utiliser exclusivement chez
elle et ainsi ne plus se rendre au centre aquatique.

\medskip

\setlength\parindent{6mm}
\begin{itemize}
\item[$\bullet~~$] Prix de la séance au centre aquatique: 15~\euro.
\item[$\bullet~~$] Prix d'achat d'un vélo de piscine pour une pratique à la maison: 999~\euro.
\end{itemize}
\setlength\parindent{0mm}

\medskip

\begin{enumerate}
\item Montrer que 10 semaines de séances au centre aquatique lui coûtent 300~\euro.
\item Que représente la solution affichée par le programme ci-après?

\begin{center}
\begin{scratch}
\blockinit{quand \greenflag est cliqué}
\blockvariable{mettre \selectmenu{x} à \txtbox{0}}
\blockinfloop{répéter jusqu'à \booloperator{\ovalvariable{x} * 2 * 15 > \txtbox{999}}}
{\blockvariable{ajouter à \selectmenu{x} \ovalnum{1}}
}
\blocklook{dire \ovaloperator{regroupe}{\txtbox{La solution est :}}\ovalvariable{x}}

\end{scratch}
\end{center}

\item  Combien de semaines faudrait-il pour que l'achat du vélo de piscine soit
rentabilisé?
\end{enumerate}

\bigskip

\end{document}