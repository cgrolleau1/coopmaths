\textbf{Exercice 5 : Changement climatique \hfill 3,5 points}

\medskip 

Le tableau ci-dessous présente l'évolution des températures minimales $\left(T_{\text{min}}\right)$ et des températures maximales $\left(T_{\text{max}}\right)$ observées en différents endroits de la Nouvelle-Calédonie au cours des quarante dernières années: 

\medskip
\begin{tabularx}{\linewidth}{|c|*{10}{>{\centering \arraybackslash}X|}}\hline
 &\tiny Nouméa&\tiny  Vaté &\tiny Thio&\tiny Nessadiou&\tiny  Houailou& \tiny Poindimié&\tiny  Koné&\tiny  Koumac&\tiny  La Roche&\tiny Ouanaham\\ \hline
$\left(T_{\text{min}}\right)\degres$~C& +1,3& +1,3&+1,2&+1,2&+1,2&+1,3&+1,2&+1,2&+1,5&+1,3\\ \hline      
$\left(T_{\text{max}}\right)\degres$~C& +1,3&+1,3&+1,0 &+0,9&+1,0&+1,0& +0,8&+0,9 &+1,0&+0,9\\ \hline   
\end{tabularx}
\medskip

\begin{enumerate}
\item Les informations de ce tableau traduisent-elles une augmentation des températures en NouvelleCalédonie? Justifier. 
\item En quel endroit la température minimale a-t-elle le plus augmenté ? 
\item Calculer l'augmentation moyenne des températures minimales et celle des températures maximales. 
\end{enumerate}

\vspace{0,5cm}

