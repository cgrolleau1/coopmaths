\textbf{Exercice 2 \hfill 4 points}

\medskip

Pour construire un mur vertical, il faut parfois utiliser un coffrage et un étayage qui maintiendra la structure verticale le temps que le béton sèche. Cet étayage peut se représenter par le schéma suivant. Les poutres de fer sont coupées et fixées de façon que :

\parbox{0.6\linewidth}{
\setlength\parindent{5mm}
\begin{itemize}
\item[$\bullet~~$] Les segments [AB] et [AE] sont perpendiculaires ; 
\item[$\bullet~~$] C est situé sur la barre [AB] ; 
\item[$\bullet~~$] D est situé sur la barre [BE] ; 
\item[$\bullet~~$] AB = 3,5 m ; AE = 2,625 m et CD = 1,5 m.
\end{itemize}\setlength\parindent{0mm}}\hfill
\parbox{0.35\linewidth}{\psset{unit=0.8cm}
\begin{pspicture}(4.5,6)
%\psgrid
\pspolygon(0.5,0.5)(4.5,0.5)(4.5,5.5)
\psline(2,2.4)(4.5,2.4)
\uput[l](0.5,0.5){E} \uput[r](4.5,0.5){A} \uput[ur](4.5,5.5){B} 
\uput[l](2,2.4){D} \uput[r](4.5,2.4){C} 
\end{pspicture}
}  

\medskip

\begin{enumerate}
\item Calculer BE. 
\item Les barres [CD] et [AE] doivent être parallèles. 

À quelle distance de B faut-il placer le point C ? 
\end{enumerate}

\bigskip

