\textbf{\textsc{Exercice 3} \hfill 5,5 points}

\medskip

\begin{enumerate}
\item  = SOMME(B2 : B14)
\item Production de l'Indonésie et de Madagascar : $\np{3200} + \np{3100} = \np{6300}$ milliers de tonnes, ce qui représente $\dfrac{\np{6300}}{\np{8342}} \times 100 \approx  75,5$\,\% de la production mondiale.

À eux deux, l'Indonésie et Madagascar produisent donc plus des trois quarts de la production mondiale de vanille.
\item Les cinq pays qui ont produit le moins de vanille en 2013 sont le Zimbabwe, le Kenya, le Malawi, les Comores et la France.

La production totale de ces cinq pays est égale à  : $11 + 15 + 22 + 35 + 79 = 162$ milliers de tonnes.

Pourcentage de la production mondiale que représente la production de vanille de ces cinq pays : 

$\dfrac{162}{\np{8342}} \times 100 \approx  2$\,\%.
\end{enumerate}

\vspace{0,5cm}

