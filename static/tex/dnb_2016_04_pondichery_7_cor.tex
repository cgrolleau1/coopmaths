\textbf{\textsc{Exercice 7 \hfill 3 points}}

\medskip

%Dans ce questionnaire à choix multiples, pour chaque question, des réponses sont proposées et
%une seule est exacte.
%
%Pour chacune des questions, écrire le numéro de la question et la lettre de la bonne réponse.
%
%Aucune justification n'est attendue.
%
%\begin{center}
%\begin{tabularx}{\linewidth}{|m{4.5cm}|*{3}{>{\centering \arraybackslash}X|}}\hline
%Questions &Réponse A &Réponse B &Réponse C\\ \hline
%1. $(2x-3)^2 = \ldots$&$4x^2 + 12x - 9$ &$4x^2 - 12x + 9$ &$4x^2 - 9$\\ \hline
%2. L'équation $(x + 1)(2x - 5) = 0$ a pour solutions \ldots&1 et 2,5&  $- 1$ et $- 2,5$ &$- 1$ et $2,5$\\ \hline
%3. Si $a > 0$ alors $\sqrt{a} + \sqrt{a} = \ldots$&\rule[-3mm]{0mm}{8mm}$a$&$2\sqrt{a}$&$\sqrt{2a}$\\ \hline
%\end{tabularx}
%\end{center}
\begin{enumerate}
\item $(2x - 3)^2 =  4x^2 + 9 - 2 \times 2x \times 3 = 4x^2 + 9 - 12x$ : réponse B.
\item $(x + 1)(2x - 5) = 0$ entraîne $\left\{\begin{array}{l c l}
x + 1&=&0\ \text{ou}\\
2x - 5&=&0
\end{array}\right.$ soit $\left\{\begin{array}{l c l}
x &=&- 1\ \text{ou}\\
x &=&\frac{5}{2}
\end{array}\right.$ Réponse C.
\item $\sqrt{a} + \sqrt{a} = 2\sqrt{a}$. Réponse B.
\end{enumerate}

\vspace{0,5cm}

