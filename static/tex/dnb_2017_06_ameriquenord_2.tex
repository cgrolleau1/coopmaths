
\medskip

Avec un logiciel de géométrie, on exécute le programme ci-dessous.

\medskip

\parbox{0.45\linewidth}{Programme de construction :}\hfill\parbox{0.53\linewidth}{\qquad \qquad Figure obtenue:}

\parbox{0.45\linewidth}{$\bullet~~$ Construire un carré ABCD ;

$\bullet~~$ Tracer le cercle de centre A et de rayon [AC] ;

$\bullet~~$Placer le point E à l'intersection du cercle et
de la demi-droite [AB) ;

$\bullet~~$Construire un carré DEFG.}\hfill
\parbox{0.53\linewidth}{ 
\psset{unit=1cm}
\begin{pspicture}(-2.2,-2)(4.2,4.2)
%\psgrid
\psframe[fillstyle=solid,fillcolor=lightgray](1.414,1.414)
\rput{55.5}(2,0){\psframe[fillstyle=solid,fillcolor=lightgray](0,0)(2.4495,2.4495)}
\psframe[fillstyle=solid,fillcolor=lightgray](1.414,1.414)
\pscircle(0,0){2}
\pspolygon[fillstyle=solid,fillcolor=gray](0,1.414)(1.414,0.414)(1.414,1.414)
\psline(4.2,0)

\uput[dl](0,0){A} \uput[d](1.414,0){B} \uput[ur](1.414,1.414){C} 
\uput[ul](0,1.414){D} \uput[dr](2,0){E} \uput[r](3.4,2){F} 
\uput[u](1.414,3.4){G}
\rput(1.1,4){} 
\end{pspicture}
}

\begin{enumerate}
\item Sur la copie, réaliser la construction avec AB $=3$~cm.
\item Dans cette question, AB $= 10~$cm.
	\begin{enumerate}
		\item Montrer que AC $= \sqrt{200}$~cm.
		\item Expliquer pourquoi AE $= \sqrt{200}$~cm.
		\item Montrer que l'aire du carré DEFG est le triple de l'aire du carré ABCD.
	\end{enumerate}
\item On admet pour cette question que pour n'importe quelle longueur du côté [AB],
l'aire du carré DEFG est toujours le triple de l'aire du carré ABCD.
	
En exécutant ce programme de construction, on souhaite obtenir un carré DEFG ayant
une aire de 48 cm$^2$.
	
Quelle longueur AB faut-il choisir au départ ?
\end{enumerate}

\vspace{0,5cm}

