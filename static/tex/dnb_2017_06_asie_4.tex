
\medskip

Margot a écrit le programme suivant. Il permet de dessiner avec trois touches du clavier.

\begin{center}
\begin{tabularx}{\linewidth}{|*{3}{>{\scriptsize\centering \arraybackslash}X|}}\hline
\begin{scratch} \blockinit{quand \greenflag est cliqué}
\blockvariable{initialisation}
\end{scratch}	&								&\\

\begin{scratch} 
\blockinit{quand \selectmenu{flèche haut} est cliqué}	
\blockmove{s'orienter à \ovalnum{0\selectarrownum}}
\blockpen{stylo en position d'écriture}
\blockmove{avancer de \ovalnum{50}}
\blockpen{relever le stylo}
\end{scratch}				&
\begin{scratch} \blockinit{quand \selectmenu{flèche droite} est cliqué}
\blockmove{s'orienter à \ovalnum{90\selectarrownum}}
\blockpen{stylo en position d'écriture}
\blockmove{avancer de \ovalnum{50}}
\blockpen{relever le stylo}
\end{scratch}				&
\begin{scratch} \blockinit{quand \selectmenu{flèche bas} est cliqué}
\blockmove{s'orienter à \ovalnum{180\selectarrownum}}
\blockpen{stylo en position d'écriture}
\blockmove{avancer de \ovalnum{50}}
\blockpen{relever le stylo}
\end{scratch}\\ \hline
\end{tabularx}
\end{center}

\begin{center}
\begin{tabularx}{\linewidth}{|*{3}{>{\centering \arraybackslash}X|}}\hline
\multicolumn{3}{|c|}{\textbf{Pour information}}\\ \hline
\begin{scratch}
\blockvariable{initialisation}
\end{scratch}

Ce bloc efface le dessin 

précédent, positionne le 
 
crayon à gauche de  
 
l'écran et relève le stylo.				&\begin{scratch}
\blockmove{s'orienter à \ovalnum{90\selectarrownum}}
\blockmove{90 à droite}
\blockmove{$- 90$ à gauche}
\blockmove{(0) vers le haut}
\blockmove{(180) vers le bas} 
\end{scratch}&\psset{unit=0.5cm}
\def\paleb{\pspolygon(0,0)(1.5,0)(0.2,-0.2)}
\def\palen{\pspolygon*(0,0)(0.2,0.2)(1.5,0)}
\begin{center}\begin{pspicture}(-1.9,-1.9)(1.9,1.9)
\multido{\n=0+90}{4}{\rput{\n}(0,0){\palen}}
\multido{\n=0+90}{4}{\rput{\n}(0,0){\paleb}}
\rput(1.9,0){90}\rput(0,1.9){0}\rput(-1.9,0){$- 90$}\rput(0,-1.9){180}
\end{pspicture}\end{center}\\ \hline
\end{tabularx}
\end{center}

\begin{enumerate}
\item Parmi les trois dessins suivants, un seul ne pourra pas être réalisé avec ce
programme. Lequel ? Expliquer.
\begin{center}
\begin{tabularx}{\linewidth}{|*{3}{>{\centering \arraybackslash}X|}}\hline
Dessin 1 		&Dessin 2 		&Dessin 3\\ 
\psset{unit=1cm}
\def\crayon{\psline(0.5,0.6)(0.05,0.05)(0,0)(0.02,0.05)(0.4,0.65)}
\begin{pspicture}(3.5,2)
\psline(0,0.6)(1.8,0.6)(1.8,1.2)(2.4,1.2)(2.4,0)(2.8,0)
\rput{-100}(2.9,0.1){\ding{46}}
\end{pspicture}	&
\psset{unit=1cm}
\def\crayon{\psline(0.5,0.6)(0.05,0.05)(0,0)(0.02,0.05)(0.4,0.65)}
\begin{pspicture}(3.5,2)
\psline(0,0.6)(1.8,0.6)(1.8,1.2)(3,1.2)(3,0)(2.4,0)(2.4,0.6)(3,0.6)
\rput{-100}(3.1,0.7){\ding{46}}
\end{pspicture}	&\psset{unit=1cm}
\def\crayon{\psline(0.5,0.6)(0.05,0.05)(0,0)(0.02,0.05)(0.4,0.65)}
\begin{pspicture}(3.5,2)
\psline(0,0)(0,0.6)(0.6,0.6)(0.6,1.2)(1.2,1.2)(1.2,1.8)(1.8,1.8)(1.8,1.2)
\rput{-100}(1.9,1.3){\ding{46}}
\end{pspicture}	\\ \hline
\end{tabularx}
\end{center}

\item  Julie a modifié le programme de Margot (voir ci-dessous). Que devient alors le
dessin 3 avec le programme modifié par Julie ?

\begin{center}
\begin{tabularx}{\linewidth}{|*{3}{>{\scriptsize \centering \arraybackslash}X}|}\hline
\multicolumn{3}{|c|}{Programme modifié par Julie}\\ \cline{2-2}
\begin{scratch} \blockinit{quand \greenflag est cliqué}
\blockvariable{initialisation}
\end{scratch}	&								&\\

\begin{scratch} 
\blockinit{quand \selectmenu{flèche haut} est cliqué}	
\blockmove{s'orienter à \ovalnum{0\selectarrownum}}
\blockpen{stylo en position d'écriture}
\blockmove{avancer de \ovalnum{50}}
\blockpen{relever le stylo}
\end{scratch}				&
\begin{scratch} \blockinit{quand \selectmenu{flèche droite} est cliqué}
\blockmove{s'orienter à \ovalnum{90\selectarrownum}}
\blockmove{avancer de \ovalnum{50}}
\end{scratch}				&
\begin{scratch} \blockinit{quand \selectmenu{flèche bas} est cliqué}
\blockmove{s'orienter à \ovalnum{180\selectarrownum}}
\blockpen{stylo en position d'écriture}
\blockmove{avancer de \ovalnum{50}}
\blockpen{relever le stylo}
\end{scratch}\\ \hline
\end{tabularx}
\end{center}

\end{enumerate}

\vspace{0,5cm}

