\textbf{Exercice 7 : Commande de livres \hfill 3 points}

\medskip

%Un collège a besoin de commander quelques livres de mathématiques et de français.
%Chaque livre de mathématiques coûte \np{3000}~F et chaque livre de français \np{2000}~F.
%
%Au total 30 livres ont été commandés pour un montant de \np{80000}F.
%
%Combien de livres de chaque sorte ont été commandés ?
%
%\emph{Dans cette question, toute trace de recherche, même incomplète ou non fructueuse, sera prise en compte dans l'évaluation.}
Si $f$ est le nombre de livres de français et $m$ le nombre de livres de mathématiques, on a donc le système :

$\left\{\begin{array}{l c l}
f + m&=&30\\
\np{2000}f + \np{3000}m&=&\np{80000}
\end{array}\right.$ ou encore en simplifiant par \np{1000} dans la seconde équation :

$\left\{\begin{array}{l c l}
f + m&=&30\\
\np{2}f + \np{3}m&=&\np{80}
\end{array}\right.$  ou encore 

$\left\{\begin{array}{l c l}
2f + 2m&=&60\\
2f + 3m&=&80
\end{array}\right.$ d’où par différence $m = 20$ et donc $f = 10$.
\vspace{0,5cm}

