\textbf{Exercice 3 :\hfill 8 points}

\medskip
 
La 24\up{e} édition du Marathon International de Moorea a eu lieu le 18 février 2012.
 
Des coureurs de différentes origines ont participé à ce marathon :

\setlength\parindent{6mm} 
\begin{itemize}
\item[$\bullet~~$] 90 coureurs provenaient de Polynésie Française dont 16 étaient des femmes 
\item[$\bullet~~$] 7 coureurs provenaient de France Métropolitaine dont aucune femme, 
\item[$\bullet~~$] 6 provenaient d'Autriche dont 3 femmes, 
\item[$\bullet~~$] 2 provenaient du Japon dont aucune femme, 
\item[$\bullet~~$] 11 provenaient d'Italie dont 3 femmes, 
\item[$\bullet~~$] 2 provenaient des Etats-Unis dont aucune femme 
\item[$\bullet~~$] Un coureur homme était Allemand.
\end{itemize}
\setlength\parindent{0mm}

\medskip
 
\begin{enumerate}
\item Compléter le tableau ci-dessous à l'aide des données de l'énoncé. 

\medskip
\begin{tabularx}{\linewidth}{|*{8}{>{\centering \arraybackslash}X|}}\cline{2-8}
\multicolumn{1}{c|}{~}	&	&	&	&Japon	&	&	&\\ \hline
Femme					&	&	&	&		&	&	&\\ \hline
\end{tabularx}
\medskip
 
\item Combien de coureurs ont participé à ce marathon ? 
\item Parmi les participants à ce marathon, quel pourcentage les femmes polynésiennes représentent-elles ? Arrondir au dixième près.
 
\hspace*{-1cm}À la fin du marathon, on interroge un coureur au hasard.
 
\item Quelle est la probabilité que ce coureur soit une femme Autrichienne ? 
\item Quelle est la probabilité que ce coureur soit une femme ? 
\item Quelle est la probabilité que ce coureur soit un homme Polynésien ? 
\item Quelle est la probabilité que ce coureur ne soit pas Japonais ? 
\item Vaitea dit que la probabilité d'interroger un coureur homme Polynésien est exactement trois fois plus grande que celle d'interroger un coureur homme non Polynésien.
 
A-t-il raison? Expliquer pourquoi. 
\end{enumerate}

\bigskip

