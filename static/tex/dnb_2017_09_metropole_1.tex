
\medskip

Un sac opaque contient $120$ boules toutes indiscernables au toucher, dont 30 sont bleues. Les autres boules sont rouges ou vertes.

On considère l'expérience aléatoire suivante :

On tire une boule au hasard, on regarde sa couleur, on repose la boule dans le sac et on mélange.

\medskip

\begin{enumerate}
\item Quelle est la probabilité de tirer une boule bleue? Écrire le résultat sous la forme d'une fraction irréductible.
\item Cécile a effectué $20$ fois cette expérience aléatoire et elle a obtenu $8$ fois une boule verte. Choisir, parmi les réponses suivantes, le nombre de boules vertes contenues dans le sac (aucune justification n'est demandée) :

\medskip
\begin{tabularx}{\linewidth}{*{4}{X}}
\textbf{a.~~} $48$ &\textbf{b.~~} $70$ &\textbf{c.~~} On ne peut pas savoir &\textbf{d.~~} $25$
\end{tabularx}
\medskip

\item  La probabilité de tirer une boule rouge est égale à $0,4$.
	\begin{enumerate}
		\item Quel est le nombre de boules rouges dans le sac ?
		\item Quelle est la probabilité de tirer une boule verte ?
	\end{enumerate}
\end{enumerate}

\vspace{0,5cm}

