\textbf{Exercice 6 \hfill 4 points}

\medskip

%Le poinçon et le faux-entrait de la figure représentant la maison de Moana sont
%à \og  angle droit \fg{} (voir la figure de l'exercice 5).
%
%On donne BA = 200 et BF = 165.
%
%\medskip

\begin{enumerate}
\item ~
\parbox{0.65\linewidth}{Indiquer les sommets du triangle rectangle ABF et coder la
figure ci-contre :}\hfill  
\parbox{0.27\linewidth}{\psset{unit=1cm}
\begin{pspicture}(3.5,2.5)
%\psgrid
\pspolygon(0.5,0.5)(3,0.5)(3,2)%BFA
\uput[dl](0.5,0.5){B}\uput[dr](3,0.5){F}\uput[ul](3,2){A}
\uput[d](1.75,0.5){165} \uput[ul](1.75,1.25){200}
\psframe(3,0.5)(2.7,0.8)
\end{pspicture}
}

\item %En utilisant le théorème de Pythagore, calculer AF. On arrondira le résultat au cm près.

D'après le théorème de Pythagore dans le triangle BAF rectangle en F :

$\text{BA}^2 = \text{BF}^2 + \text{FA}^2$, donc $200^2 = 165^2 + \text{FA}^2$ ;

donc $\text{FA}^2 = 200^2 - 165^2 = \np{40000} - \np{27225} = \np{12775}$.

Donc AF $ = \sqrt{\np{12775}} \approx 113,02$ soit 113~cm au centimètre près.
\end{enumerate}

\bigskip

