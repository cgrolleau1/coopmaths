
\medskip

%\begin{minipage}{5cm}
%Madame Martin souhaite réaliser une terrasse en béton en face de sa baie vitrée.
	
%\medskip
%	
%Elle réalise le dessin ci-contre.
%	
%\smallskip
%	
%Pour faciliter l'écoulement des eaux de pluie, le sol de la terrasse doit être incliné.
%	
%\smallskip
%	
%La terrasse a la forme d'un prisme droit dont la base est le quadrilatère ABCD et la hauteur est le segment [CG].
%	
%\smallskip
%	
%P est le point du segment [AD] tel que BCDP est un rectangle.
%\end{minipage}
%\hspace{3mm}	\begin{tikzpicture}[baseline = {(current bounding box.center)},x={(-158:0.46cm)},y = {(-10:0.54cm)},z = {(90:0.9cm)} ,> = stealth]
%\draw[dashed] (8,0,0) -- (0,0,0) -- (0,5,0) (0,0,0) -- (0,0,1);
%\draw (8,0,0) -- (8,5,0) -- (0,5,0) -- (0,5,0.65) -- (8,5,0.65) -- (8,0,1)--(0,0,1) -- (0,5,0.65) %terrasse
%(8,5,0)--(8,5,0.65) %coin de la terrasse
%(8,0,0) --(8,0,4.2) -- (0,0,4.2)--(0,0,1); %facade avant de la maison;
%\filldraw[fill = gray!50, draw=black] (0,-2.5,5.5) -- (0,0,4.2) -- (8,0,4.2) --(8,-2.5,5.5) ;%toit; 
%\filldraw[fill = gray!20, draw=black] (6,0,1) -- (6,0,3) --(2,0,3) --(2,0,1) -- cycle ;%baie vitrée;	
%\node at (4,0,2) {\parbox{1cm}{\begin{center} Baie vitrée \end{center}}};
%\draw [shift = {(-5mm,0mm)},<->] (8,0,0) -- (8,0,1) node [pos = 0.5, left] {0,27 m} ;
%\draw [shift = {(0mm,-5mm)},<->] (8,0,0) -- (8,5,0) node [pos = 0.5,sloped, below] {5 m} ;
%\draw [shift = {(0mm,-5mm)},<->] (8,5,0) -- (0,5,0) node [pos = 0.5, sloped, below ] {8 m} ;
%\draw [shift = {(5mm,0mm)},<->] (0,5,0) -- (0,5,0.65) node [pos = 0.5, right] {0,15 m} ;
%\draw[dotted] (8,0,0.65) -- (8,5,0.65);
%\filldraw[fill = gray!20, draw=black,shift = {(8,0,0)}]  (0,0,0) -- (0,0.3,0) --(0,0.3,0.2) --(0,0,0.2) -- cycle ;%angle droit;
%		\filldraw[fill = gray!20, draw=black,shift = {(8,0,0.65)}] (0,0,0) -- (0,0.3,0) --(0,0.3,0.2) --(0,0,0.2) -- cycle ;%angle droit;
%		\filldraw[fill = gray!20, draw=black,shift = {(8,4.7,0)}] (0,0,0) -- (0,0.3,0) --(0,0.3,0.2) --(0,0,0.2) -- cycle ;%angle droit;		
%		\filldraw[fill = gray!20, draw=black,shift = {(8,5,0)}] (0,0,0) -- (-0.3,0,0) --(-0.3,0,0.2) --(0,0,0.2) -- cycle ;%angle droit;
%		\draw[<-] (-0.10,2.5,0.875) -- (-0.10,3,1.5) node [above,draw,inner sep=0]{\parbox{2cm}{\begin{center} Terrasse en béton \end{center}}};
%		\node at (8,0,1) [left] {A};
%		\node at (8,5,0.65) [above] {B};
%		\node at (8,5,0) [below] {C};
%		\node at (8,0,0) [left] {D};
%		\node at (0,0,1) [above right] {E};
%		\node at (0,5,0.65) [right] {F};
%		\node at (0,5,0) [right] {G};
%		\node at (0,0,0) [below] {H};
%		\node at (8,0,0.65) [left] {P};
%		\filldraw [fill = gray!20, draw=black] (8,5,0.65) -- (8,3.,0.65) .. controls (8,3,0.72) .. (8,3.1,0.783) -- cycle;
%\end{tikzpicture}

\begin{enumerate}
	\item %L'angle $\widehat{\text{ABP}}$ doit mesurer entre 1° et 1,5°.
	
%Le projet de Madame Martin vérifie-t-il cette condition ?
Dans le triangle ABP rectangle en P, on a BP $ = 5$ ([BP] côté adjacent à l'angle $\widehat{\text{ABP}}$ et AP $ = \text{AD} - \text{PD} = \text{AD} - \text{FG} = 0,27 - 0,15 = 0,12$ ([AP] côté opposé à l'angle $\widehat{\text{ABP}}$.

On a donc par définition : $\tan \widehat{\text{ABP}} =  \dfrac{\text{AP}}{\text{BP}} = \dfrac{0,12}{5} = 0,024$.

Avec la calculatrice on obtient $\widehat{\text{ABP}} \approx 1,37\degres$. La condition est vérifiée.	
	\item %Madame Martin souhaite se faire livrer le béton nécessaire à la réalisation de sa terrasse.
	
%Elle fait appel à une entreprise spécialisée.
	
\smallskip
	
%À l'aide des informations contenues dans le tableau ci-dessous, déterminer le montant de la facture établie par l'entreprise.
%	
%\medskip
%\textit{On rappelle que toute trace de recherche, même incomplète, pourra être prise en compte dans l'évaluation}
%	
%\begin{tabularx}{\linewidth}{|>{\centering \arraybackslash}X|} \hline
%\textbf{Information 1}\\
%Distance entre l'entreprise et la maison de Madame Martin : 23 km\\ \hline
%\textbf{Information 2}\\
%\textbf{Formule du volume d'un prisme droit}\\
%Volume d'un prisme droit = Aire de la base du prisme $\times$ hauteur du prisme \\ \hline
%\textbf{Information 3}\\
%\textbf{Conditions tarifaires de l'entreprise spécialisée}
%\begin{itemize}
%\item Prix du m$^3$ de béton : 95~\euro{}.
%\item Capacité maximale du camion-toupie : 6 m$^3$.
%\item Frais de livraison : 5~\euro{} par km parcouru par le camion-toupie.
%\item L'entreprise facture les distances aller et retour (entreprise / lieu de livraison) parcourues par le camion-toupie.
%\end{itemize}\\ \hline
%\end{tabularx}
$\bullet~~$Le volume de la terrasse est celle d'un prisme droit de base ABCD et de hauteur [CG].

Son volume est donc égal à $\left(5 \times 0,15 + \dfrac{5 \times 0,12}{2}\right) \times 8 = 5 \times 1,2 + 2,4 = 8,4$~m$^3$.

$\bullet~~$Il faudra donc que le camion-toupie vienne 2 fois, ce qui représente une distance parcourue de $4 \times 23 = 92$~km.

L'entreprise facturera donc :

-- pour le béton : $8,4 \times 95 = \np{798}$~\euro ;

-- pour le transport $92 \times 5 = 460$~\euro{} soit une facture totale de :

$\np{798} + 460 = \np{1258}$~\euro.

\end{enumerate}

\vspace{0,5cm}

