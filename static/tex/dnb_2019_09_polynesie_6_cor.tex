
\medskip

%L'éco-conduite est un comportement de conduite plus responsable permettant de :
%
%\begin{itemize}
%\item réduire ses dépenses : moins de consommation de carburant et un coût d'entretien du
%véhicule réduit ;
%\item limiter les émissions de gaz à effet de serre;
%\item réduire le risque d'accident de $10$ à $15$\,\% en moyenne.
%\end{itemize}
%
%\medskip

\begin{enumerate}
\item %Un des grands principes est de vérifier la pression des pneus de son véhicule. On considère
%des pneus dont la pression recommandée par le constructeur est de $2,4$~bars.
	\begin{enumerate}
		\item %Sachant qu'un pneu perd environ $0,1$~bar par mois, en combien de mois la pression des
%pneus sera descendue à $1,9$~bar, s'il n'y a eu aucun gonflage ?
Au bout du cinquième mois.
		\item %Le graphique ci-dessous donne un pourcentage approximatif de consommation
%supplémentaire de carburant en fonction de la pression des pneus (zone grisée) :

%\begin{center}
%\psset{xunit=10cm,yunit=0.6cm,comma=true}
%\begin{pspicture}(-0.1,-2)(0.8,11)
%\psaxes[linewidth=1.2pt,Dx=10,Dy=2](0,0)(0,0)(0.8,10)
%\uput[r](0,10.5){\footnotesize Consommation supplémentaire (en \%)}
%\uput[d](0.6,-0.8){\footnotesize Pression des pneus (en bars)}
%\psset{comma=true}
%\multido{\n=0.0+0.1,\na=2.4+-0.1}{9}{\uput[d](\n,0){\np{\na}}}
%\psplot[plotpoints=2000,linewidth=1.25pt]{0}{0.8}{x dup mul 6.25 mul}
%\psplot[plotpoints=2000,linewidth=1.25pt]{0}{0.8}{x dup mul 10.3 mul}
%\pscustom[fillstyle=solid,fillcolor=lightgray]
%{\psplot[plotpoints=2000,linewidth=1.25pt]{0}{0.8}{x dup mul 6.25  mul}
%\psplot[plotpoints=2000,linewidth=1.25pt]{0.8}{0}{x dup mul 10.3 mul}
%}
%\uput[d](0.6,-1.5){source : \emph{ \blue www.eco-drive.ch}}
%\multido{\n=0.0+0.1}{9}{\psline(\n,0)(\n,10)}
%\multido{\n=0+2}{6}{\psline(0,\n)(0.8,\n)}
%\end{pspicture}
%\end{center}
%
%D'après le graphique, pour des pneus gonflés à $1,9$~bars alors que la pression
%recommandée est de $2,4$~bars, donner un encadrement approximatif du pourcentage de la
%consommation supplémentaire de carburant.
D'après le graphique pour une abscisse de 1,9 le surplus de consommation peut aller de  de 2 à un peu moins de 4,5\,\%.
	\end{enumerate}
\item  %Paul a remarqué que lorsque les pneus étaient correctement gonflés, sa voiture consommait
%en moyenne $6$~L aux $100$~km. Il décide de s'inscrire à un stage d'éco-conduite afin de diminuer
%sa consommation de carburant et donc l'émission de CO$_2$. En adoptant les principes de l'écoconduite,
%un conducteur peut diminuer sa consommation de carburant d'environ 15\,\%. Il
%souhaite, à l'issue du stage, atteindre cet objectif.
	\begin{enumerate}
		\item %Quelle sera alors la consommation moyenne de la voiture de Paul ?
		Sa consommation ne sera plus que de 85\,\% de sa consommation antérieure, soit :
		
$6 \times 0,85 = 5,1$~L/100~km.
		\item %Sachant qu'il effectue environ \np{20000}~km en une année, combien de litres de carburant
%peut-il espérer économiser ?
Comme $\np{20000} = 200 \times 100$ et qu'il économisera $6 - 5,1 = 0,8$~(L) tous les 100~km, il va donc économiser en un an :

$200 \times 0,9 = 180$~(L) de carburant.
		\item %Sa voiture roule à l'essence sans plomb. Le prix moyen est 1,35 €IL. Quel serait alors
%le montant de l'économie réalisée sur une année ?
L'économie sera donc de $160 \times 1,35 = 243$~\euro.
		\item %Ce stage lui a coûté $200$~\euro. Au bout d'un an peut-il espérer amortir cette dépense?
Comme $243 > 200$ il aura donc amorti la dépense pour le stage.
	\end{enumerate}
\end{enumerate}

\bigskip

