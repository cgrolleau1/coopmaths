\textbf{Exercice 5 \hfill 16 points}

\medskip

\begin{enumerate}
\item \textbf{PARTIE 1}
	\begin{enumerate}
		\item Dans le triangle AED rectangle en E, on a $\sin \widehat{\text{EAD}} = \dfrac{\text{ED}}{\text{AD}}$, donc $\text{AD} = \dfrac{\text{ED}}{\sin \widehat{\text{EAD}}} = \dfrac{2,53}{\sin 38} \approx 4,109$ soit AD $\approx 4,11$~(m) au centimètre près.
		\item On a $\tan \widehat{\text{EAD}} = \dfrac{\text{ED}}{\text{AE}}$, donc $\text{AE} = \dfrac{\text{ED}}{\tan \widehat{\text{EAD}}} = \dfrac{2,53}{\tan 38} \approx 3,238$, soit AE $ \approx 3,24$~(m) au centimètre près.
		\item Chaque pan du toit est un rectangle de longueur 13~m et de largeur 4,11~m, donc d'aire $13 \times 4,11 = 53,43~\left(\text{m}^2\right)$.
		
Il faut couvrir deux pans d'aire $2 \times 53,43 = 106,86$~m$^2$, donc avec 26 tuiles au m$^2$, il faudra :

$26 \times 106,86 = \np{2778,36}$, soit au moins \np{2779} tuiles d'où un coût de :

$0,65 \times \np{2779} = \np{1806,35}$~(\euro).
	\end{enumerate}

\medskip

\item \textbf{PARTIE 2}

La partie réfectoire est un pavé de dimensions : 13~(m), 5,06~(m) et 2,70~(m), donc de volume :

$13 \times 5,06 \times 2,7 = 177,606$ soit environ 178~m$^3$.

La puissance frigorifique nécessaire sera au moins de \np{18000} BTU et au plus \np{25000} BTU.

On peut choisir le Freez 8000 à \np{1050}~\euro{} mais le Air 10 pingouin un peu plus puissant ne coûte que 990~\euro.
\end{enumerate}

\bigskip

