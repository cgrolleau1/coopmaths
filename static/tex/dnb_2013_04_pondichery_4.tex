\textbf{\textsc{Exercice 4 \hfill 4 points}}

\medskip

\parbox{0.67\linewidth}{On donne la feuille de calcul ci-contre. 

La colonne B donne les valeurs de l'expression $2x^2 - 3x - 9$ pour quelques valeurs de $x$ de la colonne A.

\medskip
 
\begin{enumerate}
\item Si on tape le nombre 6 dans la cellule A 17, quelle valeur va-t-on obtenir dans la cellule B 17 ? 
\item À l'aide du tableur, trouver 2 solutions de l'équation : $2x^2 - 3x - 9 = 0$. 
\item L'unité de longueur est le cm.
 
Donner une valeur de $x$ pour laquelle l'aire du rectangle ci-dessous est égale à 5 cm$^2$. Justifier.
\end{enumerate}

\begin{center}\psset{xunit=1cm}\begin{pspicture}(5,3)
\psframe(0.5,0.5)(4.5,2.5)
\uput[ul](0.5,2.5){A} \uput[ur](4.5,2.5){B} \uput[dr](4.5,0.5){C} \uput[dl](0.5,0.5){D}  
\rput(0.5,2.5){\psframe(0.4,-0.4)}\rput(4.5,2.5){\psframe(-0.4,-0.4)}
\rput(4.5,0.5){\psframe(-0.4,0.4)}\rput(0.5,0.5){\psframe(0.4,0.4)}
\uput[u](2.5,2.5){$2x + 3$}
\uput[l](0.5,1.5){$x - 3$}
\end{pspicture}
\end{center}} \hfill
\parbox{0.3\linewidth}{$\begin{array}{|c|c|c|}\hline
	&\text{A}	&\text{B}\\ \hline
	&x			&2x^2 - 3x - 9\\ \hline
1	&- 2,5		&11\\ \hline
2	&- 2		&5\\ \hline
3	&- 1,5		&0\\ \hline
4	&- 1		&- 4\\ \hline
5	&- 0,5		&- 7\\ \hline
6	&0			&- 9\\ \hline
7	&0,5		&- 10\\ \hline
8	&1			&- 10\\ \hline
9	&1,5		&- 9\\ \hline
10	&2			&- 7\\ \hline
11	&2,5		&- 4\\ \hline
12	&3			&0\\ \hline
13	&3,5		&5\\ \hline
14	&4			&11\\ \hline
15	&4,5		&18\\ \hline
16	&5			&26\\ \hline
17	&			&\\ \hline
\end{array}$}

\bigskip

