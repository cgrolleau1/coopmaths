\documentclass[10pt]{article}
\usepackage[T1]{fontenc}
\usepackage[utf8]{inputenc}%ATTENTION codage UTF8
\usepackage{fourier}
\usepackage[scaled=0.875]{helvet}
\renewcommand{\ttdefault}{lmtt}
\usepackage{amsmath,amssymb,makeidx}
\usepackage[normalem]{ulem}
\usepackage{diagbox}
\usepackage{fancybox}
\usepackage{tabularx,booktabs}
\usepackage{colortbl}
\usepackage{pifont}
\usepackage{multirow}
\usepackage{dcolumn}
\usepackage{enumitem}
\usepackage{textcomp}
\usepackage{lscape}
\newcommand{\euro}{\eurologo{}}
\usepackage{graphics,graphicx}
\usepackage{pstricks,pst-plot,pst-tree,pstricks-add}
\usepackage[left=3.5cm, right=3.5cm, top=3cm, bottom=3cm]{geometry}
\newcommand{\R}{\mathbb{R}}
\newcommand{\N}{\mathbb{N}}
\newcommand{\D}{\mathbb{D}}
\newcommand{\Z}{\mathbb{Z}}
\newcommand{\Q}{\mathbb{Q}}
\newcommand{\C}{\mathbb{C}}
\usepackage{scratch}
\renewcommand{\theenumi}{\textbf{\arabic{enumi}}}
\renewcommand{\labelenumi}{\textbf{\theenumi.}}
\renewcommand{\theenumii}{\textbf{\alph{enumii}}}
\renewcommand{\labelenumii}{\textbf{\theenumii.}}
\newcommand{\vect}[1]{\overrightarrow{\,\mathstrut#1\,}}
\def\Oij{$\left(\text{O}~;~\vect{\imath},~\vect{\jmath}\right)$}
\def\Oijk{$\left(\text{O}~;~\vect{\imath},~\vect{\jmath},~\vect{k}\right)$}
\def\Ouv{$\left(\text{O}~;~\vect{u},~\vect{v}\right)$}
\usepackage{fancyhdr}
\usepackage[french]{babel}
\usepackage[dvips]{hyperref}
\usepackage[np]{numprint}
%Tapuscrit : Denis Vergès
%\frenchbsetup{StandardLists=true}

\begin{document}
\setlength\parindent{0mm}
% \rhead{\textbf{A. P{}. M. E. P{}.}}
% \lhead{\small Brevet des collèges}
% \lfoot{\small{Polynésie}}
% \rfoot{\small{7 septembre 2020}}
\pagestyle{fancy}
\thispagestyle{empty}
% \begin{center}
    
% {\Large \textbf{\decofourleft~Brevet des collèges Polynésie 7 septembre 2020~\decofourright}}
    
% \bigskip
    
% \textbf{Durée : 2 heures} \end{center}

% \bigskip

% \textbf{\begin{tabularx}{\linewidth}{|X|}\hline
%  L'évaluation prend en compte la clarté et la précision des raisonnements ainsi que, plus largement, la qualité de la rédaction. Elle prend en compte les essais et les démarches engagées même non abouties. Toutes les réponses doivent être justifiées, sauf mention contraire.\\ \hline
% \end{tabularx}}

% \vspace{0.5cm}\textbf{\textsc{Exercice 4} \hfill 5 points}

\medskip

\parbox{0.48\linewidth}{Pour présenter ses macarons, une boutique souhaite utiliser des présentoirs dont la forme est une pyramide régulière à base carrée de côté 30 cm et dont les
arêtes latérales mesurent 55~cm.

On a schématisé le présentoir par la figure suivante :}\hfill \parbox{0.48\linewidth}{
\psset{unit=1cm}
\begin{pspicture}(5,4.5)
\pspolygon(0.5,0.5)(3.2,0.5)(4.5,2)(2.5,4)%ABCS
\psline(3.2,0.5)(2.5,4)
\psline[linestyle=dotted](0.5,0.5)(4.5,2)(1.8,2)(3.2,0.5)%ACDB
\psline[linestyle=dotted](0.5,0.5)(1.8,2)(2.5,4)
\psline[linestyle=dashed](2.5,4)(2.5,1.3)
\psline[linewidth=0.3pt](0.5,0.5)(1,0.5)(1.2,0.8)(0.7,0.8)
\psline[linewidth=0.3pt](3.2,0.5)(2.7,0.5)(2.9,0.8)(3.4,0.8)
\psline[linewidth=0.3pt](4.5,2)(4,2)(3.75,1.7)(4.25,1.7)
\psline[linewidth=0.3pt](1.8,2)(2.3,2)(2.1,1.7)(1.6,1.7)
\uput[dl](0.5,0.5){A} \uput[dr](3.2,0.5){B} \uput[ur](4.5,2){C} \uput[ul](1.8,2){D} \uput[d](2.5,1.3){O}\uput[u](2.5,4){S} 
\end{pspicture}
}

Peut-on placer ce présentoir dans une vitrine réfrigérée parallélépipédique dont la hauteur est de
50 cm ?

\bigskip

\end{document}