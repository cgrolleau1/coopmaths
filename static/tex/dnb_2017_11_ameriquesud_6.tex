
\medskip

\begin{tabularx}{\linewidth}{|X|X|}\hline
Le bloc d'instruction \og carré\fg{} ci-dessous a été programmé puis utilisé dans les deux programmes ci-contre : 

\begin{scratch}
\initmoreblocks{définir \namemoreblocks{carré}}
\blockpen{stylo en position écriture}
\blockrepeat{répéter \ovalnum{4} fois}
	{
	\blockmove{avancer de \ovalvariable{longueur}}
	\blockmove{tourner \turnleft{} de \ovalnum{90} degrés}
	}
	\blockpen{relever le stylo}
\end{scratch}

\medskip

\textbf{Rappel : }

L'instruction \og avancer de 10 \fg{} fait avancer le lutin de 10 pixels.&

\textbf{Programme \no 1}

\begin{scratch}
\blockinit{quand \greenflag est pressé}
\blockvariable{mettre \selectmenu{longueur} à \ovalnum{10}}
%%%%%%
\blockrepeat{répéter \ovalnum{4} fois}
{
	\blockmoreblocks{carré}
	\blockvariable{mettre  \selectmenu{longueur} à \ovaloperator{\ovalvariable{longueur}+ \ovalnum {20}}} }
	\blocklook{cacher}	
\end{scratch}

\textbf{Programme \no 2}

\begin{scratch}
\blockinit{quand \greenflag est pressé}
\blockvariable{mettre \selectmenu{longueur} à \ovalnum{10}}
\blockrepeat{répéter \ovalnum{4} fois}
{
	\blockmoreblocks{carré}
	\blockvariable{mettre  \selectmenu{longueur} à \ovaloperator{\ovalvariable{longueur}* \ovalnum {2}}} }
	\blocklook{cacher}	
\end{scratch}\\ \hline
\end{tabularx} 

\medskip

\begin{enumerate}
\item Voici trois dessins :
\medskip

\begin{tabularx}{\linewidth}{|*{3}{>{\centering \arraybackslash}X|}}\hline
Dessin \no 1     &Dessin \no 2       &Dessin \no 3 \\ \hline 
\psset{unit=0.5cm}
\begin{pspicture}(4,4)
\multido{\n=1+1}{4}{\psframe(\n,\n)}
\end{pspicture}&
\psset{unit=0.9cm}
\begin{pspicture}(4,4)
\psframe(.5,.5)\psframe(1.4,1.4)\psframe(2.4,2.4)\psframe(3.4,3.4)
\end{pspicture}&
\psset{unit=1cm}
\begin{pspicture}(4,4.1)
\psframe(.5,.5)\psframe(1,1)\psframe(2,2)\psframe(4,4)
\end{pspicture}\\ \hline
\end{tabularx} 

	\begin{enumerate}
		\item Lequel de ces trois dessins obtient-on avec le programme \no 1 ? 
		\item Lequel de ces trois dessins obtient-on avec le programme \no 2 ? 
		\item Pour chacun des deux programmes, déterminer la longueur, en pixel, du côté du plus grand carré dessiné ? 
	\end{enumerate}
\end{enumerate}

\parbox{0.55\linewidth}
{	\begin{enumerate}[resume]
\item On souhaite modifier le programme \no 2 pour obtenir le dessin ci-contre.
\end{enumerate}} 
\hfill
\parbox{0.43\linewidth}
{
\psset{unit=1cm}
\begin{pspicture}(6.5,4)
\psframe(.4,.4)\psframe(0.7,0)(1.4,0.7)\psframe(1.8,0)(3.2,1.4)\psframe(3.6,0)(6.5,2.9)
\end{pspicture}
}

\medskip

\begin{enumerate}
\item[] Parmi les trois modifications suivantes, laquelle permet d'obtenir le dessin souhaité?

Aucune justification n'est attendue pour cette question. 
\end{enumerate}

\medskip

\begin{tabularx}{\linewidth}{|*{3}{>{\scriptsize\centering \arraybackslash}X|}}\hline
\textbf{Modification 1} &\textbf{Modification 2} &\textbf{Modification 3}\\ \hline
\begin{scratch}
\blockinit{quand \greenflag est pressé}
\blockvariable{mettre \selectmenu{longueur} à {10}}
\blockrepeat{répéter \ovalnum{4} fois}
{
	\blockmoreblocks{carré}
	\blockmove{avancer de \ovaloperator{\ovalvariable{longueur} + \ovalnum{10}}}
	\blockvariable{mettre  \selectmenu{longueur} à \ovaloperator{\ovalvariable{longueur} * \ovalnum {2}}} }
	\blocklook{cacher}	
\end{scratch}
&
\begin{scratch}
\blockinit{quand \greenflag est pressé}
\blockvariable{mettre \selectmenu{longueur} à \ovalnum{10}}
\blockrepeat{répéter \ovalnum{4} fois}
{
	\blockmoreblocks{carré}
	\blockvariable{mettre  \selectmenu{longueur} à \ovaloperator{\ovalvariable{longueur} * \ovalnum {2}}} 
	\blockmove{avancer de \ovaloperator{\ovalvariable{longueur} + \ovalnum{10}}}}
	\blocklook{cacher}	
\end{scratch}
&
\begin{scratch}
\blockinit{quand \greenflag est pressé}
\blockvariable{mettre \selectmenu{longueur} à \ovalnum{10}}
\blockrepeat{répéter \ovalnum{4} fois}
{
	\blockmoreblocks{carré}
	\blockvariable{mettre  \selectmenu{longueur} à \ovaloperator{\ovalvariable{longueur} * \ovalnum {2}}} 
}
	\blockmove{avancer de \ovaloperator{\ovalvariable{longueur} + \ovalnum{10}}}	
	\blocklook{cacher}	
\end{scratch}\\ \hline
\end{tabularx} 
\vspace{0,5cm}

