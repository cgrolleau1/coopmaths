\documentclass[10pt]{article}
\usepackage[T1]{fontenc}
\usepackage[utf8]{inputenc}%ATTENTION codage UTF8
\usepackage{fourier}
\usepackage[scaled=0.875]{helvet}
\renewcommand{\ttdefault}{lmtt}
\usepackage{amsmath,amssymb,makeidx}
\usepackage[normalem]{ulem}
\usepackage{diagbox}
\usepackage{fancybox}
\usepackage{tabularx,booktabs}
\usepackage{colortbl}
\usepackage{pifont}
\usepackage{multirow}
\usepackage{dcolumn}
\usepackage{enumitem}
\usepackage{textcomp}
\usepackage{lscape}
\newcommand{\euro}{\eurologo{}}
\usepackage{graphics,graphicx}
\usepackage{pstricks,pst-plot,pst-tree,pstricks-add}
\usepackage[left=3.5cm, right=3.5cm, top=3cm, bottom=3cm]{geometry}
\newcommand{\R}{\mathbb{R}}
\newcommand{\N}{\mathbb{N}}
\newcommand{\D}{\mathbb{D}}
\newcommand{\Z}{\mathbb{Z}}
\newcommand{\Q}{\mathbb{Q}}
\newcommand{\C}{\mathbb{C}}
\usepackage{scratch}
\renewcommand{\theenumi}{\textbf{\arabic{enumi}}}
\renewcommand{\labelenumi}{\textbf{\theenumi.}}
\renewcommand{\theenumii}{\textbf{\alph{enumii}}}
\renewcommand{\labelenumii}{\textbf{\theenumii.}}
\newcommand{\vect}[1]{\overrightarrow{\,\mathstrut#1\,}}
\def\Oij{$\left(\text{O}~;~\vect{\imath},~\vect{\jmath}\right)$}
\def\Oijk{$\left(\text{O}~;~\vect{\imath},~\vect{\jmath},~\vect{k}\right)$}
\def\Ouv{$\left(\text{O}~;~\vect{u},~\vect{v}\right)$}
\usepackage{fancyhdr}
\usepackage[french]{babel}
\usepackage[dvips]{hyperref}
\usepackage[np]{numprint}
%Tapuscrit : Denis Vergès
%\frenchbsetup{StandardLists=true}

\begin{document}
\setlength\parindent{0mm}
% \rhead{\textbf{A. P{}. M. E. P{}.}}
% \lhead{\small Brevet des collèges}
% \lfoot{\small{Polynésie}}
% \rfoot{\small{7 septembre 2020}}
\pagestyle{fancy}
\thispagestyle{empty}
% \begin{center}
    
% {\Large \textbf{\decofourleft~Brevet des collèges Polynésie 7 septembre 2020~\decofourright}}
    
% \bigskip
    
% \textbf{Durée : 2 heures} \end{center}

% \bigskip

% \textbf{\begin{tabularx}{\linewidth}{|X|}\hline
%  L'évaluation prend en compte la clarté et la précision des raisonnements ainsi que, plus largement, la qualité de la rédaction. Elle prend en compte les essais et les démarches engagées même non abouties. Toutes les réponses doivent être justifiées, sauf mention contraire.\\ \hline
% \end{tabularx}}

% \vspace{0.5cm}\textbf{Exercice 7 \hfill 7 points}

\medskip

%Voici les caractéristiques d'une piscine qui doit être rénovée:
%
%\begin{center}
%\begin{pspicture}(12,10)
%\psframe(6.5,10)
%\uput[r](0,9.5){\textbf{Document 1 : informations sur la piscine}}
%\uput[r](0,8){Vue aérienne de la piscine}
%\psframe(0.5,4)(5.5,7)
%\rput(3.25,5.5){piscine}
%\psline{<->}(0.5,3.6)(5.5,3.6)\uput[d](3,3.6){10 m}
%\psline{<->}(5.7,4)(5.7,7)\uput[r](5.7,5.5){4 m}
%\uput[r](0.5,2){Forme: pavé droit}
%\uput[r](0.5,1){Profondeur : 1,2 m}
%\psframe(7,6)(12,10)
%\uput[r](7,9.5){\textbf{Document 2 : information rela-}}
%\uput[r](7,8.5){\textbf{tive à la pompe de vidange}}
%\uput[r](7,7){Débit: 14 m$^3$/h}
%\psframe(7,1)(12,5.5)
%\uput[r](7,5){\textbf{Document 3 : informations sur }}
%\uput[r](7,4.5){\textbf{la peinture résine utilisée}}
%\uput[r](7,4){\textbf{ pour la rénovation}}
%\uput[r](7,3.5){-- seau de 3 litres}
%\uput[r](7,3){-- un litre recouvre une surface }
%\uput[r](7,2.5){de 6 m$^2$}
%\uput[r](7,2){-- 2 couches nécessaires}
%\uput[r](7,1.5){-- prix du seau : 69,99~\euro}
%\end{pspicture}
%\end{center}
%
%\medskip

\begin{enumerate}
\item %Le propriétaire commence par vider la piscine avec la pompe de vidange. Cette piscine est remplie à ras bord. Sera-t-elle vide en moins de 4 heures ?
$V_{\text{piscine}} = 10 \times  4 \times 1,2 = 48$. Le volume de la piscine est de 48 m$^3$.

On calcule alors : $\dfrac{48}{14} \approx 3,4$ h soit 3h 24~min.

La piscine sera donc vide en moins de 4 heures.
\item %Il repeint ensuite toute la surface intérieure de cette piscine avec de la peinture résine. Quel est le coût de la rénovation ?
On calcule la surface de la piscine :

$A_{\text{piscine}} = 10 \times 4 + 2 \times(10 \times1,2) + 2 \times(4 \times1,2)$

$A_{\text{piscine}} = 40 + 24 + 9,6$

$A_{\text{piscine}} = 73,6$~m.

La surface de la piscine est de $73,6$~m$^2$.

2 couches sont nécessaires pour peindre la piscine, il faut donc prévoir de la peinture pour
une surface de : $2 \times  73,6 = 147,2$~m$^2$.

On  calcule la quantité de peinture nécessaire : $\dfrac{147,2}{6} \approx 24,53~\ell$.

Il faudra environ 24,53 litres de peinture.

Or $\dfrac{24,53}{3} \approx  8,2$.

Les seaux contiennent 3 litres de peinture, il faudra donc 9 seaux de peinture.

$9 \times 69,99 = 629,91$.

Le coût sera donc de 629,91~\euro.
\end{enumerate}
\end{document}\end{document}