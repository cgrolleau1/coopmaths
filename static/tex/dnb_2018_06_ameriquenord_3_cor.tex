
\medskip

%\begin{minipage}[t]{0.55\linewidth}
%Deux urnes contiennent des boules numérotées indiscernables au toucher. Le schéma ci-contre représente le contenu de chacune des urnes.
%	
%\smallskip
%	
%On forme un nombre entier à deux chiffres en tirant au hasard une boule dans chaque urne :
%\begin{itemize}
%\item le chiffre des dizaines est le numéro de la boule issue de l'urne D ;
%\item le chiffre des unités est le numéro de la boule issue de l'urne U.
%\end{itemize}
%\end{minipage} \hfill
%\begin{minipage}[t]{0.4\linewidth}
%	\hfill~
%	\begin{tikzpicture}[baseline = {(current bounding box.north)},x = 1cm,y=1cm]
%	\draw[rounded corners] (0.5,2.5) -- (0.7,2.3) -- (0.5,2.1) -- (0,2.1) -- (0,0) --(2,0) node [pos = 0.5,below = 2mm] {Urne D}-- (2,2.1) -- (1.5,2.1) -- (1.3,2.3) -- (1.5,2.5) ;
%	\draw (0.35,0.27) circle (0.27) node {2};
%	\draw (0.95,0.27) circle (0.27) node {3};
%	\draw (1.65,0.27) circle (0.27) node {1};
%	\end{tikzpicture}
%	\hfill~
%	\begin{tikzpicture}[baseline = {(current bounding box.north)},x = 1cm,y=1cm]
%	\draw[rounded corners] (0.5,2.5) -- (0.7,2.3) -- (0.5,2.1) -- (0,2.1) -- (0,0) --(2,0) node [pos = 0.5,below = 2mm] {Urne U}-- (2,2.1) -- (1.5,2.1) -- (1.3,2.3) -- (1.5,2.5) ;
%	\draw (0.28,0.27) circle (0.27) node {2};
%	\draw (0.82,0.27) circle (0.27) node {6};
%	\draw (1.72,0.27) circle (0.27) node {3};
%	\draw (1.27,0.57) circle (0.27) node {5};
%	\end{tikzpicture}
%	\hfill~
%\end{minipage}
%
%\smallskip
%
%Exemple : en tirant la boule ~~\tikz[baseline = {(0,-3pt)}]{\draw (0,0) circle (0.3) node {1}}~~ de l'urne D et ensuite la boule ~~\tikz[baseline = {(0,-3pt)}]{\draw (0,0) circle (0.3) node {5}}~~ de l'urne U, on forme le nombre 15.
%	
%	\smallskip
	
	\begin{enumerate}
		\item %A-t-on plus de chance de former un nombre pair que de former un nombre impair ?
		On peut obtenir : 12, 16, 22, 26, 32, 36 soit 6 nombres pairs et 13, 15, 23, 25, 33, 35 soit 6 nombres impairs.
		
On a autant de chances de former un nombre pair que de former un nombre impair.
		\item \begin{enumerate}
			\item %Sans justifier, indiquer les nombres premiers qu'on peut former lors de cette expérience.
			On peut obtenir : 13 et 23 soit deux nombres premiers.
			\item %Montrer que la probabilité de former un nombre premier est égale à $ \dfrac{1}{6} $.
On a vu que l'on pouvait former $3 \times 4 = 12$ nombres différents.

La probabilité de former un nombre premier est égale à $ \dfrac{2}{12} = \dfrac{1}{6}$
		\end{enumerate}
		\item %Définir un évènement dont la probabilité de réalisation est égale à $ \dfrac{1}{3} $.
Par exemple l'évènement : \og obtenir un nombre inférieur à 17 \fg{} a une probabilité de $\dfrac{4}{12} = \dfrac{1}{3}$.
	\end{enumerate}

\vspace{5 mm}

