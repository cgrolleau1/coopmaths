\textbf{\textsc{Exercice 5 \hfill 8 points}}

\medskip

\begin{enumerate}
\item $V_{\text{cylindre}} = \text{aire de la base}  \times \text{hauteur} = \pi r^2 \times h$.

Donc $V_{\text{cylindre}} = =π \times5 \times 15 = 375 \pi  \approx \np{1178}$~(cm) .
\item  
	\begin{enumerate}
		\item $V_{\text{cône}} = \dfrac{\text{aire de la base}  \times \text{hauteur}}{3} = \dfrac{\pi r^2 h}{3}$.
		
D’où $V_1 = V_{\text{grand cône}} = \dfrac{\pi r^2 \times \text{SO}}{3}= \dfrac{\pi \times 5^2 \times 6}{3} = 50\pi$~cm$^2$
		\item $V_2 = V_1 - V_{\text{petit cône}}$.


Or le petit cône est une réduction du grand cône avec un rapport $k$ égal à $k = \dfrac{\text{SO}'}{\text{SO}} = \dfrac{2}{6} = \dfrac{1}{3}$.

D'où $V_{\text{petit cône}} = V_1 \times \left(\dfrac{1}{3} \right)^3 = 50 \pi \times \dfrac{1}{27} = \dfrac{50\pi}{27}$~cm$^3$. Par suite :

$V_2 = V_1 - V_{\text{petit cône}} = 50\pi − \dfrac{50\pi}{27} = \dfrac{50\pi \times 27 - 50\pi}{27}  =  \dfrac{\np{1300}\pi}{27} \approx 151$~cm$^3$.
	\end{enumerate}
\item  On peut éliminer le graphique 4 car si $h = 0$, le volume devrait être égal à $0$.

On peut éliminer le graphique 2 car le volume doit toujours augmenter si la hauteur $h$ augmente.

D’après les calculs précédents, si on remplit le bouteille jusqu’au goulot, $h$ serait alors égal à $19$~cm. Et dans ce cas, le volume du bidon serait égal à environ $\np{1178} + 151 = \np{1329}$~cm$^3$ ; ce qui correspond au graphique 1.
\end{enumerate}

\vspace{0,5cm}

