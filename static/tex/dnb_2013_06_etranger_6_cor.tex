\documentclass[10pt]{article}
\usepackage[T1]{fontenc}
\usepackage[utf8]{inputenc}%ATTENTION codage UTF8
\usepackage{fourier}
\usepackage[scaled=0.875]{helvet}
\renewcommand{\ttdefault}{lmtt}
\usepackage{amsmath,amssymb,makeidx}
\usepackage[normalem]{ulem}
\usepackage{diagbox}
\usepackage{fancybox}
\usepackage{tabularx,booktabs}
\usepackage{colortbl}
\usepackage{pifont}
\usepackage{multirow}
\usepackage{dcolumn}
\usepackage{enumitem}
\usepackage{textcomp}
\usepackage{lscape}
\newcommand{\euro}{\eurologo{}}
\usepackage{graphics,graphicx}
\usepackage{pstricks,pst-plot,pst-tree,pstricks-add}
\usepackage[left=3.5cm, right=3.5cm, top=3cm, bottom=3cm]{geometry}
\newcommand{\R}{\mathbb{R}}
\newcommand{\N}{\mathbb{N}}
\newcommand{\D}{\mathbb{D}}
\newcommand{\Z}{\mathbb{Z}}
\newcommand{\Q}{\mathbb{Q}}
\newcommand{\C}{\mathbb{C}}
\usepackage{scratch}
\renewcommand{\theenumi}{\textbf{\arabic{enumi}}}
\renewcommand{\labelenumi}{\textbf{\theenumi.}}
\renewcommand{\theenumii}{\textbf{\alph{enumii}}}
\renewcommand{\labelenumii}{\textbf{\theenumii.}}
\newcommand{\vect}[1]{\overrightarrow{\,\mathstrut#1\,}}
\def\Oij{$\left(\text{O}~;~\vect{\imath},~\vect{\jmath}\right)$}
\def\Oijk{$\left(\text{O}~;~\vect{\imath},~\vect{\jmath},~\vect{k}\right)$}
\def\Ouv{$\left(\text{O}~;~\vect{u},~\vect{v}\right)$}
\usepackage{fancyhdr}
\usepackage[french]{babel}
\usepackage[dvips]{hyperref}
\usepackage[np]{numprint}
%Tapuscrit : Denis Vergès
%\frenchbsetup{StandardLists=true}

\begin{document}
\setlength\parindent{0mm}
% \rhead{\textbf{A. P{}. M. E. P{}.}}
% \lhead{\small Brevet des collèges}
% \lfoot{\small{Polynésie}}
% \rfoot{\small{7 septembre 2020}}
\pagestyle{fancy}
\thispagestyle{empty}
% \begin{center}
    
% {\Large \textbf{\decofourleft~Brevet des collèges Polynésie 7 septembre 2020~\decofourright}}
    
% \bigskip
    
% \textbf{Durée : 2 heures} \end{center}

% \bigskip

% \textbf{\begin{tabularx}{\linewidth}{|X|}\hline
%  L'évaluation prend en compte la clarté et la précision des raisonnements ainsi que, plus largement, la qualité de la rédaction. Elle prend en compte les essais et les démarches engagées même non abouties. Toutes les réponses doivent être justifiées, sauf mention contraire.\\ \hline
% \end{tabularx}}

% \vspace{0.5cm}\textbf{\textsc{Exercice 6} \hfill 4 points}

\medskip

%\textbf{Dans cet exercice, toute trace de recherche, même incomplète, sera prise en compte dans l'évaluation.}
%
%\medskip
% 
%\emph{On considère la figure ci-dessous, qui n'est pas en vraie grandeur.}
%
%\bigskip
%
%\parbox{0.6\linewidth}{\psset{unit=0.6cm}
%\begin{pspicture}(-0.2,-0.2)(10,7.3)
%\psframe(3,7)(9,0)
%\psline(3,7)(0,7)(9,2.8)
%\uput[u](0,7){A} \uput[u](3,7){B} \uput[u](9,7){C} 
%\uput[ur](3,5.5){M} \uput[r](9,2.8){F} \uput[ur](3,0){E} 
%\uput[ur](9,0){D} \uput[u](1.5,7){3} \uput[u](6,7){6} 
%\end{pspicture}}\hfill
%\parbox{0.38\linewidth}{BCDE est un carré de 6 cm de côté.
% 
%Les points A, B et C sont alignés et AB = 3 cm.
% 
%F est un point du segment [CD].
% 
%La droite (AF) coupe le segment [BE] en M.} 
%
%\medskip
%
%Déterminer la longueur CF par calcul ou par construction pour que les longueurs BM et FD soient égales. 
Appelons $x$ les longueurs égales BM et FD.

Les droites (BM) et (CF) sont parallèles (côtés opposés du carré).

Les points A, B C d’une part, A, M, F d’autre part sont alignés dans cet ordre. Le théorème de Thalès permet d’écrire :

$\dfrac{\text{AB}}{\text{AC}} = \dfrac{\text{BM}}{\text{CF}}$.

Or CF $ = 6 - x$ ; donc $\dfrac{3}{9} = \dfrac{x}{6 - x}$ d’où $3x = 6 - x$ ou $4x  = 6$ et $x = \dfrac{3}{2} = 1,5$~cm.

Conclusion : CF $ = 6 - x = 6 - 1,5 = 4,5$~(cm).

\emph{Remarque} : méthode par construction

Si les conditions sont remplies les segments [BM] et [FD] sont parallèles et de même longueur. Le quadrilatère BMDF est donc un parallélogramme ; ses diagonales [BD] et [MF] ont donc le même milieu O centre du carré BCDE.

D'où la construction : on construit les diagonales [BD] et [CE] du carré qui se coupent en O ; la droite (AO) coupe [BE] en M et [CD] en F. On mesure CF $ = 4,5$~cm.

\begin{center}
\psset{unit=0.6cm}
\begin{pspicture}(-0.5,-0.2)(10,7.3)
%\psgrid
\psframe(3,7)(9,1)
\psline(3,7)(0,7)%(9,2.5)
\uput[u](0,7){A} \uput[u](3,7){B} \uput[u](9,7){C} 
\uput[ur](3,5.5){M} \uput[r](9,2.8){F} \uput[ur](3,0){E} 
\uput[ur](9,0){D} \uput[u](1.5,7){3} \uput[u](6,7){6}
\psline[linestyle=dotted](3,7)(9,1)
\psline[linestyle=dotted](3,1)(9,7)
\psline[linestyle=dotted](0,7)(9,2.5) 
\uput[d](6,4){O}
\end{pspicture}
\end{center}

\bigskip

\end{document}