\textbf{EXERCICE 7 : L'habitation \hfill 15 points}

\medskip

%Nolan souhaite construire une habitation.
%
%Il hésite entre une \textbf{case} et une \textbf{maison} en forme de prisme droit.
%
%La case est représentée par un cylindre droit d'axe (OO$'$) surmontée d'un cône de révolution de sommet S.
%
%Les dimensions sont données sur les figures suivantes.
%
%$x$ \textbf{représente à la fois le diamètre de la case et la longueur AB du prisme droit.}
%
%\begin{center}
%\begin{tabularx}{\linewidth}{*{2}{>{\centering \arraybackslash}X}}
%\psset{unit=0.8cm,arrowsize=2pt 3}
%\definecolor{mongris}{gray}{0.6}
%\definecolor{grisleger}{gray}{0.8}
%\begin{pspicture}(-3,-0.5)(3,5)
%\pscustom[fillstyle=solid,fillcolor=grisleger]%haut
%{\psellipticarc(0,0)(3,0.45){180}{0}
%\psline(3,2.5)(0,3.75)(-3,2.5)
%}
%\pscustom[fillstyle=solid,fillcolor=mongris]
%{\psellipticarc[linestyle=dashed](0,2.5)(3,0.45){0}{180}
%\psellipticarc(0,2.5)(3,0.5){180}{360}
%}
%\pscustom[fillstyle=solid,fillcolor=grisleger]
%{\psellipticarc(0,2.5)(3,0.5){180}{360}
%\psline(3,2.5)(3,0)
%\psellipticarc[linestyle=dashed](0,0)(3,0.49){180}{0}
%\psline(-3,0)(-3,2.5)}
%\pscustom[fillstyle=solid,fillcolor=mongris]
%{\psellipticarc[linestyle=dashed](0,0)(3,0.45){0}{180}
%\psellipticarc(0,0)(3,0.5){180}{360}
%}
%\psdots(0,0)(0,2.5)(0,3.75)
%\uput[ur](0,0){O}\uput[ur](0,2.5){O$'$}\uput[r](0,3.75){S}
%\psline{<->}(-2.5,-0.272)(2.5,0.272)\uput[dr](0,0){$x$}
%\psline(0,0)(0,3.75)\rput{90}(-0.2,1.25){2 m}\rput{90}(-0.2,3.3){1 m}
%\end{pspicture}&
%\psset{unit=0.8cm}
%\begin{pspicture}(7.3,4.3)
%\psline(0.5,0.9)(3.5,0.5)(7,1)(7,2.5)(3.5,2)(3.5,0.5)
%\psline(7,2.5)(3.9,4.1)(0.5,3.6)(3.5,2)
%\psline(0.5,0.9)(0.5,3.6)
%\uput[d](3.5,0.5){A}\uput[d](7,1){B}\uput[d](5.25,0.75){$x$}
%\uput[d](2,0.6){5 m}\uput[l](0.3,1.8){2 m}\uput[l](0.3,3.125){1 m}
%\psline[linestyle=dotted,linewidth=1.25pt](3.5,2)(0.5,2.7)
%\psline[linestyle=dotted,linewidth=1.25pt](7,1)(4,1.4)(4,4.1)
%\psline[linestyle=dotted,linewidth=1.25pt](0.5,0.9)(4,1.4)
%\psline[linewidth=0.2pt]{<->}(0.3,0.9)(0.3,2.7)
%\psline[linewidth=0.2pt]{<->}(0.3,3.6)(0.3,2.7)
%\end{pspicture}
%\end{tabularx}
%\end{center}

\textbf{Partie 1 :}

\medskip

Dans cette partie, on considère que $x = 6$ m.

\medskip

\begin{enumerate}
\item %Montrer que le volume exact de la partie cylindrique de la case est $18\pi$ m$^3$.

Le diamètre a une longueur de 6~m. Donc avec $r = 3$, le volume du cylindre est égal à :

$\pi \times 3^2 \times 2 = 18\pi$~m$^3$.
\item %Calculer le volume de la partie conique. Arrondir à l'unité.
Le volume de la partie conique est égale à : 

$\dfrac{1}{3} \times \pi \times 3^2 \times 1 = 3\pi$~m$^3$, soit $\approx 9,42$ ou 9~m$^3$ à l'unité près.
\item %En déduire que le volume total de la case est environ $66$~m$^3$.
Le volume de la case est donc égal à : 

$18\pi + 3\pi = 21\pi \approx 65,97$, soit $\approx 66$~m$^3$ à l'unité près.
\end{enumerate}

\begin{center}
\begin{tabularx}{\linewidth}{|c *{2}{>{\centering \arraybackslash}X}|}\hline
\textbf{Rappels :}&
Cylindre rayon de base $r$ et de hauteur $h$&Cône
rayon de base $r$ et de hauteur $h$\\
&\psset{unit=1cm}
\begin{pspicture}(-1.25,-0.4)(1.25,2.3)
\psellipticarc(0,0)(1.25,0.4){180}{360}
\psellipse(0,1.8)(1.25,0.4)
\pscustom[fillstyle=solid,fillcolor=lightgray]{
\psellipticarc(0,1.8)(1.25,0.4){00}{180}
\psline(-1.25,1.8)(-1.25,0)
\psellipticarc(0,0)(1.25,0.4){180}{360}
\psline(1.25,0)(1.25,1.8)
}
\uput[r](1.25,0.9){$h$}\uput[d](0.55,1.9){$r$}
\psline(0,1.8)(1.15,1.9)
\end{pspicture}&\psset{unit=1cm}
\begin{pspicture}(1.5,-0.5)(1.5,3)
\psellipticarc(0,0)(1.5,0.4){180}{360}
\pscustom[fillstyle=solid,fillcolor=lightgray]
{
\psellipticarc(0,0)(1.5,0.4){180}{360}
\psline(1.5,0)(0,2.8)(-1.5,0)
}
\psline[linestyle=dashed,linewidth=1.25pt](0,2.8)(0,0)(1.5,0)
\uput[l](0,1.4){$h$}\uput[d](0.75,0){$r$}
\psellipticarc[linestyle=dashed,linewidth=1.25pt](0,0)(1.5,0.4){0}{180}
\end{pspicture}\\
&$\text{Volume} =\pi \times r^2 \times h$&$\text{Volume} =\dfrac{1}{3} \times \pi \times r^2 \times h$\\ \hline
\end{tabularx}
\end{center}

\textbf{Partie 2 :}

\medskip

\emph{Dans cette partie, le diamètre est exprimé en mètre, le volume en m$^3$.}

\medskip

Sur l'\textbf{annexe } page \pageref{annexe1}, on a représenté la fonction qui donne le volume total de la case en fonction de son diamètre $x$.

\medskip

\begin{enumerate}
\item %Par lecture graphique, donner une valeur approchée du volume d'une case de $7$ m de diamètre.

%Tracer des pointillés permettant la lecture.
On lit sur l'annexe $V(7) \approx 90$~m$^3$.
\end{enumerate}

%La fonction qui donne le volume de la maison en forme de prisme droit est définie par 

\[V(x) = 12,5 x.\]

\begin{enumerate}[resume]
\item %Calculer l'image de 8 par la fonction $V$.
On a $V(8) = 12,5 \times 8 = 100$~m$^3$.
\item %Quelle est la nature de la fonction $V$ ?
La fonction $V$ est une fonction linéaire.
\item %Sur l'\textbf{annexe } page \pageref{annexe1}, tracer la représentation graphique de la fonction $V$.
La représentation graphique de la fonction linéaire $V$ est une droite contenant l'origine.
\end{enumerate}

%Pour des raisons pratiques, la valeur maximale de $x$ est de $6$ m. Nolan souhaite choisir la construction qui lui offre le plus grand volume.
\begin{enumerate}[resume]
\item %Quelle construction devra-t-il choisir ? Justifier.
$\bullet~~$Le plus grand volume de la maison est donc $V(6) = 12,5 \times 6 = 75$~m$^3$.

$\bullet~~$Le plus grand volume de la case est donc $V(6) \approx 66$~m$^3$.

Nolan choisira donc la maison.
\end{enumerate}

\begin{center}
    \label{annexe1}
    \textbf{\Large ANNEXE 1}
    
    \medskip
    
    \textbf{Partie 2 :} question 1 et 3
    
    \vspace{1cm}
    
    \textbf{Volume de la case en fonction de} \boldmath $x$ \unboldmath
    
    \bigskip
    
    \psset{xunit=1.2cm,yunit=0.06cm,arrowsize=2pt 4}
    \begin{pspicture}(11.4,220)
    \multido{\n=0+1}{11}{\psline[linewidth=0.2pt,linecolor=lightgray](\n,0)(\n,220)}
    \multido{\n=0+20}{10}{\psline[linewidth=0.2pt,linecolor=lightgray](0,\n)(11,\n)}
    \multido{\n=0+4}{56}{\psline[linewidth=0.4pt,linecolor=lightgray](0,\n)(11,\n)}
    \psaxes[linewidth=1.25pt,Dy=20]{->}(0,0)(0,0)(11,220)
    \psaxes[linewidth=1.25pt,Dy=20](0,0)(0,0)(11,220)
    \psplot[plotpoints=2000,linewidth=1.25pt,linecolor=red]{0}{11}{x dup mul 3.141592 mul 7 mul 12 div}
    \psplot[plotpoints=2000,linewidth=1.25pt,linecolor=blue]{0}{6}{x 12.5 mul}
    \uput[u](10.9,0){$x$}
    \uput[r](0,218){$V(x) ~\left(\text{m}^3\right)$}
    \psline[linestyle=dashed,linewidth=1.5pt, linecolor=red,ArrowInside=->]{->}(7,0)(7,89.8)(0,89.9)
    \uput[l](0,89.8){\red $\approx 90$}
    \end{pspicture}
    \end{center}

\vspace{0,5cm}

