
\medskip

\begin{enumerate}
\item Lors des Jeux Olympiques de Rio en 2016, la danoise Pernille Blume a remporté le $50$~m nage libre en $24,07$~secondes.

A-t-elle nagé plus rapidement qu'une personne qui se déplace en marchant vite, c'est-à-dire à $6$~km/h ?
\item  On donne l'expression $E = (3x + 8)^2 - 64$.
	\begin{enumerate}
		\item Développer $E$.
		\item  Montrer que $E$ peut s'écrire sous forme factorisée : $3x(3x + 16)$.
		\item  Résoudre l'équation $(3x + 8)^2 - 64 = 0$.
	\end{enumerate}
\item  La distance $d$ de freinage d'un véhicule dépend de sa vitesse et de l'état de la route.
	
On peut la calculer à l'aide de la formule suivante :
	
\begin{tabularx}{\linewidth}{m{2.3cm} X}
$d = k \times  V^2$ avec 	&$d$ : distance de freinage en m \: $V$ : vitesse du véhicule en m/s\\
							&$k$ : coefficient dépendant de l'état de la route\\
							&\multicolumn{1}{c}{$\left\{\begin{array}{l c l}
k &=& 0,14 \:\:\text{sur route mouillée}\\
k &=& 0,08 \:\:\text{sur route sèche.}
\end{array}\right.$}\\
\end{tabularx}

Quelle est la vitesse d'un véhicule dont la distance de freinage sur route mouillée est égale à $15$~m ?
\end{enumerate}

\vspace{0,5cm}

