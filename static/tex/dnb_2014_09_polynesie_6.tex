\textbf{Exercice 6 \hfill 6 points}

\bigskip

ABC est un triangle tel que AB = 5 cm, BC = 7,6 cm et AC = 9,2 cm. 

\medskip

\begin{enumerate}
\item Tracer ce triangle en vraie grandeur. 
\item ABC est-il un triangle rectangle? 
\item ~

\parbox{0.4\linewidth}{Avec un logiciel, on a construit ce triangle, puis : 

- on a placé un point P mobile sur le côté [AC] ;
 
- on a tracé les triangles ABP et BPC ;
 
- on a affiché le périmètre de ces deux triangles.}\hfill
\parbox{0.55\linewidth}{\psset{unit=0.5cm}
\begin{pspicture}(11.7,6)
\pspolygon(0.5,0.5)(11.2,0.5)(4,5.3)%ACB
\psline(4,5.3)(5.1,0.5)%BP
\uput[dl](0.5,0.5){A} \uput[ur](4,5.3){B} \uput[dr](11.2,0.5){C} \uput[d](5.1,0.5){P}
\rput(2.8,1){\scriptsize Périmètre de ABP = 13,29}
\rput(8.15,1){\scriptsize Périmètre de BPC = 17,09} 
\end{pspicture}} 
 
\medskip 
	\begin{enumerate}
		\item On déplace le point P sur le segment [AC].
		 
Où faut-il le placer pour que la distance BP soit la plus petite possible? 
		\item On place maintenant le point P à 5~cm de A. 

Lequel des triangles ABP et BPC a le plus grand périmètre ? 
		\item On déplace à nouveau le point P sur le segment [AC]. 

Où faut-il le placer pour que les deux triangles ABP et BPC aient le même périmètre ?
	\end{enumerate} 
\end{enumerate} 

\bigskip
 
