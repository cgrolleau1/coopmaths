\textbf{\textsc{Exercice 3 \hfill 6 points}}

\medskip

Un confiseur lance la fabrication de bonbons au chocolat et de bonbons au
caramel pour remplir 50 boîtes. Chaque boîte contient 10 bonbons au chocolat et 8 bonbons au caramel.

\medskip

\begin{enumerate}
\item Combien doit-il fabriquer de bonbons de chaque sorte?
\item Jules prend au hasard un bonbon dans une boite. Quelle est la probabilité qu'il
obtienne un bonbon au chocolat?
\item Jim ouvre une autre boîte et mange un bonbon. Gourmand, il en prend sans regarder un
deuxième. Est-il plus probable qu'il prenne alors un bonbon au chocolat ou un bonbon au caramel?
\item Lors de la fabrication, certaines étapes se passent mal et, au final, le confiseur a 473 bonbons
au chocolat et 387 bonbons au caramel.
	\begin{enumerate}
		\item Peut-il encore constituer des boîtes contenant 10 bonbons au chocolat et 8 bonbons au
caramel en utilisant tous les bonbons? Justifier votre réponse.
		\item Le confiseur décide de changer la composition de ses boîtes. Son objectif est de faire le plus de boîtes identiques possibles en utilisant tous ses bonbons. Combien peut-il faire de boîtes?
Quelle est la composition de chaque boîte?
	\end{enumerate}
\end{enumerate}

\vspace{0,5cm}

