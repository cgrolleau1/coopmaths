\documentclass[10pt]{article}
\usepackage[T1]{fontenc}
\usepackage[utf8]{inputenc}%ATTENTION codage UTF8
\usepackage{fourier}
\usepackage[scaled=0.875]{helvet}
\renewcommand{\ttdefault}{lmtt}
\usepackage{amsmath,amssymb,makeidx}
\usepackage[normalem]{ulem}
\usepackage{diagbox}
\usepackage{fancybox}
\usepackage{tabularx,booktabs}
\usepackage{colortbl}
\usepackage{pifont}
\usepackage{multirow}
\usepackage{dcolumn}
\usepackage{enumitem}
\usepackage{textcomp}
\usepackage{lscape}
\newcommand{\euro}{\eurologo{}}
\usepackage{graphics,graphicx}
\usepackage{pstricks,pst-plot,pst-tree,pstricks-add}
\usepackage[left=3.5cm, right=3.5cm, top=3cm, bottom=3cm]{geometry}
\newcommand{\R}{\mathbb{R}}
\newcommand{\N}{\mathbb{N}}
\newcommand{\D}{\mathbb{D}}
\newcommand{\Z}{\mathbb{Z}}
\newcommand{\Q}{\mathbb{Q}}
\newcommand{\C}{\mathbb{C}}
\usepackage{scratch}
\renewcommand{\theenumi}{\textbf{\arabic{enumi}}}
\renewcommand{\labelenumi}{\textbf{\theenumi.}}
\renewcommand{\theenumii}{\textbf{\alph{enumii}}}
\renewcommand{\labelenumii}{\textbf{\theenumii.}}
\newcommand{\vect}[1]{\overrightarrow{\,\mathstrut#1\,}}
\def\Oij{$\left(\text{O}~;~\vect{\imath},~\vect{\jmath}\right)$}
\def\Oijk{$\left(\text{O}~;~\vect{\imath},~\vect{\jmath},~\vect{k}\right)$}
\def\Ouv{$\left(\text{O}~;~\vect{u},~\vect{v}\right)$}
\usepackage{fancyhdr}
\usepackage[french]{babel}
\usepackage[dvips]{hyperref}
\usepackage[np]{numprint}
%Tapuscrit : Denis Vergès
%\frenchbsetup{StandardLists=true}

\begin{document}
\setlength\parindent{0mm}
% \rhead{\textbf{A. P{}. M. E. P{}.}}
% \lhead{\small Brevet des collèges}
% \lfoot{\small{Polynésie}}
% \rfoot{\small{7 septembre 2020}}
\pagestyle{fancy}
\thispagestyle{empty}
% \begin{center}
    
% {\Large \textbf{\decofourleft~Brevet des collèges Polynésie 7 septembre 2020~\decofourright}}
    
% \bigskip
    
% \textbf{Durée : 2 heures} \end{center}

% \bigskip

% \textbf{\begin{tabularx}{\linewidth}{|X|}\hline
%  L'évaluation prend en compte la clarté et la précision des raisonnements ainsi que, plus largement, la qualité de la rédaction. Elle prend en compte les essais et les démarches engagées même non abouties. Toutes les réponses doivent être justifiées, sauf mention contraire.\\ \hline
% \end{tabularx}}

% \vspace{0.5cm}\textbf{Exercice 8 \hfill 6 points}

\bigskip
 
%Un couple a acheté une maison avec piscine en vue de la louer. Pour cet achat, le couple a effectué un prêt auprès de sa banque. Ils louent la maison de juin à  septembre et la maison reste inoccupée le reste de l'année.
%
%\medskip
% 
%\textbf{Information 1 : Dépenses liées à  cette maison pour l'année 2013}
%
%\medskip
% 
%Le diagramme ci-dessous présente, pour chaque mois, le total des dépenses dues aux différentes taxes, aux abonnements (électricité, chauffage, eau, internet), au remplissage et au chauffage de la piscine.
% 
%\begin{center}
%\psset{xunit=0.75cm,yunit=0.008cm}
%\begin{pspicture}(-1,-110)(13,700)
%\psaxes[linewidth=1.25pt,Dx=20,Dy=100](0,0)(13,601)
%\multido{\n=0+50}{13}{\psline(0,\n)(13,\n)}
%\rput(2,650){Dépenses (en \euro)}
%\psframe[fillstyle=solid,fillcolor=lightgray](0.75,0)(1.25,250)
%\psframe[fillstyle=solid,fillcolor=lightgray](1.75,0)(2.25,250)
%\psframe[fillstyle=solid,fillcolor=lightgray](2.75,0)(3.25,250)
%\psframe[fillstyle=solid,fillcolor=lightgray](3.75,0)(4.25,250)
%\psframe[fillstyle=solid,fillcolor=lightgray](4.75,0)(5.25,450)
%\psframe[fillstyle=solid,fillcolor=lightgray](5.75,0)(6.25,550)
%\psframe[fillstyle=solid,fillcolor=lightgray](6.75,0)(7.25,550)
%\psframe[fillstyle=solid,fillcolor=lightgray](7.75,0)(8.25,550)
%\psframe[fillstyle=solid,fillcolor=lightgray](8.75,0)(9.25,550)
%\psframe[fillstyle=solid,fillcolor=lightgray](9.75,0)(10.25,300)
%\psframe[fillstyle=solid,fillcolor=lightgray](10.75,0)(11.25,150)
%\psframe[fillstyle=solid,fillcolor=lightgray](11.75,0)(12.25,150)
%\rput(1,-80){\rotatebox{45}{janvier}}
%\rput(2,-80){\rotatebox{45}{février}}
%\rput(3,-80){\rotatebox{45}{mars}}
%\rput(4,-80){\rotatebox{45}{avril}}
%\rput(5,-80){\rotatebox{45}{mai}}
%\rput(6,-80){\rotatebox{45}{juin}}
%\rput(7,-80){\rotatebox{45}{juillet}}
%\rput(8,-80){\rotatebox{45}{août}}
%\rput(9,-80){\rotatebox{45}{septembre}}
%\rput(10,-80){\rotatebox{45}{octobre}}
%\rput(11,-80){\rotatebox{45}{novembre}}
%\rput(12,-80){\rotatebox{45}{décembre}}
%\end{pspicture}
%\end{center} 

%\textbf{Information 2 : Remboursement mensuel du prêt}
%
%\medskip
% 
%Chaque mois, le couple doit verser 700~euros à  sa banque pour rembourser le prêt.
%
%\medskip
% 
%\textbf{Information 3 : Tarif de location de la maison}
%
%$\bullet~~$ Les locations se font du samedi au samedi. 
%
%$\bullet~~$Le couple loue sa maison du samedi 7 juin au samedi 27 septembre 2014. 
%
%$\bullet~~$Les tarifs pour la location de cette maison sont les suivants :
%
%\begin{center}
%\begin{tabularx}{0.85\linewidth}{|*{3}{>{\centering \arraybackslash}X|}c|}\hline 
%\textbf{Début}& \textbf{Fin}& \textbf{Nombre de semaines}&\textbf{Prix de la location}\\ \hline 
%07/06/2014& 05/07/2014& 4 semaines &750 euros par semaine\\ \hline  
%05/07/2014 &23/08/2014 &7 semaines &... euros par semaine\\ \hline  
%23/08/2014 &27/09/2014 &5 semaines &750 euros par semaine\\ \hline 
%\end{tabularx}
%\end{center}
% 
%Pour l'année 2014, avec l'augmentation des différents tarifs et taxes, le couple prévoit que le montant des dépenses liées à  la maison sera 6\,\% plus élevé que celui pour 2013.
% 
%Expliquer pourquoi le total des dépenses liées à  la maison s'élèvera à  \np{4505}~\euro{} en 2014.
Dépenses de 2013 : $4 \times 250 + 450 + 4 \times 550 + 300 + 2 \times 150 = \np{1000} + 450 + \np{2200} + 300 + 300 = \np{4250}$~(\euro).

Avec une augmentation moyenne de 6\,\% en 2014, les dépenses s'élèveront en 2014 à  : 

$\np{4250} \times 1,06 = \np{4505}$~(\euro).

%On suppose que le couple arrive à  louer sa maison durant toutes les semaines de la période de location. à€ quel tarif minimal (arrondi à  la dizaine d'euros) doit-il louer sa maison entre le 5/07 et 23/08 pour couvrir les frais engendrés par la maison sur toute l'année 2014 ? 
Soit $x$ le prix de la location semaine en juillet-août.

Le couple recevra :

$(4 + 5) \times 750 + 7x = \np{6750} + 7x$

Les frais seront couverts si :

$\np{6750} + 7x \geqslant \np{4505} + 12 \times 700$ soit $\np{6750} + 7x \geqslant \np{12905}$ ou encore  $7x \geqslant \np{6155}$ soit enfin $x \geqslant \dfrac{\np{6155}}{7}$.

Or $\dfrac{\np{6155}}{7} \approx 879,28$.

Le couple doit louer en juillet-août au tarif minimal de 880~\euro.
\end{enumerate}

%\bigskip
% 
%\textbf{Maà®trise de la langue \hfill 4 points}
\end{document}\end{document}