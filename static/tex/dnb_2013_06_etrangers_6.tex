\textbf{\textsc{Exercice 6} \hfill 4 points}

\medskip

\textbf{Dans cet exercice, toute trace de recherche, même incomplète, sera prise en compte dans l'évaluation.}

\medskip
 
\emph{On considère la figure ci-dessous, qui n'est pas en vraie grandeur.}

\bigskip

\parbox{0.6\linewidth}{\psset{unit=0.6cm}
\begin{pspicture}(-0.2,-0.2)(10,7.3)
\psframe(3,7)(9,0)
\psline(3,7)(0,7)(9,2.8)
\uput[u](0,7){A} \uput[u](3,7){B} \uput[u](9,7){C} 
\uput[ur](3,5.5){M} \uput[r](9,2.8){F} \uput[ur](3,0){E} 
\uput[ur](9,0){D} \uput[u](1.5,7){3} \uput[u](6,7){6} 
\end{pspicture}}\hfill
\parbox{0.38\linewidth}{BCDE est un carré de 6 cm de côté.
 
Les points A, B et C sont alignés et AB = 3 cm.
 
F est un point du segment [CD].
 
La droite (AF) coupe le segment [BE] en M.} 

\medskip

Déterminer la longueur CF par calcul ou par construction pour que les longueurs BM et FD soient égales. 

\bigskip

