\textbf{Exercice 2 \hfill 4 points}

\bigskip

Pour choisir un écran de télévision, d'ordinateur ou une tablette tactile, on peut s'intéresser : 
\setlength\parindent{5mm}
\begin{itemize}
\item[$\bullet~~$] à son format qui est le rapport longueur de l'écran largeur de l'écran 
\item[$\bullet~~$] à sa diagonale qui se mesure en pouces. Un pouce est égal à 2,54~cm.
\end{itemize}
\setlength\parindent{0mm}

\medskip
 
\begin{enumerate}
\item Un écran de télévision a une longueur de 80~cm et une largeur de 45~cm.
 
S'agit-il d'un écran de format $\dfrac{4}{3}$ ou $\dfrac{16}{9}$ ? 
\item Un écran est vendu avec la mention \og 15 pouces \fg. On prend les mesures suivantes : la longueur est 30,5~cm et la largeur est 22,9~cm. 

La mention \og 15 pouces \fg{} est-elle bien adaptée à cet écran ? 
\item Une tablette tactile a un écran de diagonale 7 pouces et de format $\dfrac{4}{3}$ 
Sa longueur étant égale à 14,3~cm, calculer sa largeur, arrondie au mm près.
\end{enumerate}
 
\bigskip
 
