\textbf{\textsc{Exercice 4} \hfill 6 points}

\medskip

Représentons de nouveau le triangle complété par  les longueurs  données dans l'énoncé.

\begin{center}
\psset{unit=1cm}
\begin{pspicture}(-1,0)(7,6)
\psline[linewidth=1.25pt](1,1)(1,5.8)
\psline[linewidth=1.25pt](3.4,1)(3.4,2.3)
\psline[linestyle=dotted](1,1)(4.3,1)(1,5.8)
%\uput[r](-0.8,4.8){Cristo}
%%\uput[r](-0.8,4.2){Redentor}
%\psline{->}(0,4)(1,2.9)
%\uput[l](2.4,1.7){Julien}
%\psline{->}(2.2,1.5)(3.4,1.7)
%\rput{-55}(3.9,2.){regard de Magali}
\psline[linestyle=dashed]{<->}(1,0.5)(3.4,0.5)
\psline[linestyle=dashed]{<->}(3.4,0.5)(4.3,0.5)
\uput[d](3.85,0.5){0,5~m}
\uput[d](2.2,0.5){9,5~m}
\uput[u](1,5.8){S} \uput[d](1,1){C} \uput[l](3.4,2.4){T} \uput[d](3.4,1){J} \uput[dr](4.3,1){M}
\uput[l](3.4,1.7){1,90~m} 
\end{pspicture}
\end{center} 

Pour déterminer la longueur de la hauteur [SC], il faut utiliser le théorème de Thalès. Cependant, avant de s'engager dans sa formulation, nous devons vérifier que les droites (SC) et (TJ) sont parallèles, condition à l'utilisation du théorème.

On sait que : (SC) et  (CM) d'une part  et (TJ) et  (CM) de l'autre sont perpendiculaires.

Or : si deux droites sont perpendiculaires à une même troisième, alors elles sont parallèles.
Donc : (SC) // (TJ).

On peut maintenant passer à l'énonciation du théorème de Thalès en réunissant toutes les conditions
nécessaires.

On sait que: S, T et M sont alignés ainsi que C, J et M. De plus, (SC) // (TJ).

Donc d'après Thalès  : $\dfrac{\text{MT}}{\text{MS}} = \dfrac{\text{MJ}}{\text{MC}} = \dfrac{\text{TJ}}{\text{SC}}$

Soit ici :  $\dfrac{\text{MT}}{\text{MS}} = \dfrac{0,5}{10} = \dfrac{1,9}{\text{SC}}$.

D'où avec les deux derniers quotients SC $= \dfrac{1,9 \times 10}{0,5} = 38$~m.

La statue mesure environ 38 mètres.
\vspace{0,25cm}

