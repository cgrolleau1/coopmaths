\documentclass[10pt]{article}
\usepackage[T1]{fontenc}
\usepackage[utf8]{inputenc}%ATTENTION codage UTF8
\usepackage{fourier}
\usepackage[scaled=0.875]{helvet}
\renewcommand{\ttdefault}{lmtt}
\usepackage{amsmath,amssymb,makeidx}
\usepackage[normalem]{ulem}
\usepackage{diagbox}
\usepackage{fancybox}
\usepackage{tabularx,booktabs}
\usepackage{colortbl}
\usepackage{pifont}
\usepackage{multirow}
\usepackage{dcolumn}
\usepackage{enumitem}
\usepackage{textcomp}
\usepackage{lscape}
\newcommand{\euro}{\eurologo{}}
\usepackage{graphics,graphicx}
\usepackage{pstricks,pst-plot,pst-tree,pstricks-add}
\usepackage[left=3.5cm, right=3.5cm, top=3cm, bottom=3cm]{geometry}
\newcommand{\R}{\mathbb{R}}
\newcommand{\N}{\mathbb{N}}
\newcommand{\D}{\mathbb{D}}
\newcommand{\Z}{\mathbb{Z}}
\newcommand{\Q}{\mathbb{Q}}
\newcommand{\C}{\mathbb{C}}
\usepackage{scratch}
\renewcommand{\theenumi}{\textbf{\arabic{enumi}}}
\renewcommand{\labelenumi}{\textbf{\theenumi.}}
\renewcommand{\theenumii}{\textbf{\alph{enumii}}}
\renewcommand{\labelenumii}{\textbf{\theenumii.}}
\newcommand{\vect}[1]{\overrightarrow{\,\mathstrut#1\,}}
\def\Oij{$\left(\text{O}~;~\vect{\imath},~\vect{\jmath}\right)$}
\def\Oijk{$\left(\text{O}~;~\vect{\imath},~\vect{\jmath},~\vect{k}\right)$}
\def\Ouv{$\left(\text{O}~;~\vect{u},~\vect{v}\right)$}
\usepackage{fancyhdr}
\usepackage[french]{babel}
\usepackage[dvips]{hyperref}
\usepackage[np]{numprint}
%Tapuscrit : Denis Vergès
%\frenchbsetup{StandardLists=true}

\begin{document}
\setlength\parindent{0mm}
% \rhead{\textbf{A. P{}. M. E. P{}.}}
% \lhead{\small Brevet des collèges}
% \lfoot{\small{Polynésie}}
% \rfoot{\small{7 septembre 2020}}
\pagestyle{fancy}
\thispagestyle{empty}
% \begin{center}
    
% {\Large \textbf{\decofourleft~Brevet des collèges Polynésie 7 septembre 2020~\decofourright}}
    
% \bigskip
    
% \textbf{Durée : 2 heures} \end{center}

% \bigskip

% \textbf{\begin{tabularx}{\linewidth}{|X|}\hline
%  L'évaluation prend en compte la clarté et la précision des raisonnements ainsi que, plus largement, la qualité de la rédaction. Elle prend en compte les essais et les démarches engagées même non abouties. Toutes les réponses doivent être justifiées, sauf mention contraire.\\ \hline
% \end{tabularx}}

% \vspace{0.5cm}\textbf{\textsc{Exercice 6} \hfill 6,5 points}

\medskip

%La gélule est une forme médicamenteuse utilisée quand le médicament qu'elle contient 'a
%une odeur forte ou un goût désagréable que l'on souhaite cacher.
%
%On trouve des gélules de différents calibres. Ces calibres sont numérotés de \og 000 \fg{} à \og 5\fg{} comme le montre l'illustration ci-contre (\og 000 \fg{} désignant le plus grand calibre et \og 5\fg{} désignant le plus petit) :
%
%
%\begin{center}
%\psset{unit=1cm}
%\def\gelule{\psline(-0.6,0.55)(-0.6,1.2)\psline(0.6,0.55)(0.6,1.2)\psarc(0,0.55){0.55}{180}{0}
%\psline(-0.7,2.4)(-0.7,1.2)(0.7,1.2)(0.7,2.4)\psarc(0,2.4){0.65}{0}{180}}
%\begin{pspicture}(8.6,3.8)
%%\psgrid
%\psframe(8.65,3.8)
%\rput(0.8,0.1){\gelule}\rput(0.8,2.6){000}
%\rput(2.25,0.1){\psscalebox{0.94}{\gelule}}\rput(2.25,2.45){00}
%\rput(3.4,0.1){\psscalebox{0.81}{\gelule}}\rput(3.4,2.2){0}
%\rput(4.6,0.1){\psscalebox{0.74}{\gelule}}\rput(4.6,1.95){1}
%\rput(5.65,0.1){\psscalebox{0.67}{\gelule}}\rput(5.65,1.75){2}
%\rput(6.6,0.1){\psscalebox{0.63}{\gelule}}\rput(6.6,1.6){3}
%\rput(7.5,0.1){\psscalebox{0.55}{\gelule}}\rput(7.5,1.45){4}
%\rput(8.25,0.1){\psscalebox{0.44}{\gelule}}\rput(8.25,1.15){5}
%\end{pspicture}
%\end{center}
%
%Le tableau suivant donne la longueur de ces différents calibres de gélule :
%
%\begin{center}
%\begin{tabularx}{\linewidth}{|l|*{8}{>{\centering \arraybackslash}X|}}\hline
%Calibre de la gélule 				&000 &00 	&0 		&1 		&2 		&3 	&4 		&5\\ \hline
%Longueur $L$ de la gélule (en mm) 	&26,1&23,3 	&21,7 	&19,4 	&18,0 &15,9 &14,3 &11,1\\ \hline
%\multicolumn{9}{r}{\emph{\scriptsize Source: \og Technical Reference File 1st edition CAPSUGEL - Gélules Coni-Snap}}
%\end{tabularx}
%\end{center} 
%
%\medskip
%
%\parbox{0.5\linewidth}{On considère une gélule constituée de deux demi-sphères
%identiques de diamètre 9,5 mm et d'une partie cylindrique d'une
%hauteur de 16,6 mm comme l'indique le croquis ci-contre.}\hfill
%\parbox{0.48\linewidth}{\psset{unit=1cm}
%\begin{pspicture}(-1.65,-0.6)(1.65,5.5)
%
%\rput{-25}(0,0){\scalebox{.99}[0.3]{\psarc[linewidth=1pt](0,3.3){1.15cm}{180}{0}}%
%\scalebox{.99}[0.3]{\psarc[linestyle=dashed](0,3.3){1.15cm}{0}{180}}%
%\scalebox{.99}[0.3]{ \psarc[linewidth=1pt](0,13.3){1.15cm}{180}{0}}%
%\scalebox{.99}[0.3]{\psarc[linestyle=dashed](0,13.3){1.15cm}{0}{180}}%
%\psarc(0,1){1.15cm}{-180}{0}
%\psarc(0,4){1.15cm}{0}{180}
%\psline(-1.15,1)(-1.15,4)\psline(1.15,1)(1.15,4)
%\psline{<->}(1.4,0)(1.4,4.8)
%\psline{<->}(-1.4,1)(-1.4,4)
%\psline{<->}(-1.15,5.3)(1.15,5.3)
%\uput[u](0,5.3){9,5 mm}
%\uput[r](1.4,2.4){$L$}\rput{90}(-1.55,2.45){16,6 mm}
%\psdots[dotstyle=+,dotangle=45](0,1)(0,4)}
%
%\end{pspicture}
%
%{\footnotesize Cette représentation n'est pas
%en vraie grandeur.}}

\begin{enumerate}
\item %À quel calibre correspond cette gélule ?
%Justifier votre réponse.
$16,6 + 9,5 = 26,1$ mm. Cette gélule correspond au calibre 000.
\item %Calculer le volume arrondi au mm$^3$ de cette gélule.

%\emph{On rappelle les formules suivantes :}

%\begin{center}
%\begin{tabularx}{\linewidth}{|*{3}{>{\centering \arraybackslash \scriptsize}X|}}\hline
%Volume d'un cylindre de rayon& Volume d'un cône de rayon de&  Volume d'une sphère\\
%$R$ et de hauteur $h$ & base $R$ et de hauteur $h$ & de rayon $R$ :\\
%\rule[-3mm]{0mm}{8mm}$V = \pi \times  R^2 \times h$& $ V = \dfrac{\pi \times  R^2 \times h}{3}$&$V = \dfrac{4}{3} \times \pi \times  R^3$\\ \hline
%\end{tabularx}
%\end{center}
$V_{\text{gélule}} = V_{\text{cylindre}} + V_{\text{sphère}}$.

$V_{\text{gélule}} = \pi \times 4,75^2 \times 16,6 + \dfrac{4}{3}
\times \pi \times  4,75^3$

$V_{\text{gélule}} = \np{374,5375}\pi + \dfrac{\np{428,6875}}{3}\pi$

$V_{\text{gélule}} \approx \np{1626}$ mm$^3$.

Le volume de la gélule, arrondie au mm$^3$, est de \np{1626}$ mm$^3$.
\item %Robert tombe malade et son médecin lui prescrit comme traitement une boîte
%d'antibiotique conditionné en gélules correspondant au croquis ci-dessus.

%Chaque gélule de cet antibiotique a une masse volumique de $6,15 \times  10^{-4}$ g/mm$^3$.
%La boîte d'antibiotique contient 3 plaquettes de 6 gélules.
%
%Quelle masse d'antibiotique Robert a-t-il absorbée durant son traitement ? Donner le
%résultat en grammes arrondi à l'unité.
$3 \times 6 = 18$. Dans une boîte d’antibiotique, il y a $18$ gélules.

$18 \times \np{1626} = \np{29268}~mm$^3$. 

Le volume des 18 gélules est d’environ \np{29268}~mm$^3$.

$\np{29268}  \times 6,15 \times 10^{- 4}$ \approx  18~(g).

Pendant la durée de son traitement, Robert a absorbé environ 18~g d’antibiotique.
\end{enumerate}
\end{document}\end{document}