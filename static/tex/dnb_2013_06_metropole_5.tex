\textbf{\textsc{Exercice} 5 \hfill 7 points}

\medskip

\parbox{0.5\linewidth}{Pour réaliser un abri de jardin en parpaing, un bricoleur a besoin de $300$~parpaings de dimensions 50 cm $\times$ 20 cm $\times$ 10 cm pesant chacun $10$~kg. 

Il achète les parpaings dans un magasin situé à $10$~km de sa  maison. Pour les transporter, il loue au magasin un fourgon.} \hfill \parbox{0.42\linewidth}{\psset{unit=1cm}
\begin{pspicture}(0,-0.5)(5,5)
\psline(0.1,2.2)(3.2,0.3)(4.1,0.6)(4.1,2.3)(3.2,2)(3.2,0.3)
\psline(0.1,2.2)(0.1,3.9)(1,4.2)(4.1,2.3)
\psline(0.1,3.9)(3.2,2)
\psline[linewidth=0.6pt,arrowsize=3pt 3]{<->}(0.1,1.9)(3.2,0)\rput(1.5,0.6){50 cm}
\psline[linewidth=0.6pt,arrowsize=3pt 3]{<->}(4.4,0.6)(4.4,2.3)
\uput[r](4.4,1.2){20 cm}
\psline[linewidth=0.6pt,arrowsize=3pt 3]{<->}(0.1,4.1)(1,4.4)\rput(0.4,4.6){10 cm}
\end{pspicture}}

\vspace{0,5cm}

\textbf{Information 1} : Caractéristiques du fourgon :

\bigskip
 
\begin{itemize}
\item 3 places assises. 
\item Dimensions du volume transportable (L $\times   l \times h$) : 

2,60 m $\times$ 1,56 m $\times$ 1,84 m. 
\item Charge pouvant être transportée : $1,7$ tonne.
\item Volume réservoir : $80$ litres. 
\item Diesel (consommation : $8$ litres aux $100$ km). 
\end{itemize}

\bigskip

\textbf{Information} 2 : Tarifs de location du fourgon

\medskip

\begin{tabularx}{\linewidth}{|*{5}{>{\centering \arraybackslash}X|}}\hline 
1 jour			& 1 jour 			&1 jour			&1 jour			& km\\
30 km maximum 	&50 km maximum 		&100 km maximum &200 km maximum	&supplémentaire\\ \hline 
48~\euro 		&55~\euro 			&61~\euro 		&78~\euro		&2~\euro\\ \hline
\multicolumn{5}{l}{\emph{Ces prix comprennent le kilométrage indiqué hors carburant}}\\
\end{tabularx} 

\bigskip

\textbf{Information} 3 : Un litre de carburant coûte $1,50$~\euro.
 
\medskip
 
\begin{enumerate}
\item Expliquer pourquoi il devra effectuer deux aller-retour pour transporter les $300$~parpaings jusqu'à sa maison. 
\item Quel sera le coût total du transport ? 
\item Les tarifs de location du fourgon sont-ils proportionnels à la distance maximale autorisée par jour ? 
\end{enumerate}

\bigskip

