
\medskip

\begin{enumerate}
\item Comme OC = 3OA, le rapport de l'homthétie permettant de passet de la figure A à la figure C est 3.
\item Comme $\dfrac{3}{5} = 3 \times \dfrac{1}{5}$ et que OD = 5OA :

l'homothétie de centre O et de rapport $\dfrac{1}{5}$ permet de passer de la figure E à la figure A, puis 
l'homothétie de centre O et de rapport $3$ permet de passer de la figure A à la figure C. On est donc passé de la figure E à la figure C.
\item Si l'aire est quatre fois plus grande, c'est que les longueurs sont deux fois plus grandes : c'est donc la figure B donc l'aire est quatre fois celle de la figure A.
\end{enumerate}

\bigskip

