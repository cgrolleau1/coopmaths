
\begin{center}
\medskip

Cet exercice est un QCM (questionnaire à choix multiple).

Pour chaque ligne du tableau, une seule affirmation est juste.

\medskip
\end{center}
\textbf{Indiquer le numéro de la question et recopier l'affirmation juste sur votre copie.\\
Aucune justification n'est attendue. Aucun point n'est retiré en cas de mauvaise
réponse.}

\begin{center}
\begin{tabular}{|m{6cm}|c|c|c|}\hline
%\begin{tabularx}{\linewidth}{|m{5.75cm}|*{3}{>{\centering arraybackslash}X|}}\hline
\textbf{Questions}&\multicolumn{3}{|c|}{\textbf{Affirmations}}\\ \hline
			&A 		&B 		&C\\ \hline
\textbf{1.} Combien faut-il environ de CD de $700$ Mégaoctets pour stocker autant de données qu'une clé de 32 Gigaoctets ?&46 	&\np{4600} 	&\np{4600000}\\ \hline 
\textbf{2.} La diagonale d'un rectangle de 10 cm par 20 cm est d'environ:	&15 cm &22 cm &30 cm\\ \hline
\textbf{3.}   Une solution de l'équation 

$2x + 3 = 7x - 4$ est: &$\dfrac{5}{7}$&1,4 &$- 0,7$\\ \hline
\textbf{4.}  La fraction irréductible de la fraction $\dfrac{882}{\np{1134}}$ est : \rule[-3mm]{0mm}{8mm}&$\dfrac{14}{9}$& $\dfrac{63}{81}$& $\dfrac{7}{9}$\\ \hline
\textbf{5.}   On considère la fonction
 
$f \: : x \longmapsto  3x + 4$.

Quelle formule doit-on entrer en B2 puis recopier vers la droite afin de calculer les
images des nombres de la ligne 1 par la fonction $f$ ?

\begin{tabularx}{\linewidth}{|c|*{4}{>{\centering \arraybackslash}X|}}\hline
B2	&		&$f_x$	&	& \\ \hline
	&A		&B		&C	&D\\ \hline
1	&$x$	&5		&6	&7\\ \hline
2	&$f(x)$	&		&	&\\ \hline
3	&		&		&	&\\ \hline
\end{tabularx}
&$=3\star \text{A}1 + 4$& $= 3 \star 5 + 4$ & $=3\star \text{B}1 + 4$\\ \hline
\end{tabular}
\end{center}

\bigskip

