\textbf{\textsc{Exercice 6 \hfill 6 points}}

\medskip

\begin{enumerate}
  \item $12,5 + 10 = 22,5$\\
  La distance d'arrêt du scooter est donc de 22,5 m à 45 km/h.
  \item
  \begin{enumerate}
    \item D'après le graphique, si la distance de réaction est de 15 m, la vitesse est de 55 km/h.
    \item La distance de freinage n'est pas proportionnelle à la vitesse car la représentation graphique n'est pas une droite.
    \item D'après le graphique, si une voiture roule à 90 km/h, alors :
    \begin{itemize}
      \item la distance de réaction est de 25 m ;
      \item la distance de freinage est de 40 m ;
    \end{itemize}
    La distance d'arrêt est donc de 40 + 25 = 65 m.
  \end{enumerate}
  \item $\dfrac{110^2}{152,4} \approx 79$\\
  La distance de freinage sur route mouillée à 110 km/h est donc d'environ 79~m.
\end{enumerate}

\medskip

