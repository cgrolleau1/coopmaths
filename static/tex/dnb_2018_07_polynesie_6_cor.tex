
\medskip

\begin{enumerate}
\item Résultat 1 prend la valeur : $2 \times 3 + 3 = 6 + 3 = 9$, puis Résultat 1 prend la valeur : $9 \times 9 = 81$.

Résultat 2 prend la valeur $3 \times 3 = 9$, puis la valeur $9 \times 4 = 36$, puis la valeur $36 + 12 \times 3 = 36 + 36 = 72$ et enfin la valeur $72 + 9 = 81$.
\item
	\begin{enumerate}
		\item En remplaçant 3 par $x$, Résultat 1 prend la valeur : $2 \times x + 3 = 2x + 3$, puis Résultat 1 prend la valeur : $(2x + 3)\times (2x + 3) = (2x + 3)^2$.
		\item Résultat 2 prend la valeur $x \times x = x^2$, puis la valeur $x^2 \times 4 = 4x^2$, puis la valeur $4x^2 + 12 \times x = 4x^2 + 12x$ et enfin la valeur $4x^2 + 12x + 9$.
		\item On a vu dans la question précédente que pour un nombre choisi $x$, le Résultat 2 est 
		
$4x^2 + 12x + 9$.
		
Il faut donc trouver $x$ tel que :
		
$4x^2 + 12x + 9 = 9$, soit $4x^2 + 12x = 0$ ou en factorisant :
		
$4x(x + 3) = 0$ : il y a donc deux possibilités :
		
$x = 0$ ou $x + 3 = 0$, soit  $x = - 3$.
		
Conclusion : Alice a introduit $0$ ou $- 3$.
	\end{enumerate}
\end{enumerate}
