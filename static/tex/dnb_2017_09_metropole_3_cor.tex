
\medskip

%Voici trois figures différentes, aucune n'est à  l'échelle indiquée dans l'exercice :
%
%\begin{center}
%\begin{tabularx}{\linewidth}{*{3}{>{\centering \arraybackslash}X}}
%\psset{unit=0.9cm}
%\begin{pspicture}(4.9,4.3)
%\psline[linecolor=blue](0,0)(0,4.3)(3.8,4.3)(3.8,1)(1,1)(1,3.3)(2.7,3.3)(2.7,2.2)(2.2,2.2)
%\end{pspicture}&\psset{unit=0.8cm}
%\begin{pspicture}(4.9,4.3)
%\psline[linecolor=blue](4.9,2.3)(3.6,0)(0.9,0)(0,2.2)(1,3.4)(2.4,3.4)(3,2.3)(2.7,1.7)(2,1.7)
%\end{pspicture}&\psset{unit=0.9cm}
%\begin{pspicture}(4.9,4.3)
%\psline[linecolor=blue](0.5,0.5)(0.5,3.4)(3.4,3.4)(3.4,1.4)(1.2,1.4)(1.2,2.7)(2.7,2.7)(2.7,1.9)(1.9,1.9)
%\end{pspicture}\\
%figure 1 &figure 2 &figure 3\\
%\end{tabularx}
%\end{center}
%
%Le programme ci-dessous contient une variable nommée \og \textbf{longueur} \fg.
%
%\begin{minipage}{.4\linewidth}
%Script
%\begin{scratch}
%	\blockinit{Quand \greenflag est cliqué}
%	\blocklook{cacher}
%	\blockmove{aller à  x: \ovalnum{0} y: \ovalnum{0}}
%	\blockmove{s’orienter à  \ovalnum{90\selectarrownum} degrés}
%	\blockvariable{mettre \selectmenu{longueur} à    \ovalnum{30} }
%	\blocklook{effacer tout}	
%	\blocklook{mettre la taille du stylo à  \ovalnum{3}}
%	\blocklook{stylo en position d'écriture}		
%		\blockrepeat{répéter \ovalnum{2} fois}
%	{
%		\blockevent{un tour}
%		\blockvariable{ajouter  à  \selectmenu{longueur}    \ovalnum{30} }
%	}
%\end{scratch}
%\end{minipage}
%\begin{minipage}{.2\linewidth}
%\end{minipage}
%\begin{minipage}{.4\linewidth}
%{Le~bloc~:~\textbf{un~tour}}
%\begin{scratch}
%	\blockinit{Définir un tour}
%		\blockrepeat{répéter \ovalnum{2} fois}
%		{\blockmove{avancer de \selectmenu{longueur} }
%		\blockmove{tourner de \ovalnum{90\selectarrownum} degrés}
%		}		
%	\blockvariable{ajouter à   \selectmenu{longueur}  \ovalnum{30} }
%		\blockrepeat{répéter \ovalnum{2} fois}
%		{
%		\blockmove{avancer de \selectmenu{longueur} }
%		\blockmove{tourner de \ovalnum{90\selectarrownum} degrés}									}
%\end{scratch}
%\end{minipage}
%
%On rappelle que l'instruction \begin{scratch} \blockmove{s’orienter à  \ovalnum{90\selectarrownum} degrés}  \end{scratch} signifie que l'on s'oriente vers la droite avec le stylo.
%
%\medskip

\begin{enumerate}
\item 
	\begin{enumerate}
		\item %Dessiner la figure obtenue avec le bloc \og un tour\fg{} donné dans le cadre de droite ci-dessus, pour une longueur de départ égale à  30, étant orienté vers la droite avec le stylo, en début de tracé. On prendra 1~cm pour 30~unités de longueur, c'est-à -dire 30 pixels.
		~
\begin{center}
\psset{unit=1cm}
\begin{pspicture}(-2.1,-2.1)(2.1,2.1)
\psline(0,0)(1,0)(1,1)(-1,1)(-1,-1)%(2,-1)(2,2)(-1,2)(-1,-1)
\end{pspicture}		
\end{center}
		\item %Comment est-on orienté avec le stylo après ce tracé ? (aucune justification n'est demandée)
On a tourné quatre fois de 90\degres, donc fait un tour : le stylo est encore orienté vers la droite.
	\end{enumerate}
\item %Laquelle des figures 1 ou 3 le programme ci-dessus permet-il d'obtenir ? Justifier votre réponse.
Ce ne peut être la figure 1 puisque l'on déplace de 30 puis de 60, alors que dans le tour on répète deux déplacements de 30.

Ce ne peut être la figure 2 puisque l’on tourne après chaque déplacement de 60\degres.

Il ne reste donc que la figure 3.
\item %Quelle modification faut-il apporter au bloc \og \textbf{un tour}\fg{} pour obtenir la figure 2 ci-dessus ?
Les déplacements augmentent bien de longueur à  chaque fois ; il suffit donc de tourner de 60\degres{}  pour obtenir la figure 2.
\end{enumerate}

\vspace{0,5cm}

