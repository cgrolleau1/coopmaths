
\medskip 

%Pour chacune des affirmations suivantes, indiquer sur la copie, si elle est vraie ou fausse.
%
%On rappelle que chaque réponse doit être justifiée.

\begin{itemize}[label=$\bullet~~$]
\item \textbf{Affirmation \no 1} : fausse

On a $94 - 18 = 76$.%\og Dans la série de valeurs ci-dessous, l'étendue est 25. 
%
%Série : 37~;~20~;~18~;~25~;~45~;~94~;~62 \fg.

\item \textbf{Affirmation \no 2} : vraie

%\og Les nombres 70 et 90 ont exactement deux diviseurs premiers en commun \fg.
On a $70 = 7 \times 10 = 7 \times 2 \times 5 = 2 \times 5 \times 7$ ;

$90 = 9 \times 10 = 2 \times 3^2 \times 5$.

70 et 90 ont deux facteurs premiers en commun : 2 et 5.

\item \textbf{Affirmation \no 3} : fausse

%\parbox{0.55\linewidth}
%{
%\og À partir du quadrilatère BUTS, on a obtenu le quadrilatère VRAC par une translation \fg.
%}
%\hfill 
%\parbox{0.43\linewidth}
%{
%\psset{unit=1cm}
%\def\poly{\pspolygon[fillstyle=solid,fillcolor=lightgray](0,0)(1,0)(1,2)(0,1)}
%\begin{pspicture}(5.5,3.8)
%\rput(0.3,0.3){\poly}\rput{180}(4.6,3.3){\poly}
%\uput[dl](0.3,0.3){C}\uput[dr](1.3,0.3){A} \uput[ur](1.3,2.3){R} \uput[ul](0.3,1.3){V} 
%\uput[ur](4.6,3.3){U} \uput[ul](3.6,3.3){B} \uput[dl](3.6,1.3){S} \uput[dr](4.6,2.3){T} 
%\end{pspicture}
%}
Les deux quadrilatères n'ont pas la même orientation.
\item \textbf{Affirmation \no 4} : vraie

%\og Quand on multiplie l'arête d'un cube par 3, son volume est multiplié par 27 \fg.
Chaque dimension étant multipliée par 3, le volume est multiplié par $3 \times 3 \times 3 = 3^3 = 27$.
\end{itemize}

\vspace{0,5cm}

