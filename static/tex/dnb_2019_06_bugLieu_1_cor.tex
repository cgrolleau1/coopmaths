\documentclass[10pt]{article}
\usepackage[T1]{fontenc}
\usepackage[utf8]{inputenc}%ATTENTION codage UTF8
\usepackage{fourier}
\usepackage[scaled=0.875]{helvet}
\renewcommand{\ttdefault}{lmtt}
\usepackage{amsmath,amssymb,makeidx}
\usepackage[normalem]{ulem}
\usepackage{diagbox}
\usepackage{fancybox}
\usepackage{tabularx,booktabs}
\usepackage{colortbl}
\usepackage{pifont}
\usepackage{multirow}
\usepackage{dcolumn}
\usepackage{enumitem}
\usepackage{textcomp}
\usepackage{lscape}
\newcommand{\euro}{\eurologo{}}
\usepackage{graphics,graphicx}
\usepackage{pstricks,pst-plot,pst-tree,pstricks-add}
\usepackage[left=3.5cm, right=3.5cm, top=3cm, bottom=3cm]{geometry}
\newcommand{\R}{\mathbb{R}}
\newcommand{\N}{\mathbb{N}}
\newcommand{\D}{\mathbb{D}}
\newcommand{\Z}{\mathbb{Z}}
\newcommand{\Q}{\mathbb{Q}}
\newcommand{\C}{\mathbb{C}}
\usepackage{scratch}
\renewcommand{\theenumi}{\textbf{\arabic{enumi}}}
\renewcommand{\labelenumi}{\textbf{\theenumi.}}
\renewcommand{\theenumii}{\textbf{\alph{enumii}}}
\renewcommand{\labelenumii}{\textbf{\theenumii.}}
\newcommand{\vect}[1]{\overrightarrow{\,\mathstrut#1\,}}
\def\Oij{$\left(\text{O}~;~\vect{\imath},~\vect{\jmath}\right)$}
\def\Oijk{$\left(\text{O}~;~\vect{\imath},~\vect{\jmath},~\vect{k}\right)$}
\def\Ouv{$\left(\text{O}~;~\vect{u},~\vect{v}\right)$}
\usepackage{fancyhdr}
\usepackage[french]{babel}
\usepackage[dvips]{hyperref}
\usepackage[np]{numprint}
%Tapuscrit : Denis Vergès
%\frenchbsetup{StandardLists=true}

\begin{document}
\setlength\parindent{0mm}
% \rhead{\textbf{A. P{}. M. E. P{}.}}
% \lhead{\small Brevet des collèges}
% \lfoot{\small{Polynésie}}
% \rfoot{\small{7 septembre 2020}}
\pagestyle{fancy}
\thispagestyle{empty}
% \begin{center}
    
% {\Large \textbf{\decofourleft~Brevet des collèges Polynésie 7 septembre 2020~\decofourright}}
    
% \bigskip
    
% \textbf{Durée : 2 heures} \end{center}

% \bigskip

% \textbf{\begin{tabularx}{\linewidth}{|X|}\hline
%  L'évaluation prend en compte la clarté et la précision des raisonnements ainsi que, plus largement, la qualité de la rédaction. Elle prend en compte les essais et les démarches engagées même non abouties. Toutes les réponses doivent être justifiées, sauf mention contraire.\\ \hline
% \end{tabularx}}

% \vspace{0.5cm}\textbf{\textsc{Exercice 1 \hfill 12 points}}

\medskip

%Mathilde fait tourner deux roues de loterie A et B comportant chacune quatre secteurs numérotés comme sur le schéma ci-dessous:
%
%\begin{center}
%
%\psset{unit=1cm} 
%\begin{pspicture}(-2,-2.5)(2,2)
%\pscircle(0,0){2}
%\psline(-2,0)(2,0)\psline(0,2)(0,-2)
%\uput[d](0,-2){Roue A}
%\rput(1;135){1} \rput(1;45){2} \rput(1;-45){3} \rput(1;-135){4} 
%\pspolygon[fillstyle=solid,fillcolor=gray](1.6;60)(2.3;55)(2.3;65)
%\end{pspicture}\hspace{1.5cm}
%\begin{pspicture}(-2,-2.5)(2,2)
%\pscircle(0,0){2}
%\psline(-2,0)(2,0)\psline(0,2)(0,-2)
%\uput[d](0,-2){Roue B}
%\rput(1;135){6} \rput(1;45){7} \rput(1;-45){8} \rput(1;-135){9} 
%\pspolygon[fillstyle=solid,fillcolor=gray](1.6;60)(2.3;55)(2.3;65)
%\end{pspicture}
%\end{center} 
%
%La probabilité d'obtenir chacun des secteurs d'une roue est la même. Les flèches indiquent les deux secteurs obtenus. 
%
%L'expérience de Mathilde est la suivante: elle fait tourner les deux roues pour obtenir un nombre à deux chiffres. Le chiffre obtenu avec la roue A est le chiffre des dizaines et celui avec la roue B est le chiffre des unités. 
%
%\emph{Dans l'exemple ci-dessus, elle obtient le nombre $27$ (Roue A : $2$ et Roue B : $7$)}.

\medskip
 
\begin{enumerate}
\item %Écrire tous les nombres possibles issus de cette expérience.Avec 1 ; 2 ; 3 ; 4 en premier et 6 ; 7 ; 8 ou 9 en second on peut obtenir les nombres :

16 ; 17 ; 18 ; 19 ; 26 ; 27 ; 28 ; 29 ; 36 ; 37 ; 38 ; 39 ; 46 ; 47 ; 48 ; 49, soit 16 nombres. 
\item %Prouver que la probabilité d'obtenir un nombre supérieur à 40 est 0,25.
Il y a 4 nombres supérieurs à 40 sur 16 ; la probabilité est donc égale à $\dfrac{4}{16} = \dfrac{1}{4} = \dfrac{25}{100} = 0,25$.
 
\item %Quelle est la probabilité que Mathilde obtienne un nombre divisible par 3 ? 
Les nombres divisibles par 3 sont : 18 ; 27 ; 36 ; 39 ; 48 : il y en a 5 sr 16 ; la probabilité est donc égale à $\dfrac{5}{16} = \np{0,3125}$.
\end{enumerate} 

\vspace{0.5cm}

\end{document}