\textbf{\textsc{Exercice 3 \hfill 5 points}}

\medskip

\begin{enumerate}
\item 
	\begin{enumerate}
		\item La station a vendu le plus de forfaits de ski durant le mois de février (\np{148901}).
		\item $\np{60457} + \np{60457} + \np{148901} + \np{100058} + \np{10035} = \np{379908}$.
		
Durant la saison, \np{379908} forfaits ont été vendus.
		
$\dfrac{\np{148901}}{\np{379908}} \approx 0,39$ et $\dfrac{1}{3} \approx  0,33$.

Ninon a raison, la station vend plus d'un tiers des forfaits au mois de février.
	\end{enumerate}
\item Il faut saisir la formule : =SOMME(B2:F2).
\item Je calcule : $m = \dfrac{\np{379908}}{5} = \np{75981,6} \approx  \np{75982}$.
	
Le nombre moyen de forfaits vendus par mois est d'environ \np{75982}.
\end{enumerate}


\vspace{0,5cm}

