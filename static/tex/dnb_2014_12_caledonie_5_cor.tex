\textbf{Exercice 5 : Changement climatique}

\begin{enumerate}
\item En différents endroits de Nouvelle-Calédonie, les températures minimales et les températures maximales ont augmenté. Ces informations traduisent une augmentation des températures dans chacun de ces endroits. 
\item C'est à La Roche que la température minimale a le plus augmenté \linebreak (augmentation de 1,5 \degres C). 
\item Augmentation moyenne des températures minimales : \\[2mm]
$\dfrac{5\times1,2+4\times1,3+1,5}{10}=1,27$ \\[2mm]
Les températures minimales ont augmenté en moyenne de 1,27 °C. \\
Augmentation moyenne des températures maximales : \\[2mm]
$\dfrac{0,8+3\times0,9+4\times1,0+2\times1,3}{10}=1,01$ \\[2mm]
Les températures maximales ont augmenté en moyenne de 1,01 °C.
\end{enumerate}

\newpage

