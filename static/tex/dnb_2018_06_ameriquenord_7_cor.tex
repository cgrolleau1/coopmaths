
\medskip
%Les trois questions suivantes sont indépendantes.

\begin{enumerate}
	\item %$\text{A} = 2x(x - 1) - 4 (x - 1)$.
	
%Développer et réduire l'expression A.
$\text{A} = 2x(x - 1) - 4 (x - 1) = 2x^2 - 2x  - 4x + 4 = 2x^2 - 6x + 4$.
	\item %Montrer que le nombre $-5$ est une solution de l'équation \quad $(2x + 1) \times (x-2) = 63$.
$(2\times {\blue - 5}+ 1) \times ({\blue - 5}-2) = (- 10 + 1) \times (- 7) = - 9 \times (- 7) = 63$.	
	\item %On considère la fonction $f$ définie par \quad $f(x) = -3x + 1,5$.
		\begin{enumerate}
			\item %Parmi les deux graphiques ci-dessous, quel est celui qui représente la fonction $f$ ?
L'ordonnée à l'origine est égale à $1,5$.

De plus le coefficient directeur est égal à $- 3$. C'est donc la droite (d$_2$) qui représente la fonction $f$.	
			\item %Justifiez votre choix.
			Voir ci-dessus.
		\end{enumerate}
	\end{enumerate}

%	\begin{tabularx}{\linewidth}{XX}
%		\begin{tikzpicture}[>=stealth]
%		\node at(1,-2.75){Graphique A};
%		 \clip (-2.25,-2.25) rectangle (4.25,4.25);
%		 \draw[xstep = 0.5,ystep = 0.5, gray!70] (-2.25,-2.25) grid (4.25,4.25);
%		 \draw [->] (-2.25,0) --(4.25,0);
%		 \draw [->] (0,-2.25) --(0,4.25);
%		 \foreach \n in {-2,-1,1,2,3,4}{
%			\node at (\n,0) [below ]{\np{\n}}; 
%			\node at (0,\n) [left]{\np{\n}};
%	 	}
% 		\node at (0,0) [below left] {0};
% 		\draw[domain = 0.5:3.5,line width=1.25pt] plot (\x,{3*\x-4.5});
% 		\node at (3.45,3.75) {(d$ _1 $)};
%		\end{tikzpicture}&
%				\hfill \begin{tikzpicture}[>=stealth]
%		\node at(1,-2.75){Graphique B};
%		\clip (-2.25,-2.25) rectangle (4.25,4.25);
%		\draw[xstep = 0.5,ystep = 0.5, gray!70] (-2.25,-2.25) grid (4.25,4.25);
%		\draw [->] (-2.25,0) --(4.25,0);
%		\draw [->] (0,-2.25) --(0,4.25);
%		\foreach \n in {-2,-1,1,2,3,4}{
%			\node at (\n,0) [below ]{\np{\n}}; 
%			\node at (0,\n) [left]{\np{\n}};
%		}
%		\node at (0,0) [below left] {0};
%		\draw[domain = -1.5:1.5,line width=1.25pt] plot (\x,{-3*\x+1.5});
%		\node at (-1.35,3.75) {(d$ _2 $)};
%		\end{tikzpicture}
%	\end{tabularx}

\vspace{5mm}

