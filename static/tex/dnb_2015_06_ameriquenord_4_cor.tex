\textbf{\textsc{Exercice 4} \hfill 4 points}

\medskip

%Trouver le nombre auquel je pense.
%
%\setlength\parindent{1.5cm}
%\begin{itemize}
%\item[$\bullet~~$] Je pense à un nombre.
%\item[$\bullet~~$] Je lui soustrais $10$.
%\item[$\bullet~~$] J'élève le tout au carré.
%\item[$\bullet~~$] Je soustrais au résultat le carré du nombre auquel j'ai pensé.
%\item[$\bullet~~$] J'obtiens alors : $- 340$.
%\end{itemize}
%\setlength\parindent{0cm}
Notons $x$ le nombre auquel l'on pense.

$\bullet~~$	$x$

$\bullet~~$	$x - 10$

$\bullet~~$	$(x - 10)^2 = (x - 10)(x - 10) = x^2 - 10x - 10x + 100 = x^2 - 20x + 100$

$\bullet~~$	$x^2 - 20x + 100 - x^2 = - 20x + 100$

Le résultat obtenu est : $- 20x + 100$.

On résout l'équation :	$- 20x + 100	=	- 340$

				$- 20x	=	-440$
				
				$20x	=	440$
				
				$x	=	22$.
				
Le nombre auquel on pense au départ est donc $22$.
\vspace{0,5cm}

