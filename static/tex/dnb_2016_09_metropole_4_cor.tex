\textbf{\textsc{Exercice 4} \hfill 4,5 points}

\medskip

Question 1 : Le nombre 2 est solution de l'inéquation : \textbf{c.}\quad  $5x - 4 \leqslant  7$.

Question 2 : La fonction $f$ qui à tout nombre $x$ associe le nombre $2 x - 8$ est représentée par le graphique \textbf{c.}

Question 3 : Un coureur qui parcourt 100 mètres en 10 secondes a une vitesse égale à : \textbf{b.} \quad 36~km/h

\vspace{0,5cm}

