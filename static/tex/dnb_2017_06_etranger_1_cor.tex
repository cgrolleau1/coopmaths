\documentclass[10pt]{article}
\usepackage[T1]{fontenc}
\usepackage[utf8]{inputenc}%ATTENTION codage UTF8
\usepackage{fourier}
\usepackage[scaled=0.875]{helvet}
\renewcommand{\ttdefault}{lmtt}
\usepackage{amsmath,amssymb,makeidx}
\usepackage[normalem]{ulem}
\usepackage{diagbox}
\usepackage{fancybox}
\usepackage{tabularx,booktabs}
\usepackage{colortbl}
\usepackage{pifont}
\usepackage{multirow}
\usepackage{dcolumn}
\usepackage{enumitem}
\usepackage{textcomp}
\usepackage{lscape}
\newcommand{\euro}{\eurologo{}}
\usepackage{graphics,graphicx}
\usepackage{pstricks,pst-plot,pst-tree,pstricks-add}
\usepackage[left=3.5cm, right=3.5cm, top=3cm, bottom=3cm]{geometry}
\newcommand{\R}{\mathbb{R}}
\newcommand{\N}{\mathbb{N}}
\newcommand{\D}{\mathbb{D}}
\newcommand{\Z}{\mathbb{Z}}
\newcommand{\Q}{\mathbb{Q}}
\newcommand{\C}{\mathbb{C}}
\usepackage{scratch}
\renewcommand{\theenumi}{\textbf{\arabic{enumi}}}
\renewcommand{\labelenumi}{\textbf{\theenumi.}}
\renewcommand{\theenumii}{\textbf{\alph{enumii}}}
\renewcommand{\labelenumii}{\textbf{\theenumii.}}
\newcommand{\vect}[1]{\overrightarrow{\,\mathstrut#1\,}}
\def\Oij{$\left(\text{O}~;~\vect{\imath},~\vect{\jmath}\right)$}
\def\Oijk{$\left(\text{O}~;~\vect{\imath},~\vect{\jmath},~\vect{k}\right)$}
\def\Ouv{$\left(\text{O}~;~\vect{u},~\vect{v}\right)$}
\usepackage{fancyhdr}
\usepackage[french]{babel}
\usepackage[dvips]{hyperref}
\usepackage[np]{numprint}
%Tapuscrit : Denis Vergès
%\frenchbsetup{StandardLists=true}

\begin{document}
\setlength\parindent{0mm}
% \rhead{\textbf{A. P{}. M. E. P{}.}}
% \lhead{\small Brevet des collèges}
% \lfoot{\small{Polynésie}}
% \rfoot{\small{7 septembre 2020}}
\pagestyle{fancy}
\thispagestyle{empty}
% \begin{center}
    
% {\Large \textbf{\decofourleft~Brevet des collèges Polynésie 7 septembre 2020~\decofourright}}
    
% \bigskip
    
% \textbf{Durée : 2 heures} \end{center}

% \bigskip

% \textbf{\begin{tabularx}{\linewidth}{|X|}\hline
%  L'évaluation prend en compte la clarté et la précision des raisonnements ainsi que, plus largement, la qualité de la rédaction. Elle prend en compte les essais et les démarches engagées même non abouties. Toutes les réponses doivent être justifiées, sauf mention contraire.\\ \hline
% \end{tabularx}}

% \vspace{0.5cm}\textbf{\textsc{Exercice 1} \hfill 6 points}

\medskip

%Pour chacune des affirmations suivantes, dire si elle est vraie ou fausse. 
%
%Chaque réponse doit être justifiée. 
%
%\medskip

%\parbox{0.55\linewidth}{\textbf{Affirmation 1 :} 

%Un menuisier prend les mesures suivantes dans le coin d'un mur à 1 mètre au-dessus du sol pour construire une étagère $ABC$ : 

%$AB = 65$ cm ; $AC = 72$ cm et $BC = 97$ cm 

%Il réfléchit quelques minutes et assure que l'étagère a un angle droit.}
%\hfill
%\parbox{0.45\linewidth}{\psset{unit=.9cm}
%\begin{pspicture}(-.5,-.5)(6.5,7)
%\pspolygon[fillstyle=solid,fillcolor=lightgray](0,0)(0,5)(3.7,6.5)(3.7,1.5)
%\pspolygon[fillstyle=solid,fillcolor=lightgray](3.7,6.5)(3.7,1.5)(6.5,0.5)(6.5,5.5)
%\pstGeonode[PointName=none,PointSymbol=none](0,0){b}
%\pstGeonode[PointName=none,PointSymbol=none](0,5){a}
%\pstGeonode[PointName=none,PointSymbol=none](3.7,1.5){c}
%\pstGeonode[PointName=none,PointSymbol=none](6.5,.5){d}
%\pstTranslation[PointName=none,PointSymbol=none]{b}{a}{c}[f]
%\pstTranslation[PointName=none,PointSymbol=none]{b}{a}{d}[e]
%\pstLineAB{b}{c}
%\pstLineAB{d}{c}
%\pstLineAB{b}{a}
%\pstLineAB{d}{e}
%\pstLineAB{a}{f}
%\pstLineAB{f}{c}
%\pstLineAB{f}{e}
%\pstHomO[dotscale=2,PointSymbol=x,HomCoef=.34,PosAngle=120]{c}{f}[A]
%\pstTranslation[PointName=none,PointSymbol=none]{f}{e}{A}[h]
%\pstHomO[dotscale=2,PointSymbol=x,HomCoef=.32,PosAngle=20]{A}{h}[C]
%\pstTranslation[PointName=none,PointSymbol=none]{f}{a}{A}[g]
%\pstHomO[dotscale=2,PointSymbol=x,HomCoef=.36,PosAngle=150]{A}{g}[B]
%\pstLineAB{A}{B}
%\pstLineAB{C}{B}\pstLineAB{A}{C}
%
%\end{pspicture} }
\textbf{Affirmation 1 :}

Seul le côté le plus long peut être l'hypoténuse. Or :

$97^2 = (100 - 3)^2 = 100^2 + 3^2 - 2\times 100 \times 3 = \np{10000} + 9 - 600 = \np{9409}$ ;

$65^2 + 72^2 = \np{4225} + \np{5184} = \np{9409}$.

Donc $\np{9403} = \np{4225} + \np{5180}$, soit $\text{BC}^2 = \text{BA}^2 + \text{AC}^2$ : la réciproque du théorème de Pythagore est  vraie, donc ABC est  rectangle en A.
\medskip

\textbf{Affirmation 2 :} 

%Les normes de construction imposent que la pente d'un toit représentée ici par l'angle $\widehat{CAH}$ doit avoir une mesure comprise entre 30$^\circ$ et 35$^\circ$. \\

%\parbox{0.43\linewidth}{Une coupe du toit est représentée ci-contre : 
%
%$AC = 6$ m et $AH = 5$ m. 
%
% 
%
%$H$ est le milieu de $[AB]$. 
%
% }
%\hfill
%\parbox{0.56\linewidth}{\psset{unit=1.5cm}
%\begin{pspicture}(-3,-.5)(3,1.3)
%\pstGeonode[dotscale=2,PointSymbol=x,PosAngle=-90](0,0){H}
%\pstGeonode[dotscale=2,PointSymbol=x,PosAngle=170](-2.6,0){A}
%\pstGeonode[dotscale=2,PointSymbol=x,PosAngle=10](2.6,0){B}
%\pstGeonode[dotscale=2,PointSymbol=x,PosAngle=90](0,1.3){C}
%\pstGeonode[PointName=none,PointSymbol=none](-.6,0){d}
%\pstGeonode[PointName=none,PointSymbol=none](.6,0){e}
%\pstProjection[PointName=none,PointSymbol=none]{A}{C}{d}[g]
%\pstProjection[PointName=none,PointSymbol=none]{B}{C}{e}[f]
%\pstLineAB{A}{B}
%\pstLineAB{A}{C}
%\pstLineAB{C}{B}
%\pstLineAB[linestyle=dashed]{C}{H}
%\pstLineAB{C}{d}
%\pstLineAB{d}{g}
%\pstLineAB{C}{e}
%\pstLineAB{e}{f}
%\pstRightAngle[RightAngleSize=0.15]{B}{H}{C}
%\end{pspicture} }
%Le charpentier affirme que sa construction respecte la norme.
On a par définition $\cos \widehat{\text{CAH}} = \dfrac{\text{AH}}{\text{AC}} = \dfrac{5}{6}$.

La calculatrice donne $\widehat{\text{CAH}} \approx 33,6$\degres.

\medskip

\textbf{Affirmation 3 : }

%Un peintre souhaite repeindre les volets d'une maison. Il constate qu'il utilise $\dfrac{1}{6}$ du pot pour mettre une couche de peinture sur l'intérieur et l'extérieur d'un volet. Il doit peindre ses 4 paires de volets et mettre sur chaque volet 3 couches de peinture. 
%
% Il affirme qu'il lui faut 2 pots de peinture.
Il y a 8 volets ; il faut trois couches sur chacun d'eux et pour chaque couche il utilise $\dfrac{1}{6}$ de pot ; il lui faut donc :

$8 \times 3 \times \dfrac{1}{6} = \dfrac{24}{6} = 4$ pots de peinture.

\bigskip

\end{document}