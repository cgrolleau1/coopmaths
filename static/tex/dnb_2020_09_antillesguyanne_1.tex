
\medskip

\parbox{0.49\linewidth}{La figure ci-contre est dessinée à main levée. On donne les informations suivantes :

\begin{itemize}[label=\textbullet]
\item ABC est un triangle tel que :
AC = 10,4 cm, AB = 4 cm et BC = 9,6 cm ;
\item les points A, L et C sont alignés ;
\item les points B, K et C sont alignés ;
\item la droite (KL) est parallèle à la droite (AB) ;
\item CK = 3~cm.
\end{itemize}}\hfill
\parbox{0.49\linewidth}{\psset{unit=1cm}
\begin{pspicture}(6,2.8)
%\psgrid
\pslineByHand(0.5,0.5)(5.5,1)(4.75,2.5)(0.5,0.5)%CBA
\uput[ur](4.75,2.5){A} \uput[d](5.5,1){B} \uput[l](0.5,0.5){C} \uput[ul](1.9,1.1){L} \uput[dl](2.2,0.65){K} 
\pslineByHand(2.5,0)(1.5,2)%LK

\end{pspicture}
}
\medskip

\begin{enumerate}
\item À l'aide d'instruments de géométrie, construire la figure en vraie grandeur sur la copie en laissant apparents les traits de construction.
\item Prouver que le triangle ABC est rectangle en B.
\item Calculer la longueur CL en cm.
\item À l'aide de la calculatrice, calculer une valeur approchée de la mesure de l'angle $\widehat{\text{CAB}}$, au degré près.
\end{enumerate}

\bigskip

