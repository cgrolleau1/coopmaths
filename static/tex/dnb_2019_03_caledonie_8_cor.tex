
\medskip

Déterminons l'aire de chacun des trois modèles. Dans les trois cas, le triangle est rectangle, donc on choisira comme base l'un des côté adjacent à l'angle droit, et la hauteur correspondante sera alors l'autre côté adjacent à l'angle droit.


\begin{enumerate}
\item L'aire du modèle 1 est : $\mathcal{A}_1 = \dfrac{\text{ES}\times \text{EL}}{2} = \dfrac{4\times 3,5}{2} = \np[m^2]{7}$. 
	
Le modèle 1 ne convient pas.
	
	\item Le modèle 2 est un triangle rectangle en P{}, donc, d'après le théorème de Pythagore :
	
$\text{PT}^2 = \text{OT}^2 - \text{PO}^2 = 5^2-3^2 = 25 - 9 = \np[m^2]{16}$.
	
Donc $\text{PT} = \sqrt{16} = \np[m]{4}$.
	
On en déduit $\mathcal{A}_2 = \dfrac{\text{PO}\times \text{PT}}{2}=\dfrac{3\times 4}{2}= \np[m^2]{6}$.
	
Le modèle 2 ne convient pas non plus.
	
\item Le triangle est rectangle, on pourrait utiliser ici le théorème de Pythagore, mais, pour changer, on va utiliser la trigonométrie.
	
Dans le triangle rectangle MUR, on a : $\cos \widehat{\text{MRU}} = \dfrac{\text{UR}}{\text{MR}}$, soit $\cos(45) = \dfrac{\text{UR}}{6}$.
	
Donc on en déduit $\text{UR} = 6\cos(45) = 6\times \dfrac{\sqrt{2}}{2} = 3\sqrt{2}\approx \np[m]{4,24}$.
	
Comme le triangle est isocèle en U, on a : $\text{MU} = \text{UR}$
	
On a alors : $\mathcal{A}_3 = \dfrac{\text{MU}\times\text{UR}}{2} = \dfrac{3\sqrt{2}\times3\sqrt{2}}{2} = \np[m^2]{9}$.
	
Le modèle 3 convient.
\end{enumerate}

\medskip
 
Finalement, seul le modèle 3 convient à Lisa.
