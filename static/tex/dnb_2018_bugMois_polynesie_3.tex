\documentclass[10pt]{article}
\usepackage[T1]{fontenc}
\usepackage[utf8]{inputenc}%ATTENTION codage UTF8
\usepackage{fourier}
\usepackage[scaled=0.875]{helvet}
\renewcommand{\ttdefault}{lmtt}
\usepackage{amsmath,amssymb,makeidx}
\usepackage[normalem]{ulem}
\usepackage{diagbox}
\usepackage{fancybox}
\usepackage{tabularx,booktabs}
\usepackage{colortbl}
\usepackage{pifont}
\usepackage{multirow}
\usepackage{dcolumn}
\usepackage{enumitem}
\usepackage{textcomp}
\usepackage{lscape}
\newcommand{\euro}{\eurologo{}}
\usepackage{graphics,graphicx}
\usepackage{pstricks,pst-plot,pst-tree,pstricks-add}
\usepackage[left=3.5cm, right=3.5cm, top=3cm, bottom=3cm]{geometry}
\newcommand{\R}{\mathbb{R}}
\newcommand{\N}{\mathbb{N}}
\newcommand{\D}{\mathbb{D}}
\newcommand{\Z}{\mathbb{Z}}
\newcommand{\Q}{\mathbb{Q}}
\newcommand{\C}{\mathbb{C}}
\usepackage{scratch}
\renewcommand{\theenumi}{\textbf{\arabic{enumi}}}
\renewcommand{\labelenumi}{\textbf{\theenumi.}}
\renewcommand{\theenumii}{\textbf{\alph{enumii}}}
\renewcommand{\labelenumii}{\textbf{\theenumii.}}
\newcommand{\vect}[1]{\overrightarrow{\,\mathstrut#1\,}}
\def\Oij{$\left(\text{O}~;~\vect{\imath},~\vect{\jmath}\right)$}
\def\Oijk{$\left(\text{O}~;~\vect{\imath},~\vect{\jmath},~\vect{k}\right)$}
\def\Ouv{$\left(\text{O}~;~\vect{u},~\vect{v}\right)$}
\usepackage{fancyhdr}
\usepackage[french]{babel}
\usepackage[dvips]{hyperref}
\usepackage[np]{numprint}
%Tapuscrit : Denis Vergès
%\frenchbsetup{StandardLists=true}

\begin{document}
\setlength\parindent{0mm}
% \rhead{\textbf{A. P{}. M. E. P{}.}}
% \lhead{\small Brevet des collèges}
% \lfoot{\small{Polynésie}}
% \rfoot{\small{7 septembre 2020}}
\pagestyle{fancy}
\thispagestyle{empty}
% \begin{center}
    
% {\Large \textbf{\decofourleft~Brevet des collèges Polynésie 7 septembre 2020~\decofourright}}
    
% \bigskip
    
% \textbf{Durée : 2 heures} \end{center}

% \bigskip

% \textbf{\begin{tabularx}{\linewidth}{|X|}\hline
%  L'évaluation prend en compte la clarté et la précision des raisonnements ainsi que, plus largement, la qualité de la rédaction. Elle prend en compte les essais et les démarches engagées même non abouties. Toutes les réponses doivent être justifiées, sauf mention contraire.\\ \hline
% \end{tabularx}}

% \vspace{0.5cm}\textbf{Exercice 3 \hfill 16 points}

\medskip

Une personne s'intéresse à un magazine sportif qui parait une fois par semaine. Elle
étudie plusieurs formules d'achat de ces magazines qui sont détaillées ci-après.

\begin{center}
\begin{tabularx}{\linewidth}{|X|}\hline
$\bullet~~$ Formule A - Prix du magazine à l'unité: 3,75~\euro{} ;\\
$\bullet~~$ Formule B - Abonnement pour l'année: 130~\euro{} ;\\
$\bullet~~$ Formule C - Forfait de 30~\euro{} pour l'année et 2,25~\euro{} par magazine.\\ \hline
\end{tabularx}
\end{center}

On donne ci-dessous les représentations graphiques qui correspondent à ces trois
formules.

\begin{center}
\psset{xunit=0.225cm,yunit=0.05cm}
\begin{pspicture}(-1,-5)(49,150)
\multido{\n=0+2}{25}{\psline[linewidth=0.3pt](\n,0)(\n,150)}
\multido{\n=0+20}{8}{\psline[linewidth=0.3pt](0,\n)(49,\n)}
\psaxes[linewidth=1.25pt,Dx=2,Dy=20]{->}(0,0)(0,0)(49,150)
\psaxes[linewidth=1.25pt,Dx=2,Dy=20](0,0)(0,0)(49,150)
\uput*[u](41,0){Nombre de magazines}
\uput*[r](0,145){Coût en euros}
\psline(40,150)
\psline(0,130)(49,130)
\psplot[plotpoints=3000,linewidth=1.25pt]{0}{48}{2.25 x mul 30 add}
\uput[d](30,95){(D$_1$)} \uput[u](30,130){(D$_2$)} \uput[u](30,112){(D$_3$)} 
\end{pspicture}
\end{center}

\begin{enumerate}
\item Sur votre copie, recopier le contenu du cadre ci-dessous et relier par un trait
chaque formule d'achat avec sa représentation graphique.

\begin{center}
\newcolumntype{Y}{>{\raggedleft \arraybackslash}X}
\begin{tabularx}{0.6\linewidth}{|X Y|}\hline
Formule A  \psdots[dotstyle=+,dotangle=45](0.1,0.1)&\psdots[dotstyle=+,dotangle=45](0,0.1)~(D1)
\\
Formule B \psdots[dotstyle=+,dotangle=45](0.1,0.1)&\psdots[dotstyle=+,dotangle=45](0,0.1)~(D2)\\
Formule C \psdots[dotstyle=+,dotangle=45](0.1,0.1)&\psdots[dotstyle=+,dotangle=45](0,0.1)~(D3)\\ \hline
\end{tabularx}
\end{center}

\item  En utilisant le graphique, répondre aux questions suivantes.

\emph{Les traits de construction devront apparaitre sur le graphique en ANNEXE qui est à rendre avec la copie.}
	\begin{enumerate}
		\item En choisissant la formule A, quelle somme dépense-t-on pour acheter
16 magazines dans l'année ?
		\item Avec $120$~\euro, combien peut-on acheter de magazines au maximum dans
une année avec la formule C ?
		\item Si on décide de ne pas dépasser un budget de $100$~\euro{} pour l'année, quelle est alors la formule qui permet d'acheter le plus grand nombre de
magazines ?
	\end{enumerate}
\item  Indiquer la formule la plus avantageuse selon le nombre de magazines achetés
dans l'année.
\end{enumerate}

\bigskip

\end{document}