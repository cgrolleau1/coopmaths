\textbf{Exercice 6 : La géode \hfill 6 points}

\medskip

La géode, située à la Cité des Sciences de la Villette à Paris, est une
structure sphérique.

\medskip

\begin{enumerate}
\item La salle de projection, située à l'intérieur de la géode, est une
demi-sphère de diamètre 26~m.

Calculer le volume de cette salle. Donner la réponse en m$^3$ arrondie à l'unité.
\item La surface extérieure est en partie recouverte de triangles équilatéraux de $120$~cm de coté.
	\begin{enumerate}
		\item Montrer que la hauteur d'un de ces triangles est $104$~cm (arrondie à l'unité).
		\item En déduire que l'aire d'un triangle est d'environ \np{6240}cm$^2$.
 	\end{enumerate}
\item Il a fallu \np{6433} triangles pour recouvrir la partie extérieure de la Géode.
	
Quelle est l'aire de la surface recouverte par ces triangles ? Donner la réponse en m$^2$ arrondie à l'unité.
\end{enumerate}

\medskip

\begin{tabularx}{\linewidth}{|l X|}\hline
Formulaire :& Volume d'une sphère : $S = \dfrac{4}{3} \times \pi \times r^3$ où $r$ est le rayon de la sphère.\rule[-3mm]{0mm}{9mm}\\
&Aire d'un triangle :  $A = \dfrac{b \times h}{2}$ où $b$ est l'aire d'une base et $h$ sa hauteur associée.\rule[-3mm]{0mm}{9mm}\\ \hline
\end{tabularx}

\vspace{0,5cm}

