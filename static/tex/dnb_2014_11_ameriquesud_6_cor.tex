\textbf{\textsc{Exercice 6} \hfill 6 points}

\medskip

%Lors d'une activité sportive, il est recommandé de surveiller son rythme cardiaque.
%
%Les médecins calculaient autrefois, la fréquence cardiaque maximale recommandée $f_m$ exprimée en battements par minute, en soustrayant à 220 l'âge $a$ de la personne exprimé en années. 
%
%\medskip

\begin{enumerate}
\item %Traduire cette dernière phrase par une relation mathématique.
$f_m = 220 - a$. 
\item %Des recherches récentes ont montré que cette relation devait être légèrement modifiée. 

%La nouvelle relation utilisée par les médecins est: 

%\[\text{Fréquence cardiaque maximale recommandée}\: = 208 - (0,75 \times a).\] 

	\begin{enumerate}
		\item %Calculer la fréquence cardiaque maximale à 60~ans recommandée aujourd'hui par les médecins.
$f_{60} =  208 - (0,75 \times 60) = 208 - 45 = 163$.
		\item %Déterminer l'âge pour lequel la fréquence cardiaque maximale est de 184 battements par minute.
		$f_a = 208 - (0,75 \times a) = 184$ si $208 - 184 = 0,75a$ ou $24 = 0,75a$ d'où finalement $a = 32$. 
		\item %Sarah qui a vingt ans court régulièrement.
		
%Au cours de ses entraînements, elle surveille son rythme cardiaque. 

%Elle a ainsi déterminé sa fréquence cardiaque maximale recommandée et a obtenu 193~battements par minute. Quand elle aura quarante ans, sa fréquence cardiaque maximale sera de 178~battements par minute. 

%Est-il vrai que sur cette durée de vingt ans sa fréquence cardiaque maximale aura diminué d'environ 8\,\% ?
On calcule $\dfrac{193 - 178}{193} \times 100 = \dfrac{15}{193} \times 100 \approx 7,77\,\%$ soit effectivement à peu près 8\,\% à l'unité près.
	\end{enumerate} 
\end{enumerate}

\vspace{0,5cm}

