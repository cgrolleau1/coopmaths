\textbf{\textsc{Exercice 6} \hfill 4 points}

\medskip

A l'issue de la 18\up{e} étape du tour de France cycliste 2014, les coureurs ont parcouru
\np{3260,5}~kilomètres depuis le départ. Le classement général des neuf premiers coureurs est
le suivant :

\begin{center}
\begin{tabularx}{\linewidth}{|c|*{3}{X|}}\hline
Classement	& NOM Prénom		& Pays d'origine	&Temps de course de chaque coureur\\ \hline
1.			& NIBALI Vincenzo	& Italie 			&80~h 45~min\\ \hline
2.			& PINOT Thibaut		& France 			&80~h 52~min\\ \hline
3.			& PÉRAUD Jean-Christophe& France		&80~h 53~min\\ \hline
4.			& VALVERDE Alejandro& Espagne 			&80~h 53~min\\ \hline
5.			& BARDET Romain		& France 			&80~h 55~min\\ \hline
6.			& VAN GARDEREN Tejay& Etats-Unis		&80~h 57~min\\ \hline
7.			& MOLLEMA Bauke		& Pays Bas 			&80~h 59~min\\ \hline
8.			& TEN DAM Laurens	& Pays-Bas 			&81~h 00~min\\ \hline
9.			& KONIG Leopold		& République Tchèque&81~h 00~min\\ \hline
\multicolumn{4}{r}{\emph{Source : letour.fr}}\\
\end{tabularx}
\end{center}

\medskip

\begin{enumerate}
\item Calculer la différence entre le temps de course de Leopold Konig et celui de
Vincenzo Nibali.
\item On considère la série statistique des temps de course.
	\begin{enumerate}
		\item Que représente pour la série statistique la différence calculée à la
question 1. ?
		\item Quelle est la médiane de cette série statistique ? Vous expliquerez votre
démarche.
		\item Quelle est la vitesse moyenne en km.h$^{-1}$ du premier français Thibaut Pinot ?
		
Arrondir la réponse à l'unité.
	\end{enumerate}
\end{enumerate}

\vspace{0,5cm}

