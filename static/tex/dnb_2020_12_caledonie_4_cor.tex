\textbf{EXERCICE 4 : La régate \hfill 16 points}

\medskip

Dans la figure suivante, on donne les distances en mètres :

AB $= 400$, AC $= 300$, BC $= 500$ et CD $= 700$.

\medskip
\begin{center}
\begin{tabularx}{\linewidth}{lX} 
\psset{unit=1cm}
\begin{pspicture}(7,5.2)
%\psgrid
\pspolygon(1.5,3.2)(0.5,0.5)(6.2,5.14)(5,1.9)%ABDE
\uput[u](1.5,3.2){A} \uput[l](0.5,0.5){B} \uput[u](3,2.6){C} \uput[r](6.2,5.14){D} \uput[r](5,1.9){E} 
\end{pspicture}&\vspace{-2.5cm}\begin{tabular}{|l|} \hline

Les droites (AE) et (BD) se coupent en C \\
~\\
Les droites (AB) et (DE)sont parallèles\\ \hline
\end{tabular}\\
\end{tabularx}

\end{center}

\begin{enumerate}
\item %Calculer la longueur DE.
Les droites (AB) et (DE) étant parallèles, on peut écrire d'après le théorème de Thalès : 

$\dfrac{\text{DE}}{\text{AB}} = \dfrac{\text{CD}}{\text{BC}}$ soit $\dfrac{\text{DE}}{400} = \dfrac{700}{500}$,d'où en multipliant par 400 : $\text{DE} = 400 \times \dfrac{700}{500} = 400 \times \dfrac{7}{5} = 560$~(m).
\item %Montrer que le triangle ABC est rectangle.
On a BC$^2 = 500^2 = \np{25000}$ et $\text{AB}^2 + \text{AC}^2 = 400^2 + 300^2 = \np{16000} + \np{9000} = \np{25000}$.

On a donc  $\text{AB}^2 + \text{AC}^2 = \text{BC}^2$ : d'après la réciproque du théorème de Pythagore, le triangle ABC est rectangle en A.
\item %Calculer la mesure de l'angle $\widehat{\text{ABC}}$. Arrondir au degré.
Par définition du cosinus d'un angle aigu, dans le triangle ABC rectangle en A :

$\cos \widehat{\text{ABC}} = \dfrac{\text{AB}}{\text{BC}} = \dfrac{400}{500} = \dfrac{4}{5} = 0,8$.

La calculatrice donne, en mode degré : $\widehat{\text{ABC}} \approx 36,8$,soit 37\degres au degré près.
\end{enumerate}

%Lors d'une course les concurrents doivent effectuer plusieurs tours du parcours représenté ci-dessus. Ils partent du point A, puis passent par les points B, C, D et E dans cet ordre puis de nouveau par le point C pour ensuite revenir au point A.

%\smallskip
%
%Maltéo, le vainqueur, a mis 1~h 48~min pour effectuer les $5$ tours du parcours. La distance parcourue pour faire un tour est \np{2880}~m.

\begin{enumerate}[resume]
\item %Calculer la distance totale parcourue pour effectuer les $5$~tours du parcours.
\emph{Remarque : non demandé} : 

Pour calculer la longueur d'un parcours, il reste à calculer CE.

Or les droites (AB) et (DE) étant parallèles, la droite (AC) perpendiculaire à (AB) est aussi perpendiculaire à (DE), donc le triangle CDE est rectangle en E.

D'après le théorème de Pythagore :

$\text{CE}^2 + \text{ED}^2 = \text{CD}^2$ ou $\text{CE}^2 + 560^2 = 700^2$,soit $\text{CE}^2 = 700^2 - 560^2 = (700+560)\times (700-560) = \np{1260} \times 140 = \np{176400} $.

D'où CE $= \sqrt{\np{176400}} = 420$~(m).

Longueur d'un parcours : AB + BC + CD + DE + EC + CA $= 400 + 500 + 700 + 560 + 420 + 300 = \np{2880}$.

Les 5 tours représentent donc une longueur de $5 \times \np{2880} = \np{14400}$~(m) ou 14,4~(km).
\item %Calculer la vitesse moyenne de Maltéo. Arrondir à l'unité.
1~h 48~min = 60 + 48 = 108~min.
La vitesse moyenne est égale au quotient de la distance parcourue par le le temps mis pour faire les 5 tours :

$v = \dfrac{\np{14400}}{108}= \dfrac{\np{1600}}{12}= \dfrac{400}{3} \approx 133,33$~(m/min) soit $\approx 60 \times 133,33 = \np{7999,8}$~(m/h), soit enfin à peu près 8~(km/h).
\end{enumerate}

\vspace{0,5cm}

