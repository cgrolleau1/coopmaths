\documentclass[10pt]{article}
\usepackage[T1]{fontenc}
\usepackage[utf8]{inputenc}%ATTENTION codage UTF8
\usepackage{fourier}
\usepackage[scaled=0.875]{helvet}
\renewcommand{\ttdefault}{lmtt}
\usepackage{amsmath,amssymb,makeidx}
\usepackage[normalem]{ulem}
\usepackage{diagbox}
\usepackage{fancybox}
\usepackage{tabularx,booktabs}
\usepackage{colortbl}
\usepackage{pifont}
\usepackage{multirow}
\usepackage{dcolumn}
\usepackage{enumitem}
\usepackage{textcomp}
\usepackage{lscape}
\newcommand{\euro}{\eurologo{}}
\usepackage{graphics,graphicx}
\usepackage{pstricks,pst-plot,pst-tree,pstricks-add}
\usepackage[left=3.5cm, right=3.5cm, top=3cm, bottom=3cm]{geometry}
\newcommand{\R}{\mathbb{R}}
\newcommand{\N}{\mathbb{N}}
\newcommand{\D}{\mathbb{D}}
\newcommand{\Z}{\mathbb{Z}}
\newcommand{\Q}{\mathbb{Q}}
\newcommand{\C}{\mathbb{C}}
\usepackage{scratch}
\renewcommand{\theenumi}{\textbf{\arabic{enumi}}}
\renewcommand{\labelenumi}{\textbf{\theenumi.}}
\renewcommand{\theenumii}{\textbf{\alph{enumii}}}
\renewcommand{\labelenumii}{\textbf{\theenumii.}}
\newcommand{\vect}[1]{\overrightarrow{\,\mathstrut#1\,}}
\def\Oij{$\left(\text{O}~;~\vect{\imath},~\vect{\jmath}\right)$}
\def\Oijk{$\left(\text{O}~;~\vect{\imath},~\vect{\jmath},~\vect{k}\right)$}
\def\Ouv{$\left(\text{O}~;~\vect{u},~\vect{v}\right)$}
\usepackage{fancyhdr}
\usepackage[french]{babel}
\usepackage[dvips]{hyperref}
\usepackage[np]{numprint}
%Tapuscrit : Denis Vergès
%\frenchbsetup{StandardLists=true}

\begin{document}
\setlength\parindent{0mm}
% \rhead{\textbf{A. P{}. M. E. P{}.}}
% \lhead{\small Brevet des collèges}
% \lfoot{\small{Polynésie}}
% \rfoot{\small{7 septembre 2020}}
\pagestyle{fancy}
\thispagestyle{empty}
% \begin{center}
    
% {\Large \textbf{\decofourleft~Brevet des collèges Polynésie 7 septembre 2020~\decofourright}}
    
% \bigskip
    
% \textbf{Durée : 2 heures} \end{center}

% \bigskip

% \textbf{\begin{tabularx}{\linewidth}{|X|}\hline
%  L'évaluation prend en compte la clarté et la précision des raisonnements ainsi que, plus largement, la qualité de la rédaction. Elle prend en compte les essais et les démarches engagées même non abouties. Toutes les réponses doivent être justifiées, sauf mention contraire.\\ \hline
% \end{tabularx}}

% \vspace{0.5cm}\textbf{\textsc{Exercice 1 \hfill 15 points}}

\medskip

\begin{enumerate}
\item $28= 4\times 7 = 2^2 \times 7$ : Réponse C 

A et B contiennent des facteurs non premiers.

\item Le nouveau prix est égal à :
$58 \times \left(1 - \dfrac{20}{100}\right)
= 58 \times \dfrac{80}{100} = 58 \times 0,8 = 46,4$, soit 46,40~\euro : 
Réponse B
\item Dans le triangle ABC rectangle en  A, on a 
$\tan 15 = \dfrac{\text{AC}}{\text{AB}} = \dfrac{\text{AC}}{25}$, d'où en multipliant chaque membre par 25 :

$\text{AC} = 25 \times \tan 15 \approx 6,698$ : réponse la plus proche 
\item Rangés dans l'ordre croissant les termes de la série sont : 2 ; 3 ; 5 ; 6 ; 8 ; 12.

Il y a 6 termes, donc la médiane est tout nombre compris entre le  3\up{e} et le 4\up{e} terme, donc en particulier la moyenne des deux nombres soit 5,5 : réponse A.
\item Les dimensions du carré B sont deux fois plus petites que celles du carré A : le rapport d'homothétie est donc égal à $+ 0,5$ ou $- 0,5$.

Avec A comme centre d'homothétie le rapport est égal à $- 0,5$ ; réponse a.

Avec B comme centre d'homothétie le rapport est égal à $0,5$ : réponse b. 

\psset{unit=0.8cm}
\begin{center}
\begin{pspicture}(-2,-2)(3,3.35)
%\psgrid
\psframe(1,1)(3,3)\psframe(0,0)(1,1)
\rput(2,2){\small carré A}\rput(0.5,0.5){\scriptsize carré B}
\psline[linestyle=dotted,dotsep=1.2pt](3,1)(-1,-1)(1,3)(3,3)(-1,-1)
\uput[ul](1,1){A} \uput[dl](-1,-1){B}
\end{pspicture} 
\end{center}
\end{enumerate}
\vspace{0,25cm}

\end{document}