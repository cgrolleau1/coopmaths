\textbf{Exercice 1 : Questionnaire à choix multiples \hfill 5 points}

\medskip

%Cet exercice est un questionnaire à choix multiple (QCM). Pour chaque question, une seule des trois réponses proposées est exacte. Sur la copie, indiquer le numéro de la question et la réponse choisie.
%
%On ne demande pas de justifier. Aucun point ne sera enlevé en cas de mauvaise réponse.
%
%\medskip
%
%\begin{center}
%\begin{tabularx}{\linewidth}{|m{6cm}|*{3}{>{\centering \arraybackslash}X|}}\hline
%\multirow{2}{6cm}{Questions posées}&\multicolumn{3}{|c|}{Réponses proposées}\\ \cline{2-4}
%&A& B& C\\ \hline
%\textbf{1.} Marc a 10 ans et il pèse 30 kg. Quel sera son poids à 20 ans?&60 kg &40 kg& On ne peut  pas savoir\\ \hline
%\textbf{2.} Quelle est la largeur d'un rectangle de longueur 8 cm et de périmètre 24 cm ?& 3 cm &4 cm &16 cm\\ \hline
%\textbf{3.} Si je réponds à cette question au hasard, quelle est la probabilité que ma réponse soit juste ?&$\dfrac{1}{3}$&$\dfrac{1}{2}$&On ne peut pas savoir\\ \hline
%\textbf{4.} Quel est le volume, arrondi à l'unité, d'une boule de rayon 3 cm ?&113 cm$^3$& 19 m$^3$& 28 cm$^2$\\ \hline 
%\textbf{5.} Quelles sont les solutions de l'équation
%$(x + 1)(5x-10) = 0$ ?&$- 1$ et $- 2$ &1 et 2 &$- 1$ et $2$\\ \hline
%\end{tabularx}
%\end{center}
\begin{enumerate}
\item Le poids n'a pas de lien direct avec l'âge : réponse C.
\item Le demi-périmètre est égal à 12~cm donc la largeur mesure 4~cm : réponse B.
\item Il y a trois réponses proposées, donc la probabilité est de $\dfrac{1}{3}$ : réponse A.
\item Le volume est égal à $\dfrac{4}{3}\pi \times 3^3 = 4\pi \times 3^2 = 36\pi\approx 113,09$ soit 113~cm$^3$ à l'unité près.
\item $x - 1 = 0$ ou $5x - 10 = 0$ soit $x = - 1$ ou $x = 2$ : réponse C.
\end{enumerate}

\vspace{0,5cm}

