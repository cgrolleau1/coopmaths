\textbf{\textsc{Exercice 1} \hfill 4 points}

\medskip

Une société commercialise des composants électroniques qu'elle fabrique dans deux usines. Lors d'un contrôle de qualité, 500 composants sont prélevés dans chaque usine et sont examinés pour déterminer s'ils sont \og bons\fg{} ou \og défectueux \fg. 

Résultats obtenus pour l'ensemble des \np{1000} composants prélevés : 

\begin{center}
\begin{tabularx}{0.65\linewidth}{|*{3}{>{\centering \arraybackslash}X|}}\cline{2-3}
\multicolumn{1}{c|}{~}&  Usine A&   Usine B\\ \hline     
Bons   &473  & 462 \\ \hline   
Défectueux   &27   &38 \\ \hline
\end{tabularx}
\end{center}
   
\begin{enumerate}
\item Si on prélève un composant au hasard parmi ceux provenant de l'usine A, quelle est la probabilité qu'il soit défectueux ? 
\item Si on prélève un composant au hasard parmi ceux qui sont défectueux, quelle est la probabilité qu'il provienne de l'usine A? 
\item Le contrôle est jugé satisfaisant si le pourcentage de composants défectueux est inférieur à 7\,\% dans chaque usine. Ce contrôle est-il satisfaisant ? 
\end{enumerate}

\bigskip

