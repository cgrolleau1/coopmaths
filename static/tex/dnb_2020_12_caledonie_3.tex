
\medskip

On donne les deux programmes de calcul suivants :

\begin{center}
\begin{tabularx}{\linewidth}{|X|X|}\hline 
\multicolumn{1}{|c|}{\textbf{Programme A}}&\multicolumn{1}{|c|}{\textbf{Programme B}}\\
~&~\\
$\bullet~~$Choisir un nombre								&$\bullet~~$Choisir un nombre\\
$\bullet~~$Soustraire $5$ à ce nombre						&$\bullet~~$Mettre ce nombre au carré\\
$\bullet~~$Multiplier le résultat par le nombre de départ	&$\bullet~~$Soustraire 4 au résultat\\ \hline
\end{tabularx}
\end{center}

\medskip

\begin{enumerate}
\item Alice choisit le nombre 4 et applique le programme A. 

Montrer qu'elle obtiendra $- 4$.
\item Lucie choisit le nombre $- 3$ et applique le programme B. 

Quel résultat va-t-elle obtenir ?
\end{enumerate}

Tom souhaite trouver un nombre pour lequel des deux programmes de calculs donneront le même résultat.

Il choisit $x$ comme nombre de départ pour les deux programmes.
\begin{enumerate}[resume]
\item Montrer que le résultat du programme A peut s'écrira $x^2 - 5x$.
\item Exprimer en fonction de $x$ le résultat obtenu avec le programme B.
\item Quel est le nombre que Tom cherche ?
\end{enumerate}

\textbf{Toute trace de recherche même non aboutie sera prise, en compte dans la notation.}

\vspace{0,5cm}

