
\medskip

%Marc et Jim, deux amateurs de course à pied, s'entrainent sur une piste d'athlétisme dont la longueur du tour mesure $400$~m. 

\smallskip

%Marc fait un temps moyen de $2$ minutes par tour. 
%
%Marc commence son entrainement par un échauffement d'une longueur d'un kilomètre. 
%
%\medskip

\begin{enumerate}
\item %Combien de temps durera l'échauffement de Marc?
En supposant que Marc coure  à la vitesse de 2 minutes pour faire 400 m, il mettra 1 minute pour faire 200 m, donc 5~minutes pour faire $5 \times 200 = \np{1000}$~m
\item %Quelle est la vitesse moyenne de course de Marc en km/h ? 
1~km en 5~min représente une vitesse de $12 \times 1 = 12$~(km/h) en $12 \times 5 = 60$~min = 1 h. 
\end{enumerate}

%À la fin de l'échauffement, Marc et Jim décident de commencer leur course au même point de départ A et vont effectuer un certain nombre de tours. 

%Jim a un temps moyen de 1 minute et 40 secondes par tour. 

%Le schéma ci-dessous représente la piste d'athlétisme de Marc et Jim constituée de deux segments [AB] et [CD] et de deux demi-cercles de diamètre [AD] et [BC). 
%
%(\emph{Le schéma n'est pas à l'échelle et les longueurs indiquées sont arrondies à l'unité.}) 
%
%\medskip
%
%\begin{center}
%\psset{unit=0.75cm}
%\begin{pspicture}(18,6)
%\psline(3,0.5)(9.7,0.5)\psline(3,5.7)(9.7,5.7)
%\psline[linestyle=dotted](3,0.5)(3,5.7)
%\psline[linestyle=dotted](9.7,0.5)(9.7,5.7)
%\psarc(3,3.1){2.6}{90}{270}
%\psarc(9.7,3.1){2.6}{-90}{90}
%\uput[u](3,5.7){A} \uput[u](9.7,5.7){B} \uput[d](9.7,0.5){C} \uput[d](3,0.5){D} 
%\psframe(3,5.7)(3.2,5.5)\psframe(9.7,5.7)(9.5,5.5)\psframe(9.7,0.5)(9.5,0.7)
%\psframe(3,0.5)(3.2,0.7)
%\psline[linestyle=dashed]{<->}(3,5.9)(9.7,5.9)\uput[u](6.35,5.9){90 m}
%\psline[linestyle=dashed]{<->}(3.4,5.7)(3.4,0.5)\uput[r](3.4,3.1){70 m}
%\rput(15.5,3){ABCD est un rectangle}
%\rput(15.5,2.2){AB = 90 m et AD = 70 m}
%\end{pspicture}
%\end{center}

\begin{enumerate}[resume]
\item %Calculer le temps qu'il faudra pour qu'ils se retrouvent ensemble, au même moment, et pour la première fois au point A. 
Un tour de piste a pour longueurs la longueur des deux lignes droites et la longueur d'un cercle de diamètre [AD].

Longueur d'un tour  : $2 \times 90 + 70 \times \pi = 180 + 70\pi \approx 399,911$ ce qui correspond bien à l'unité près à 400~m.

$\bullet~~$Marc passe donc au point A toutes les deux minutes soit toutes les 120 secondes ;

$\bullet~~$Jim passe au point toute les 1 µin 40, soit toutes les 100 secondes.

Ils repasseront la première fois ensemble au point A au bout d'un temps égal au plus petit multiple commun à 100 et à 120.

$100 = 10 \times 10 = 2^2 \times 5^2$ et 

$120 = 12 \times 10 = 2^3 \times 3 \times 5$.

Le p.p.c.m. à 100 et 120 est $2^3 \times 3 \times 5^2  = 24 \times 25 = 600$~(s)

%Puis déterminer combien de tours de piste cela représentera pour chacun d'entre eux. 
$\bullet~~$Marc aura donc fait $\dfrac{600}{100} = 6$~tours et 

$\bullet~~$Jim aura fait $\dfrac{600}{120} = \dfrac{60}{12} = 5$~tours.
\smallskip

%\emph{Toute trace de recherche, même non aboutie, devra apparaitre sur la copie. Elle sera prise en compte dans l'évaluation.} 
\end{enumerate}

\vspace{0.5cm}

