\textbf{\textsc{Exercice 4} \hfill 7 points}

\medskip 
 
Le nombre d'abonnés à une revue dépend du prix de la revue.
 
Pour un prix $x$ compris entre 0 et 20~\euro, le nombre d'abonnés est donné par la fonction $A$ telle que : $A(x) = - 50 x + \np{1250}$.
 
La recette, c'est-à-dire le montant perçu par l'éditeur de cette revue, est donnée par la fonction $R$ telle que : $R(x) = - 50 x^2 + \np{1250} x$. 


\begin{center}\textbf{Représentation graphique de la fonction} \boldmath $A$ \unboldmath 

\medskip

\psset{xunit=0.4cm,yunit=0.004cm}
\begin{pspicture}(-1,-100)(25,1400)
\multido{\n=0+1}{25}{\psline[linestyle=dashed,linecolor=orange,linewidth=0.2pt](\n,0)(\n,1400)}
\multido{\n=0+100}{15}{\psline[linestyle=dashed,linecolor=orange,linewidth=0.2pt](0,\n)(24,\n)}
\psaxes[linewidth=1.5pt,Dx=2,Dy=200]{->}(0,0)(0,0)(24,1400)
\psplot[plotpoints=100,linewidth=1.25pt]{0}{24}{1250 50 x mul sub}
\uput[u](19,0){prix de la revue en euros}
\uput[r](0,1350){nombre d'abonnés}
\end{pspicture}

\vspace{1cm}

\textbf{Représentation graphique de la fonction } \boldmath $R$ \unboldmath

\vspace{0,5cm}

\psset{xunit=0.5cm,yunit=0.001cm}
\begin{pspicture}(-1,-400)(20,8400)

\multido{\n=0+1}{21}{\psline[linestyle=dashed,linecolor=orange,linewidth=0.2pt](\n,0)(\n,8400)}
\multido{\n=0+400}{21}{\psline[linestyle=dashed,linecolor=orange,linewidth=0.2pt](0,\n)(20,\n)}
\psaxes[linewidth=1.5pt,Dx=2,Dy=20000]{->}(0,0)(0,0)(20,8400)
\multido{\n=0+2000}{5}{\uput[l](0,\n){\np{\n}}}
\psplot[plotpoints=100,linewidth=1.25pt,linecolor=blue]{0}{20}{25  x  sub 50 mul x mul}
\uput[u](16.5,0){prix de la revue en euros}
\uput[r](0,8350){recette en euros}
\end{pspicture} 
\end{center}

\medskip
 
\begin{enumerate}
\item Le nombre d'abonnés est-il proportionnel au prix de la revue ? Justifier. 
\item Vérifier, par le calcul, que $A(10) = 750$ et interpréter concrètement ce résultat. 
\item La fonction $R$ est-elle affine ? Justifier. 
\item Déterminer graphiquement pour quel prix la recette de l'éditeur est maximale. 
\item Déterminer graphiquement les antécédents de \np{6800} par $R$. 
\item Lorsque la revue coûte $5$~euros, déterminer le nombre d'abonnés et la recette. 
\end{enumerate}

\bigskip

