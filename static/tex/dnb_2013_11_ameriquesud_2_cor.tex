\textbf{Exercice 2 \hfill 6 points}

\medskip 

%On considère les deux programmes de calcul suivants : 
%
%\medskip
%
%\begin{tabularx}{\linewidth}{|m{6cm}|X|}\hline
%\textbf{Programme A}&   \textbf{Programme B}\\   
%$\bullet~~$ Choisir un nombre de départ &$\bullet~~$Choisir un nombre de départ \\  
%$\bullet~~$  Soustraire 1 au nombre choisi&$\bullet~~$Calculer le carré du nombre choisi  \\ 
%$\bullet~~$  Calculer le carré de la différence obtenue&$\bullet~~$Ajouter 1 au résultat   \\
%$\bullet~~$  Ajouter le double du nombre de départ au résultat&$\bullet~~$Écrire le résultat obtenu\\   
%$\bullet~~$  Écrire le résultat obtenu & \\ \hline
%\end{tabularx}
%
%\medskip

\begin{enumerate}
\item %Montrer que, lorsque le nombre de départ est 3, le résultat obtenu avec le programme A est 10.
$3 \to 3 - 1 = 2 \to 2^2 = 4 \to 4 + 2\times 3 = 4 + 6 = 10$. 
\item %Lorsque le nombre de départ est 3, quel résultat obtient-on avec le programme B ? 
$3 \to 3^2 = 9 \to 9 + 1 = 10$.
\item %Lorsque le nombre de départ est $- 2$, quel résultat obtient-on avec le programme A ?
$- 2 \to -2 - 1 = - 3 \to (- 3)^2 = 9 \to 9 + 2 \times (- 2) = 9 - 4 = 5$. 
\item %Quel(s) nombre(s) faut-il choisir au départ pour que le résultat obtenu avec le programme B soit 5 ? 
Avec $x$ au départ le programme B donne $x^2 + 1$.

On a $x^2 + 1 = 5$ ou $x^2 = 4$ ou $x^2 - 4 = 0$ ou $(x + 2)(x - 2) = 0$, d'où $x = 2$ ou $x = - 2$ (ce dernier vu à la question précédente.)
\item %Henri prétend que les deux programmes de calcul fournissent toujours des résultats identiques. 
%A-t-il raison ? Justifier la réponse.
Avec $x$ au départ, le programme donne $(x - 1)^2+ 2x = x^2 + 1 - 2x + 2x = x^2 + 1$ : c'est effectivement ce que donne le programme B. 
\end{enumerate}

\bigskip

