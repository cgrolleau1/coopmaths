
\medskip

%\emph{Pour illustrer l'exercice, la figure ci-dessous a été faite à  main levée.}
%
%\begin{center}
%\psset{unit=1cm}
%\begin{pspicture}(8,5)
%%\psgrid
%\pslineByHand(0.5,2.5)(7.5,4)
%\pslineByHand(2,0.5)(6,4.8)
%\pslineByHand(2,0.5)(0.5,2.5)
%\pslineByHand(6,4.8)(7.5,4)
%\pslineByHand(3,3)(4,2.6)
%\uput[ul](4.75,3.4){A}\uput[r](7.5,4){B}\uput[u](6,4.8){C}
%\uput[l](0.5,2.5){D} \uput[d](2,0.5){E}\uput[ul](3,3){F}\uput[dr](3.8,2.6){G}
%\rput{-50}(1.2,1.2){8,1 cm}\rput{-27}(3.3,2.6){3 cm}
%\rput{46}(3.2,1.4){6,8 cm}\rput{45}(4.4,2.78){4 cm}
%\rput{15}(3.6,3.4){5 cm}\rput{45}(5.2,4.3){5 cm}
%\rput{15}(6,3.4){6,25 cm}
%\end{pspicture}
%\end{center}
%
%Les points D, F, A et B sont alignés, ainsi que les points E, G, A et C.
%
%De plus, les droites (DE) et (FG) sont parallèles.
%
%\medskip

\begin{enumerate}
\item %Montrer que le triangle AFG est un triangle rectangle.
On a $\text{AF}^2 = 5^2 = 25$ ;

$\text{AG}^2 + \text{GF}^2 = 4^2 + 3^2 = 16 + 9 = 25$, soit :

$\text{AF}^2 =\text{AG}^2 + \text{GF}^2$ : d'après la réciproque du théorème de Pythagore le triangle AGF est rectangle en G.
\item %Calculer la longueur du segment [AD]. En déduire la longueur du segment [FD].
Les droites (FG) et (AE) sont parallèles ; comme la droite (AG) est perpendiculaire à  la droite (FG), elle est aussi perpendiculaire à  la droite (ED) : le triangle AED est donc rectangle en E.

Le théorème de Pythagore appliqué à  ce triangle s'écrit :

$\text{AE}^2 + \text{ED}^2 = \text{AD}^2$ soit $(6,8 + 4)^2 + 8,1^2 = \text{AD}^2$ ; donc 

$\text{AD}^2 = 116,64 + 65,61 = 182,25 = 13,5^2$ ;  $\text{AD} = 13,5$~(cm).

On a donc $\text{FD} = \text{AD} - \text{AF} = 13,5 - 5 = 8,5$~(cm).
\item %Les droites (FG) et (BC) sont-elles parallèles ? Justifier.
On a $\dfrac{\text{AG}}{\text{AC}} =  \dfrac{4}{5} = 0,8$ ; $\dfrac{\text{AF}}{\text{AB}} =  \dfrac{5}{6,25} = 0,8$.

Comme $\dfrac{\text{AG}}{\text{AC}} = \dfrac{\text{AF}}{\text{AB}}$, que les points G, A, C d'une part, F, A et B d'autre part sont alignés d'après la réciproque de la propriété de Thalès on en déduit que les droites (FG) et (BC) sont parrallèles.
\end{enumerate}

\vspace{0,5cm}

