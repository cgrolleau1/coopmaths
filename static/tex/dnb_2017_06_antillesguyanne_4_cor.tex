
\medskip

\begin{enumerate}
\item Mai 2015 correspond à la période du 01/04/15 au 30/06/15. Pour une
puissance de $28$~kW, le prix d'achat du kWh en centimes d'euros est $13,95$, soit
\np{0,1395} ~\euro. 

Je calcule ainsi le prix de \np{31420}~kWh :

$\np{31420} \times \np{0,1395} = \np{4383,09}$.

Le prix d'achat de 31 420 kWh est d'environ \np{4383}~\euro.
\item  ABC est un triangle rectangle en B tel que
BC = 4,5 m et AC $= 7  - 4,8  = 2,2$~m.

On a donc : $\tan \widehat{\text{ABC}} = \dfrac{\text{AC}}{\text{BC}}$, c'est-à-dire

$\tan \widehat{\text{ABC}} = \dfrac{2,2}{4,5}$. La calculatrice donne $\widehat{\text{ABC}} \approx  26$\degres.

Le pan sud du toit forme un angle d'environ 26\degres avec l'horizontale.
\item 
	\begin{enumerate}
		\item ABC est un triangle rectangle en B, donc d'après le théorème de
Pythagore, on a :
		
$\text{AB}^2 = \text{AC}^2 + \text{BC}^2$
		
$\text{AB}^2 = 2,2^2 + 4,5^2,$
		
$\text{AB}^2 = 4,84 + 20,25$
		
$\text{AB}^2 = 25,09$
		
Donc $\text{AB} = \sqrt{25,09} \approx  5$~m.
		\item 1 carré de 1 m de côté a une aire de 1 m$^2$. 20 panneaux occupent alors une
surface de 20 m$^2$.
		
$7,5  \times 5  = 37,5$~m$^2$ Le pan sud du toit a une aire d'environ $37,5$ m$^2$.
		
$\dfrac{20}{37,5} \times 100 \approx 53$.

Environ $53\,\%$ du pan sud du toit sera donc recouvert par les panneaux solaires.
		\item Si on enlève l'espace utilisé pour les bordures, celui disponible pour disposer
les 20 panneaux est un rectangle de dimensions:
		
longueur $= 7,5  - 2 \times 0,3  = 7,5  - 0,6 = 6,9$~(m) ;
		
largeur $= 5 - 2 \times 0,3= 5 - 0,6 = 4,4$~m.
		
Le propriétaire peut donc installer jusqu'à $6 \times 4 = 24$ panneaux de $1$~m de côté. Il
pourra donc aisément installer ses $20$ panneaux solaires.
 	\end{enumerate}
\end{enumerate}

\bigskip

