
\medskip

Le baklava est une pâtisserie traditionnelle dans plusieurs pays comme la Bulgarie ou
le Maroc. Il s'agit d'un dessert long à préparer, à base de pâte feuilletée, de miel, de
noix ou de pistaches ou de noisettes, selon les régions.
Dans un sachet non transparent, on a sept baklavas indiscernables au toucher portant
les lettres du mot BAKLAVA.
\begin{center}
\begin{pspicture}(10,1)
\multido{\n=1.0+1.2}{7}{\pscircle(\n,0.5){0.4}}
\rput(1,0.5){B}\rput(2.2,0.5){A}\rput(3.4,0.5){K}
\rput(4.6,0.5){L}\rput(5.8,0.5){A}\rput(7,0.5){V}\rput(8.2,0.5){A}
\end{pspicture}
\end{center}

On tire au hasard un gâteau dans ce sachet et on regarde la lettre inscrite sur le
gâteau.

\medskip

\begin{enumerate}
\item Quelles sont les issues de cette expérience ?
\item Déterminer les probabilités suivantes:
	\begin{enumerate}
		\item La lettre tirée est un L.
		\item La lettre tirée n'est pas un A.
	\end{enumerate}
\item  Enzo achète un sachet contenant $10$ baklavas tous indiscernables au toucher.
	
Ce sachet contient $2$ baklavas à base de pistaches, $4$ baklavas à base de noisettes
et les autres baklavas sont à base de noix.
	
Enzo pioche au hasard un gâteau et le mange ; c'est un gâteau à base de noix. 

Il souhaite en manger un autre.

Son amie Laura affirme que, s'il veut maintenant prendre un nouveau gâteau, il
aura plus de chances de piocher un gâteau à base de noix
A-t-elle raison ? Justifier la réponse.
\end{enumerate}

\bigskip

