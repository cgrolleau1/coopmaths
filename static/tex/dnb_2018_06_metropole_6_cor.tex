
\medskip

\begin{enumerate}
\item~
	\begin{enumerate}
		\item~
\psset{unit=0.5cm}
\begin{center}
\begin{pspicture}(5,5)
\psframe(5,5)
\psdots(0,0)(5,0)(0,5)(5,5)(0;60)
\psarc(0,0){5}{50}{70}
\psarc(5,0){5}{110}{130}
\psline(0,0)(2.5,4.33)(5,0)
\end{pspicture}
\end{center}
\begin{tabularx}{\linewidth}{X X}
\begin{scratch}
\blockinit{quand \greenflag est cliqué}
\blockmove{aller à x: \ovalnum 0 y: \ovalnum 0}
\blockpen{stylo en position d’écriture}
\blockmove{s'orienter à \ovalnum{90\selectarrownum} degrés}
\blockvariable{mettre \ovalvariable{Longueur \selectarrownum} à \ovalnum{300}}
\blockevent{Carré}
\blockevent{Triangle}
\blockmove{avancer de \ovalvariable{\ovalnum{Longueur} / \ovalnum{6}}}
{
\blockvariable{mettre \ovalvariable{Longueur \selectarrownum} à \ovaloperator{~~~~~~~~~~~~~}}
\blockevent{Carré}
\blockevent{Triangle}
}

\end{scratch}
&
\begin{scratch}
\initmoreblocks{définir \namemoreblocks{Carré}}
\blockrepeat{répéter \ovalnum{4} fois}
{\blockmove{avancer de \ovalnum{Longueur}}
\blockmove{tourner \turnleft{} de \ovalnum{90} degrés}
}
\end{scratch}

\begin{scratch}
\initmoreblocks{définir \namemoreblocks{Triangle}}
\blockrepeat{répéter \ovalnum{3} fois}
{\blockmove{avancer de \ovalnum{Longueur}}
\blockmove{tourner \turnleft{} de \ovalnum{120} degrés}
}
\end{scratch}\\
\end{tabularx}

		\item Après l'exécution de la ligne 8, le stylo sera à $x = 50$ et $y = 0$.
	\end{enumerate}
\item Pour tracer la figure intérieure on doit se décaler de $50$ de chaque côté. Donc le côté intérieur sera de $300 - 2 \times 50 = 200$.
\item 
	\begin{enumerate}
		\item Il s'agit d'une homothétie de rapport :

\[\dfrac{200}{300} = \dfrac{2}{3}.\]

		\item  Par définition, si $k$ est le rapport de réduction des longueurs, $k^2$ sera le rapport de réduction pour les aires. Donc :

\[k^2 = \left(\dfrac{2}{3}\right)^2 = \dfrac{4}{9}.\]

	\end{enumerate}
\end{enumerate}

\bigskip

