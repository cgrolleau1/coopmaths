\textbf{\textsc{Exercice 2} \hfill 3 points}

\medskip
 
À Pise vers 1200 après J. C. (problème attribué à Léonard de Pise, dit Fibonacci, 
mathématicien italien du moyen âge).

%\medskip

\parbox{0.6\linewidth}{Une lance, longue de 20 pieds, est posée verticalement 
le long d'une tour considérée comme perpendiculaire au sol. Si on éloigne l'extrémité de la lance qui repose sur le sol de 12 pieds de la tour, de combien descend l'autre 
extrémité de la lance le long du mur ? 

* Un pied est une unité de mesure anglo-saxonne valant environ 30 cm.} \hfill
\parbox{0.36\linewidth}{\psset{unit=1cm}
\begin{pspicture}(4,5)
\psframe(4,0.3)
\psframe(0.5,0.3)(2.5,3.8)
\psframe*(2.52,0.3)(2.62,3)
\pspolygon*(3.8,0.3)(3.9,0.35)(2.7,2.55)(2.6,2.5)
\psarc{->}(2.65,2.525){1.5cm}{-90}{-62}
\psline(2.52,2.56)(3,2.56)
\psline(2.52,3)(3,3)
\psline{<->}(3,2.56)(3,3)\uput[r](3,2.78){$h$}
\rput(1.5,3){Tour}
\psline{<->}(2.62,0.15)(3.9,0.15)
\rput(3.2,0.4){12 pieds}
\rput{-90}(2.2,1.65){lance}
\end{pspicture}}

\textit{Il est utile de tracer un schéma simplifié du problème. }

\parbox{0.3\linewidth}{\psset{unit=1cm}
\begin{pspicture}(2,0)(4,3.4)
%\psgrid
\psline(2.7,0.3)(2.7,3)
\pspolygon(3.8,0.3)(2.7,2.55)(2.7,0.3)
\psline(2.52,2.56)(3,2.56)
\psline(2.52,3)(3,3)
\psline{<->}(3,2.56)(3,3)\uput[r](3,2.78){$h$}
\rput(3.2,-0.2){12}
\rput(3.8,1.5){20}
\uput[-135](2.7,0.2){$A$}
\uput[-45](3.8,0.3){$B$}
\uput[180](2.7,2.55){$C$}
\uput[90](2.7,3){$D$}
\end{pspicture}}
\hfill
\parbox{0.65\linewidth}{\psset{unit=.8cm}
\textit{Dans le schéma, $AD=CB=20$ pieds.}

\textit{Le triangle, $ABC$ étant rectangle en $A$, d'après le théorème de Pythagore, on a :}

$AC^2=CB^2-AB^2=20^2-12^2=400-144=256$. D'où \fbox{$AC=16$ pieds.}

La lance descend de $h=20 -16 =4$ pieds.
}

\vspace{0,5cm}

