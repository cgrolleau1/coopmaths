\textbf{\textsc{Exercice 1} \hfill 4 points}

\medskip

\emph{Pour chacune des quatre questions suivantes, plusieurs propositions de réponse sont faites. Une seule des propositions est exacte. Aucune justification n'est attendue. Une bonne réponse rapporte $1$ point. Une mauvaise réponse ou une absence de réponse rapporte $0$ point. Reporter sur votre copie le numéro de la question et donner la bonne réponse.}

\medskip
 
\begin{enumerate}
\item L'arbre ci-dessous est un arbre de probabilité.

\begin{center}
\psset{nrot=:U}
\pstree[treemode=R]{\Tdot}
{\Tdot~[tnpos=r]{}\naput{$\frac{1}{9}$}
 \Tdot~[tnpos=r]{}\nbput{\psset{unit=1cm}\begin{pspicture}(-0.25,0)(0.25,0.25)\pscurve*(-0.25,0.25)(-0.15,0.45)(0,0.5)(0.25,0.5)(0.25,0.25)(0.1,0.05)(0,0.1)(-0.15,0.15)(-0.25,0.25)\end{pspicture}}
 \Tdot~[tnpos=r]{}\nbput{$\frac{1}{3}$}
 } 
 \end{center}

La probabilité manquante sous la tache est: 

\medskip
\begin{tabularx}{\linewidth}{*{3}{X}} 
\textbf{a.~~} $\dfrac{7}{9}$ &\textbf{b.~~} $\dfrac{5}{12}$ &\textbf{c.~~} $\dfrac{5}{9}$
\end{tabularx}
\medskip
  
\item Dans une salle, il y a des tables à 3 pieds et à 4 pieds. Léa compte avec les yeux bandés 169 pieds. Son frère lui indique qu'il y a 34 tables à 4 pieds. Sans enlever son bandeau, elle parvient à donner le nombre de tables à 3 pieds qui est de :
 
\medskip
\begin{tabularx}{\linewidth}{*{3}{X}} 
\textbf{a.~~} 135&\textbf{b.~~} 11&\textbf{c.~~} 166 
\end{tabularx}

\medskip
\item 90\,\% du volume d'un iceberg est situé sous la surface de l'eau.
 
La hauteur totale d'un iceberg dont la partie visible est 35~m est d'environ: 

\medskip
\begin{tabularx}{\linewidth}{*{3}{X}}
\textbf{a.~~}  350 m&\textbf{b.~~} \np{3500} m&\textbf{c.~~} 31,5 m
\end{tabularx}

\medskip 
\item \psset{unit=0.6cm}\begin{pspicture}(3.1,2.3)
\psline(1.2,0.6)(0,0.6)(0,2.1)(1.2,2.1)
\psline(2.4,2.1)(3.1,2.1)(3.1,0.6)(2.4,0.6)
\psarc(1.8,2.1){0.6}{-180}{0}
\psarc(1.8,0.6){0.6}{-180}{0}
\end{pspicture} a le même périmètre que: 

\medskip
\begin{tabularx}{\linewidth}{*{3}{X}} 
\textbf{a.~~}  &\textbf{b.~~}  &\textbf{c.~~}\\
\psset{unit=0.6cm}\begin{pspicture}(3.2,2.5) 
\psframe(0,0)(3.1,2.1)\end{pspicture}&
\psset{unit=0.6cm}\begin{pspicture}(3.1,2.3)\psline(1.2,0.6)(0,0.6)(0,2.1)(1.2,2.1)
\psline(2.4,2.1)(3.1,2.1)(3.1,0.6)(2.4,0.6)
\psarc(1.8,2.1){0.6}{0}{180}
\psarc(1.8,0.6){0.6}{180}{0}
\end{pspicture}&\psset{unit=0.6cm}\begin{pspicture}(3.1,2.3)
\psline(1.2,0.6)(0,0.6)(0,2.1)(3.1,2.1)(3.1,0.6)(2.4,0.6)
\psarc(1.8,0.6){0.6}{180}{0}
\end{pspicture} 
\end{tabularx}

\medskip 
\end{enumerate}

\bigskip

