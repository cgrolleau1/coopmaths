
\medskip

%L'épreuve du marathon consiste à parcourir le plus rapidement possible la distance de
%42,195 km en course à pied. Cette distance se réfère historiquement à l'exploit effectué
%par le Grec Phillipidès, en 490 av. J-C, pour annoncer la victoire des Grecs contre
%les Perses. Il s'agit de la distance entre Marathon et Athènes.
%
%\medskip

\begin{enumerate}
\item %En 2014, le kényan Dennis Kimetto a battu l'ancien record du monde en
%parcourant cette distance en 2~h 2~min 57~s. Quel est alors l'ordre de grandeur
%de sa vitesse moyenne : 5 km/h,\: 10 km/h ou 20 km/h ?
À peu près 40 km en 2 h, donc 20 km en une heure.
\item %Lors de cette même course, le britannique Scott Overall a mis 2 h 15 min
%pour réaliser son marathon. Calculer sa vitesse moyenne en km/h. Arrondir la
%valeur obtenue au centième de km/h.
Il a couru en $2 \times 60 + 15 = 135$~(min). Sa vitesse moyenne est $v_{\text{Overall}} = \dfrac{42,195}{135}$~(km/min) $ = \dfrac{42,195}{135} \times 60 \approx 18,75$~(km/h).
\item %Dans cette question, on considérera que Scott Overall court à une vitesse
%constante. Au moment où Dennis Kimetto franchit la ligne d'arrivée,
%déterminer:
	\begin{enumerate}
		\item %le temps qu'il reste à courir à Scott Overall ;
		Scott Overall a mis 2 h 15 min $-$ 2 h 2 min 57 s = 12 min 3 s ou $12 \times 60 + 3 = 723$~(s).
		\item %la distance qu'il lui reste à parcourir. Arrondir le résultat au mètre près.
		À la vitesse moyenne calculée dans la question précédente soit $\dfrac{42,195}{135}$~(km/min) ou $\dfrac{42,195}{135 \times 60}$~(km/s) il lui reste donc à parcourir
		
$\dfrac{42,195}{135 \times 60} \times 723 \approx \np{3,7656}$ soit à peu près \np{3766}~(m).	
	\end{enumerate}
\end{enumerate}

\vspace{0,5cm}

