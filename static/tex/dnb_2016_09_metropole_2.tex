\textbf{\textsc{Exercice 2} \hfill 6 points}

\medskip

\parbox{0.47\textwidth}{Sur la figure ci-contre. le point J appartient au
segment [IM] et le point K appartient au segment [IL].

Sur la figure, les longueur sont données en mètres.

\medskip

\begin{enumerate}
\item Montrer que IKJ est un triangle rectangle.
\item Montrer que LM est égal à 3,75~m.
\item Calculer la longueur KM au centimètre près.
\end{enumerate}}
\hfill
\parbox{0.47\textwidth}{
\psset{unit=0.5cm}
\begin{pspicture}(12,9)
%\psgrid
\pspolygon(1,1)(11.4,3.7)(6.3,8.3)
\psline(4.42,5.62)(7.8,2.72)
\psline[linestyle=dashed](4.42,5.62)(11.4,3.7)
\psset{arrowsize=3pt 5}
\psline[linewidth=0.5pt]{<->}(0.8,1.2)(4.2,5.8)\rput(2.3,3.8){\small 3,2}
\psline[linewidth=0.5pt]{<->}(4.2,5.8)(6.2,8.4)\rput(5,7.4){\small 1,8}
\psline[linewidth=0.5pt]{<->}(4.5,5.82)(7.9,3)\rput(6.4,4.75){\small 2,4}
\psline[linewidth=0.5pt]{<->}(1.2,0.8)(7.8,2.6)\rput(4.6,1.3){\small 4}
\uput[dl](1,1){I} \uput[ul](4.42,5.62){K}\uput[u](6.3,8.3){L}
\uput[r](11.4,3.7){M}\uput[dr](7.8,2.72){J}
\rput{-130}(6.3,8.3){\psframe(0.3,0.3)}
\end{pspicture}}

\vspace{0,5cm}

