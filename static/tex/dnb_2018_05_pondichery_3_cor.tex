
\medskip

%Cet exercice est un Q. C. M. (Questionnaire à choix multiples).
%
%Pour chacune des questions, quatre réponses sont proposées et une seule est exacte. Une réponse fausse ou absente n'enlève pas de point.
%
%\smallskip
%
%Pour chacune des trois questions, écrire sur votre copie le numéro de la question et la lettre correspondant à
%la bonne réponse.
%
%\begin{center}
%\begin{tabularx}{\linewidth}{|c|m{2cm}|*{3}{>{\footnotesize\centering \arraybackslash}X|}m{2cm}|}\cline{3-6}
%\multicolumn{2}{c|}{~}&Réponse a& Réponse b& Réponse c& Réponse d\\ \hline
%1 &$2,53 \times  10^{15} =$&\np{2,530 000 000 000 000 00}&\np{2530000000000000}&\np{253000000000000000}& \footnotesize 37,95\\ \hline
%2&\footnotesize La latitude de l'équateur est :  &$0\degres$& $90\degres$ Est& $90\degres$ Nord &\footnotesize$90\degres$ Sud\\ \hline
%3&$\dfrac{\frac{2}{3} + \frac{5}{6}}{7} = $&$\dfrac{3}{14}$&$\dfrac{1}{9}$ &\np{0,214285714}&\footnotesize \np{0,111 111 111}\\ \hline
%\end{tabularx}
%\end{center}

\begin{enumerate}
\item $2,53 \times 10^{15} = \np{2530000000000000}$ : réponse b.
\item La latitude de l'équateur est $0$\degres : réponse a.
\item $\dfrac{\frac{2}{3} + \frac{5}{6}}{7} = \dfrac{\frac{4}{6} + \frac{5}{6}}{7} =\dfrac{\frac{9}{6}}{7}
= \dfrac{\frac{3}{2}}{7} = \frac{3}{2} \times \frac{1}{7} = \dfrac{3}{14}$ : réponse a.
\end{enumerate}

\vspace{0,5cm}

