\textbf{Exercice 3 \hfill 3,5 points}

\medskip

%Le paraha peue (le \og  poisson lune \fg{} local ou platax orbicularis) est
%l'espèce prioritaire du développement de la pisciculture lagonnaire en
%Polynésie française, avec l'émergence d'une filière de production
%locale.
%
%La courbe ci-dessous représente la croissance du paraha peue en cage.
%
%\begin{center}
%\psset{xunit=0.024cm,yunit=0.006cm}
%\begin{pspicture}(-10,-100)(460,1200)
%\multido{\n=0+50}{10}{\psline[linestyle=dotted](\n,0)(\n,1200)}
%\multido{\n=0+200}{6}{\psline[linestyle=dotted](0,\n)(460,\n)}
%\pscurve(0,0)(50,25)(100,100)(150,200)(200,315)(250,485)(300,680)(350,900)(400,1110)
%\psaxes[linewidth=1.25pt,Dx=50,Dy=200]{->}(0,0)(460,1200)
%\uput[r](0,1150){Masse en g}
%\uput[u](390,0){Nombre de jours en cage}
%\end{pspicture}
%\end{center}

\begin{enumerate}
\item %Compléter le tableau suivant en utilisant le tracé.

\begin{center}
\begin{tabularx}{\linewidth}{|m{3cm}|*{4}{>{\centering \arraybackslash}X|}}\hline
Nombre de jours en cage	&30	&150	&225	&365\\ \hline
Masse en g				& 7	&200	&400	&\np{1000}\\ \hline
\end{tabularx}
\end{center}

\item  %Les grandeurs, nombre de jours en cage et masse, sont-elles proportionnelles? (justifier la réponse)
Non : au bout de 100 jours la masse est d'environ 100 g et en 200 jours la masse n'est pas le double mais à peu près 300~g.
\hfill

\end{enumerate}
 
\bigskip

