
\medskip

\begin{enumerate}
\item 
	\begin{enumerate}
		\item Le diamètre de $C_2$ est $1,5$ cm. Son rayon est donc $\dfrac{1,5}{2} = 0,75 $~cm.
		
L'aire B de sa base est $\pi \times r^2 = \pi \times  0,75^2$.

Son volume est $V = B \times h = \pi \times 0,752 \times 4,2$.

Le volume de sable est $\dfrac{2}{3}\times \pi \times 0,75^2 \times 4,2$, soit environ $4,95$ cm$^3$.

L'aire d'un disque de rayon $r$ est $\pi \times r^2$.
		\item  On a : volume = vitesse d'écoulement $\times$ temps.
		
Donc le temps d'écoulement est  $\dfrac{\text{volume}}{\text{vitesse d'écoulement}}= \dfrac{4,95}{1,98} = 2,5$.

Le temps d'écoulement est $2,5$ minutes, soit 2 minutes 30 secondes.
	\end{enumerate}
\item 
	\begin{enumerate}
		\item On a : $1 + 1 + 2 + 6 + 3 + 7 + 6 + 3 + 1 + 2 + 3 + 2 + 3 = 40$.
		
On a effectué 40 tests.
		\item  
		
$\bullet~~$ La plus grande valeur est 2 min 38 s et le plus petite est 2 min 22 s.

La différence (étendue de la série) est de $16$ secondes, inférieure à $20$ s.

$\bullet~~$ La médiane est la moyenne entre la 20\up{e} valeur de la série ordonnée et la 21\up{e} valeur.

Or, on a $1 + 1 + 2 + 6 + 3 + 7 = 20$, donc la 20\up{e} valeur est 2 min 29 s et la 21\up{e} est 2 min 30. 

La médiane est bien comprise entre 2 min 29 s et 2 min 31 s.

$\bullet~~$Comme tous les temps commencent par 2 min, il suffit de faire la moyenne des secondes en faisant :

$\dfrac{1 \times 22 + 1 \times 24 + \ldots  + 2 \times 35 + 3 \times 38}{40} = \dfrac{\np{1204}}{40}  = 30,1$.

-- Le temps moyen d'écoulement est 2 min 30,1 s.

-- La moyenne est entre 2 min 28 s et 2 min 32 s.

-- Le sablier testé ne sera pas rejeté.
	\end{enumerate}
\end{enumerate}
\bigskip

