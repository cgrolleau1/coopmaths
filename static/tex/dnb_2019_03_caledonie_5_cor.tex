
\medskip

\begin{enumerate}
	\item 
		\begin{enumerate}
			\item On compte 21 pays dans la série statistique présentée, qui ont accumulé :
		
		$14 + 14 + 11 + 9 + 8 + 7 + 5 + 5 5 + 5 + 4 + 3 + 2 + 2 + 2+ 1+1+1+1+1+1 = 102$ médailles d'or.
		
		Le nombre moyen de médaille d'or des pays en ayant obtenu s'établit donc à $\dfrac{34}{7}\approx 4,9$ médailles (au dixième près).
		
			\item Il y a 21 pays dans notre série statistique, donc comme $\dfrac{21 + 1}{2} = 11$, la médiane est la 11\up{e} valeur de la série, rangée dans l'ordre (croissant, ou décroissant, peu importe). Ici, la médiane est donc de 4. 
		
			\item Au moins la moitié des pays ont un nombre de médailles inférieur ou égal à 4. 
		
En effet, il y en a 11 (Norvège, Allemagne, Canada, États-Unis, Pays-Bas, Suède, République de Corée, Suisse, France, Autriche et Japon), et 11 est supérieur à $\dfrac{21}{2}$.
		
\medskip

De façon analogue, on peut aussi dire \og Au moins la moitié des pays ont un nombre de médailles supérieur ou égal à 4.\fg
		
Là encore, 11 pays ont un nombre de médailles inférieur ou égal à 4 (Japon, Italie, Russie, République Tchèque, Bélarus, Chine, Slovaquie, Finlande, Grande Bretagne, Pologne et Hongrie), et 11 est supérieur à $\dfrac{21}{2}$.
		\end{enumerate}
	\item Comme il faut additionner tous les effectifs, la formule la plus efficace est : \textsf{=SOMME(B2:~K2)}.
	\item Dans cette expérience aléatoire, on choisit un pays au hasard, donc on est en situation d'équiprobabilité, avec vingt et une issues possibles, donc :
		\begin{enumerate}
			\item Il y a six issues favorables à l'évènement (Chine, Slovaquie, Finlande, Grande Bretagne, Pologne et Hongrie), donc la probabilité de l'évènement est $\dfrac{6}{21}$.
			\item Il y a dix issues favorables à l'évènement (Norvège, Allemagne, Canada, États-Unis, Pays-Bas, Suède, République de Corée, Suisse, France et Autriche), donc la probabilité de l'évènement est $\dfrac{10}{21}$.
		\end{enumerate}
\end{enumerate}

\vspace{0,5cm}

