\textbf{\textsc{Exercice 4} \hfill 3 points}

\medskip

Durant un parcours sur le Canal du Midi partant de l'écluse de Renneville jusqu'à l'écluse de Gay, on a relevé les hauteurs de chaque écluse franchie depuis le départ dans la feuille de calcul donnée en annexe 1.
 
Les hauteurs franchies de manière ascendante sont notées positivement, celles de manière descendante négativement.

\medskip
 
\begin{enumerate}
\item \textit{La formule que l'on doit saisir dans la cellule M5 pour obtenir la valeur du dénivelé* du parcours est} \fbox{=SOMME(B3:L3)}
\item \textit{La valeur du dénivelé* du parcours est la somme des dénivelés intermédiaires soit} \fbox{$-21,47$~m} 
\item \textit{Le parcours est  globalement \fbox{descendant} car le dénivelé est négatif}.
 
* \emph{Le dénivelé du parcours représente la différence de niveau (hauteur) entre les écluses.}
\end{enumerate}
 
\vspace{0,5cm}

