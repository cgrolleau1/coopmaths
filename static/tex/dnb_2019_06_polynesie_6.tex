
\medskip

Le tableau ci-dessous regroupe les résultats de la finale du $200$~m hommes des Jeux
Olympiques de Rio de Janeiro en 2016, remporté par Usain Bolt en $19,78$ secondes.

\begin{center}
\begin{tabularx}{\linewidth}{|c|*{3}{>{\centering \arraybackslash}X|}}\hline
\textbf{Rang}& \textbf{Athlète}& \textbf{Nation}& \textbf{Performance en seconde}\\ \hline
1 &U. Bolt& Jamaïque& 19,78\\ \hline
2 &A. De Grasse& Canada& 20,02\\ \hline
3 &C. Lemaitre& France& 20,12\\ \hline
4 &A. Gemili& Grande-Bretagne& 20,12\\ \hline
5 &C. Martina& Hollande& 20,13\\ \hline
6 &L. Merritt& USA& 20,19\\ \hline
7 &A. Edward& Panama& 20,23\\ \hline
8 &R. Guliyev& Turquie& 20,43\\ \hline
\end{tabularx}
\end{center}

\medskip

\begin{enumerate}
\item Calculer la vitesse moyenne en m/s de l'athlète le plus rapide. Arrondir au centième.
\item Calculer la moyenne des performances des athlètes. Arrondir au centième.
\item En 1964 à Tokyo, la moyenne des performances des athlètes sur le $200$ m hommes était de $20,68$~s et l'étendue était de $0,6$ s. En comparant ces résultats à ceux de 2016, qu'observe-t-on ? 
\end{enumerate}

\bigskip

