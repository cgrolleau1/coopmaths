\textbf{Exercice 6 \hfill 4 points}

\medskip

M. Durand doit changer de voiture. Il choisit un modèle PRIMA qui existe en deux versions:
ESSENCE ou DIESEL. Il dispose des informations suivantes :

\begin{center}
\begin{tabularx}{\linewidth}{|*{2}{X|}}\hline
\multicolumn{2}{|c|}{Modèle PRIMA}\\ \hline
\multicolumn{1}{|c|}{Version ESSENCE}&\multicolumn{1}{|c|}{Version DIESEL}\\
$\bullet~~$ Consommation moyenne :	&$\bullet~~$ Consommation moyenne\\
6,2 L pour 100 km					&5,2 L pour 100 km\\
$\bullet~~$ Type de moteur: essence	&$\bullet~~$ Type de moteur : diesel\\
$\bullet~~$ Carburant: SP 95		&$\bullet~~$ Carburant: gazole\\
$\bullet~~$ Prix d'achat: \np{21550}~\euro&$\bullet~~$ Prix d'achat : \np{23950}~\euro\\ \hline
\end{tabularx}
\end{center}

\begin{center}
\begin{tabularx}{\linewidth}{|X|}\hline
\multicolumn{1}{|c|}{Estimation du prix des carburants par}\\\multicolumn{1}{|c|}{
M. Durand en 2015}\\
$\bullet~~$ Prix d'un litre de SP 95 : 1,415~\euro\\
$\bullet~~$ Prix d'un litre de gazole: 1,224~\euro\\\hline
\end{tabularx}
\end{center}

Durant les dernières années, M. Durand a parcouru en moyenne \np{22300}~km par an.

Pour choisir entre les deux modèles, il décide de réaliser le tableau comparatif ci-dessous, établi pour \np{22300}~km  parcourus en un an.

\begin{center}
\begin{tabularx}{\linewidth}{|l|*{2}{>{\centering \arraybackslash}X|}}\hline
&Version ESSENCE 				&Version DIESEL\\ \hline
Consommation de carburant (en L)& \np{1383}&\\ \hline
Budget de carburant (en \euro) 	&\np{1957}&\\ \hline
\end{tabularx}
\end{center}

\medskip

\begin{enumerate}
\item Recopier et compléter le tableau sur la copie en écrivant les calculs effectués.
\item M. Durand choisit finalement la version DIESEL.

En considérant qu'il parcourt \np{22300}~km tous les ans et que le prix du carburant ne varie pas, dans combien d'années l'économie réalisée sur le carburant compensera-t-elle la différence de prix d'achat entre les deux versions ?
\end{enumerate}

\bigskip

