\documentclass[a4paper,11pt,fleqn]{article}
%%%%%%%%%%%%%%%%%%%%%%%%%%%%%%%%%%%%%%%%%%%%
%%% Préambule pour le fichier Coop-Maths %%%
%%%%%%%%%%%%%%%%%%%%%%%%%%%%%%%%%%%%%%%%%%%%



% Remarques : 
%		
%		* il faut penser à  doubler l'interligne le plus souvent possible avec \begin{spacing}{2} ou \begin{enumerate}[itemsep=1em]
%
%

% Prérequis : 
%		* Compiler avec XeTeX
%		* Avoir le répertoire images avec les entêtes, pieds de pages et icones
%
%
% Utilisation : 
%		* La fiche commence avec \theme{nombres|gestion|grandeurs|geo|algo}{Texte (entrainement, évaluation, mise en route...}{numéro de version ou vide}{titre du thême et niveau}
%
%		* On peut préférer \themelight{couleur_theme}{couleur_numerotation} notamment pour faire des corrigés à  décalquer
%
%
%
%		* Chaque exercice commence par \exo{titre ou vide}. Le numéro de l'exercice utilise le compteur de section
%		
%		* \begin{correction}...\end{correction} remet le compteur d'exercice à  0, met le cadre adéquat et change la couleur du thême en vert 
%
%		* Plusieurs environnements sont définis : objectif, methode (dans lequel il faut faire précéder le titre d'un \iconemethode
%
%		* Une macro \cfbox permet de faire la même chose que \fbox mais avec la couleur du thême
%
%		* Une macro \reference pour donner la référence d'un objectif dans la couleur du thême. Par exemple \reference{G01}
%
%       * Une macro \motimportant qui met en gras dans la couleur du thême.
%
%		* Une macro \mathimportant qui met en gras dans la couleur du thême à  l'intérieur d'un mode math.
%
%		* Une macro \mathimportantcorr qui met en gras dans la couleur de la correction à  l'intérieur d'un mode math.
%

%		* Une macro \pointillés qui met des pointillés sur la longueur choisie
%
%		* Une macro \point{x}{y}{nom} qui trace un point dans le repêre au couleur du thême
%		
%		* Une macro \pointpos{x}{y}{nom}{position} qui trace un point dans le repêre au couleur du thême
%
%		* Une macro \pointcorr{x}{y}{nom} qui fait la même chose dans la couleur de la correction
%
%		* Une macro \pointcorrpos{x}{y}{nom}{position} qui trace un point dans le repêre au couleur du thême
%
%		* Une macro \tampon(nom) qui insêre le graphique tampon-nom en haut à  droite de la feuille
%
%	* Une macro \version(v) qui insêre le numéro de version en haut à  droite de la feuille (à  ne pas utiliser en même temps que tampon)
%
%		* Une macro \mathunderline pour souligner en couleur correction une partie d'une expression mathématique
%
%	* Une macro \deg pour le signe des degrés
%
%	* Une macro \exalea{id} qui met l'image du dé et le lien en gras vers l'exercice 
%

\usepackage[left=1.5cm,right=1.5cm,top=3.5cm,bottom=2cm]{geometry}
%\usepackage[utf8]{inputenc}		        % Accents, encodage utf8
% Inutile avec XeTeX ?
%\usepackage[T1]{fontenc}		        		% Encodage des caractêres
\usepackage{lmodern}			        		% Choix de la fonte (Latin Modern de D. Knuth)
\usepackage[french]{babel}	        		% Les rêgles de typo. franà§aises
\usepackage{multicol} 					% Multi-colonnes
\usepackage{calc} 						% Calculs 
\usepackage{enumerate}					% Pour modifier les numérotations
\usepackage{enumitem}
\usepackage{graphicx}					% Pour insérer des images
\usepackage{tabularx}					% Pour faire des tableaux
\usepackage{pgf,tikz}					% Pour les images et figures géométriques
\usetikzlibrary{arrows,calc,fit,patterns,plotmarks,shapes.geometric,shapes.misc,shapes.symbols,shapes.arrows,
shapes.callouts, shapes.multipart, shapes.gates.logic.US,shapes.gates.logic.IEC, er, automata,backgrounds,chains,topaths,trees,petri,mindmap,matrix, calendar, folding,fadings,through,positioning,scopes,decorations.fractals,decorations.shapes,decorations.text,decorations.pathmorphing,decorations.pathreplacing,decorations.footprints,decorations.markings,shadows}
\usepackage{tkz-euclide}				% Géométrie euclidienne avec TikZ
\usepackage{pgfplots}					% Représentations graphiques de fonctions
\usepackage{amsmath,amsfonts,amssymb,mathrsfs}  % Spécial math
\usepackage[squaren]{SIunits}			% Pour les unités (gêre le conflits avec  \square de l'extension amssymb)
\usepackage{pifont}						% Pour les symboles "ding"
\usepackage{bbding}						% Pour les symboles
\usepackage[misc]{ifsym}						% Pour les symboles
\usepackage{cancel}						% Pour pouvoir barrer les nombres
\usepackage{url} 			        		% Pour afficher correctement les url
 
\usepackage{eurosym}						% Pour utiliser la commande \euro
\usepackage{fancyhdr,lastpage}          	% En-têtes et pieds
 \pagestyle{fancy}                      	% de pages personnalisés
\usepackage{fancybox}					% Pour les encadrés
\usepackage{xlop}						% Pour les calculs posés
\usepackage{standalone}					% Pour avoir un apercu d'un fichier qui sera utilisé avec un input
\usepackage{multido}						% Pour faire des boucles
\usepackage[hidelinks]{hyperref}					% Pour gérer les liens hyper-texte
\renewcommand{\url}[1]{\textcolor{couleur_theme}{\href{http://#1}{#1}}} 	% Pour ajouter automatiquement le http
\usepackage{fourier}
\usepackage{colortbl} 					% Pour des tableaux en couleur
\usepackage{setspace}					% Pour \begin{spacing}{2.0} \end{spacing}
\usepackage{multirow}					% Pour des cellules multilignes dans un tableau
\usepackage{import}						% Equivalent de input mais en spécifiant le répertoire de travail
\usepackage[]{qrcode}					% Pour la commande \qrcode
\usepackage{pdflscape}
\usepackage[framemethod=tikz]{mdframed} % Pour les cadres
\usepackage{tikzsymbols}
\usepackage{tasks}						% Pour les listes horizontales


\graphicspath{{./images/}}				% Le répertoire


%%% Choix de la police


\usepackage{mathspec}
\setmainfont{Myriad Pro}
\setmathrm{Myriad Pro}
\setmathsf{Myriad Pro}
\setmathtt{Myriad Pro}
%\setboldmathrm[BoldFont={Optima ExtraBlack}]{Optima Bold}
\setmathfont(Digits){Myriad Pro}
\setmathfont(Latin){Palatino}

\spaceskip=2\fontdimen2\font plus 3\fontdimen3\font minus3\fontdimen4\font\relax %Pour doubler l'espace entre les mots


%%% Mise en forme

\setlength{\parindent}{0mm}				% Pas de retrait en début de paragraphe
\setlength\multicolsep{0pt} 				% Pour que l'environnement multicols ne commence pas par un saut de ligne
\renewcommand{\arraystretch}{1.5}		% Interligne dans les tableaux
\renewcommand{\labelenumi}{\textbf{\theenumi{}.}}		% Numérotation en gras
\renewcommand{\labelenumii}{\textbf{\theenumii{}.}}		% Numérotation de niveau 2 en gras
\renewcommand{\thesection}{\Roman{section}.}				% Numérotation des sections en chiffres romains
\renewcommand{\thesubsection}{\alph{subsection})}		% Numérotation des sous-sections en lettres
\setlength{\columnsep}{1cm}								% Séparation des colonnes
\setlength{\columnseprule}{1.5pt}
\renewcommand{\columnseprulecolor}{\color{couleur_theme}} % Trait de séparation des colones en couleur


\setlength\arrayrulewidth{1.5pt} 	% Epaisseur des les filets des tableaux
\arrayrulecolor{couleur_theme}		% Couleur des filets des tableaux


\renewcommand{\headrulewidth}{0pt} % Pour enlever les traits en en-tête et en pied de page
\renewcommand{\footrulewidth}{0pt}
\fancyhead[L]{}
\fancyhead[R]{}



%%% Couleurs

\definecolor{nombres}{cmyk}{0,.8,.95,0}
\definecolor{gestion}{cmyk}{.75,1,.11,.12}
\definecolor{gestionbis}{cmyk}{.75,1,.11,.12}
\definecolor{grandeurs}{cmyk}{.02,.44,1,0}
\definecolor{geo}{cmyk}{.62,.1,0,0}
\definecolor{algo}{cmyk}{.69,.02,.36,0}
\definecolor{correction}{cmyk}{.63,.23,.93,.06}



%%% Environnements


\newmdenv[linecolor=couleur_theme, linewidth=3pt,topline=true,rightline=false,bottomline=false,frametitlerule=false,frametitlefont={\color{couleur_theme}\bfseries},frametitlerulewidth=1pt]{methode}


\newmdenv[startcode={\setlength{\multicolsep}{0cm}\setlength{\columnsep}{.2cm}\setlength{\columnseprule}{0pt}\vspace{0cm}},linecolor=white, linewidth=3pt,innerbottommargin=10pt,innertopmargin=5pt,innerrightmargin=20pt,splittopskip=20pt,splitbottomskip=10pt,everyline=true,tikzsetting={draw=couleur_theme,line width=4pt,dashed,dash pattern= on 10pt off 10pt},frametitleaboveskip=-.6cm,frametitle={\tikz\node[anchor= east,rectangle,fill=white]{\textcolor{couleur_theme}{\raisebox{-.3\height}{\includegraphics[width=.8cm]{\iconeobjectif}}\; \bfseries \Large Objectifs}};}]{objectif}

\newmdenv[startcode={\colorlet{couleur_numerotation}{correction}\renewcommand{\columnseprulecolor}{\color{correction}}
\setcounter{section}{0}\arrayrulecolor{correction}},linecolor=white, linewidth=4pt,innerbottommargin=10pt,innertopmargin=5pt,splittopskip=20pt,splitbottomskip=10pt,everyline=true,frametitle=correction,tikzsetting={draw=correction,line width=3pt,dashed,dash pattern= on 15pt off 10pt},frametitleaboveskip=-.4cm,frametitle={\tikz\node[anchor= east,rectangle,fill=white]{\; \textcolor{correction}{\raisebox{-.3\height}{\includegraphics[width=.6cm]{icone-correction}}\; \bfseries \Large Corrections}};}]{correction}

\newmdenv[roundcorner=0,linewidth=0pt,frametitlerule=false, backgroundcolor=gray!40,leftmargin=8cm]{remarque}



%%% Macros



\newcommand{\theme}[4]
{
	%\theme{nombres|gestion|grandeurs|geo|algo}{Texte (entrainement, évaluation, mise en route...}{numéro de version ou vide}{titre du thême et niveau}
	\fancyhead[C]{
	\begin{tikzpicture}[remember picture,overlay]
	  \node[anchor=north east,inner sep=0pt] at ($(current page.north east)+(0,-.8cm)$) {\includegraphics{header-#1}};
	  \node[anchor=east, fill=white] at ($(current page.north east)+(-2,-1.4cm)$) {\Huge \textcolor{couleur_theme}{\bfseries \#} #2 \textcolor{couleur_theme}{\bfseries \MakeUppercase{#3}}};
	  \node[anchor=center, color=white] at ($(current page.north)+(0,-2.65cm)$) {\Large \bfseries \MakeUppercase{#4}};
	\end{tikzpicture}
	}
	\fancyfoot[C]{
	\begin{tikzpicture}[remember picture,overlay]
	  \node[anchor=south west,inner sep=0pt] at ($(current page.south west)+(0,0)$) {\includegraphics{footer-#1}};
	\end{tikzpicture} 
	}
	\colorlet{couleur_theme}{#1}
	\colorlet{couleur_numerotation}{couleur_theme}
	\def\iconeobjectif{icone-objectif-#1}
	\def\urliconeomethode{icone-methode-#1}
}

\newcommand{\themelight}[2]
{
\pagestyle{empty}
\colorlet{couleur_numerotation}{#2}
\colorlet{couleur_theme}{#1}
\geometry{top=1cm,bottom=1cm}

}




\newcommand{\numb}[1]{ % Dessin autour du numéro d'exercice
\begin{tikzpicture}[overlay,yshift=-.3cm,scale=.8]
	\draw[fill=couleur_numerotation,couleur_numerotation](-.3,0)rectangle(.5,.8);
	\draw[line width=.05cm,couleur_numerotation,fill=white] (0,0)--(.5,.5)--(1,0)--(.5,-.5)--cycle;
	\node[couleur_numerotation]  at (.5,0) { \large \bfseries #1};
		\draw (-.4,.8) node[white,anchor=north west]{\bfseries EX}; 
\end{tikzpicture}
}


\usepackage{titlesec} % Le titre de section est en fait un numéro d'exercice avec sa consigne alignée à  gauche.


\titleformat{\section}{}{\numb{\arabic{section}}}{1cm}{\hspace{0em}}{}


\newcommand{\exo}[1]{ % Un titre d'exercice est en f	ait une nouvelle section avec la consigne écrite en caractêres normaux
\section{\textmd{#1}}
\medskip
}

\newcommand{\iconemethode}[0]{ % Pour l'icone des méthodes dans la bonne couleur
\raisebox{-.3\height}{\includegraphics[width=1cm]{\urliconeomethode}}\quad
}


\newcommand{\cfbox}[1]{% Cadre dans la couleur du thême
	\setlength{\fboxrule}{2pt}
    \colorlet{currentcolor}{.}%
    {\color{couleur_theme}%
    \fbox{\color{currentcolor}#1}}%
}

\newcommand{\corrfbox}[1]{% Cadre dans la couleur de la correction
	\setlength{\fboxrule}{2pt}
    \colorlet{currentcolor}{.}%
    {\color{correction}%
    \fbox{\color{currentcolor}#1}}%
}


\newcommand{\reference}[1]{\ignorespaces
\textcolor{couleur_theme}{\textbf{#1 -}}\xspace
}

\newcommand{\refexo}[1]{\ignorespaces
\textcolor{couleur_theme}{\textbf{\;Ex #1}}\xspace
}

\newcommand{\motimportant}[1]{\ignorespaces
\textcolor{couleur_theme}{\textbf{#1}}\xspace
}

\newcommand{\mathimportant}[1]{
\mathbin{\color{couleur_theme}\pmb{#1}}
}

\newcommand{\mathimportantcorr}[1]{
{\color{correction}\pmb{#1}}
}

\newcommand{\motimportantcorr}[1]{\ignorespaces
\textcolor{correction}{\textbf{#1}}\xspace
}

\newcommand{\pointilles}[1]{
\makebox[#1]{\dotfill}
}

\newcommand{\point}[3]{
\draw[couleur_theme,ultra thick] (#1,#2)--++(.3,.3)--++(-.6,-.6)--++(.3,.3)--++(-.3,.3)--++(.6,-.6);
\draw[couleur_theme] (#1,#2)  node[above right=.1,black]{$#3$};
}

\newcommand{\pointpos}[4]{
\draw[couleur_theme,ultra thick] (#1,#2)--++(.3,.3)--++(-.6,-.6)--++(.3,.3)--++(-.3,.3)--++(.6,-.6);
\draw[couleur_theme] (#1,#2)  node[#4=.1,black]{$#3$};
}


\newcommand{\pointcorr}[3]{
\draw[correction,ultra thick] (#1,#2)--++(.3,.3)--++(-.6,-.6)--++(.3,.3)--++(-.3,.3)--++(.6,-.6);
\draw[correction] (#1,#2)  node[above right=.1,black]{$#3$};
}

\newcommand{\pointcorrpos}[4]{
\draw[correction,ultra thick] (#1,#2)--++(.3,.3)--++(-.6,-.6)--++(.3,.3)--++(-.3,.3)--++(.6,-.6);
\draw[correction] (#1,#2)  node[#4=.1,black]{$#3$};
}



\newcommand{\tampon}[1]{
	\fancyhead[R]{
	\begin{tikzpicture}[remember picture,overlay]
		\node[anchor=north east,inner sep=0pt] at ($(current page.north east)+(-.3,-.3cm)$) {\includegraphics[width=1.6cm]{tampon-#1}};
	\end{tikzpicture}
	}
}

\newcommand{\version}[1]{
	\fancyhead[R]{
	\begin{tikzpicture}[remember picture,overlay]
		\node[anchor=north east,inner sep=0pt] at ($(current page.north east)+(-.5,-.5cm)$) {\large \textcolor{couleur_theme}{\bfseries V#1}};
	\end{tikzpicture}
	}
}

\newcommand{\exalea}[1]{
	\raisebox{-.5\height}{\includegraphics[height=1cm]{exalea}} \textcolor{couleur_theme}{\bfseries \url{coopmaths.fr/ex#1}}
}

\newcommand{\youtube}[1]{
	\raisebox{-.3\height}{\includegraphics[height=.6cm]{youtube}} \textcolor{couleur_theme}{\bfseries \url{coopmaths.fr/video#1}}
}


\def\mathunderline#1{\color{correction}\underline{{\color{black}#1}}\color{black}}

\renewcommand{\deg}{\ensuremath{^\circ}}


\theme{nombres}{Exemples}{}{Exercices générés avec Maths.ALÉA()}
\begin{document}

\version{1}

\exo{Calculer}

\begin{multicols}{2}
\begin{enumerate}[itemsep=2em]
	\item $ 7 \times 10 = \dotfill $
	\item $ 7 \times 7 = \dotfill $
	\item $ 80 \times 40 = \dotfill $
	\item $ 70 \times \nombre{3000} = \dotfill $
	\item $ 3 \times 2 = \dotfill $
	\item $ 20 \times \nombre{6000} = \dotfill $
	\item $ 700 \times 900 = \dotfill $
	\item $ 900 \times \nombre{9000} = \dotfill $
	\item $ 900 \times 6 = \dotfill $
	\item $ 50 \times 800 = \dotfill $
\end{enumerate}
\end{multicols}

\exo{Calculer}

\begin{multicols}{2}
\begin{enumerate}[itemsep=2em]
	\item $ (+19) + (-13) = \dotfill $
	\item $ (-9) + (+2) = \dotfill $
	\item $ (-11) + (-1) = \dotfill $
	\item $ (-5) + (-18) = \dotfill $
	\item $ (+10) + (-8) = \dotfill $
	\item $ (-2) + (+5) = \dotfill $
	\item $ (-7) + (+4) = \dotfill $
	\item $ (-2) + (+18) = \dotfill $
	\item $ (-6) + (+14) = \dotfill $
	\item $ (+5) + (-12) = \dotfill $
\end{enumerate}
\end{multicols}

\exo{Calculer}

\begin{multicols}{2}
\begin{enumerate}[itemsep=2em]
	\item $ -13+8 = \dotfill $
	\item $ 15-19 = \dotfill $
	\item $ 16-10 = \dotfill $
	\item $ -17-8 = \dotfill $
	\item $ -16+14 = \dotfill $
	\item $ -6-19 = \dotfill $
	\item $ -14+11 = \dotfill $
	\item $ 3-20 = \dotfill $
	\item $ -10+12 = \dotfill $
	\item $ 7-16 = \dotfill $
\end{enumerate}
\end{multicols}

\newpage
\exo{Compléter.}

\begin{multicols}{2}
\begin{enumerate}[itemsep=2em]
	\item $ \dfrac{3}{5} = \dfrac{\phantom{000000000000}}{\phantom{000000000000}} = \dfrac{}{45} $
	\item $ \dfrac{1}{2} = \dfrac{\phantom{000000000000}}{\phantom{000000000000}} = \dfrac{}{22} $
	\item $ \dfrac{9}{10} = \dfrac{\phantom{000000000000}}{\phantom{000000000000}} = \dfrac{}{100} $
	\item $ \dfrac{1}{5} = \dfrac{\phantom{000000000000}}{\phantom{000000000000}} = \dfrac{}{50} $
	\item $ \dfrac{3}{10} = \dfrac{\phantom{000000000000}}{\phantom{000000000000}} = \dfrac{}{80} $
	\item $ \dfrac{3}{8} = \dfrac{\phantom{000000000000}}{\phantom{000000000000}} = \dfrac{}{64} $
	\item $ \dfrac{1}{6} = \dfrac{\phantom{000000000000}}{\phantom{000000000000}} = \dfrac{}{18} $
	\item $ \dfrac{2}{5} = \dfrac{\phantom{000000000000}}{\phantom{000000000000}} = \dfrac{}{40} $
	\item $ \dfrac{2}{7} = \dfrac{\phantom{000000000000}}{\phantom{000000000000}} = \dfrac{}{63} $
	\item $ \dfrac{1}{8} = \dfrac{\phantom{000000000000}}{\phantom{000000000000}} = \dfrac{}{48} $
\end{enumerate}
\end{multicols}

\exo{Simplifier les fractions suivantes.}

\begin{multicols}{2}
\begin{enumerate}[itemsep=2em]
	\item $ \dfrac{40}{70} = \dfrac{\phantom{00000000000000}}{} = \dfrac{\phantom{0000}}{} $
	\item $ \dfrac{6}{12} = \dfrac{\phantom{00000000000000}}{} = \dfrac{\phantom{0000}}{} $
	\item $ \dfrac{30}{36} = \dfrac{\phantom{00000000000000}}{} = \dfrac{\phantom{0000}}{} $
	\item $ \dfrac{40}{56} = \dfrac{\phantom{00000000000000}}{} = \dfrac{\phantom{0000}}{} $
	\item $ \dfrac{33}{77} = \dfrac{\phantom{00000000000000}}{} = \dfrac{\phantom{0000}}{} $
	\item $ \dfrac{6}{27} = \dfrac{\phantom{00000000000000}}{} = \dfrac{\phantom{0000}}{} $
	\item $ \dfrac{6}{21} = \dfrac{\phantom{00000000000000}}{} = \dfrac{\phantom{0000}}{} $
	\item $ \dfrac{25}{40} = \dfrac{\phantom{00000000000000}}{} = \dfrac{\phantom{0000}}{} $
	\item $ \dfrac{77}{110} = \dfrac{\phantom{00000000000000}}{} = \dfrac{\phantom{0000}}{} $
	\item $ \dfrac{44}{99} = \dfrac{\phantom{00000000000000}}{} = \dfrac{\phantom{0000}}{} $
\end{enumerate}
\end{multicols}

\exo{Écrire sous la forme de la somme d'un nombre entier et d'une fraction inférieure à 1 puis donner l'écriture décimale}

\begin{multicols}{2}
\begin{enumerate}[itemsep=2em]
	\item $ \dfrac{17}{4} = \phantom{0000} + \dfrac{\phantom{00000000}}{} =  $
	\item $ \dfrac{19}{4} = \phantom{0000} + \dfrac{\phantom{00000000}}{} =  $
	\item $ \dfrac{9}{2} = \phantom{0000} + \dfrac{\phantom{00000000}}{} =  $
	\item $ \dfrac{25}{8} = \phantom{0000} + \dfrac{\phantom{00000000}}{} =  $
	\item $ \dfrac{13}{5} = \phantom{0000} + \dfrac{\phantom{00000000}}{} =  $
	\item $ \dfrac{11}{10} = \phantom{0000} + \dfrac{\phantom{00000000}}{} =  $
	\item $ \dfrac{13}{4} = \phantom{0000} + \dfrac{\phantom{00000000}}{} =  $
	\item $ \dfrac{13}{10} = \phantom{0000} + \dfrac{\phantom{00000000}}{} =  $
	\item $ \dfrac{3}{2} = \phantom{0000} + \dfrac{\phantom{00000000}}{} =  $
	\item $ \dfrac{22}{5} = \phantom{0000} + \dfrac{\phantom{00000000}}{} =  $
\end{enumerate}
\end{multicols}

\exo{Compléter}

\begin{multicols}{2}
\begin{enumerate}[itemsep=2em]
	\item $ 0,5~\text{hm} = \dotfill ~\text{m}$
	\item $ 6,52~\text{kL} = \dotfill ~\text{L}$
	\item $ 0,8~\text{dam} = \dotfill ~\text{m}$
	\item $ 12,8~\text{kL} = \dotfill ~\text{L}$
	\item $ 0,03~\text{kg} = \dotfill ~\text{g}$
	\item $ 2,52~\text{hg} = \dotfill ~\text{g}$
	\item $ 2,58~\text{km} = \dotfill ~\text{m}$
	\item $ 7,23~\text{dag} = \dotfill ~\text{g}$
	\item $ 0,09~\text{dag} = \dotfill ~\text{g}$
	\item $ 0,3~\text{dam} = \dotfill ~\text{m}$
\end{enumerate}
\end{multicols}

\exo{Compléter}

\begin{multicols}{2}
\begin{enumerate}[itemsep=2em]
	\item $ 3~\text{hm}^2 = \dotfill ~\text{m}^2$
	\item $ 300~\text{dam}^2 = \dotfill ~\text{m}^2$
	\item $ 900~\text{hm}^2 = \dotfill ~\text{m}^2$
	\item $ 91~\text{dam}^2 = \dotfill ~\text{m}^2$
	\item $ 1~\text{hm}^2 = \dotfill ~\text{m}^2$
	\item $ 30~\text{hm}^2 = \dotfill ~\text{m}^2$
	\item $ 9~\text{km}^2 = \dotfill ~\text{m}^2$
	\item $ 50~\text{km}^2 = \dotfill ~\text{m}^2$
	\item $ 8~\text{km}^2 = \dotfill ~\text{m}^2$
	\item $ 3~\text{km}^2 = \dotfill ~\text{m}^2$
\end{enumerate}
\end{multicols}

\exo{Pour chacune des figures, calculer son périmètre puis son aire (valeur exacte et si nécessaire valeur approchée au dixième près).}

\begin{multicols}{2}
\begin{enumerate}
	\item Un rectangle $TUVW$ de $10$ cm de longueur et de $5$ cm de largeur.
	\item Un rectangle $QRST$ de $7$ cm de longueur et de $3$ cm de largeur.
	\item Un carré $NOPQ$ de $4$ cm de côté .
	\item Un cercle de $10$ cm de rayon.
	\item Un cercle de $16$ cm de diamètre.
	\item Un cercle de $3$ cm de rayon.
	\item Un rectangle $CDEF$ tel que $CD = 11$ cm et $DE = 2$ cm.
	\item Un rectangle $EFGH$ de $10$ cm de longueur et de $7$ cm de largeur.
	\item Un triangle $STU$ rectangle en $T$ tel que $ST = 3$ cm, TU = 4 cm et TU = 5 cm.
	\item Un rectangle $EFGH$ tel que $EF = 7$ cm et $FG = 2$ cm.
\end{enumerate}
\end{multicols}


\newpage
\setcounter{section}{0}
\version{2}

\exo{Calculer}

\begin{multicols}{2}
\begin{enumerate}[itemsep=2em]
	\item $ 400 \times 2 = \dotfill $
	\item $ \nombre{8000} \times 4 = \dotfill $
	\item $ 300 \times 10 = \dotfill $
	\item $ \nombre{4000} \times \nombre{5000} = \dotfill $
	\item $ \nombre{4000} \times \nombre{1000} = \dotfill $
	\item $ \nombre{4000} \times 7 = \dotfill $
	\item $ 600 \times 300 = \dotfill $
	\item $ \nombre{6000} \times \nombre{5000} = \dotfill $
	\item $ \nombre{9000} \times 300 = \dotfill $
	\item $ \nombre{7000} \times 30 = \dotfill $
\end{enumerate}
\end{multicols}

\exo{Calculer}

\begin{multicols}{2}
\begin{enumerate}[itemsep=2em]
	\item $ (-2) + (+10) = \dotfill $
	\item $ (-12) + (+1) = \dotfill $
	\item $ (-7) + (-10) = \dotfill $
	\item $ (-3) + (+15) = \dotfill $
	\item $ (+9) + (-8) = \dotfill $
	\item $ (-19) + (+4) = \dotfill $
	\item $ (-1) + (-5) = \dotfill $
	\item $ (-4) + (+19) = \dotfill $
	\item $ (-11) + (+6) = \dotfill $
	\item $ (-18) + (-4) = \dotfill $
\end{enumerate}
\end{multicols}

\exo{Calculer}

\begin{multicols}{2}
\begin{enumerate}[itemsep=2em]
	\item $ 7-20 = \dotfill $
	\item $ -2-13 = \dotfill $
	\item $ 6-17 = \dotfill $
	\item $ 11-2 = \dotfill $
	\item $ 8-18 = \dotfill $
	\item $ -11+10 = \dotfill $
	\item $ 4-10 = \dotfill $
	\item $ 8-10 = \dotfill $
	\item $ -14-18 = \dotfill $
	\item $ -14+11 = \dotfill $
\end{enumerate}
\end{multicols}

\newpage
\exo{Compléter.}

\begin{multicols}{2}
\begin{enumerate}[itemsep=2em]
	\item $ \dfrac{8}{9} = \dfrac{\phantom{000000000000}}{\phantom{000000000000}} = \dfrac{}{99} $
	\item $ \dfrac{7}{9} = \dfrac{\phantom{000000000000}}{\phantom{000000000000}} = \dfrac{}{45} $
	\item $ \dfrac{4}{5} = \dfrac{\phantom{000000000000}}{\phantom{000000000000}} = \dfrac{}{50} $
	\item $ \dfrac{2}{9} = \dfrac{\phantom{000000000000}}{\phantom{000000000000}} = \dfrac{}{36} $
	\item $ \dfrac{1}{3} = \dfrac{\phantom{000000000000}}{\phantom{000000000000}} = \dfrac{}{15} $
	\item $ \dfrac{9}{10} = \dfrac{\phantom{000000000000}}{\phantom{000000000000}} = \dfrac{}{50} $
	\item $ \dfrac{1}{5} = \dfrac{\phantom{000000000000}}{\phantom{000000000000}} = \dfrac{}{35} $
	\item $ \dfrac{1}{7} = \dfrac{\phantom{000000000000}}{\phantom{000000000000}} = \dfrac{}{14} $
	\item $ \dfrac{4}{7} = \dfrac{\phantom{000000000000}}{\phantom{000000000000}} = \dfrac{}{56} $
	\item $ \dfrac{6}{7} = \dfrac{\phantom{000000000000}}{\phantom{000000000000}} = \dfrac{}{70} $
\end{enumerate}
\end{multicols}

\exo{Simplifier les fractions suivantes.}

\begin{multicols}{2}
\begin{enumerate}[itemsep=2em]
	\item $ \dfrac{2}{10} = \dfrac{\phantom{00000000000000}}{} = \dfrac{\phantom{0000}}{} $
	\item $ \dfrac{12}{28} = \dfrac{\phantom{00000000000000}}{} = \dfrac{\phantom{0000}}{} $
	\item $ \dfrac{6}{48} = \dfrac{\phantom{00000000000000}}{} = \dfrac{\phantom{0000}}{} $
	\item $ \dfrac{16}{72} = \dfrac{\phantom{00000000000000}}{} = \dfrac{\phantom{0000}}{} $
	\item $ \dfrac{10}{40} = \dfrac{\phantom{00000000000000}}{} = \dfrac{\phantom{0000}}{} $
	\item $ \dfrac{77}{88} = \dfrac{\phantom{00000000000000}}{} = \dfrac{\phantom{0000}}{} $
	\item $ \dfrac{5}{30} = \dfrac{\phantom{00000000000000}}{} = \dfrac{\phantom{0000}}{} $
	\item $ \dfrac{55}{77} = \dfrac{\phantom{00000000000000}}{} = \dfrac{\phantom{0000}}{} $
	\item $ \dfrac{10}{20} = \dfrac{\phantom{00000000000000}}{} = \dfrac{\phantom{0000}}{} $
	\item $ \dfrac{32}{56} = \dfrac{\phantom{00000000000000}}{} = \dfrac{\phantom{0000}}{} $
\end{enumerate}
\end{multicols}

\exo{Écrire sous la forme de la somme d'un nombre entier et d'une fraction inférieure à 1 puis donner l'écriture décimale}

\begin{multicols}{2}
\begin{enumerate}[itemsep=2em]
	\item $ \dfrac{17}{8} = \phantom{0000} + \dfrac{\phantom{00000000}}{} =  $
	\item $ \dfrac{3}{2} = \phantom{0000} + \dfrac{\phantom{00000000}}{} =  $
	\item $ \dfrac{14}{5} = \phantom{0000} + \dfrac{\phantom{00000000}}{} =  $
	\item $ \dfrac{5}{4} = \phantom{0000} + \dfrac{\phantom{00000000}}{} =  $
	\item $ \dfrac{7}{4} = \phantom{0000} + \dfrac{\phantom{00000000}}{} =  $
	\item $ \dfrac{43}{10} = \phantom{0000} + \dfrac{\phantom{00000000}}{} =  $
	\item $ \dfrac{15}{4} = \phantom{0000} + \dfrac{\phantom{00000000}}{} =  $
	\item $ \dfrac{11}{5} = \phantom{0000} + \dfrac{\phantom{00000000}}{} =  $
	\item $ \dfrac{47}{10} = \phantom{0000} + \dfrac{\phantom{00000000}}{} =  $
	\item $ \dfrac{5}{2} = \phantom{0000} + \dfrac{\phantom{00000000}}{} =  $
\end{enumerate}
\end{multicols}

\exo{Compléter}

\begin{multicols}{2}
\begin{enumerate}[itemsep=2em]
	\item $ 1,52~\text{daL} = \dotfill ~\text{L}$
	\item $ 3,67~\text{dam} = \dotfill ~\text{m}$
	\item $ 2,9~\text{hL} = \dotfill ~\text{L}$
	\item $ 0,09~\text{kL} = \dotfill ~\text{L}$
	\item $ 0,4~\text{dam} = \dotfill ~\text{m}$
	\item $ 0,7~\text{dam} = \dotfill ~\text{m}$
	\item $ 0,1~\text{k€} = \dotfill ~\text{€}$
	\item $ 18,5~\text{dag} = \dotfill ~\text{g}$
	\item $ 0,5~\text{dam} = \dotfill ~\text{m}$
	\item $ 9,3~\text{dag} = \dotfill ~\text{g}$
\end{enumerate}
\end{multicols}

\exo{Compléter}

\begin{multicols}{2}
\begin{enumerate}[itemsep=2em]
	\item $ 30~\text{km}^2 = \dotfill ~\text{m}^2$
	\item $ 5~\text{hm}^2 = \dotfill ~\text{m}^2$
	\item $ 8~\text{dam}^2 = \dotfill ~\text{m}^2$
	\item $ 10~\text{hm}^2 = \dotfill ~\text{m}^2$
	\item $ 9~\text{km}^2 = \dotfill ~\text{m}^2$
	\item $ 7~\text{hm}^2 = \dotfill ~\text{m}^2$
	\item $ 90~\text{dam}^2 = \dotfill ~\text{m}^2$
	\item $ 80~\text{hm}^2 = \dotfill ~\text{m}^2$
	\item $ 1~\text{dam}^2 = \dotfill ~\text{m}^2$
	\item $ 300~\text{dam}^2 = \dotfill ~\text{m}^2$
\end{enumerate}
\end{multicols}

\exo{Pour chacune des figures, calculer son périmètre puis son aire (valeur exacte et si nécessaire valeur approchée au dixième près).}

\begin{multicols}{2}
\begin{enumerate}
	\item Un rectangle $UVWX$ tel que $UV = 7$ cm et $VW = 3$ cm.
	\item Un carré $ABCD$ de $5$ cm de côté .
	\item Un cercle de $8$ cm de rayon.
	\item Un carré $CDEF$ de $9$ cm de côté .
	\item Un triangle $STU$ rectangle en $T$ tel que $ST = 24$ cm, TU = 10 cm et TU = 26 cm.
	\item Un carré $OPQR$ de $4$ cm de côté .
	\item Un triangle $TUV$ rectangle en $U$ tel que $TU = 21$ cm, UV = 20 cm et UV = 29 cm.
	\item Un rectangle $STUV$ de $11$ cm de longueur et de $5$ cm de largeur.
	\item Un cercle de $20$ cm de diamètre.
	\item Un carré $NOPQ$ tel que $NO = 4$ cm.
\end{enumerate}
\end{multicols}


\newpage
\setcounter{section}{0}
\version{3}

\exo{Calculer}

\begin{multicols}{2}
\begin{enumerate}[itemsep=2em]
	\item $ 800 \times 10 = \dotfill $
	\item $ 4 \times 80 = \dotfill $
	\item $ 300 \times 500 = \dotfill $
	\item $ 60 \times 800 = \dotfill $
	\item $ 9 \times 40 = \dotfill $
	\item $ 9 \times 8 = \dotfill $
	\item $ 3 \times \nombre{3000} = \dotfill $
	\item $ 7 \times 200 = \dotfill $
	\item $ 50 \times 4 = \dotfill $
	\item $ \nombre{2000} \times 2 = \dotfill $
\end{enumerate}
\end{multicols}

\exo{Calculer}

\begin{multicols}{2}
\begin{enumerate}[itemsep=2em]
	\item $ (-11) + (+2) = \dotfill $
	\item $ (+4) + (-6) = \dotfill $
	\item $ (-10) + (-11) = \dotfill $
	\item $ (-13) + (-18) = \dotfill $
	\item $ (-20) + (-10) = \dotfill $
	\item $ (-4) + (+1) = \dotfill $
	\item $ (-6) + (-10) = \dotfill $
	\item $ (-19) + (+20) = \dotfill $
	\item $ (-4) + (-17) = \dotfill $
	\item $ (-10) + (+2) = \dotfill $
\end{enumerate}
\end{multicols}

\exo{Calculer}

\begin{multicols}{2}
\begin{enumerate}[itemsep=2em]
	\item $ -13+17 = \dotfill $
	\item $ -20+7 = \dotfill $
	\item $ -5+4 = \dotfill $
	\item $ 9-10 = \dotfill $
	\item $ 6-13 = \dotfill $
	\item $ -17-12 = \dotfill $
	\item $ 19-11 = \dotfill $
	\item $ -3-10 = \dotfill $
	\item $ -14-2 = \dotfill $
	\item $ -9-20 = \dotfill $
\end{enumerate}
\end{multicols}

\exo{Compléter.}

\begin{multicols}{2}
\begin{enumerate}[itemsep=2em]
	\item $ \dfrac{1}{4} = \dfrac{\phantom{000000000000}}{\phantom{000000000000}} = \dfrac{}{32} $
	\item $ \dfrac{9}{10} = \dfrac{\phantom{000000000000}}{\phantom{000000000000}} = \dfrac{}{90} $
	\item $ \dfrac{1}{5} = \dfrac{\phantom{000000000000}}{\phantom{000000000000}} = \dfrac{}{30} $
	\item $ \dfrac{1}{2} = \dfrac{\phantom{000000000000}}{\phantom{000000000000}} = \dfrac{}{10} $
	\item $ \dfrac{4}{7} = \dfrac{\phantom{000000000000}}{\phantom{000000000000}} = \dfrac{}{42} $
	\item $ \dfrac{1}{6} = \dfrac{\phantom{000000000000}}{\phantom{000000000000}} = \dfrac{}{30} $
	\item $ \dfrac{3}{4} = \dfrac{\phantom{000000000000}}{\phantom{000000000000}} = \dfrac{}{20} $
	\item $ \dfrac{7}{10} = \dfrac{\phantom{000000000000}}{\phantom{000000000000}} = \dfrac{}{80} $
	\item $ \dfrac{2}{9} = \dfrac{\phantom{000000000000}}{\phantom{000000000000}} = \dfrac{}{36} $
	\item $ \dfrac{7}{8} = \dfrac{\phantom{000000000000}}{\phantom{000000000000}} = \dfrac{}{80} $
\end{enumerate}
\end{multicols}

\exo{Simplifier les fractions suivantes.}

\begin{multicols}{2}
\begin{enumerate}[itemsep=2em]
	\item $ \dfrac{21}{56} = \dfrac{\phantom{00000000000000}}{} = \dfrac{\phantom{0000}}{} $
	\item $ \dfrac{6}{14} = \dfrac{\phantom{00000000000000}}{} = \dfrac{\phantom{0000}}{} $
	\item $ \dfrac{27}{30} = \dfrac{\phantom{00000000000000}}{} = \dfrac{\phantom{0000}}{} $
	\item $ \dfrac{4}{10} = \dfrac{\phantom{00000000000000}}{} = \dfrac{\phantom{0000}}{} $
	\item $ \dfrac{48}{54} = \dfrac{\phantom{00000000000000}}{} = \dfrac{\phantom{0000}}{} $
	\item $ \dfrac{56}{72} = \dfrac{\phantom{00000000000000}}{} = \dfrac{\phantom{0000}}{} $
	\item $ \dfrac{9}{36} = \dfrac{\phantom{00000000000000}}{} = \dfrac{\phantom{0000}}{} $
	\item $ \dfrac{9}{15} = \dfrac{\phantom{00000000000000}}{} = \dfrac{\phantom{0000}}{} $
	\item $ \dfrac{56}{80} = \dfrac{\phantom{00000000000000}}{} = \dfrac{\phantom{0000}}{} $
	\item $ \dfrac{32}{40} = \dfrac{\phantom{00000000000000}}{} = \dfrac{\phantom{0000}}{} $
\end{enumerate}
\end{multicols}

\exo{Écrire sous la forme de la somme d'un nombre entier et d'une fraction inférieure à 1 puis donner l'écriture décimale}

\begin{multicols}{2}
\begin{enumerate}[itemsep=2em]
	\item $ \dfrac{7}{2} = \phantom{0000} + \dfrac{\phantom{00000000}}{} =  $
	\item $ \dfrac{25}{8} = \phantom{0000} + \dfrac{\phantom{00000000}}{} =  $
	\item $ \dfrac{21}{10} = \phantom{0000} + \dfrac{\phantom{00000000}}{} =  $
	\item $ \dfrac{9}{4} = \phantom{0000} + \dfrac{\phantom{00000000}}{} =  $
	\item $ \dfrac{15}{4} = \phantom{0000} + \dfrac{\phantom{00000000}}{} =  $
	\item $ \dfrac{12}{5} = \phantom{0000} + \dfrac{\phantom{00000000}}{} =  $
	\item $ \dfrac{43}{10} = \phantom{0000} + \dfrac{\phantom{00000000}}{} =  $
	\item $ \dfrac{39}{10} = \phantom{0000} + \dfrac{\phantom{00000000}}{} =  $
	\item $ \dfrac{41}{10} = \phantom{0000} + \dfrac{\phantom{00000000}}{} =  $
	\item $ \dfrac{27}{8} = \phantom{0000} + \dfrac{\phantom{00000000}}{} =  $
\end{enumerate}
\end{multicols}

\exo{Compléter}

\begin{multicols}{2}
\begin{enumerate}[itemsep=2em]
	\item $ 0,07~\text{hg} = \dotfill ~\text{g}$
	\item $ 5,16~\text{hg} = \dotfill ~\text{g}$
	\item $ 0,7~\text{dag} = \dotfill ~\text{g}$
	\item $ 0,09~\text{hm} = \dotfill ~\text{m}$
	\item $ 4,67~\text{kL} = \dotfill ~\text{L}$
	\item $ 1,8~\text{hg} = \dotfill ~\text{g}$
	\item $ 1,41~\text{daL} = \dotfill ~\text{L}$
	\item $ 12,4~\text{kL} = \dotfill ~\text{L}$
	\item $ 2,62~\text{km} = \dotfill ~\text{m}$
	\item $ 0,2~\text{km} = \dotfill ~\text{m}$
\end{enumerate}
\end{multicols}

\exo{Compléter}

\begin{multicols}{2}
\begin{enumerate}[itemsep=2em]
	\item $ 34~\text{hm}^2 = \dotfill ~\text{m}^2$
	\item $ 3~\text{dam}^2 = \dotfill ~\text{m}^2$
	\item $ 43~\text{dam}^2 = \dotfill ~\text{m}^2$
	\item $ 10~\text{km}^2 = \dotfill ~\text{m}^2$
	\item $ 74~\text{dam}^2 = \dotfill ~\text{m}^2$
	\item $ 11~\text{dam}^2 = \dotfill ~\text{m}^2$
	\item $ 8~\text{km}^2 = \dotfill ~\text{m}^2$
	\item $ 2~\text{dam}^2 = \dotfill ~\text{m}^2$
	\item $ 80~\text{hm}^2 = \dotfill ~\text{m}^2$
	\item $ 40~\text{hm}^2 = \dotfill ~\text{m}^2$
\end{enumerate}
\end{multicols}

\exo{Pour chacune des figures, calculer son périmètre puis son aire (valeur exacte et si nécessaire valeur approchée au dixième près).}

\begin{multicols}{2}
\begin{enumerate}
	\item Un triangle $STU$ rectangle en $T$ tel que $ST = 15$ cm, TU = 8 cm et TU = 17 cm.
	\item Un rectangle $DEFG$ tel que $DE = 6$ cm et $EF = 5$ cm.
	\item Un triangle rectangle $KLM$ a pour côtés : $21$ cm, $29$ cm et $20$ cm.
	\item Un carré $MNOP$ tel que $MN = 10$ cm.
	\item Un cercle de $6$ cm de diamètre.
	\item Un rectangle $KLMN$ de $6$ cm de longueur et de $3$ cm de largeur.
	\item Un carré $KLMN$ de $2$ cm de côté .
	\item Un carré $NOPQ$ de $5$ cm de côté .
	\item Un rectangle $PQRS$ tel que $PQ = 4$ cm et $QR = 2$ cm.
	\item Un cercle de $16$ cm de diamètre.
\end{enumerate}
\end{multicols}



\newpage
\version{1}
\begin{correction}\exo{}

\begin{multicols}{2}
\begin{enumerate}
	\item $ 7 \times 10 = 70 $
	\item $ 7 \times 7 = 49 $
	\item $ 80 \times 40 = \nombre{3200} $
	\item $ 70 \times \nombre{3000} = \nombre{210000} $
	\item $ 3 \times 2 = 6 $
	\item $ 20 \times \nombre{6000} = \nombre{120000} $
	\item $ 700 \times 900 = \nombre{630000} $
	\item $ 900 \times \nombre{9000} = \nombre{8100000} $
	\item $ 900 \times 6 = \nombre{5400} $
	\item $ 50 \times 800 = \nombre{40000} $
\end{enumerate}
\end{multicols}

\exo{}

\begin{multicols}{2}
\begin{enumerate}
	\item $ (+19) + (-13) = (+6) $
	\item $ (-9) + (+2) = (-7) $
	\item $ (-11) + (-1) = (-12) $
	\item $ (-5) + (-18) = (-23) $
	\item $ (+10) + (-8) = (+2) $
	\item $ (-2) + (+5) = (+3) $
	\item $ (-7) + (+4) = (-3) $
	\item $ (-2) + (+18) = (+16) $
	\item $ (-6) + (+14) = (+8) $
	\item $ (+5) + (-12) = (-7) $
\end{enumerate}
\end{multicols}

\exo{}

\begin{multicols}{2}
\begin{enumerate}
	\item $ -13+8 = -5 $
	\item $ 15-19 = -4 $
	\item $ 16-10 = 6 $
	\item $ -17-8 = -25 $
	\item $ -16+14 = -2 $
	\item $ -6-19 = -25 $
	\item $ -14+11 = -3 $
	\item $ 3-20 = -17 $
	\item $ -10+12 = 2 $
	\item $ 7-16 = -9 $
\end{enumerate}
\end{multicols}

\exo{}

\begin{multicols}{2}
\begin{enumerate}[itemsep=2em]
	\item $ \dfrac{3}{5} = \dfrac{9 \times 3}{9 \times 5} = \dfrac{27}{45} $
	\item $ \dfrac{1}{2} = \dfrac{11 \times 1}{11 \times 2} = \dfrac{11}{22} $
	\item $ \dfrac{9}{10} = \dfrac{10 \times 9}{10 \times 10} = \dfrac{90}{100} $
	\item $ \dfrac{1}{5} = \dfrac{10 \times 1}{10 \times 5} = \dfrac{10}{50} $
	\item $ \dfrac{3}{10} = \dfrac{8 \times 3}{8 \times 10} = \dfrac{24}{80} $
	\item $ \dfrac{3}{8} = \dfrac{8 \times 3}{8 \times 8} = \dfrac{24}{64} $
	\item $ \dfrac{1}{6} = \dfrac{3 \times 1}{3 \times 6} = \dfrac{3}{18} $
	\item $ \dfrac{2}{5} = \dfrac{8 \times 2}{8 \times 5} = \dfrac{16}{40} $
	\item $ \dfrac{2}{7} = \dfrac{9 \times 2}{9 \times 7} = \dfrac{18}{63} $
	\item $ \dfrac{1}{8} = \dfrac{6 \times 1}{6 \times 8} = \dfrac{6}{48} $
\end{enumerate}
\end{multicols}

\exo{}

\begin{multicols}{2}
\begin{enumerate}[itemsep=2em]
	\item $ \dfrac{40}{70} = \dfrac{10 \times 4}{10 \times 7} = \dfrac{4}{7} $
	\item $ \dfrac{6}{12} = \dfrac{6 \times 1}{6 \times 2} = \dfrac{1}{2} $
	\item $ \dfrac{30}{36} = \dfrac{6 \times 5}{6 \times 6} = \dfrac{5}{6} $
	\item $ \dfrac{40}{56} = \dfrac{8 \times 5}{8 \times 7} = \dfrac{5}{7} $
	\item $ \dfrac{33}{77} = \dfrac{11 \times 3}{11 \times 7} = \dfrac{3}{7} $
	\item $ \dfrac{6}{27} = \dfrac{3 \times 2}{3 \times 9} = \dfrac{2}{9} $
	\item $ \dfrac{6}{21} = \dfrac{3 \times 2}{3 \times 7} = \dfrac{2}{7} $
	\item $ \dfrac{25}{40} = \dfrac{5 \times 5}{5 \times 8} = \dfrac{5}{8} $
	\item $ \dfrac{77}{110} = \dfrac{11 \times 7}{11 \times 10} = \dfrac{7}{10} $
	\item $ \dfrac{44}{99} = \dfrac{11 \times 4}{11 \times 9} = \dfrac{4}{9} $
\end{enumerate}
\end{multicols}

\exo{}

\begin{multicols}{2}
\begin{enumerate}[itemsep=2em]
	\item $ \dfrac{17}{4} = 4+\dfrac{1}{4} = 4,25 $
	\item $ \dfrac{19}{4} = 4+\dfrac{3}{4} = 4,75 $
	\item $ \dfrac{9}{2} = 4+\dfrac{1}{2} = 4,5 $
	\item $ \dfrac{25}{8} = 3+\dfrac{1}{8} = 3,125 $
	\item $ \dfrac{13}{5} = 2+\dfrac{3}{5} = 2,6 $
	\item $ \dfrac{11}{10} = 1+\dfrac{1}{10} = 1,1 $
	\item $ \dfrac{13}{4} = 3+\dfrac{1}{4} = 3,25 $
	\item $ \dfrac{13}{10} = 1+\dfrac{3}{10} = 1,3 $
	\item $ \dfrac{3}{2} = 1+\dfrac{1}{2} = 1,5 $
	\item $ \dfrac{22}{5} = 4+\dfrac{2}{5} = 4,4 $
\end{enumerate}
\end{multicols}

\exo{}

\begin{multicols}{2}
\begin{enumerate}
	\item $ 0,5~\text{hm} =  0,5\times100~\text{m} = 50~\text{m}$
	\item $ 6,52~\text{kL} =  6,52\times\nombre{1000}~\text{L} = \nombre{6520}~\text{L}$
	\item $ 0,8~\text{dam} =  0,8\times10~\text{m} = 8~\text{m}$
	\item $ 12,8~\text{kL} =  12,8\times\nombre{1000}~\text{L} = \nombre{12800}~\text{L}$
	\item $ 0,03~\text{kg} =  0,03\times\nombre{1000}~\text{g} = 30~\text{g}$
	\item $ 2,52~\text{hg} =  2,52\times100~\text{g} = 252~\text{g}$
	\item $ 2,58~\text{km} =  2,58\times\nombre{1000}~\text{m} = \nombre{2580}~\text{m}$
	\item $ 7,23~\text{dag} =  7,23\times10~\text{g} = 72,3~\text{g}$
	\item $ 0,09~\text{dag} =  0,09\times10~\text{g} = 0,9~\text{g}$
	\item $ 0,3~\text{dam} =  0,3\times10~\text{m} = 3~\text{m}$
\end{enumerate}
\end{multicols}

\exo{}

\begin{multicols}{2}
\begin{enumerate}
	\item $ 3~\text{hm}^2 =  3\times\nombre{10000}~\text{m}^2 = \nombre{30000}~\text{m}^2$
	\item $ 300~\text{dam}^2 =  300\times100~\text{m}^2 = \nombre{30000}~\text{m}^2$
	\item $ 900~\text{hm}^2 =  900\times\nombre{10000}~\text{m}^2 = \nombre{9000000}~\text{m}^2$
	\item $ 91~\text{dam}^2 =  91\times100~\text{m}^2 = \nombre{9100}~\text{m}^2$
	\item $ 1~\text{hm}^2 =  1\times\nombre{10000}~\text{m}^2 = \nombre{10000}~\text{m}^2$
	\item $ 30~\text{hm}^2 =  30\times\nombre{10000}~\text{m}^2 = \nombre{300000}~\text{m}^2$
	\item $ 9~\text{km}^2 =  9\times\nombre{1000000}~\text{m}^2 = \nombre{9000000}~\text{m}^2$
	\item $ 50~\text{km}^2 =  50\times\nombre{1000000}~\text{m}^2 = \nombre{50000000}~\text{m}^2$
	\item $ 8~\text{km}^2 =  8\times\nombre{1000000}~\text{m}^2 = \nombre{8000000}~\text{m}^2$
	\item $ 3~\text{km}^2 =  3\times\nombre{1000000}~\text{m}^2 = \nombre{3000000}~\text{m}^2$
\end{enumerate}
\end{multicols}

\exo{}

\begin{multicols}{2}
\begin{enumerate}
	\item $\mathcal{P}_{TUVW}=(10~\text{cm}+5~\text{cm})\times2=30~\text{cm}$\\
$\mathcal{A}_{TUVW}=10~\text{cm}\times5~\text{cm}=50~\text{cm}^2$
	\item $\mathcal{P}_{QRST}=(7~\text{cm}+3~\text{cm})\times2=20~\text{cm}$\\
$\mathcal{A}_{QRST}=7~\text{cm}\times3~\text{cm}=21~\text{cm}^2$
	\item $\mathcal{P}_{NOPQ}=4\times4~\text{cm}=16~\text{cm}$\\
$\mathcal{A}_{NOPQ}=4~\text{cm}\times4~\text{cm}=16~\text{cm}^2$
	\item $\mathcal{P}=2\times10\times\pi~\text{cm}=20\pi~\text{cm}\approx62,8~\text{cm}$\\
$\mathcal{A}=10\times10\times\pi~\text{cm}^2=100\pi~\text{cm}^2\approx314,2~\text{cm}^2$
	\item Le diamètre est de $16$ cm donc le rayon est de $8$ cm.\\
$\mathcal{P}=2\times8\times\pi~\text{cm}=16\pi~\text{cm}\approx50,3~\text{cm}$\\
$\mathcal{A}=8\times8\times\pi~\text{cm}^2=64\pi~\text{cm}^2\approx201,1~\text{cm}^2$
	\item $\mathcal{P}=2\times3\times\pi~\text{cm}=6\pi~\text{cm}\approx18,8~\text{cm}$\\
$\mathcal{A}=3\times3\times\pi~\text{cm}^2=9\pi~\text{cm}^2\approx28,3~\text{cm}^2$
	\item $\mathcal{P}_{CDEF}=(11~\text{cm}+2~\text{cm})\times2=26~\text{cm}$\\
$\mathcal{A}_{CDEF}=11~\text{cm}\times2~\text{cm}=22~\text{cm}^2$
	\item $\mathcal{P}_{EFGH}=(10~\text{cm}+7~\text{cm})\times2=34~\text{cm}$\\
$\mathcal{A}_{EFGH}=10~\text{cm}\times7~\text{cm}=70~\text{cm}^2$
	\item $\mathcal{P}_{STU}=3~\text{cm}+4~\text{cm}+5~\text{cm}=12~\text{cm}$\\
$\mathcal{A}_{STU}=3~\text{cm}\times4~\text{cm}\div2=6~\text{cm}^2$
	\item $\mathcal{P}_{EFGH}=(7~\text{cm}+2~\text{cm})\times2=18~\text{cm}$\\
$\mathcal{A}_{EFGH}=7~\text{cm}\times2~\text{cm}=14~\text{cm}^2$
\end{enumerate}
\end{multicols}


\end{correction}

\newpage
\version{2}
\begin{correction}\exo{}

\begin{multicols}{2}
\begin{enumerate}
	\item $ 400 \times 2 = 800 $
	\item $ \nombre{8000} \times 4 = \nombre{32000} $
	\item $ 300 \times 10 = \nombre{3000} $
	\item $ \nombre{4000} \times \nombre{5000} = \nombre{20000000} $
	\item $ \nombre{4000} \times \nombre{1000} = \nombre{4000000} $
	\item $ \nombre{4000} \times 7 = \nombre{28000} $
	\item $ 600 \times 300 = \nombre{180000} $
	\item $ \nombre{6000} \times \nombre{5000} = \nombre{30000000} $
	\item $ \nombre{9000} \times 300 = \nombre{2700000} $
	\item $ \nombre{7000} \times 30 = \nombre{210000} $
\end{enumerate}
\end{multicols}

\exo{}

\begin{multicols}{2}
\begin{enumerate}
	\item $ (-2) + (+10) = (+8) $
	\item $ (-12) + (+1) = (-11) $
	\item $ (-7) + (-10) = (-17) $
	\item $ (-3) + (+15) = (+12) $
	\item $ (+9) + (-8) = (+1) $
	\item $ (-19) + (+4) = (-15) $
	\item $ (-1) + (-5) = (-6) $
	\item $ (-4) + (+19) = (+15) $
	\item $ (-11) + (+6) = (-5) $
	\item $ (-18) + (-4) = (-22) $
\end{enumerate}
\end{multicols}

\exo{}

\begin{multicols}{2}
\begin{enumerate}
	\item $ 7-20 = -13 $
	\item $ -2-13 = -15 $
	\item $ 6-17 = -11 $
	\item $ 11-2 = 9 $
	\item $ 8-18 = -10 $
	\item $ -11+10 = -1 $
	\item $ 4-10 = -6 $
	\item $ 8-10 = -2 $
	\item $ -14-18 = -32 $
	\item $ -14+11 = -3 $
\end{enumerate}
\end{multicols}

\exo{}

\begin{multicols}{2}
\begin{enumerate}[itemsep=2em]
	\item $ \dfrac{8}{9} = \dfrac{11 \times 8}{11 \times 9} = \dfrac{88}{99} $
	\item $ \dfrac{7}{9} = \dfrac{5 \times 7}{5 \times 9} = \dfrac{35}{45} $
	\item $ \dfrac{4}{5} = \dfrac{10 \times 4}{10 \times 5} = \dfrac{40}{50} $
	\item $ \dfrac{2}{9} = \dfrac{4 \times 2}{4 \times 9} = \dfrac{8}{36} $
	\item $ \dfrac{1}{3} = \dfrac{5 \times 1}{5 \times 3} = \dfrac{5}{15} $
	\item $ \dfrac{9}{10} = \dfrac{5 \times 9}{5 \times 10} = \dfrac{45}{50} $
	\item $ \dfrac{1}{5} = \dfrac{7 \times 1}{7 \times 5} = \dfrac{7}{35} $
	\item $ \dfrac{1}{7} = \dfrac{2 \times 1}{2 \times 7} = \dfrac{2}{14} $
	\item $ \dfrac{4}{7} = \dfrac{8 \times 4}{8 \times 7} = \dfrac{32}{56} $
	\item $ \dfrac{6}{7} = \dfrac{10 \times 6}{10 \times 7} = \dfrac{60}{70} $
\end{enumerate}
\end{multicols}

\exo{}

\begin{multicols}{2}
\begin{enumerate}[itemsep=2em]
	\item $ \dfrac{2}{10} = \dfrac{2 \times 1}{2 \times 5} = \dfrac{1}{5} $
	\item $ \dfrac{12}{28} = \dfrac{4 \times 3}{4 \times 7} = \dfrac{3}{7} $
	\item $ \dfrac{6}{48} = \dfrac{6 \times 1}{6 \times 8} = \dfrac{1}{8} $
	\item $ \dfrac{16}{72} = \dfrac{8 \times 2}{8 \times 9} = \dfrac{2}{9} $
	\item $ \dfrac{10}{40} = \dfrac{10 \times 1}{10 \times 4} = \dfrac{1}{4} $
	\item $ \dfrac{77}{88} = \dfrac{11 \times 7}{11 \times 8} = \dfrac{7}{8} $
	\item $ \dfrac{5}{30} = \dfrac{5 \times 1}{5 \times 6} = \dfrac{1}{6} $
	\item $ \dfrac{55}{77} = \dfrac{11 \times 5}{11 \times 7} = \dfrac{5}{7} $
	\item $ \dfrac{10}{20} = \dfrac{10 \times 1}{10 \times 2} = \dfrac{1}{2} $
	\item $ \dfrac{32}{56} = \dfrac{8 \times 4}{8 \times 7} = \dfrac{4}{7} $
\end{enumerate}
\end{multicols}

\exo{}

\begin{multicols}{2}
\begin{enumerate}[itemsep=2em]
	\item $ \dfrac{17}{8} = 2+\dfrac{1}{8} = 2,125 $
	\item $ \dfrac{3}{2} = 1+\dfrac{1}{2} = 1,5 $
	\item $ \dfrac{14}{5} = 2+\dfrac{4}{5} = 2,8 $
	\item $ \dfrac{5}{4} = 1+\dfrac{1}{4} = 1,25 $
	\item $ \dfrac{7}{4} = 1+\dfrac{3}{4} = 1,75 $
	\item $ \dfrac{43}{10} = 4+\dfrac{3}{10} = 4,3 $
	\item $ \dfrac{15}{4} = 3+\dfrac{3}{4} = 3,75 $
	\item $ \dfrac{11}{5} = 2+\dfrac{1}{5} = 2,2 $
	\item $ \dfrac{47}{10} = 4+\dfrac{7}{10} = 4,7 $
	\item $ \dfrac{5}{2} = 2+\dfrac{1}{2} = 2,5 $
\end{enumerate}
\end{multicols}

\exo{}

\begin{multicols}{2}
\begin{enumerate}
	\item $ 1,52~\text{daL} =  1,52\times10~\text{L} = 15,2~\text{L}$
	\item $ 3,67~\text{dam} =  3,67\times10~\text{m} = 36,7~\text{m}$
	\item $ 2,9~\text{hL} =  2,9\times100~\text{L} = 290~\text{L}$
	\item $ 0,09~\text{kL} =  0,09\times\nombre{1000}~\text{L} = 90~\text{L}$
	\item $ 0,4~\text{dam} =  0,4\times10~\text{m} = 4~\text{m}$
	\item $ 0,7~\text{dam} =  0,7\times10~\text{m} = 7~\text{m}$
	\item $ 0,1~\text{k€} =  0,1\times\nombre{1000}~\text{€} = 100~\text{€}$
	\item $ 18,5~\text{dag} =  18,5\times10~\text{g} = 185~\text{g}$
	\item $ 0,5~\text{dam} =  0,5\times10~\text{m} = 5~\text{m}$
	\item $ 9,3~\text{dag} =  9,3\times10~\text{g} = 93~\text{g}$
\end{enumerate}
\end{multicols}

\exo{}

\begin{multicols}{2}
\begin{enumerate}
	\item $ 30~\text{km}^2 =  30\times\nombre{1000000}~\text{m}^2 = \nombre{30000000}~\text{m}^2$
	\item $ 5~\text{hm}^2 =  5\times\nombre{10000}~\text{m}^2 = \nombre{50000}~\text{m}^2$
	\item $ 8~\text{dam}^2 =  8\times100~\text{m}^2 = 800~\text{m}^2$
	\item $ 10~\text{hm}^2 =  10\times\nombre{10000}~\text{m}^2 = \nombre{100000}~\text{m}^2$
	\item $ 9~\text{km}^2 =  9\times\nombre{1000000}~\text{m}^2 = \nombre{9000000}~\text{m}^2$
	\item $ 7~\text{hm}^2 =  7\times\nombre{10000}~\text{m}^2 = \nombre{70000}~\text{m}^2$
	\item $ 90~\text{dam}^2 =  90\times100~\text{m}^2 = \nombre{9000}~\text{m}^2$
	\item $ 80~\text{hm}^2 =  80\times\nombre{10000}~\text{m}^2 = \nombre{800000}~\text{m}^2$
	\item $ 1~\text{dam}^2 =  1\times100~\text{m}^2 = 100~\text{m}^2$
	\item $ 300~\text{dam}^2 =  300\times100~\text{m}^2 = \nombre{30000}~\text{m}^2$
\end{enumerate}
\end{multicols}

\exo{}

\begin{multicols}{2}
\begin{enumerate}
	\item $\mathcal{P}_{UVWX}=(7~\text{cm}+3~\text{cm})\times2=20~\text{cm}$\\
$\mathcal{A}_{UVWX}=7~\text{cm}\times3~\text{cm}=21~\text{cm}^2$
	\item $\mathcal{P}_{ABCD}=4\times5~\text{cm}=20~\text{cm}$\\
$\mathcal{A}_{ABCD}=5~\text{cm}\times5~\text{cm}=25~\text{cm}^2$
	\item $\mathcal{P}=2\times8\times\pi~\text{cm}=16\pi~\text{cm}\approx50,3~\text{cm}$\\
$\mathcal{A}=8\times8\times\pi~\text{cm}^2=64\pi~\text{cm}^2\approx201,1~\text{cm}^2$
	\item $\mathcal{P}_{CDEF}=4\times9~\text{cm}=36~\text{cm}$\\
$\mathcal{A}_{CDEF}=9~\text{cm}\times9~\text{cm}=81~\text{cm}^2$
	\item $\mathcal{P}_{STU}=24~\text{cm}+10~\text{cm}+26~\text{cm}=60~\text{cm}$\\
$\mathcal{A}_{STU}=24~\text{cm}\times10~\text{cm}\div2=120~\text{cm}^2$
	\item $\mathcal{P}_{OPQR}=4\times4~\text{cm}=16~\text{cm}$\\
$\mathcal{A}_{OPQR}=4~\text{cm}\times4~\text{cm}=16~\text{cm}^2$
	\item $\mathcal{P}_{TUV}=21~\text{cm}+20~\text{cm}+29~\text{cm}=70~\text{cm}$\\
$\mathcal{A}_{TUV}=21~\text{cm}\times20~\text{cm}\div2=210~\text{cm}^2$
	\item $\mathcal{P}_{STUV}=(11~\text{cm}+5~\text{cm})\times2=32~\text{cm}$\\
$\mathcal{A}_{STUV}=11~\text{cm}\times5~\text{cm}=55~\text{cm}^2$
	\item Le diamètre est de $20$ cm donc le rayon est de $10$ cm.\\
$\mathcal{P}=2\times10\times\pi~\text{cm}=20\pi~\text{cm}\approx62,8~\text{cm}$\\
$\mathcal{A}=10\times10\times\pi~\text{cm}^2=100\pi~\text{cm}^2\approx314,2~\text{cm}^2$
	\item $\mathcal{P}_{NOPQ}=4\times4~\text{cm}=16~\text{cm}$\\
$\mathcal{A}_{NOPQ}=4~\text{cm}\times4~\text{cm}=16~\text{cm}^2$
\end{enumerate}
\end{multicols}


\end{correction}

\newpage
\version{3}
\begin{correction}\exo{}

\begin{multicols}{2}
\begin{enumerate}
	\item $ 800 \times 10 = \nombre{8000} $
	\item $ 4 \times 80 = 320 $
	\item $ 300 \times 500 = \nombre{150000} $
	\item $ 60 \times 800 = \nombre{48000} $
	\item $ 9 \times 40 = 360 $
	\item $ 9 \times 8 = 72 $
	\item $ 3 \times \nombre{3000} = \nombre{9000} $
	\item $ 7 \times 200 = \nombre{1400} $
	\item $ 50 \times 4 = 200 $
	\item $ \nombre{2000} \times 2 = \nombre{4000} $
\end{enumerate}
\end{multicols}

\exo{}

\begin{multicols}{2}
\begin{enumerate}
	\item $ (-11) + (+2) = (-9) $
	\item $ (+4) + (-6) = (-2) $
	\item $ (-10) + (-11) = (-21) $
	\item $ (-13) + (-18) = (-31) $
	\item $ (-20) + (-10) = (-30) $
	\item $ (-4) + (+1) = (-3) $
	\item $ (-6) + (-10) = (-16) $
	\item $ (-19) + (+20) = (+1) $
	\item $ (-4) + (-17) = (-21) $
	\item $ (-10) + (+2) = (-8) $
\end{enumerate}
\end{multicols}

\exo{}

\begin{multicols}{2}
\begin{enumerate}
	\item $ -13+17 = 4 $
	\item $ -20+7 = -13 $
	\item $ -5+4 = -1 $
	\item $ 9-10 = -1 $
	\item $ 6-13 = -7 $
	\item $ -17-12 = -29 $
	\item $ 19-11 = 8 $
	\item $ -3-10 = -13 $
	\item $ -14-2 = -16 $
	\item $ -9-20 = -29 $
\end{enumerate}
\end{multicols}

\exo{}

\begin{multicols}{2}
\begin{enumerate}[itemsep=2em]
	\item $ \dfrac{1}{4} = \dfrac{8 \times 1}{8 \times 4} = \dfrac{8}{32} $
	\item $ \dfrac{9}{10} = \dfrac{9 \times 9}{9 \times 10} = \dfrac{81}{90} $
	\item $ \dfrac{1}{5} = \dfrac{6 \times 1}{6 \times 5} = \dfrac{6}{30} $
	\item $ \dfrac{1}{2} = \dfrac{5 \times 1}{5 \times 2} = \dfrac{5}{10} $
	\item $ \dfrac{4}{7} = \dfrac{6 \times 4}{6 \times 7} = \dfrac{24}{42} $
	\item $ \dfrac{1}{6} = \dfrac{5 \times 1}{5 \times 6} = \dfrac{5}{30} $
	\item $ \dfrac{3}{4} = \dfrac{5 \times 3}{5 \times 4} = \dfrac{15}{20} $
	\item $ \dfrac{7}{10} = \dfrac{8 \times 7}{8 \times 10} = \dfrac{56}{80} $
	\item $ \dfrac{2}{9} = \dfrac{4 \times 2}{4 \times 9} = \dfrac{8}{36} $
	\item $ \dfrac{7}{8} = \dfrac{10 \times 7}{10 \times 8} = \dfrac{70}{80} $
\end{enumerate}
\end{multicols}

\exo{}

\begin{multicols}{2}
\begin{enumerate}[itemsep=2em]
	\item $ \dfrac{21}{56} = \dfrac{7 \times 3}{7 \times 8} = \dfrac{3}{8} $
	\item $ \dfrac{6}{14} = \dfrac{2 \times 3}{2 \times 7} = \dfrac{3}{7} $
	\item $ \dfrac{27}{30} = \dfrac{3 \times 9}{3 \times 10} = \dfrac{9}{10} $
	\item $ \dfrac{4}{10} = \dfrac{2 \times 2}{2 \times 5} = \dfrac{2}{5} $
	\item $ \dfrac{48}{54} = \dfrac{6 \times 8}{6 \times 9} = \dfrac{8}{9} $
	\item $ \dfrac{56}{72} = \dfrac{8 \times 7}{8 \times 9} = \dfrac{7}{9} $
	\item $ \dfrac{9}{36} = \dfrac{9 \times 1}{9 \times 4} = \dfrac{1}{4} $
	\item $ \dfrac{9}{15} = \dfrac{3 \times 3}{3 \times 5} = \dfrac{3}{5} $
	\item $ \dfrac{56}{80} = \dfrac{8 \times 7}{8 \times 10} = \dfrac{7}{10} $
	\item $ \dfrac{32}{40} = \dfrac{8 \times 4}{8 \times 5} = \dfrac{4}{5} $
\end{enumerate}
\end{multicols}

\exo{}

\begin{multicols}{2}
\begin{enumerate}[itemsep=2em]
	\item $ \dfrac{7}{2} = 3+\dfrac{1}{2} = 3,5 $
	\item $ \dfrac{25}{8} = 3+\dfrac{1}{8} = 3,125 $
	\item $ \dfrac{21}{10} = 2+\dfrac{1}{10} = 2,1 $
	\item $ \dfrac{9}{4} = 2+\dfrac{1}{4} = 2,25 $
	\item $ \dfrac{15}{4} = 3+\dfrac{3}{4} = 3,75 $
	\item $ \dfrac{12}{5} = 2+\dfrac{2}{5} = 2,4 $
	\item $ \dfrac{43}{10} = 4+\dfrac{3}{10} = 4,3 $
	\item $ \dfrac{39}{10} = 3+\dfrac{9}{10} = 3,9 $
	\item $ \dfrac{41}{10} = 4+\dfrac{1}{10} = 4,1 $
	\item $ \dfrac{27}{8} = 3+\dfrac{3}{8} = 3,375 $
\end{enumerate}
\end{multicols}

\exo{}

\begin{multicols}{2}
\begin{enumerate}
	\item $ 0,07~\text{hg} =  0,07\times100~\text{g} = 7~\text{g}$
	\item $ 5,16~\text{hg} =  5,16\times100~\text{g} = 516~\text{g}$
	\item $ 0,7~\text{dag} =  0,7\times10~\text{g} = 7~\text{g}$
	\item $ 0,09~\text{hm} =  0,09\times100~\text{m} = 9~\text{m}$
	\item $ 4,67~\text{kL} =  4,67\times\nombre{1000}~\text{L} = \nombre{4670}~\text{L}$
	\item $ 1,8~\text{hg} =  1,8\times100~\text{g} = 180~\text{g}$
	\item $ 1,41~\text{daL} =  1,41\times10~\text{L} = 14,1~\text{L}$
	\item $ 12,4~\text{kL} =  12,4\times\nombre{1000}~\text{L} = \nombre{12400}~\text{L}$
	\item $ 2,62~\text{km} =  2,62\times\nombre{1000}~\text{m} = \nombre{2620}~\text{m}$
	\item $ 0,2~\text{km} =  0,2\times\nombre{1000}~\text{m} = 200~\text{m}$
\end{enumerate}
\end{multicols}

\exo{}

\begin{multicols}{2}
\begin{enumerate}
	\item $ 34~\text{hm}^2 =  34\times\nombre{10000}~\text{m}^2 = \nombre{340000}~\text{m}^2$
	\item $ 3~\text{dam}^2 =  3\times100~\text{m}^2 = 300~\text{m}^2$
	\item $ 43~\text{dam}^2 =  43\times100~\text{m}^2 = \nombre{4300}~\text{m}^2$
	\item $ 10~\text{km}^2 =  10\times\nombre{1000000}~\text{m}^2 = \nombre{10000000}~\text{m}^2$
	\item $ 74~\text{dam}^2 =  74\times100~\text{m}^2 = \nombre{7400}~\text{m}^2$
	\item $ 11~\text{dam}^2 =  11\times100~\text{m}^2 = \nombre{1100}~\text{m}^2$
	\item $ 8~\text{km}^2 =  8\times\nombre{1000000}~\text{m}^2 = \nombre{8000000}~\text{m}^2$
	\item $ 2~\text{dam}^2 =  2\times100~\text{m}^2 = 200~\text{m}^2$
	\item $ 80~\text{hm}^2 =  80\times\nombre{10000}~\text{m}^2 = \nombre{800000}~\text{m}^2$
	\item $ 40~\text{hm}^2 =  40\times\nombre{10000}~\text{m}^2 = \nombre{400000}~\text{m}^2$
\end{enumerate}
\end{multicols}

\exo{}

\begin{multicols}{2}
\begin{enumerate}
	\item $\mathcal{P}_{STU}=15~\text{cm}+8~\text{cm}+17~\text{cm}=40~\text{cm}$\\
$\mathcal{A}_{STU}=15~\text{cm}\times8~\text{cm}\div2=60~\text{cm}^2$
	\item $\mathcal{P}_{DEFG}=(6~\text{cm}+5~\text{cm})\times2=22~\text{cm}$\\
$\mathcal{A}_{DEFG}=6~\text{cm}\times5~\text{cm}=30~\text{cm}^2$
	\item $\mathcal{P}_{KLM}=21~\text{cm}+20~\text{cm}+29~\text{cm}=70~\text{cm}$\\
$\mathcal{A}_{KLM}=21~\text{cm}\times20~\text{cm}\div2=210~\text{cm}^2$
	\item $\mathcal{P}_{MNOP}=4\times10~\text{cm}=40~\text{cm}$\\
$\mathcal{A}_{MNOP}=10~\text{cm}\times10~\text{cm}=100~\text{cm}^2$
	\item Le diamètre est de $6$ cm donc le rayon est de $3$ cm.\\
$\mathcal{P}=2\times3\times\pi~\text{cm}=6\pi~\text{cm}\approx18,8~\text{cm}$\\
$\mathcal{A}=3\times3\times\pi~\text{cm}^2=9\pi~\text{cm}^2\approx28,3~\text{cm}^2$
	\item $\mathcal{P}_{KLMN}=(6~\text{cm}+3~\text{cm})\times2=18~\text{cm}$\\
$\mathcal{A}_{KLMN}=6~\text{cm}\times3~\text{cm}=18~\text{cm}^2$
	\item $\mathcal{P}_{KLMN}=4\times2~\text{cm}=8~\text{cm}$\\
$\mathcal{A}_{KLMN}=2~\text{cm}\times2~\text{cm}=4~\text{cm}^2$
	\item $\mathcal{P}_{NOPQ}=4\times5~\text{cm}=20~\text{cm}$\\
$\mathcal{A}_{NOPQ}=5~\text{cm}\times5~\text{cm}=25~\text{cm}^2$
	\item $\mathcal{P}_{PQRS}=(4~\text{cm}+2~\text{cm})\times2=12~\text{cm}$\\
$\mathcal{A}_{PQRS}=4~\text{cm}\times2~\text{cm}=8~\text{cm}^2$
	\item Le diamètre est de $16$ cm donc le rayon est de $8$ cm.\\
$\mathcal{P}=2\times8\times\pi~\text{cm}=16\pi~\text{cm}\approx50,3~\text{cm}$\\
$\mathcal{A}=8\times8\times\pi~\text{cm}^2=64\pi~\text{cm}^2\approx201,1~\text{cm}^2$
\end{enumerate}
\end{multicols}


\end{correction}

\end{document}