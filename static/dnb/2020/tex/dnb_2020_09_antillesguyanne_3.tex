
\medskip

Dans cet exercice, le carré ABCD n'est pas représenté en vraie grandeur.

Aucune justification n'est attendue pour les questions 1. et 2. On attend des réponses justifiées pour la question 3.

\medskip

\begin{enumerate}
\item ~

\parbox{0.68\linewidth}{On considère le carré ABCD de centre O représenté ci-contre, partagé en quatre polygones superposables, numérotés \textcircled{1}, \textcircled{2}, \textcircled{3}, et \textcircled{4}.
	\begin{enumerate}
		\item Quelle est l'image du polygone \textcircled{1} par la symétrie centrale de centre O ?
		\item Quelle est l'image du polygone \textcircled{4} par la rotation de centre O qui transforme le polygone \textcircled{1} en le polygone \textcircled{2} ?
	\end{enumerate}}\hfill
\parbox{0.28\linewidth}{
\psset{unit=1cm}
\begin{pspicture}(-0.1,-0.1)(3,3)
\def\motif{\psline(0,1.4)(0,1.05)(-0.6,0.7)(0,0.35)(0,0)(0.35,0)(0.7,0.6)(1.05,0)(1.4,0)}
\psframe(0,0)(2.8,2.8)
\uput[ul](0,2.8){A}\uput[ur](2.8,2.8){B}\uput[dr](2.8,0){C}\uput[dl](0,0){D}
\rput(1.4,1.4){\motif}
\rput{90}(1.4,1.4){\motif}
\rput{180}(1.4,1.4){\motif}
\rput(0.5,2.2){\textcircled{1}}\rput(2.2,2.2){\textcircled{2}}
\rput(2.2,0.5){\textcircled{3}}\rput(0.5,0.5){\textcircled{4}}
\uput[dr](1.4,1.4){O}
\end{pspicture}
}
	
\item La figure ci-dessous est une partie de pavage dont un motif de base est le carré ABCD de la question 1.

Quelle transformation partant du polygone \textcircled{1} permet d'obtenir le polygone \textcircled{5} ?

\def\zig{\psset{unit=0.8cm}
\def\motif{\psline(0,1.4)(0,1.05)(-0.6,0.7)(0,0.35)(0,0)(0.35,0)(0.7,0.6)(1.05,0)(1.4,0)}
\rput(1.4,1.4){\motif}
\rput{90}(1.4,1.4){\motif}
\rput{180}(1.4,1.4){\motif}
}
\begin{center}
\psset{unit=0.8cm}
\begin{pspicture}(14,8.4)
\psframe(14,8.4)
%\psgrid
\multido{\n=0.0+2.8}{5}{
	\multido{\na=0.0+2.8}{3}{
	\rput(\n,\na){\zig}
	}
	}
\multido{\n=0.0+2.8}{6}
{\psline(\n,0)(\n,8.4)}
\multido{\n=0.0+2.8}{4}
{\psline(0,\n)(14,\n)}
\rput(0.6,7.7){\textcircled{1}}\rput(3.4,7.7){\textcircled{5}}
\uput[ul](0,8.4){A}\uput[ur](2.8,8.4){B}\uput[dl](0,5.6){D}\uput[dr](2.8,5.6){C}
\uput[dr](1.4,7){O}
\end{pspicture}
\end{center}
\item On souhaite faire imprimer ces motifs sur un tissu rectangulaire de longueur $315$ cm et de largeur $270$ cm.

On souhaite que le tissu soit entièrement recouvert par les carrés identiques à ABCD, sans découpe et de sorte que le côté du carré mesure un nombre entier de centimètres.
	\begin{enumerate}
		\item Montrer qu'on peut choisir des carrés de 9~cm de côté.
		\item Dans ce cas, combien de carrés de 9~cm de côté seront imprimés sur le tissu?
	\end{enumerate}
\end{enumerate}

\bigskip

