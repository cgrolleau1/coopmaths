
\medskip
 
 On considère le programme de calcul suivant :
 
\begin{center}
\begin{tabularx}{0.65\linewidth}{|X|}\hline
$\bullet~~$ Choisir un nombre;

$\bullet~~$ Ajouter $7$ à ce nombre;

$\bullet~~$ Soustraire $7$ au nombre choisi au départ;

$\bullet~~$ Multiplier les deux résultats précédents;

$\bullet~~$ Ajouter $50$.\\ \hline
\end{tabularx}
\end{center}

\medskip

\begin{enumerate}
\item Montrer que si le nombre choisi au départ est 2, alors le résultat obtenu est $5$.
\item Quel est le résultat obtenu avec ce programme si le nombre choisi au départ est $-10$ ?
\item Un élève s'aperçoit qu'en calculant le double de $2$ et en ajoutant $1$, il obtient $5$, le même résultat que celui qu'il a obtenu à la question 1.

Il pense alors que le programme de calcul revient à calculer le double du nombre de départ et à ajouter 1.

A-t-il raison ?
\item Si $x$ désigne le nombre choisi au départ, montrer que le résultat du programme de calcul est $x^2 + 1$.
\item Quel(s) nombre(s) doit-on choisir au départ du programme de calcul pour obtenir $17$ comme résultat ?
\end{enumerate}

\bigskip

