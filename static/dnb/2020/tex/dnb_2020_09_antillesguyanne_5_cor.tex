
\medskip

%On dispose de deux urnes:
%
%\begin{itemize}
%\item une urne bleue contenant trois boules bleues numérotées: \textcircled{2},\textcircled{3} et \textcircled{4}.
%\item une urne rouge contenant quatre boules rouges numérotées: \textcircled{2},\textcircled{3}, \textcircled{4} et \textcircled{5}.
%\end{itemize}
%
%Dans chaque urne, les boules sont indiscernables au toucher et ont la même probabilité d'être tirées.
%
%\begin{center}
%\begin{tabularx}{\linewidth}{|*{2}{>{\centering \arraybackslash}X|}}\hline
%Urne bleue &Urne rouge\\
%\textcircled{2} \textcircled{3}  \textcircled{4}& \textcircled{2} \textcircled{3} \textcircled{4}  \textcircled{5}\\ \hline
%\end{tabularx}
%\end{center}
%\medskip
%
%On s'intéresse à l'expérience aléatoire suivante :
%
%\og On tire au hasard une boule bleue et on note son numéro, puis on tire au hasard une boule rouge et on note son numéro. \fg
%
%\emph{Exemple} : si on tire la boule bleue numérotée \textcircled{3}, puis la boule rouge numérotée \textcircled{4},le tirage obtenu sera noté (3~;~4).
%
%On précise que le tirage (3~;~4) est différent du tirage (4~;~3).
%
%\medskip

\begin{enumerate}
\item %On définit les deux évènements suivants:

%\og On obtient deux nombres premiers \fg{} et \og La somme des deux nombres est égale à 12 \fg
	\begin{enumerate}
		\item %Pour chacun des deux évènements précédents, dire s'il est possible ou impossible lorsqu'on effectue l'expérience aléatoire.
$\bullet~~$Il est possible de tirer deux nombres premiers : (2~;~2), (2~;3), (2~;~5), (3~;~2), (3~;~3), (3~;~5).

$\bullet~~$La somme la plus grande est $4 + 5 = 9$. 12 est donc impossible à atteindre.
		\item %Déterminer la probabilité de l'évènement \og On obtient deux nombres premiers \fg.
Il y a $3 \times 4 = 12$ tirages différents et on a vu qu'il y en avait 6 donnant deux nombres premiers. La probabilité est donc égale à $\dfrac{6}{12} = \dfrac{1}{2} = 0,5$.
	\end{enumerate}
\item %On obtient un \og double \fg{} lorsque les deux boules tirées portent le même numéro. 

%Justifier que la probabilité d'obtenir un \og double \fg{} lors de cette expérience, est $\dfrac{1}{4}$.
On peut obtenir les doubles (2~;~2), (3~;~3) et (4~;~4), donc 3 doubles sur 12 tirages possibles. La probabilité de tirer un double est donc égale à $\dfrac{3}{12} = \dfrac{1}{4}$.
\item %Dans cette question, aucune justification n'est attendue. 

%On souhaite simuler cette expérience \np{1000} fois.

%Pour cela, on a commencé à écrire un programme, à ce stade, encore incomplet. Voici des copies d'écran :

%\begin{center}
%{\footnotesize
%\begin{tabular}{|c |c|}\hline
%\textbf{Script principal}& \textbf{Bloc \og Tirer deux boules \fg}\\
%\begin{scratch}
%\blockinit{quand \greenflag est cliqué},
%\blockrepeat{répéter \ovalnum{A} fois}
%		{
%		\blockif{si \booloperator{\ovalmove{Boule bleue} = \ovalmove{Boule rouge}} alors}
%{\blockmove{ajouter à \selectmenu{Nombre de doubles} \ovalnum{1}}}		
%		}	
%\end{scratch}
%&
%\begin{scratch}	
%\initmoreblocks{définir \namemoreblocks{Tirer deux boules}}
%\blockmove{mettre \selectmenu{Boule bleue} à {nombre aléatoire entre \ovalnum{2} et \ovalnum{B}}}
%\blockmove{mettre \selectmenu{Boule rouge} à {nombre aléatoire entre \ovalnum{2} et \ovalnum{C}}}
%\end{scratch}
%\\ 
%\multicolumn{2}{|c|}{Boule bleue, Boule rouge et Nombre de doubles 
% sont des variables.}\\
%\multicolumn{2}{|c|}{Le bloc \begin{scratch}\blockmove{Tirer deux boules}\end{scratch} est à insérer dans le script principal.}\\ \hline
%\end{tabular}}
%\end{center}
	\begin{enumerate}
		\item %Par quels nombres faut-il remplacer les lettres A, B et C ?
Il faut remplacer A par \np{1000}, B par 4 et C par 5.
		\item %Dans le script principal, indiquer où placer le bloc \begin{scratch}\blockmove{Tirer deux boules}\end{scratch}
Il faut insérer le bloc après répéter \np{1000} fois.
		\item %Dans le script principal, indiquer où placer le bloc 
%\begin{scratch}
%\blockmove{mettre \selectmenu{Nombre de doubles}à \ovalnum{0} }
%\end{scratch}
Il faut insérer le bloc avant répéter \np{1000} fois.
		\item %On souhaite obtenir la fréquence d'apparition du nombre de \og doubles \fg{} obtenus.
		
%Parmi les instructions ci-dessous, laquelle faut-il placer à la fin du script principal après la
%boucle \og répéter \fg{} ?

%\begin{center}
%{\footnotesize
%\begin{tabular}{|ccc|}\hline
%Proposition \textcircled{1}&Proposition \textcircled{2} & Proposition \textcircled{3}\\
%\begin{scratch}
%\blockmove{dire \ovaloperator{Nombre de doubles}} 
%\end{scratch}&
%\begin{scratch} 
%\blockmove{dire \ovaloperator{Nombre de doubles}/\ovalnum{1000}}\end{scratch}&\begin{scratch} 
%\blockmove{dire \ovaloperator{Nombre de doubles}/\ovalnum{2}}
%\end{scratch}\\ \hline
%\end{tabular}}
%\end{center}
Il faut placer à la fin la proposition \textcircled{2}.
	\end{enumerate}
\end{enumerate}
