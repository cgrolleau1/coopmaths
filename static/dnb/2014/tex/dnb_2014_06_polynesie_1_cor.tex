
\medskip

%On place des boules toutes indiscernables au toucher dans un sac. Sur chaque boule colorée est inscrite une lettre. Le tableau suivant présente la répartition des boules :
%
%\begin{center}
%\begin{tabularx}{0.8\linewidth}{|c|*{3}{>{\centering \arraybackslash}X|}}\hline 
%\diagbox{Lettre}{Couleur}&Rouge&Vert&Bleu\\ \hline
%A& 3&5& 2\\ \hline 
%B& 2&2& 6\\ \hline
%\end{tabularx}
%\end{center} 
 
\begin{enumerate}
\item %Combien y a-t-il de boules dans le sac ?
Il y a :

$3 + 5 + 2 + 2 + 2 + 6 = 20$ boules dans le sac. 
\item %On tire une boule au hasard, on note sa couleur et sa lettre.
	\begin{enumerate}
		\item %Vérifier qu'il y a une chance sur dix de tirer une boule bleue portant la lettre A.
Il y a 2 boules bleues portant la lettre A sur les 20, la probabilité est donc égale à 

$\dfrac{2}{20} = \dfrac{1}{10} = 0,1$. 		 
		\item %Quelle est la probabilité de tirer une boule rouge ?
Il y a 5 boules rouges, donc la probabilité est égale à $\dfrac{5}{20} = \dfrac{1}{4}$. 
		\item %A-t-on autant de chance de tirer une boule portant la lettre A que de tirer une boule portant la lettre B ?
		Il y a 10 boules portant la lettre A et donc autant portant la lettre B. On a donc effectivement autant de chance de tirer une boule portant la lettre A que de tirer une boule portant la lettre B.
	\end{enumerate}
\end{enumerate}
 
\bigskip

