
\medskip 

Les éoliennes sont construites de manière à avoir la même mesure d'angle entre chacune de leurs pales. 

\medskip

\begin{enumerate}
\item Une éolienne a trois pales. Quelle est la mesure de l'angle entre deux de ses pales ? 
\item Pour réduire le bruit provoqué par les éoliennes, il faut augmenter le nombre de pales. 

Sur l'annexe 1, on a représenté le mât d'une éolienne à six pales par le segment [AB]. En prenant le point A pour centre des pales, compléter la construction avec des pales de 5~cm. 
\item On estime qu'à 80 m du centre des pales d'une éolienne le niveau sonore est juste suffisant pour que l'on puisse entendre le bruit qu'elle produit. 
\end{enumerate}

Un randonneur dont les oreilles sont à 1,80~m du sol se déplace vers une éolienne dont le mât mesure 35~m de haut. Il s'arrête dès qu'il entend le bruit qu'elle produit (voir le schéma ci-dessous). 

À quelle distance du mât de l'éolienne (distance BC) se trouve-t-il ? Arrondir le résultat à l'unité. 

\begin{center}
\psset{unit=0.75cm}
\def\pale{\pspolygon[fillstyle=solid,fillcolor=lightgray](0,0)(3,0)(3,0.2)(0.6,0.3)(0.3,0.2)(0,0.2)\pspolygon[fillstyle=solid,fillcolor=black](3,0)(3.5,0.1)(3.5,0.2)(3,0.2)\qdisk(3.5,0.15){1.5pt}}
\begin{pspicture}(-1.5,-2)(14.5,9)
%\psgrid
\pspolygon[fillstyle=solid,fillcolor=lightgray](1.85,0)(2.15,0)(2.1,5)(1.9,5)
\pscircle[fillstyle=solid,fillcolor=lightgray](2,5.3){3mm}
\psline[arrowsize=3pt 3]{<->}(-1.4,0)(-1.4,5.3)\rput{90}(-1.6,2.65){35~m}
\psdots(2,0)(14,0)
\psline[linewidth=2pt](2,0)(14,0)
\psline(2,0)(-1.5,0)
\psline[arrowsize=3pt 3]{<->}(14,0)(14,1.3)
\rput{90}(14.5,0.55){1,80 m}
\rput{-30}(2,5.3){\pale}
\rput{-150}(2,5.3){\pale}
\rput{90}(2,5.3){\pale}
\uput[r](14,1.4){oreilles}
\pscircle[fillstyle=solid,fillcolor=gray](2,5.3){2mm}
\rput(4,7.5){Une pale}\psline[arrowsize=3pt 3]{->}(3.9,7.2)(2,7)
\rput(4,6){Centre des pales}\psline[arrowsize=3pt 3]{->}(3.9,5.6)(2,5.3) 
\rput(4,2){Mât}\psline[arrowsize=3pt 3]{->}(3.5,2)(2.1,2)
\uput[d](2,0){B} \uput[ul](1.8,5.3){A} \uput[d](14,0){C} 
\psline[linewidth=3pt,linestyle=dotted](2,5.3)(14,1.3)
\rput{-18}(8,3.6){80 m}
\rput(6.5,-0.75){Sol} 
\rput(6.5,-1.5){\emph{La figure n'est pas à l'échelle}} 
\end{pspicture}
\end{center} 

\begin{center}

	\textbf{ANNEXE 1 - Exercice 6}
	
	\vspace{1cm}
	
	\psset{unit=1cm}
	\begin{pspicture}(10,14)
	\psline(5,2)(5,10)
	\psdots(5,2)(5,10)
	\uput[l](5,2){B}\uput[l](5,10){A}
	\end{pspicture} 
	
	\end{center}
\vspace{0,5cm}

