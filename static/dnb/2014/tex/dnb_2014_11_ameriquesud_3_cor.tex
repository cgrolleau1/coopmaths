
\medskip 

%Le document ci-dessous indique les tarifs postaux pour un envoi depuis la France métropolitaine d'une lettre ou d'un paquet en mode \og lettre prioritaire \fg. 
%
%Ces tarifs sont fonction du poids de la lettre.
%
%\begin{center} 
%
%\psset{unit=1cm}
%\begin{pspicture}(10,2)
%\psframe(10,2)
%\psframe[fillstyle=solid,fillcolor=lightgray](4,2)
%\rput(2,1){LETTRE PRIORITAIRE}\rput(7,1){service urgent d’envoi de courrier}
%\end{pspicture} 
%\end{center}
%
%\setlength\parindent{6mm}
%\begin{itemize}
%\item[$\bullet~~$] \textbf{Pour les envois vers :} La France, Monaco, Andorre et secteurs postaux (armée). Complément d’affranchissement aérien vers l'Outre-mer pour les envois de plus de 20~g 
%\item[$\bullet~~$] \textbf{Service universel:} Jusqu'à 2~kg 
%\item[$\bullet~~$] \textbf{Délai:} J + 1,  indicatif 
%\item[$\bullet~~$] \textbf{Dimensions:} Minimales : $14 \times 9$ cm, maximales : L + l + H = 100 cm, avec L $< 60$ cm 
%\item[$\bullet~~$] \textbf{Complément aérien :} 
%		\begin{itemize}
%			\item Vers zone OM1 : Guyane, Guadeloupe, Martinique, La Réunion, St Pierre et Miquelon, St-Barthélémy, St-Martin et Mayotte : 0,05~\euro{} par tranche de 10~g. 
%			\item Vers zone OM2 : Nouvelle-Calédonie, Polynésie française, Wallis-et~Futuna, TAAF. : 0,11~\euro{} par tranche de 10~g
%		\end{itemize}
%		 \item[$\bullet~~$] \textbf{Exemple de complément :} Pour un envoi de 32~g vers la Guadeloupe : 1,10\euro{} + $4 \times 0,05$\euro{} = 1,3\euro. 
%\end{itemize}
%\setlength\parindent{0mm}
%
%\begin{center}
%\begin{tabularx}{\linewidth}{|l *{2}{>{\centering \arraybackslash}X|}}\hline
%\multicolumn{2}{|c|}{POIDS JUSQU'À}&TARIFS NETS\\ \hline
%20~g	&     	&0,66\euro \\ \hline  
%50~g	&     	&1,10\euro \\ \hline   
%100~g	&     	&1,65\euro \\ \hline   
%250~g	&     	&2,65\euro \\ \hline   
%500~g	&     	&3,55\euro \\ \hline   
%1~kg	&     	&4,65\euro \\ \hline   
%2~kg	&     	&6,00\euro \\ \hline   
%3~kg	&   	&7,00O\euro \\ \hline
%\end{tabularx}
%\end{center}   

\begin{enumerate}
\item %Expliquer pourquoi le coût d'un envoi vers la France Métropolitaine, en \og lettre prioritaire \fg, d'une lettre de 75~g est de 1,65\euro.
De 51 à 100 g le montant de l'affranchissement est égal à 1,65~\euro. 
\item %Montrer que le coût d'un envoi à Mayotte, en \og lettre prioritaire \fg, d'une lettre de 109~g est de 3,20~\euro.
Pour Mayotte le montant est de 2,65~\euro{} plus un complément aérien de 

$11 \times 0,05 = 0,55$~\euro{} soit au total 3,20~\euro. 

%\textbf{Dans cette question ci-dessous, il sera tenu compte de toute trace de réponse même incomplète dans l'évaluation.}
 
\item %Au moment de poster son courrier à destination de Wallis-et-Futuna, Loïc s'aperçoit qu'il a oublié sa carte de crédit et qu'il ne lui reste que 6,76~\euro{} dans son porte-monnaie. 

%Il avait l'intention d'envoyer un paquet de 272 g, en \og lettre prioritaire \fg. 

%Peut-il payer le montant correspondant ?
Le montant initial est 3,55~\euro{} auquel il faut ajouter le complément aérien de $28 \times 0,11 = 3,08$~\euro{} soit au total  6,63~\euro. Il peut payer l'envoi.
\item %Le paquet a les dimensions suivantes : L = 55~cm l = 30~cm et h = 20~cm. Le guichetier de l'agence postale le refuse. Pourquoi ? 
On a L + l + H $= 105 > 100$ : la somme des dimensions dépasse 100 cm ; le paquet est refusé.
\end{enumerate}

\vspace{0,5cm}

