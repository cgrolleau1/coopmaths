
\medskip

L'Association des Enfants Heureux organise une course. Chaque
enfant a un vélo ou un tricycle. 

L'organisateur a compté $64$ enfants et
$151$~roues.

\medskip

\begin{enumerate}
\item% Combien de vélos et combien de tricycles sont engagés dans cette course ?
Soit $v$ le nombre de vélos et $t$ le nombre de tricycles.

Un vélo possède 2 roues, un tricycles en possède 3; il y a en tout 151 roues donc $2v+3t=151$.

Il y a 64 enfants donc le nombre total de cycles est 64: $v+t=64$.

On résout le système $\left\lbrace 
\begin{array}{r !{=} l}
2v+3t & 151\\
v+t & 64
\end{array}
\right.$

$\left\lbrace 
\begin{array}{r !{=} l}
2v+3t & 151\\
v+t & 64
\end{array}
\right. 
\iff
\left\lbrace 
\begin{array}{r !{=} l}
2v+3t & 151\\
v & 64-t
\end{array}
\right.
\iff
\left\lbrace 
\begin{array}{r !{=} l}
2(64-t)+3t & 151\\
v & 64-t
\end{array}
\right.\\
\phantom{\left\lbrace 
\begin{array}{r !{=} l}
2v+3t & 151\\
v+t & 64
\end{array}
\right.}
\iff
\left\lbrace 
\begin{array}{r !{=} l}
128-2t+3t & 151\\
v & 64-t
\end{array}
\right.
\iff
\left\lbrace 
\begin{array}{r !{=} l}
t & 151-128\\
v & 64-t
\end{array}
\right.
\iff
\left\lbrace 
\begin{array}{r !{=} l}
t & 23\\
v & 41
\end{array}
\right.
$ 

Dans cette course, il y a 41 vélos et 23 tricycles engagés.

\item Chaque vélo engagé rapporte 500~F et chaque tricycle 400~F. %Calculer la somme que l'association des Enfants Heureux recevra.

Les vélos rapportent $41\times 500=\np{20500}$~F;
les tricycles rapportent $23\times 400=\np{9200}$~F.

L'association recevra pour cette course $\np{20500}+ \np{9200} = \np{29700}$~F.

%\emph{Dans cet exercice, tout essai, toute idée exposée et toute démarche, même non aboutis ou mal formulés seront pris en compte pour l'évaluation.}

\end{enumerate}

\vspace{0,5cm}

