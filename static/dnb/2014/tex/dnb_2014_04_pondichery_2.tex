
\medskip

\emph{Cet exercice est un questionnaire \`{a} choix multiple (QCM). Pour chaque ligne du tableau, trois réponses sont proposées, mais une seule est exacte.\\ Toute réponse exacte vaut 1 point. \\Toute réponse inexacte ou toute absence de réponse n'enlève pas de point.}

\medskip
 
Indiquez sur votre copie le numéro de la question et, sans justifier, recopier la réponse exacte (A ou B ou C).

\medskip

\begin {tabularx}{\linewidth}{|m{4cm}|*{3}{>{\centering \arraybackslash}X|}}\hline 
&A &B &C\\ \hline 
\rule[1mm]{0mm}{4mm}\textbf{1.}~~ $\sqrt{(- 5)^2}$
&n'existe pas &est égal \`{a} $- 5$ &est égal \`{a} $5$\\ \hline 
\textbf{2.}~~Si deux surfaces ont la même aire alors&elles sont superposables& elles ont le même  périmètre&leurs périmètres ne sont pas forcément égaux.\\ \hline 
\textbf{3.}~~Soit $f$ la fonction définie par: ${f(x) = 3x - (2x + 7) + (3x + 5)}$
& $f$ est une fonction affine
& $f$ est une fonction linéaire
& $f$ n'est pas une fonction affine.\\ \hline 
\textbf{4.}~~Hicham a récupéré les résultats d'une enquête sur les numéros qui sont sortis ces dernières années au loto. Il souhaite jouer lors du prochain tirage.&
Il vaut mieux qu'il joue les numéros qui sont souvent sortis&
Il vaut mieux qu'il joue les numéros qui ne sont pas souvent sortis.&
L'enquête ne peut pas l'aider.\\ \hline 
\textbf{5.}~~Une expression factorisée de $(x - 1)^2 - 16$ est ...& 
$(x + 3)(x - 5)$& 
$(x - 4)(x + 4)$& 
$x^2 - 2x - 15$\\ \hline
\end{tabularx} 

\vspace{0,5cm}

