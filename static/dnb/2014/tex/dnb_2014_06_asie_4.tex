
\medskip

Dans chaque cas, dire si l'affirmation est vraie ou fausse. 

\textbf{Justifier vos réponses.}

\medskip
 
\textbf{Cas 1 }: À l'entrée d'un cinéma, on peut lire les tarifs ci-dessous pour une place de cinéma.

\begin{center}
\begin{tabularx}{0.45\linewidth}{|l X|}\hline
\multicolumn{2}{|c|}{Tarif d'une place de cinéma :}\\ 
Plein tarif :			& 9,50~\euro\\ 
Enfants ($- 12$ ans) :	& 5,20~\euro\\
Étudiants :				& 6,65~\euro \\
Séniors :				& 7,40~\euro \\\hline
\end{tabularx}
\end{center}
  
\textbf{Affirmation 1 }: Les étudiants bénéficient d'une réduction de 30\,\% sur le plein tarif.

\medskip
 
\textbf{Cas 2 }: $a$ et $b$ désignent des entiers positifs avec $a > b$ 

\textbf{Affirmation 2 } : PGCD$(a~;~b) = a - b$.

\medskip
 
\textbf{Cas 3 }: $A$ est égale au produit de la somme de $x$ et de $5$ par la différence entre $2x$ et $1$. $x$ désigne un nombre relatif. 

\textbf{Affirmation 3 }: $A = 2x^2 + 9x - 5$. 

\bigskip

