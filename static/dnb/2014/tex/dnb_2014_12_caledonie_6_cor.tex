

\begin{enumerate}
\item La mesure de l'angle entre deux pales d'une éolienne est $\dfrac{360\degres}{3}=120$\degres.
\item La mesure de l'angle entre deux pales d'une éolienne (6 pales) est $\dfrac{360\degres}{6}=60$\degres.
\item ~\\
\psset{unit=0.75cm}
\def\pale{\pspolygon[fillstyle=solid,fillcolor=lightgray](0,0)(3,0)(3,0.2)(0.6,0.3)(0.3,0.2)(0,0.2)\pspolygon[fillstyle=solid,fillcolor=black](3,0)(3.5,0.1)(3.5,0.2)(3,0.2)\qdisk(3.5,0.15){1.5pt}}
\begin{pspicture}(-1.5,-2)(14.5,9)
\pspolygon[fillstyle=solid,fillcolor=lightgray](1.85,0)(2.15,0)(2.1,5)(1.9,5)
\pscircle[fillstyle=solid,fillcolor=lightgray](2,5.3){3mm}
\psline[arrowsize=3pt 3]{<->}(-1.4,0)(-1.4,5.3)\rput{90}(-1.6,2.65){35~m}
\psdots(2,0)(14,0)
\psline[linewidth=2pt](2,0)(14,0)
\psline(2,0)(-1.5,0)
\psline[arrowsize=3pt 3]{<->}(14,0)(14,1.3)
\rput{90}(14.5,0.55){1,80 m}
\rput{-30}(2,5.3){\pale}
\rput{-150}(2,5.3){\pale}
\rput{90}(2,5.3){\pale}
\uput[r](14,1.4){oreilles}
\psline[linewidth=3pt,linestyle=dotted](2,1.3)(14,1.3) % Tracé de [DE]
\pscircle[fillstyle=solid,fillcolor=gray](2,5.3){2mm}
\rput(4,7.5){Une pale}\psline[arrowsize=3pt 3]{->}(3.9,7.2)(2,7)
\rput(4,6){Centre des pales}\psline[arrowsize=3pt 3]{->}(3.9,5.6)(2,5.3) 
\rput(4,2){Mât}\psline[arrowsize=3pt 3]{->}(3.5,2)(2.1,2)
\uput[d](2,0){$B$} \uput[ul](1.8,5.3){$A$} \uput[d](14,0){$C$} 
\rput[bl](1.5,1.15){$D$} \rput[bl](13.6,0.9){$E$}
\psline[linewidth=3pt,linestyle=dotted](2,5.3)(14,1.3)
\rput{-18}(8,3.6){80 m}
\rput(6.5,-0.75){Sol} 
\end{pspicture}

Sur la figure, qui n'est pas à  l'échelle, $AB=35$ m, $AE=80$ m et $CE=1,80$ m.

$BCED$ est un rectangle, donc $DB = CE = 1,80$ m.

$D$ appartient à $[AB]$, donc $AD = 35 - 1,80 = 33,20$ m.

Le triangle $ADE$ est rectangle en $D$, donc d'après le théorème de Pythagore : \\[1mm]
$AE^2=AD^2+DE^2$ \\[1mm]
$80^2=33,20^2+DE^2$ \\[1mm]
$6~400=1~102,24+DE^2$ \\[1mm]
$DE^2=6~400-1~102,24$ \\[1mm]
$DE^2=5~297,76$ \\[1mm]
$DE=\sqrt{5~297,76}$ \\[1mm]
$DE\approx73$ m \\[1mm]
Comme $BC=DE$, le randonneur se trouve à environ 73 m du mât de \linebreak l'éolienne. 


\begin{center}

\textbf{ANNEXE 1 - Exercice 6}

\vspace{0.5cm}

%\psset{xunit=1.0cm,yunit=1.0cm,algebraic=true,dimen=middle,dotstyle=o,dotsize=3pt 0,linewidth=0.8pt,arrowsize=3pt 2,arrowinset=0.25}
%\begin{pspicture*}(-0.336923259784,1.63443796416)(10.6143060724,15.3532334047)
%\psline(5.,10.)(5.,2.)
%\pscircle[linestyle=dashed,dash=4pt 4pt](5.,10.){5.}
%\psline(5.,10.)(8.40302322754,6.33674558448)
%\psline(5.,10.)(9.87398299814,11.115477357)
%\psline(5.,10.)(6.4709597706,14.7787317725)
%\psline(5.,10.)(1.59697677246,13.6632544155)
%\psline(5.,10.)(0.126017001863,8.88452264305)
%\psline(5.,10.)(3.5290402294,5.22126822752)
%\pscustom[linecolor=blue]{
%\parametricplot{-0.8222087235326183}{0.22498882766397932}{0.892763260773*cos(t)+5.|0.892763260773*sin(t)+10.}
%\lineto(5.,10.)\closepath}
%\parametricplot[linecolor=blue]{-0.8222087235326183}{0.22498882766397932}{0.892763260773*cos(t)+5.|0.892763260773*sin(t)+10.}
%\psline[linecolor=blue](5.78843128396,9.84518603196)(5.96363823595,9.81078292795)
%\psline[linecolor=blue](5.73901453489,9.68464500726)(5.90323998709,9.61456611999)
%\pscustom[linecolor=blue]{
%\parametricplot{0.2249888276639793}{1.272186378860577}{1.19035101436*cos(t)+5.|1.19035101436*sin(t)+10.}
%\lineto(5.,10.)\closepath}
%\parametricplot[linecolor=blue]{0.2249888276639793}{1.272186378860577}{1.19035101436*cos(t)+5.|1.19035101436*sin(t)+10.}
%\psline[linecolor=blue](5.72395086789,10.8296147359)(5.84134830593,10.9641468553)
%\psline[linecolor=blue](5.88061740695,10.6609678145)(6.0234202297,10.7681517844)
%\pscustom[linecolor=blue]{
%\parametricplot{1.272186378860577}{2.3193839300571746}{0.892763260773*cos(t)+5.|0.892763260773*sin(t)+10.}
%\lineto(5.,10.)\closepath}
%\parametricplot[linecolor=blue]{1.272186378860577}{2.3193839300571746}{0.892763260773*cos(t)+5.|0.892763260773*sin(t)+10.}
%\psline[linecolor=blue](4.7398571872,10.7602085051)(4.68204767325,10.9291437284)
%\psline[linecolor=blue](4.90359816748,10.7976828574)(4.88217553803,10.9749457145)
%\pscustom[linecolor=blue]{
%\parametricplot{2.3193839300571746}{3.366581481253772}{1.19035101436*cos(t)+5.|1.19035101436*sin(t)+10.}
%\lineto(5.,10.)\closepath}
%\parametricplot[linecolor=blue]{2.3193839300571746}{3.366581481253772}{1.19035101436*cos(t)+5.|1.19035101436*sin(t)+10.}
%\psline[linecolor=blue](3.91955712939,10.2121524747)(3.7443501774,10.2465555787)
%\psline[linecolor=blue](3.98727637811,10.4321531382)(3.82305092591,10.5022320255)
%\pscustom[linecolor=blue]{
%\parametricplot{-2.9166038259258142}{-1.8694062747292166}{0.892763260773*cos(t)+5.|0.892763260773*sin(t)+10.}
%\lineto(5.,10.)\closepath}
%\parametricplot[linecolor=blue]{-2.9166038259258142}{-1.8694062747292166}{0.892763260773*cos(t)+5.|0.892763260773*sin(t)+10.}
%\psline[linecolor=blue](4.47171152884,9.39460546297)(4.3543140908,9.26007334363)
%\psline[linecolor=blue](4.35738729763,9.51767213539)(4.21458447488,9.41048816547)
%\pscustom[linecolor=blue]{
%\parametricplot{-1.8694062747292166}{-0.8222087235326183}{1.19035101436*cos(t)+5.|1.19035101436*sin(t)+10.}
%\lineto(5.,10.)\closepath}
%\parametricplot[linecolor=blue]{-1.8694062747292166}{-0.8222087235326183}{1.19035101436*cos(t)+5.|1.19035101436*sin(t)+10.}
%\psline[linecolor=blue](5.35649200272,8.95823278935)(5.41430151668,8.789297566)
%\psline[linecolor=blue](5.13210621494,8.90687904733)(5.15352884439,8.72961619014)
%\psdots[dotstyle=x,linecolor=darkgray](5.,10.)
%\rput[bl](5.1,10.2){\darkgray{$A$}}
%\psdots[dotstyle=x,linecolor=darkgray](5.,2.)
%\rput[bl](5.1,2){\darkgray{$B$}}
%\rput[bl](6.1,9.6){\blue{$60\textrm{\degre}$}}
%\end{pspicture*}
%\end{center} 




%%%%%%%%%%%%%%%%%% éolienne à 6 pales
%\begin{center}
\psset{unit=0.75cm}
\def\pale{\pspolygon[fillstyle=solid,fillcolor=lightgray](0,0)(3,0)(3,0.2)(0.6,0.3)(0.3,0.2)(0,0.2)\pspolygon[fillstyle=solid,fillcolor=black](3,0)(3.5,0.1)(3.5,0.2)(3,0.2)\qdisk(3.5,0.15){1.5pt}}
\begin{pspicture}(-1.5,-1)(6,9)
%\psgrid
\pspolygon[fillstyle=solid,fillcolor=lightgray](1.85,0)(2.15,0)(2.1,5)(1.9,5)
\pscircle[fillstyle=solid,fillcolor=lightgray](2,5.3){3mm}

\rput{-20}(2,5.3){\pale}
\rput{40}(2,5.3){\pale}
\rput{100}(2,5.3){\pale}
\rput{160}(2,5.3){\pale}
\rput{220}(2,5.3){\pale}
\rput{280}(2,5.3){\pale}
\uput[r](14,1.4){oreilles}
\rput(6.5,-1.5){\emph{La figure n'est pas à l'échelle}} 
\end{pspicture}
\end{center} 
\end{enumerate}
%%%%%%%%%%%%%%%%%%%%%%%%%%%%%

