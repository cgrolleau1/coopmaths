
\bigskip

La figure ci-dessous, qui n'est pas dessinée en vraie grandeur, représente un cercle $(C)$ et plusieurs segments. On dispose des informations suivantes :

\parbox{0.5\linewidth}{\begin{itemize}
\item[$\bullet~~$] [AB] est un diamètre du cercle $(C)$ de centre O et de rayon 7,5 cm. 
\item[$\bullet~~$]K et F sont deux points extérieurs au cercle $(C)$. 
\item[$\bullet~~$]Les segments [AF] et [BK] se coupent en un point T situé sur le cercle $(C)$. 
\item[$\bullet~~$]AT = 12 cm, BT = 9 cm, TF = 4 cm, TK = 3 cm.
\end{itemize} 
 
\begin{enumerate}
\item Démontrer que le triangle ATB est rectangle. 
\item Calculer la mesure de l'angle $\widehat{\text{BAT}}$ arrondie au degré près. 
\item Les droites (AB) et (KF) sont-elles parallèles ? 
\item Calculer l'aire du triangle TKF.
\end{enumerate}} \hfill
\parbox{0.45\linewidth}{\psset{unit=0.6cm}
\begin{pspicture}(9,7)
\pscircle(3.4,3){3}
\pspolygon(8.7,3.4)(1.6,5.4)(5.2,0.6)(6.7,6)%FABKF
\uput[dl](3.4,3){O} \uput[ul](1.6,5.4){A} \uput[dr](5.2,0.6){B} 
\uput[d](8.7,3.4){F} \uput[u](6.7,6){K} \uput[dr](6.2,4.1){T} 
\uput[l](0.6,4.3){$(C)$}
\psdots[dotstyle=+,dotangle=45](3.4,3) 
\end{pspicture}}

\bigskip
 
