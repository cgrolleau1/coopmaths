Dans ce questionnaire à choix multiples, pour chaque question, des réponses sont proposées, une seule est exacte. Pour chacune des questions, écrire le numéro de la question et recopier la bonne réponse. Aucune justification n'est attendue.
\subsubsection*{Question 1: Réponse d).}
Quand on double le rayon $R$ d'une boule, son volume $V$ est multiplié par 8: \[
V=\dfrac{4}{3}\pi (2R)^3=\dfrac{4}{3}\pi 2^3R^3=\dfrac{4}{3}\pi 8R^3=8V
\]

\subsubsection*{Question 2: Réponse a).}
Une vitesse égale à 36~km.h$^{-1}$ correspond à $10~\text{m}.s^{-1}$.
\[
36~\text{km}\to 1~\text{heure}\Longleftrightarrow \nombre{36000}~\text{m}\to \nombre{3600}~\text{secondes}\Longleftrightarrow \frac{\nombre{36000}}{\nombre{3600}}=10\to 1~\text{seconde}
\]
\subsubsection*{Question 3: Réponse c).}
Quand on divise $\sqrt{525}$ par 5, on obtient $\sqrt{21}$:
\[
\frac{\sqrt{525}}{5}=\frac{\sqrt{5^2\times 21}}{5}=\frac{5\sqrt{21}}{5}=\sqrt{21}
\]

\subsubsection*{Question 4: Réponse a).}
On partage un disque dur de 1,5~To en dossiers de 60~Go chacun.

Le nombre de dossier obtenus est égal à 25:
\[
1,5~\text{To}=1,5\times 10^{12}~\text{octets}=1,5\times 10^{3}~\text{Go}\Longrightarrow
\frac{1,5\times 10^3}{60}=\frac{\nombre{1500}}{60}=25
\]

\vspace{0,5cm}

