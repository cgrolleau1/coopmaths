
\medskip

$\bullet~~$ Recherche de la longueur du parcours ACDA :

Dans le triangle ACD rectangle en C, d'après le théorème de Pythagore, on a :

$\text{AD}^2 = \text{AC}^2 + \text{DC}^2$.

D'où AD$^2 = 1,4^2 + 1,05^2 = \np{3,0625}$ ; par suite, AD $= \sqrt{\np{3,0625}} = 1,75$~(km). 

Or $\text{AC} + \text{CD} + \text{DA} = 1,4 + 1,05 + 1,75 = 4,2$.

Donc la longueur du parcours ACDA est de $4,2$~km.

$\bullet~~$  Recherche de la longueur du parcours AEFA :

Dans le triangle AEF, E$'$ appartient à [AE], F$'$ appartient à [AF] et les droites (E$'$F$'$) et (EF) sont parallèles. D'après le théorème de Thalès, on a : 

$\dfrac{\text{AE}}{\text{AE'}} = \dfrac{\text{AF}}{\text{AF'}} = \dfrac{\text{EF}}{\text{E'}\text{F'}}$, c'est-à-dire, $\dfrac{1,3}{0,5} = \dfrac{1,6}{\text{AF'}} = \dfrac{\text{EF}}{0,4}$. 

D'où : $\dfrac{1,3}{0,5} = \dfrac{\text{EF}}{0,4}$. Ainsi EF $= \dfrac{1,3 \times 0,4}{0,5}  =1,04$~(km).

Or $\text{AE} + \text{EF}+ \text{FA} = 1,3 + 1,04 + 1,6 = 3,94$.

Donc la longueur du parcours AEFA est de $3,94$~km.

$\bullet~~$  Comparaison des deux parcours :

$4,2 - 4 = 0,2$ et $4 - 3,94 = 0,06$.

La commune choisira donc le parcours AEFA car sa longueur s'approche le plus possible de $4$~km.

\vspace{0,5cm}

