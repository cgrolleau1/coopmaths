
\medskip


Question \no 1 : Réponse A directement en utilisant la calculatrice. \\
Commentaires sur la question \no 1 : Par le calcul.\\[2mm]
$\dfrac{4}{5} + \dfrac{1}{5}\times \dfrac{2}{3}=\dfrac{4}{5} + \dfrac{1\times2}{5\times 3}$ \quad Priorité de la multiplication sur l'addition. \\[2mm]
\phantom{$\dfrac{4}{5} + \dfrac{1}{5}\times \dfrac{2}{3}$} $=\dfrac{4}{5} + \dfrac{2}{15}$ \\[2mm]
\phantom{$\dfrac{4}{5} + \dfrac{1}{5}\times \dfrac{2}{3}$} $=\dfrac{4\times 3}{5\times 3} + \dfrac{2}{15}$  \\[2mm]
\phantom{$\dfrac{4}{5} + \dfrac{1}{5}\times \dfrac{2}{3}$} $=\dfrac{12}{15} + \dfrac{2}{15}$  \\[2mm]
\phantom{$\dfrac{4}{5} + \dfrac{1}{5}\times \dfrac{2}{3}$} $=\dfrac{12+2}{15}$  \\[2mm]
\phantom{$\dfrac{4}{5} + \dfrac{1}{5}\times \dfrac{2}{3}$} $=\dfrac{14}{15}$  \\[2mm]

Question \no 2 : Réponse C directement en utilisant la calculatrice.\\
Commentaires sur la question \no 2 : Par le calcul.\\[2mm]
$\sqrt{25}=5$ et $\sqrt{3}^2 =3$, donc $\sqrt{25} \times \sqrt{3}^2 =5 \times 3=15$ \\

Question \no 3 : Réponse A.\\
Commentaires sur la question \no 3 : \\
Par le calcul : \quad 5\,\% de 650 correspond à $\dfrac{5}{100}\times650=32,5$. \\[2mm]
Mentalement : \quad  5\,\% signifie 5 pour 100, donc $6\times5$ pour $6\times100$, soit 30 pour 600. \\
\phantom{Mentalement : \quad} La moitié de 5 pour la moitié de 100, donc 2,5 pour 50.\\
Au total : $30+2,5$ pour $600+50$, soit 32,5 pour 650. \\

Question \no 4 : Réponse B.\\
Commentaires sur la question \no 4 : \\
On élimine la réponse A, car un véhicule est considéré comme un poids lourd à partir du moment où son poids total autorisé en charge (PTAC) excède 3,5 tonnes. Les véhicules qui disposent de quatre essieux ou plus, ainsi que les autobus articulés ont un PTAC maximal de 32 tonnes. \\
On élimine la réponse C, $7 \times 10^{-15}$ g est inférieur à 1 g.

\medskip

\vspace{0,5cm}

