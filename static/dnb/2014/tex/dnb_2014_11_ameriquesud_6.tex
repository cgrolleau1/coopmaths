
\medskip

Lors d'une activité sportive, il est recommandé de surveiller son rythme cardiaque.

Les médecins calculaient autrefois, la fréquence cardiaque maximale recommandée $f_m$ exprimée en battements par minute, en soustrayant à 220 l'âge $a$ de la personne exprimé en années. 

\medskip

\begin{enumerate}
\item Traduire cette dernière phrase par une relation mathématique. 
\item Des recherches récentes ont montré que cette relation devait être légèrement modifiée. 

La nouvelle relation utilisée par les médecins est: 

\[\text{Fréquence cardiaque maximale recommandée}\: = 208 - (0,75 \times a).\] 

	\begin{enumerate}
		\item Calculer la fréquence cardiaque maximale à 60~ans recommandée aujourd'hui par les médecins. 
		\item Déterminer l'âge pour lequel la fréquence cardiaque maximale est de 184 battements par minute. 
		\item Sarah qui a vingt ans court régulièrement.
		
Au cours de ses entraînements, elle surveille son rythme cardiaque. 

Elle a ainsi déterminé sa fréquence cardiaque maximale recommandée et a obtenu 193~battements par minute. Quand elle aura quarante ans, sa fréquence cardiaque maximale sera de 178~battements par minute. 

Est-il vrai que sur cette durée de vingt ans sa fréquence cardiaque maximale aura diminué d'environ 8\,\% ?
	\end{enumerate} 
\end{enumerate}

\vspace{0,5cm}

