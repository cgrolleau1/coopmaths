
\medskip 

Le principe d'un vaccin est d'inoculer (introduire dans l'organisme) à une personne saine, en très faible quantité, une bactérie, ce qui permet à l'organisme de fabriquer des anticorps. Ces anticorps permettront de combattre la maladie par la suite si la personne souffre de cette maladie. 

Lors de la visite médicale de Pablo le jeudi 16 octobre, le médecin s'aperçoit qu'il n'est pas à jour de ses vaccinations contre le tétanos. Il  réalise alors une première injection d'anatoxine tétanique et lui indique qu'un rappel sera nécessaire. 

On réalise des prises de sang quotidiennes pour suivre la réaction de l'organisme aux injections. 

\begin{center}
Évolution du taux d'anticorps en fonction du temps lors de deux injections anatoxine tétanique* 

\psset{xunit=0.25cm,yunit=0.01cm}
\begin{pspicture}(-5,-100)(42,1000)
\multido{\n=0+1}{43}{\psline[linewidth=0.3pt](\n,0)(\n,1000)}
\multido{\n=0+50}{21}{\psline[linewidth=0.3pt](0,\n)(42,\n)}
\psaxes[linewidth=1.25pt,Dx=5,Dy=100](0,0)(42,1000)
\pscurve[linewidth=1.5pt,linecolor=blue](0,0)(1.5,0)(3,40)(4,70)(5,90)(6,70)(7,41)(10,9)(13,0)(24,0)(29,0)(30,6)(31,100)(32,300)(33,650)(33.7,800)(34.5,850)(35.5,810)(36,790)(37,770)(41,725)
\uput[d](21,-50){temps (jours)}
\rput{90}(-4,450){taux d'anticorps (unité arbitraire)}
\end{pspicture}
\end{center}  

*anatoxine tétanique (AT) : substance inactivée provenant de la bactérie responsable du tétanos et servant à la fabrication du vaccin.

\medskip 

\begin{enumerate}
\item Combien de jours faut-il attendre, après la première injection, pour constater une présence d'anticorps ? 
\item Quelle est la valeur maximale du taux d'anticorps atteinte après la première injection ? 

À quel jour de la semaine correspond cette valeur ? 
\item Au bout de combien de jours approximativement, après la première injection, Pablo n'a t-il plus d'anticorps dans son organisme ? 
\item Durant combien de jours environ le taux d'anticorps est supérieur à 800 ? 
\end{enumerate}

\vspace{0,5cm}

