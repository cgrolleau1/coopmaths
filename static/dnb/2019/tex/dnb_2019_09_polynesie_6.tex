
\medskip

L'éco-conduite est un comportement de conduite plus responsable permettant de :

\begin{itemize}
\item réduire ses dépenses : moins de consommation de carburant et un coût d'entretien du
véhicule réduit ;
\item limiter les émissions de gaz à effet de serre;
\item réduire le risque d'accident de $10$ à $15$\,\% en moyenne.
\end{itemize}

\medskip

\begin{enumerate}
\item Un des grands principes est de vérifier la pression des pneus de son véhicule. On considère
des pneus dont la pression recommandée par le constructeur est de $2,4$~bars.
	\begin{enumerate}
		\item Sachant qu'un pneu perd environ $0,1$~bar par mois, en combien de mois la pression des pneus sera descendue à $1,9$~bar, s'il n'y a eu aucun gonflage ?
		\item Le graphique ci-dessous donne un pourcentage approximatif de consommation
supplémentaire de carburant en fonction de la pression des pneus (zone grisée) :

\begin{figure}
\begin{center}
\psset{xunit=1cm,yunit=1cm,comma=true,labelFontSize=\scriptstyle}
\begin{pspicture}(-0.1,-1)(8,6)
\multido{\n=0+1,\na=2.4+-0.1}{9}{\uput[d](\n,0){\scriptsize\np{\na}}}
\multido{\n=0+1,\na=0+2}{6}{\uput[l](0,\n){\scriptsize\na}}
\def\f{x 0.0325 x mul 0.2775 add mul}
\def\g{x x 7 add mul 60 div}
\psplot[plotpoints=2000,linewidth=1.25pt,linecolor=lightgray]{0}{8}{\f}
\psplot[plotpoints=2000,linewidth=1.25pt,linecolor=lightgray]{0}{8}{\g}
\pscustom[fillstyle=solid,fillcolor=lightgray!50,linecolor=lightgray]
{\psplot[plotpoints=2000,linewidth=0pt]{0}{8}{\f}
\psplot[plotpoints=2000,linewidth=0pt]{8}{0}{\g}
}
\uput[r](0,5.3){\footnotesize Consommation supplémentaire (en \%)}
\uput[d](6.5,-0.4){\footnotesize Pression des pneus (en bar)}
\uput[d](6.5,-0.8){\footnotesize source : \emph{ \blue www.eco-drive.ch}}
\psgrid[subgriddiv=1, gridlabels=0, gridcolor=lightgray](0,0)(8,5)
\psplot[plotpoints=2000]{0}{8}{\f}
\psplot[plotpoints=2000]{0}{8}{\g}
\psaxes[linewidth=1pt,labels=none,ticksize=0pt 0pt](0,0)(0,0)(8,5)
\end{pspicture}
\end{center}
\end{figure}

D'après le graphique, pour des pneus gonflés à $1,9$~bar alors que la pression recommandée est de $2,4$~bars, donner un encadrement approximatif du pourcentage de la consommation supplémentaire de carburant.
	\end{enumerate}
\item  Paul a remarqué que lorsque les pneus étaient correctement gonflés, sa voiture consommait
en moyenne $6$~L aux $100$~km. Il décide de s'inscrire à un stage d'éco-conduite afin de diminuer
sa consommation de carburant et donc l'émission de CO$_2$. En adoptant les principes de l'écoconduite, un conducteur peut diminuer sa consommation de carburant d'environ 15\,\%. Il
souhaite, à l'issue du stage, atteindre cet objectif.
	\begin{enumerate}
		\item Quelle sera alors la consommation moyenne de la voiture de Paul ?
		\item Sachant qu'il effectue environ \np{20000}~km en une année, combien de litres de carburant
peut-il espérer économiser ?
		\item Sa voiture roule à l'essence sans plomb. Le prix moyen est 1,35 €/L. Quel serait alors le montant de l'économie réalisée sur une année ?
		\item Ce stage lui a coûté $200$~\euro. Au bout d'un an peut-il espérer amortir cette dépense ?
	\end{enumerate}
\end{enumerate}

\bigskip

