
\medskip

%Hugo a téléchargé des titres musicaux sur son téléphone. Il les a classés par genre musical
%comme indiqué dans le tableau ci-dessous :
%
\begin{center}
\begin{tabularx}{0.7\linewidth}{l*{4}{>{\centering \arraybackslash}X}}\toprule
Genre musical 		&Pop 	&Rap 	&Techno &Variété\\ \midrule
Nombre de titres 	&35 	&23 	&14 	&28\\ \bottomrule
\end{tabularx}
\end{center}

\begin{enumerate}
\item %Combien de titres a-t-il téléchargés?
Hugo a téléchargé :

$35 + 23 + 14 + 28 = 100$ titres.
\item %Il souhaite utiliser la fonction \og lecture aléatoire\fg{} de son téléphone qui consiste à choisir au hasard parmi tous les titres musicaux téléchargés, un titre à diffuser. Tous les titres
%sont différents et chaque titre a autant de chances d'être choisi. On s'intéresse au genre
%musical du premier titre diffusé.
	\begin{enumerate}
		\item %Quelle est la probabilité de l'évènement: \og Obtenir un titre Pop\fg{} ?
La probabilité est égale à $\dfrac{35}{100} = 0,35$.
		\item %Quelle est la probabilité de l'évènement \og Le titre diffusé n'est pas du Rap \fg{} ?
La probabilité est égale à $\dfrac{100 - 23}{100} = \dfrac{77}{100} = 0,77$.	
		\item %Un fichier musical audio a une taille d'environ 4 Mo (Mégaoctets). Sur le téléphone
%d'Hugo, il reste $1,5$ Go (Gigaoctet) disponible. 

%Il souhaite télécharger de nouveaux titres musicaux. Combien peut-il en télécharger au maximum ?
1,5 Go = \np{1500}~Mo. Il peut donc encore télécharger $\dfrac{\np{1500}}{4} = 375$~titres.
%\emph{Rappel}: 1 Go = \np{1000}~Mo
	\end{enumerate}
\end{enumerate}

\vspace{0,5cm}

