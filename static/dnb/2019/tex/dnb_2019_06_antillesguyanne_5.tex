
\medskip

Voici deux programmes de calcul:

\medskip

\parbox{0.48\linewidth}{\begin{center}\textbf{PROGRAMME A}

\psset{unit=1cm}
\begin{pspicture}(0,0.5)(6,6)
%\psgrid
\rput(3,5){Choisir un nombre}\psframe(1.4,4.7)(4.6,5.2)\psline{->}(2.5,4.7)(2,3.3)
\rput(2,3){Multiplier par 4}\psframe(0.8,2.8)(3.3,3.3)\psline{->}(3,4.7)(4.,4.3)
\rput(4,4){Soustraire 2}\psframe(3,3.8)(5,4.3)\psline{->}(2,2.8)(2.5,1.2)
\rput(4,2){Élever au carré}\psframe(2.8,1.7)(5.2,2.2)\psline{->}(4,3.8)(4,2.2)
\rput(3,1){ Ajouter les deux nombres}\psframe(1,0.7)(5,1.2)\psline{->}(4,1.7)(3.5,1.2)
\end{pspicture}
\end{center}}\hfill
\parbox{0.48\linewidth}{\begin{center}\textbf{PROGRAMME B}
\vspace{1.5cm}
\psset{unit=1cm}
\begin{pspicture}(6,4)
%\psgrid
\uput[r](1,3.5){$\bullet~~$Choisir un nombre}
\uput[r](1,2.5){$\bullet~~$Calculer son carré}
\uput[r](1,1.5){$\bullet~~$Ajouter 6 au résultat.}
\psframe(1,3.8)(4.6,1.2)
\end{pspicture}
\end{center}}

\medskip

\begin{enumerate}
\item 
	\begin{enumerate}
		\item Montrer que, si l'on choisit le nombre $5$, le résultat du programme A est $29$.
		\item Quel est le résultat du programme B si on choisit le nombre 5 ?
	\end{enumerate}
\item Si on nomme $x$ le nombre choisi, expliquer pourquoi le résultat du programme A peut s'écrire $x^2 + 4$.
\item Quel est le résultat du programme B si l'on nomme $x$ le nombre choisi ?
\item Les affirmations suivantes sont-elles vraies ou fausses? Justifier les réponses et écrire les étapes des éventuels calculs :
	\begin{enumerate}
		\item \og Si l'on choisit le nombre $\dfrac{2}{3}$, le résultat du programme B est $\dfrac{58}{9}$. \fg
		\item \og Si l'on choisit un nombre entier, le résultat du programme B est un nombre entier impair. \fg
		\item \og Le résultat du programme B est toujours un nombre positif. \fg
		\item \og Pour un même nombre entier choisi, les résultats des programmes A et B sont ou bien tous les deux des entiers pairs, ou bien tous les deux des entiers impairs. \fg
	\end{enumerate}
\end{enumerate}

\bigskip

