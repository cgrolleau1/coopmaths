
\medskip

\begin{enumerate}
\item %Calculer $5x^2 - 3(2x+1)$ pour $x = 4$.
$5 \times 4^2 - 3(2\times 4+1) = 5 \times 16 - 3\times 9 = 80 - 27 = 53$.
\item %Montrer que, pour toute valeur de $x$, on a: $5x^2 - 3(2x + 1) = 5x^2 - 6x - 3$. 
$5x^2 - 3(2x + 1) = 5x^2 - 3 \times 2x - 3 \times 1 = 5x^2 - 6x - 3$.
\item %Trouver la valeur de $x$ pour laquelle $5x^2 - 3(2x+1)= 5x^2 - 4x +1$.
D'après la question précédente : $5x^2 - 3(2x+1)= 5x^2 - 4x +1$ peut s'écrire :

$5x^2 - 6x - 3 = 5x^2 - 4x +1$ ou en ajoutant $- 5x^2$ à chaque membre :

$- 6x - 3 = - 4x + 1$ et en ajoutant $6x$ à chaque membre :

$- 3 = 2x + 1$ et en ajoutant $- 1 $ à chaque membre :

$- 4 = 2x$ et en multipliant chaque membre par $\dfrac{1}{2}$ :

$- 2 =x$.
(\emph{Rem.} : $5\times (- 2)^2 - 3(2 \times (- 2) + 1) = 20 + 9 = 29 $ et 
$5\times (- 2)^2 - 4\times (- 2) +1 = 20 + 8 + 1 = 29$.)
\end{enumerate}



\vspace{0,5cm}

