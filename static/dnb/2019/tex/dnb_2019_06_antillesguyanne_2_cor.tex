
\medskip

%\og S'orienter à 90 \fg{} signifie que l'on se tourne vers la droite.
%
%Mathieu, Pierre et Elise souhaitent tracer le motif ci-dessous à l'aide de leur ordinateur. Ils commencent tous par le \textbf{script commun} ci-dessous, mais écrivent un script \textbf{Motif} différent.
%\medskip
%
%\parbox{0.4\linewidth}{\textbf{Script commun} aux trois élèves
%
%{\small \begin{scratch}
%\blockinit{Quand  \greenflag est cliqué}
%\blockmove{aller à x: \ovalnum{-160} y: \ovalnum{-100}}
%\blockmove{s’orienter à \ovalnum{90\selectarrownum}}
%\blockpen{effacer tout}
%\blockpen{mettre la taille du stylo à \ovalnum{4}}
%\blockpen{stylo en position d’écriture}
%\blocklook{Motif}
%\end{scratch}}
%} \hfill
%\parbox{0.57\linewidth}{\hspace{3cm}\textbf{Motif}
%
%\psset{unit=0.5cm}
%\begin{pspicture}(10,9)
%\multido{\n=6+1}{5}{\psline[linecolor=blue](\n,3)(\n,8)}
%\multido{\n=3+1}{6}{\psline[linecolor=blue](6,\n)(10,\n)}
%\pspolygon[linewidth=2.3pt](7,5)(8,5)(8,4)(9,4)(9,7)(8,7)(8,6)(7,6)
%\rput(2,2){\footnotesize Point de départ}
%\rput(8,2.5){\footnotesize Le quadrillage a des}
%\rput(8,2){\footnotesize carreaux qui mesurent}
%\rput(8,1.5){\footnotesize 10 pixels de côté.}
%\psline{->}(3,2.2)(8,4)
%\end{pspicture}}
% 
%\bigskip
%
%\begin{tabularx}{\linewidth}{|*{3}{>{\centering \arraybackslash}X|}}\hline
%\textbf{Motif de Mathieu}& \textbf{Motif de Pierre}& \textbf{Motif d'Elise}\\
%{\small \begin{scratch}
%\initmoreblocks{définir \namemoreblocks{Motif}}
%\blockmove{avancer de \ovalnum{10}}
%\blockmove{tourner \turnleft{} de \ovalnum{90} degrés}
%\blockmove{avancer de \ovalnum{30}}
%\blockmove{tourner \turnleft{} de \ovalnum{90} degrés}
%\blockmove{avancer de \ovalnum{20}}
%\blockrepeat{répéter \ovalnum{2} fois}
%{\blockmove{tourner \turnleft{} de \ovalnum{90} degrés}
%\blockmove{avancer de \ovalnum{10}}
%}
%\blockmove{tourner \turnright{} de \ovalnum{90} degrés}
%\blockmove{avancer de \ovalnum{20}}
%\end{scratch}}
%&
%{\small \begin{scratch}
%\initmoreblocks{définir \namemoreblocks{Motif}}
%\blockmove{avancer de \ovalnum{10}}
%\blockmove{tourner \turnleft{} de \ovalnum{90} degrés}
%\blockmove{avancer de \ovalnum{30}}
%\blockrepeat{répéter \ovalnum{2} fois}
%{\blockmove{tourner \turnleft{} de \ovalnum{90} degrés}
%\blockmove{avancer de \ovalnum{10}}
%\blockmove{tourner \turnleft{} de \ovalnum{90} degrés}
%\blockmove{avancer de \ovalnum{10}}
%\blockmove{tourner \turnleft{} de \ovalnum{90} degrés}
%\blockmove{avancer de \ovalnum{10}}
%}
%\blockmove{tourner \turnleft{} de \ovalnum{90} degrés}
%\end{scratch}}&
%{\small \begin{scratch}
%\initmoreblocks{définir \namemoreblocks{Motif}}
%\blockmove{avancer de \ovalnum{10}}
%\blockmove{tourner \turnleft{} de \ovalnum{90} degrés}
%\blockmove{avancer de \ovalnum{30}}
%\blockrepeat{répéter \ovalnum{2} fois}
%{
%\blockmove{tourner \turnleft{} de \ovalnum{90} degrés}
%\blockmove{avancer de \ovalnum{10}}
%\blockmove{tourner \turnleft{} de \ovalnum{90} degrés}
%\blockmove{avancer de \ovalnum{10}}
%\blockmove{tourner \turnright{} de \ovalnum{90} degrés}
%\blockmove{avancer de \ovalnum{10}}
%}
%\blockmove{tourner \turnleft{} de \ovalnum{90} degrés}
%\end{scratch}}\\ \hline
%\end{tabularx}
%
%\medskip

\begin{enumerate}
\item ~%Tracer le motif de Mathieu en prenant comme échelle : 1 cm pour 10 pixels.
\begin{center}
\psset{unit=1cm}
\begin{pspicture}(3,3)
\psline(1,0)(2,0)(2,3)(0,3)(0,2)(1,2)(1,0)
\end{pspicture}
\end{center}
\item %Quel élève a un script permettant d'obtenir le motif souhaité ? On ne demande pas de justifier.
C'est le motif d'Élise.
\item%~
	\begin{enumerate}
		\item La rotation centrée au point commun des quatre motifs (au centre de la figure et  de $+ 90\degres$ permet de passer de 1 à 2 de 2 à 3 de 3 à 4.
		\item\begin{scratch}
\blockrepeat{répéter \ovalnum{4} fois}
{\blocklook{Motif}
\blockmove{tourner \turnleft{} de \ovalnum{90} degrés}
}
\end{scratch}	
	\end{enumerate}

%\hspace{0.5cm}\parbox{0.6\linewidth}{ 
%On utilise ce motif pour obtenir la figure ci-contre.

%Quelle transformation du plan permet de passer à la fois du
%motif 1 au motif 2, du motif 2 au motif 3 et du motif 3 au
%motif 4 ?
%Modifier le \textbf{script commun} à partir de la ligne 7 incluse
%pour obtenir la figure voulue. On écrira sur la copie
%uniquement la partie modifiée. Vous pourrez utiliser
%certaines ou toutes les instructions suivantes :

%\parbox{0.31\linewidth}{\psset{unit=0.5cm}
%\begin{pspicture}(8,8)
%\psgrid[gridlabels=0pt,subgriddiv=1,gridcolor=blue]
%\def\Te{\pspolygon[linewidth=1.6pt](0,0)(3,0)(3,1)(2,1)(2,2)(1,2)(1,1)(0,1)}
%\rput(4,3){\Te}\rput{90}(5,4){\Te}\rput{-90}(3,4){\Te}\rput{-180}(4,5){\Te}
%\rput(2.5,4.5){2}
%\rput(4.5,5.5){1}
%\rput(3.5,2.5){3}
%\rput(5.5,3.5){4}
%\end{pspicture}
%}
%
%\medskip
%
%\begin{scratch}
%\blockrepeat{répéter \ovalnum{2} fois}
%{}
%\end{scratch} \begin{scratch}\initmoreblocks{\namemoreblocks{Motif}}\end{scratch}
%
%\begin{scratch}\blockmove{tourner \turnleft{} de \ovalnum{} degrés}\end{scratch}
%\begin{scratch}\blockmove{avancer de \ovalnum{}}\end{scratch}
%\begin{scratch}\blockmove{tourner \turnright{} de \ovalnum{} degrés}\end{scratch}

\item  %Un élève trace les deux figures A et B que vous trouverez en \textbf{ANNEXE 1.1}

%Placer sur cette annexe, \textbf{qui est à rendre avec la copie}, le centre O  de la symétrie centrale qui transforme la figure A en figure B.
Voir sur l'annexe le point en rouge.
\end{enumerate}
\begin{center}
\textbf{\large ANNEXE 1 - A rendre avec la copie}

\vspace{2cm}


\textbf{ANNEXE 1.1}

\bigskip

\psset{unit=0.5cm}
\begin{pspicture}(16,10)
\multido{\n=0+1}{17}{\psline[linewidth=0.3pt,linecolor=blue](\n,0)(\n,10)}
\multido{\n=0+1}{11}{\psline[linewidth=0.3pt,linecolor=blue](0,\n)(16,\n)}
\def\Te{\pspolygon[linewidth=1.6pt](0,0)(3,0)(3,1)(2,1)(2,2)(1,2)(1,1)(0,1)}
\rput(4,3){\Te}\rput{90}(5,4){\Te}\rput{-90}(3,4){\Te}\rput{-180}(4,5){\Te}
\rput(12,5){\Te}\rput{90}(13,6){\Te}\rput{-90}(11,6){\Te}\rput{-180}(12,7){\Te}
\rput(3.5,0.5){Figure A}
\rput(11.5,2.5){Figure B}
\psdots[dotstyle=+,dotangle=45,dotscale=2,linecolor=red,linewidth=2pt](8,5)
\end{pspicture}


\end{center}

