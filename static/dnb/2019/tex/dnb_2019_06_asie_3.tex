
\medskip

\parbox{0.8\linewidth}{Un programme permet à un robot de se déplacer sur les cases
d'un quadrillage. Chaque case atteinte est colorée en gris. Au
début d'un programme, toutes les cases sont blanches, le robot se
positionne sur une case de départ indiquée par un \og d \fg{} et la
colore aussitôt en gris.}\hfill
\parbox{0.18\linewidth}{
\psset{unit=1cm}
\begin{pspicture}(-1.5,-1.4)(1.5,1.4)
\uput[r](0.8,0){E}\uput[u](0,0.8){N}\uput[l](-0.8,0){W}\uput[d](0,-0.8){S}
%%%%%%%%
\psset{unit=0.15cm}
\multido{\iA=0+90}{4}{
\rput{\iA}(0,0){\pspolygon[fillstyle=solid,fillcolor=black](0,0)(0,-6.5)(0.9,-0.9)}
\rput{\iA}(0,0){\pspolygon[fillstyle=solid,fillcolor=white](0,0)(0,-6.5)(-0.9,-0.9)}
}%%%%%%%%
\end{pspicture}
}

\medskip
%
Voici des exemples de programmes et leurs effets:
%
\medskip

\begin{tabularx}{\linewidth}{|l|X|>{\centering \arraybackslash}X|}\hline
$\bullet~~$ 1W& Le robot avance de 1 case vers l'ouest.&\psset{unit=0.5cm}
\begin{pspicture*}(4.5,3)
\psframe[fillstyle=solid,fillcolor=lightgray](1,1)(3,2)
\psgrid[gridlabels=0,subgriddiv=1,gridwidth=0.4pt]
\rput(2.5,1.5){\textbf{d}}
\end{pspicture*} \\ \hline
$\bullet~~$ 2E 1W 2N &Le robot avance de 2 cases vers l'est, puis de 1 case vers l'ouest,
puis de 2 cases vers le nord.&\psset{unit=0.5cm}
\begin{pspicture*}(5,5)
\pspolygon[fillstyle=solid,fillcolor=lightgray](1,1)(4,1)(4,2)(3,2)(3,4)(2,4)(2,2)(1,2)
\psgrid[gridlabels=0,subgriddiv=1,gridwidth=0.4pt]
\rput(1.5,1.5){\textbf{d}}
\end{pspicture*}\\ \hline
$\bullet~~$ 3 (1S 2E)& Le robot répète 3 fois le déplacement suivant:

\og avancer de 1 case vers le sud puis de 2 cases vers l'est \fg, 

Soit 3 fois :

\psset{unit=0.5cm}
\begin{pspicture*}(5,3.5)
\pspolygon[fillstyle=solid,fillcolor=lightgray](1,1)(4,1)(4,2)(2,2)(2,3)(1,3)
\psgrid[gridlabels=0,subgriddiv=1,gridwidth=0.4pt]
\rput(1.5,2.5){\textbf{d}}
\end{pspicture*}& \psset{unit=0.5cm}
\begin{pspicture*}(8.5,4.5)
\pspolygon[fillstyle=solid,fillcolor=lightgray](0,4)(1,4)(1,3)(3,3)(3,2)(5,2)(5,1)(7,1)(7,0)(4,0)(4,1)(2,1)(2,2)(0,2)
\psgrid[gridlabels=0,subgriddiv=1,gridwidth=0.4pt]
\rput(0.5,3.5){\textbf{d}}
\end{pspicture*} \\ \hline
\end{tabularx}

\medskip

\begin{enumerate}
\item Voici un programme :

\textbf{Programme}  : 1W 2N 2E 4S 2W

On souhaite dessiner le motif obtenu avec ce programme.

Sur votre copie, réaliser ce motif en utilisant des carreaux, comme dans les exemples
précédents. On marquera un \og \textbf{d} \fg{} sur la case de départ.
\item  Voici deux programmes:

\textbf{Programme \no 1}  : 1S 3(1N 3E 2S)

\textbf{Programme \no 2}  : 3(1S 1N 3E 1S)

\parbox{0.4\linewidth}{\textbf{a.~} Lequel des deux programmes permet d'obtenir le motif ci-contre ?

\textbf{b.~} Expliquer pourquoi l'autre programme ne permet pas d'obtenir le motif ci-contre.}\hfill
\parbox{0.4\linewidth}{\psset{unit=0.5cm}
\begin{pspicture}(11,5)
\pspolygon[fillstyle=solid,fillcolor=lightgray](0,3)(1,3)(1,4)(3,4)(3,2)(4,2)(4,3)(6,3)(6,1)(7,1)(7,2)(9,2)(9,1)(10,1)(10,3)(7,3)(7,4)(4,4)(4,5)(0,5)
\psgrid[gridlabels=0,subgriddiv=1,gridwidth=0.2pt]
\rput(0.5,4.5){\textbf{d}}
\end{pspicture}}

\item  Voici un autre programme:

\textbf{Programme \no 3} : 4(1S 1E 1N)

Il permet d'obtenir le résultat suivant:

\psset{unit=0.5cm}
\begin{pspicture*}(6,2.5)
\psframe[fillstyle=solid,fillcolor=lightgray](5,2)
\psgrid[gridlabels=0,subgriddiv=1,gridwidth=0.2pt]
\rput(0.5,1.5){\textbf{d}}
\end{pspicture*}

Réécrire ce programme \no 3 en ne modifiant qu'une seule instruction afin d'obtenir ceci :

\psset{unit=0.5cm}
\begin{pspicture}(10,2.5)
\pspolygon[fillstyle=solid,fillcolor=lightgray](0,2)(1,2)(1,1)(2,1)(2,2)(3,2)(3,1)(4,1)(4,2)(5,2)(5,1)(6,1)(6,2)(7,2)(7,1)(8,1)(8,2)(9,2)(9,0)(0,0)
\psgrid[gridlabels=0,subgriddiv=1,gridwidth=0.2pt]
\rput(0.5,1.5){\textbf{d}}
\end{pspicture}
\end{enumerate}

\vspace{0,5cm}

