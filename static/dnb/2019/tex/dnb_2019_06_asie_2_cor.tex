
\medskip

%Le tableau ci-dessous présente les émissions de gaz à effet de serre pour la France et
%l'Union Européenne, en millions de tonnes équivalent CO$_2$, en 1990 et 2013.
%
%\begin{center}
%\begin{tabularx}{\linewidth}{|l|*{2}{>{\centering \arraybackslash}X|}}\cline{2-3}
%\multicolumn{1}{c|}{~}&1990 (en millions de tonnes équivalent CO$_2$)&2013
% (en millions de tonnes équivalent CO$_2$)\\ \hline
%France &549,4 &490,2\\ \hline
%Union Européenne &\np{5680,9}& \\ \hline
%\multicolumn{3}{r}{\small \emph{Source: Agence européenne pour l'environnement, $2015$}}
%\end{tabularx}
%\end{center}
%\smallskip

\begin{enumerate}
\item %Entre 1990 et 2013, les émissions de gaz à effet de serre dans l'Union Européenne
%ont diminué de 21\,\%.

%Quelle est la quantité de gaz à effet de serre émise en 2013 par l'Union Européenne ?

%Donner une réponse à $0,1$ million de tonnes équivalent CO$_2$ près.
Baisser de 21\,\% c'est multiplier par $\left(1 - \dfrac{21}{100}  \right) = \dfrac{100 - 21}{100} = \dfrac{79}{100} = 0,79$, donc la quantité de gaz à effet de serre émise en 2013 par l'Union Européenne est égale à :

$\np{5680,9} \times 0,79 = \np{4487,91} \approx \np{4487,9}$ millions de tonnes à 0,1 près.
\item  %La France s'est engagée d'ici 2030 à diminuer de $\dfrac{2}{5}$ ses émissions de gaz à effet de serre par rapport à 1990.

%Justifier que cela correspond pour la France à diminuer d'environ $\dfrac{1}{3}$ ses émissions de gaz à effet de serre par rapport à 2013.
Diminuer de $\dfrac{2}{5}$ ses émissions de 1990 revient à produire encore $1 - \dfrac{2}{5} =  \dfrac{3}{5} =  \dfrac{6}{10} = 0,6$.

La France devra donc produire en 2030 au plus :

$549,4 \times 0,6 = 329,64$.

De même diminuer de $\dfrac{1}{3}$ ses émissions de 2013 revient à produire encore $1 - \dfrac{1}{3} =  \dfrac{2}{3}$.

La France devra donc produire en 2030 au plus :

$490,2 \times \dfrac{2}{3} = 326,8$. À 3 près l'affirmation est correcte.
\end{enumerate}

\bigskip

