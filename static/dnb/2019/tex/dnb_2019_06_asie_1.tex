
\medskip

Nina et Claire ont chacune un programme de calcul.

\begin{center}
\begin{tabularx}{\linewidth}{|X|X|}\hline
\textbf{Programme de Nina}&\textbf{Programme de Claire}\\
Choisir un nombre de départ&Choisir un nombre de départ\\
Soustraire 1.&Multiplier ce nombre par $- \dfrac{1}{2}$\\
Multiplier le résultat par $-2$&Ajouter 1 au résultat\\
Ajouter 2.&\\ \hline
\end{tabularx}
\end{center}
\smallskip

\begin{enumerate}
\item Montrer que si les deux filles choisissent 1 comme nombre de départ, Nina
obtiendra un résultat final 4 fois plus grand que celui de Claire.
\item  Quel nombre de départ Nina doit-elle choisir pour obtenir $0$ à la fin ?
\item  Nina dit à Claire: \og Si on choisit le même nombre de départ, mon résultat sera
toujours quatre fois plus grand que le tien \fg.

A-t-elle raison ?
\end{enumerate}

\bigskip

