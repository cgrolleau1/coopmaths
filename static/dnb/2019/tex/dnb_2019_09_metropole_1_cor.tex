
\medskip

%Michel participe à un rallye VIT sur un parcours balisé. Le trajet est représenté en traits pleins.
%
%Le départ du rallye est en A et l'arrivée est en G.
%
%\begin{center}
%\psset{unit=1cm}
%\begin{pspicture}(10,8)
%%\psgrid
%\psline[linewidth=1.25pt](0.5,7.5)(4.5,7.5)(9,0.5)(11,0.5)%AB(D)FG
%\uput[u](0.5,7.5){A} \uput[u](4.5,7.5){B}\uput[u](6,7.5){C}
%\uput[ur](6,5){D}\uput[d](6,0.5){E}\uput[d](9,0.5){F}
%\uput[d](11,0.5){G}
%\psline[linestyle=dashed,linewidth=1.25pt](4.5,7.5)(6,7.5)(6,0.5)(9,0.5)%BCEF
%\uput[u](5.25,7.5){1,5 km}\uput[u](2.5,7.5){7 km}
%\uput[r](6,6.25){2 km}\uput[l](6,2.75){5 km}\uput[d](10,0.5){3,5 km}
%\uput[r](0,4.7){Le dessin n'est pas à l'échelle.}
%\uput[r](0,4){Les points A, B et C sont alignés.}
%\uput[r](0,3.3){Les points C, D et E sont alignés.}
%\uput[r](0,2.6){Les points B, D et F sont alignés.}
%\uput[r](0,1.9){Les points E, F et G sont alignés.}
%\uput[r](0,1.2){Le triangle BCD est rectangle en C.}
%\uput[r](0,0.5){Le triangle DEF est rectangle en E.}
%\end{pspicture}
%\end{center}
%
%\medskip

\begin{enumerate}
\item %Montrer que la longueur BD est égale à $2,5$~km.
Le triangle BCD est rectangle en C. Le théorème de Pythagore permet d'écrire :

$\text{BD}^2 = \text{BC}^2 + \text{CD}^2$, soit $\text{BD}^2 = 1,5^2 + 2^2 = 2,25 + 4 = 6,25 = 2,5^2$.

Donc BD $ = 2,5$~km.
\item %Justifier que les droites (BC) et (EF) sont parallèles.
C, D et E sont alignés ; le triangle  BCD est rectangle en C, donc la droite (BC) est perpendiculaire à la droite (CE).

Le triangle DEF est rectangle en E, donc la droite (EF) est perpendiculaire à la droite (CE).

Conclusion : les droites (BC) et (EF) étant perpendiculaires à la droite (CE) sont parallèles.  
\item %Calculer la longueur DF.
D'après le résultat précédent on peut appliquer le théorème de Thalès :

$\dfrac{\text{DF}}{\text{DB}} = \dfrac{\text{DE}}{\text{DC}}  = \dfrac{\text{EF}}{\text{BC}}$, soit  

$\dfrac{\text{DF}}{\text{2,5}} = \dfrac{\text{5}}{\text{2}}$, d'où en multipliant chaque membre par 2,5 :

DF $ = 2,5 \times 2,5 = 6,25$~km.
\item %Calculer la longueur totale du parcours.
La longueur totale du parcours est égale à :

$\text{AB} + \text{BD} + \text{DF} + \text{FG} = 7 + 2,5 + 6,25 + 3,5 = 19,25$~km.
\item %Michel roule à une vitesse moyenne de $16$~km/h pour aller du point A au point B.

%Combien de temps mettra-t-il pour aller du point A au point B ?

%Donner votre réponse en minutes et secondes.
Michel parcourt 16 km en 60~min ou 4~km en 15~min ou 1~km en $\dfrac{15}{4}$~min.

Pour parcourir 7~km, il mettra donc $7 \times \dfrac{15}{4} = \dfrac{105}{4}$~min soit $\dfrac{105}{4} \times 60 = \np{1575}~$s soit 26~min 15~s.
\end{enumerate}

\bigskip

