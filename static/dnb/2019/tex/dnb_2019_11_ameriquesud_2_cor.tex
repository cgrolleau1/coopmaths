
\medskip

%On a saisi dans un tableur les dépenses liées au transport des familles françaises pour les années 2013 et 2015. Ces dépenses sont exprimées en milliards d'euros.
%
%Pour l'année 2013, on a aussi saisi dans ce tableur les dépenses totales annuelles qui correspondent aux dépenses liées au logement, au transport, à la santé, à l'éducation, etc.
%
%Voici une copie de l'écran obtenu.
%
%Par exemple : en 2015, les dépenses annuelles des familles françaises, liées à l'achat de carburant, ont été de $34$ milliards d'euros.
%
%\begin{center}
%\begin{tabularx}{\linewidth}{|c|m{5cm}|*{2}{>{\centering \arraybackslash}X|}}\hline
%&\multicolumn{1}{|c|}{A}				&B &C\\ \hline
%1&Dépenses annuelles liées au transport	&Année 2013	&Année 2015\\ \hline
%2&Achat de véhicules particuliers 		&38			&39\\ \hline
%3&Frais d'entretien des véhicules 		&45 		&51\\ \hline
%4& Achat de carburant 					&39 		&34\\ \hline
%5&Achat de services de transports 
%(avion; train, etc.)					&26 		&28\\ \hline
%6&Total pour le budget transport		&148		&152\\ \hline
%7&										&			&\\ \hline
%8&Dépenses totales annuelles			&\np{1498}	& \\ \hline
%\multicolumn{4}{r}{\emph{D'après une source: INSEE}}
%\end{tabularx}
%\end{center}
%
%\smallskip

\begin{enumerate}
\item %Pour l'année 2015, quelle est la dépense des familles françaises liée aux frais d'entretien des véhicules?
Les frais d'entretien des véhicules ont représenté en 2015 51 milliards d'euros.
\item %Quelle formule peut-on saisir dans la cellule B6 avant de l'étirer dans la cellule C6 ?
=SOMME(B2:B5).
\item %À la lecture du tableau, les dépenses annuelles liées à l'achat de carburant ont-elles baissé de 5\,\% entre 2013 et 2015 ?
La baisse des dépenses de carburant est égale à :

$\dfrac{39 - 34}{39} \times 100 = \dfrac{5}{39}\times 100$, soit environ 12,8\,\%, donc beaucoup plus de 5\,\%.
\item %En 2015, les dépenses des familles françaises liées aux transports correspondaient à environ 9,87\,\% des dépenses totales annuelles.

%Quelles étaient alors les dépenses totales annuelles des familles françaises en 2015 ?
Si $t$ est le montant des dépenses totales en 2015, on a :

$\dfrac{9,87}{100} \times t = 152$ soit en multipliant chaque membre par $\dfrac{100}{9,87}$ :

$t = 152 \times \dfrac{100}{9,87} = \dfrac{\np{15200}}{9,87}\approx \np{1540,0}$ milliards d'euros.
\end{enumerate}

\vspace{0,5cm}

