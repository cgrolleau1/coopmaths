
\medskip

Hugo a téléchargé des titres musicaux sur son téléphone. Il les a classés par genre musical
comme indiqué dans le tableau ci-dessous :

\begin{center}
\begin{tabularx}{0.7\linewidth}{|l|*{4}{>{\centering \arraybackslash}X|}}\hline
Genre musical &Pop &Rap &Techno &Variété\\ \hline
Nombre de titres &35 &23 &14 &28\\ \hline
\end{tabularx}
\end{center}

\begin{enumerate}
\item Combien de titres a-t-il téléchargés?
\item  Il souhaite utiliser la fonction \og lecture aléatoire\fg{} de son téléphone qui consiste à choisir
au hasard parmi tous les titres musicaux téléchargés, un titre à diffuser. Tous les titres
sont différents et chaque titre a autant de chances d'être choisi. On s'intéresse au genre
musical du premier titre diffusé.
	\begin{enumerate}
		\item Quelle est la probabilité de l'évènement: \og Obtenir un titre Pop\fg{} ?
		\item Quelle est la probabilité de l'évènement \og Le titre diffusé n'est pas du Rap \fg{} ?
		\item Un fichier musical audio a une taille d'environ 4 Mo (Mégaoctets). Sur le téléphone
d'Hugo, il reste $1,5$ Go (Gigaoctet) disponible. 

Il souhaite télécharger de nouveaux titres musicaux. Combien peut-il en télécharger au maximum ?

\emph{Rappel}: 1 Go = \np{1000}~Mo
	\end{enumerate}
\end{enumerate}


