
\medskip

%Nina et Claire ont chacune un programme de calcul.
%
%\begin{center}
%\begin{tabularx}{\linewidth}{|X|X|}\hline
%\textbf{Programme de Nina}&\textbf{Programme de Claire}\\
%Choisir un nombre de départ&Choisir un nombre de départ\\
%Soustraire 1.&Multiplier ce nombre par $- \dfrac{1}{2}$\\
%Multiplier le résultat par $-2$&Ajouter 1 au résultat\\
%Ajouter 2.&\\ \hline
%\end{tabularx}
%\end{center}
%\smallskip

\begin{enumerate}
\item %Montrer que si les deux filles choisissent 1 comme nombre de départ, Nina obtiendra un résultat final 4 fois plus grand que celui de Claire.
$\bullet~~$Nina obtient successivement : $1 \to 1 - 1 = 0 \to 0 \times (- 2) = 0 \to 2$ ;

$\bullet~~$Claire obtient successivement : $1 \to 1 \times \left(- \dfrac{1}{2} \right) = - \dfrac{1}{2} \to - \dfrac{1}{2} + 1 =  \dfrac{1}{2}$.
Or $2 = 4 \times \left(\dfrac{1}{2} \right)$ : le résultat de Nina est quatre fois plus grand que elui de Claire.
\item  %Quel nombre de départ Nina doit-elle choisir pour obtenir $0$ à la fin ?
En partant de $0$ et en faisant les opérations inverses du programme on obtient :

$0 \gets 0 - 2 = - 2 \gets - 2 \times \left(- \dfrac{1}{2}\right) = 1 \gets 1 + 1 = 2$.

En partant de 2 Nina obtiendra 0.
\item  %Nina dit à Claire: \og Si on choisit le même nombre de départ, mon résultat sera
%toujours quatre fois plus grand que le tien \fg.

%A-t-elle raison ?
$\bullet~~$En partant de $x$ quelconque Nina obtient successivement :

$x \to x - 1 \to -2(x - 1) = - 2x + 2 \to - 2x + 2 + 2 = 4 - 2x$.

$\bullet~~$En partant de $x$ quelconque Claire obtient successivement :

$x \to x \times \left(- \dfrac{1}{2}\right) \to 1 - \dfrac{x}{2}$.

Or $4\left(1 - \dfrac{x}{2} \right) = 4 - 2x $. Nina a  raison.
\end{enumerate}

\bigskip

