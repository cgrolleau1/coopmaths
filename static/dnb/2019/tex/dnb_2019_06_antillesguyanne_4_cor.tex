
\medskip

%Leila est en visite à Paris. Aujourd'hui, elle est au Champ de Mars où l'on peut voir la tour Eiffel dont la hauteur totale BH est $324$~m.
%
%Elle pose son appareil photo au sol à une distance AB = 600 m du monument et le programme pour prendre une photo (voir le dessin ci-dessous).
%
%\medskip

\begin{enumerate}
\item %Quelle est la mesure, au degré près, de l'angle $\widehat{\text{HAB}}$ ?
La Tour Eiffel est en principe verticale : le triangle ABH est donc rectangle en B et dans ce triangle on a $\tan \widehat{\text{HAB}} = \dfrac{\text{côté opposé}}{\text{côté adjacent}} = \dfrac{324}{600} = \dfrac{6 \times 54}{6 \times 100} = \dfrac{54}{100} = 0,54$.

La calculatrice donne $\widehat{\text{HAB}} \approx 28,369$, soit 28\degres au degré près.
\item %Sachant que Leila mesure $1,70$ m, à quelle distance AL de son appareil doit-elle se placer pour paraître aussi grande que la tour Eiffel sur sa photo ?
Leila étant en position verticale le segment la représentant est parallèle au segment [BH].

On peut donc d'après la propriété de Thalès :

$\dfrac{\text{hauteur de Leila}}{\text{BH}} = \dfrac{\text{AL}}{\text{AB}}$, soit

$\dfrac{1,70}{324} = \dfrac{\text{AL}}{600}$. on a donc :

AL $ = 600\times \dfrac{1,70}{324} \approx 3,148$~(m) soit 3,15~m au centimètre près.

%Donner une valeur approchée du résultat au centimètre près.
\end{enumerate}

%\begin{center}
%\psset{unit=0.9cm,arrowsize=2pt 3}
%\begin{pspicture}(10,5)
%\pspolygon(1,1)(9,1)(9,4.5)%ABL
%\uput[dl](1,1){A}\uput[d](9,1){B}\uput[ur](9,4.5){H}
%\uput[u](1,1){appareil photo}\uput[u](3.,1){Leila}\rput{90}(9.5,2){Tour Eiffel}
%\uput[d](3,1){L}
%\psline(3,1)(3,1.88)
%\rput(5,-0.2){Le dessin n'est pas à l'échelle}
%\uput[u](4.5,0.4){AB  = 600 m}\rput{90}(9.5,3.75){324~m}
%\psline[linewidth=0.3pt]{<->}(1,0.4)(9,0.4)
%\end{pspicture}
%\end{center}

\bigskip

