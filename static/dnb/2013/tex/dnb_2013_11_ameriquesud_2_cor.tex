
\medskip 

%Jean-Michel est propriétaire d'un champ, représenté par le triangle ABC ci-dessous. Il achète à son voisin le champ adjacent, représenté par le triangle ADC. On obtient ainsi un nouveau champ formé par le quadrilatère ABCD. 
%
%\medskip
%
%\parbox{0.6\linewidth}{Jean Michel sait que le périmètre de son champ ABC est de 154 mètres et que BC = 56 m. 
%
%Son voisin l'informe que le périmètre du champ ADC est de 144 mètres et que AC = 65 m. 
%
%De plus, il sait que AD = 16 m.} \hfill
%\parbox{0.38\linewidth}{\psset{unit=1.25cm}
%\begin{pspicture}(-0.8,0)(3,2.5)
%\pspolygon(0.3,0.2)(2.6,0.2)(1.6,2.1)(0.3,1.55)%DCBA
%\psline (2.6,0.2)(0.3,1.55)
%\uput[ul](0.3,1.55){A} \uput[u](1.6,2.1){B} \uput[dr](2.6,0.2){C} \uput[dl](0.3,0.2){D} 
%\end{pspicture}}
%
%\medskip 

\begin{enumerate}
\item 
	\begin{enumerate}
		\item %Justifier que les longueurs AB et DC sont respectivement égales à 33 m et 63 m.
On a AB + BC + CA = 154 soit AB $+ 56 + 65 = 154$, d'où AB $ = 154 - 121 = 33$~m.

De même AD + DC + CA = 144 soit $16 + \text{DC} + 65 = 144$, d'où DC $ = 144 - 81 = 63$~m. 
		\item %Calculer le périmètre du champ ABCD.
Le périmètre du champ ABCD est égal à AB + BC + CD + DA $ = 33 + 56 + 63 + 16 = 168$~m.
	\end{enumerate}
\item %Démontrer que le triangle ADC est rectangle en D.
On a $\text{AD}^2 + \text{DC}^2  = 16^2 + 63^2 = 256 + \np{3969} = \np{4225}$.

D'autre part $\text{AC}^2 = 65^2 = \np{4225}$.

On a donc $\text{AD}^2 + \text{DC}^2  = \text{AC}^2$ ce qui signifie d'après la réciproque de Pythagore que le triangle ADC est rectangle en D.

%On admet que le triangle ABC est rectangle en B. 
\item %Calculer l'aire du champ ABCD.
On a $\mathcal{A}(\text{ABC}) = \dfrac{1}{2} \times \text{AB} \times \text{BC} = \dfrac{33 \times 56}{2} = 33 \times 28 = 924$.

De même  $\mathcal{A}(\text{ADC}) = \dfrac{1}{2} \times \text{AD} \times \text{DC} = \dfrac{16 \times 63}{2} = 63 \times 8 = 504$.

Donc  $\mathcal{A}(\text{ABCD}) = 924 + 504 = \np{1428}$~m$^2$.
\item %Jean-Michel veut clôturer son champ avec du grillage. Il se rend chez son commerçant habituel et tombe sur l'annonce suivante: 

%\begin{center}\psset{unit=1cm} \begin{pspicture}(5,1)
%\psframe(5,1) \rput(2.5,0.5){Grillage : 0,85~\euro{} par mètre}
%\end{pspicture}
%\end{center} 
%
%Combien va-t-il payer pour clôturer son champ?
Il doit payer au moins :  $\np{168} \times 0,85 = \np{142,80}$~\euro.
\end{enumerate}

\bigskip

