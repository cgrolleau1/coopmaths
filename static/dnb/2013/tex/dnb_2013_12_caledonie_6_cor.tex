
\bigskip 

%Un vendeur de bain moussant souhaite faire des coffrets pour les fêtes de fin d'année. 
% 
%En plus du traditionnel \og pavé moussant \fg, il veut positionner par dessus une \og pyramide moussante\fg{} qui ait le même volume que le pavé. 
%
%Les schémas suivants donnent les dimensions ($h$ désigne la hauteur de la pyramide) :  
%
%\begin{center}
%\psset{unit=0.8cm}
%\begin{tabularx}{\linewidth}{*{3}X}
%\begin{pspicture}(5,4)
%\psframe(0,0.4)(2.7,1.3)
%\psline(2.7,0.4)(4.3,1.2)(4.3,2.1)(2.7,1.3)
%\psline(4.3,2.1)(1.6,2.1)(0,1.3)
%\rput(1.35,0.85){\small Pavé moussant}
%\uput[d](1.35,0.4){20 cm}
%\uput[dr](3.4,0.8){20 cm}
%\uput[r](4.3,1.6){8 cm}
%\end{pspicture}&\begin{pspicture}(5,4)
%\pspolygon(0,0.4)(2.8,0.4)(4.4,1.2)(2.3,3.1)
%\psline(2.8,0.4)(2.3,3.1)
%\psline[linestyle=dashed](0,0.4)(4.4,1.2)
%\psline[linestyle=dashed](2.8,0.4)(1.6,1.2)(2.3,3.1)
%\psline[linestyle=dashed](0,0.4)(1.6,1.2)(4.4,1.2)
%\psline[linestyle=dashed](2.3,3.1)(2.25,0.85)
%\uput[d](1.2,0.4){20 cm}
%\uput[dr](3.5,0.8){20 cm}\uput[l](2.3,1.6){$h$}
%\rput{48}(1.2,2.2){\small Pyramide moussante}
%\end{pspicture}&\begin{pspicture}(5,4)
%\psframe(0,0)(3,1)
%\psline(3,0)(4.8,0.9)(4.8,1.9)(3,1)
%\psline(4.8,1.9)(2.4,3.9)(3,1)
%\psline(2.4,3.9)(0,1)
%\rput(1.5,0.5){\small Pavé moussant}
%\rput{50}(1.5,2.4){\small Pyramide}
%\rput{50}(1.9,2){\small moussante}
%\end{pspicture}\\
%\end{tabularx}
%\end{center}
% 
%On rappelle les formules suivantes: 
%
%\setlength\parindent{8mm}
%\begin{itemize}
%\item[$\bullet~~$] $V_{\text{pavé}} = \text{Longueur} \times \text{largeur} \times \text{hauteur}$  
%\item[$\bullet~~$] $V_{\text{pyramide}} = \dfrac{\text{aire de la base} \times \text{hauteur}}{3}$ 
%\end{itemize}
%\setlength\parindent{0mm}
%
%\medskip

\begin{enumerate}
\item %Calculer le volume d'un \og pavé moussant \fg.
D'après la formule $20 \times 20 \times 8 = \np{3200}$~cm$^3$. 
\item %Montrer que le volume d'une \og pyramide moussante \fg{} est égal à $\dfrac{400 h}{3}$ cm$^3$.
le volume de la pyramide est égal à $\dfrac{20 \times 20 \times h}{3} = \dfrac{400h}{3}$~cm$^3$. 
\item %En déduire la hauteur qu'il faut à une pyramide pour qu'elle ait le même volume qu'un pavé.
Les deux volumes sont égaux si :

$\np{3200} =  \dfrac{400h}{3}$ soit $400h = 3 \times \np{3200}$ ou $h = 3 \times 8 = 24$~cm.

Le chapeau sera trois fois plus haut que le pavé !
\end{enumerate} 

\bigskip 

