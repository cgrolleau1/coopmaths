
\medskip

%Caroline souhaite s'équiper pour faire du roller.
% 
%Elle a le choix entre une paire de rollers gris à 87~\euro{} ? et une paire de rollers noirs à 99~\euro.
% 
%Elle doit aussi acheter un casque et hésite entre trois modèles qui coûtent respectivement 45~\euro, 22~\euro{} et 29~\euro.
%
%\medskip
 
\begin{enumerate}
\item ~
%Si elle choisit son équipement (un casque et une paire de rollers) au hasard, quelle est la probabilité pour que l'ensemble lui coûte moins de 130~\euro{}?
\begin{center}
\pstree[treemode=R,nodesepA=0pt,nodesepB=2.5pt]{\TR{}}
{\pstree{\TR{87}}
	{\TR{45 $\to$ 132}
	\TR{22   $\to$ 109}
	\TR{29 $\to$ 116}
	}
\pstree{\TR{99}}
	{\TR{45  $\to$ 144}
	\TR{22  $\to$ 121}
	\TR{29  $\to$ 128}
	}
}
\end{center}
Sur les six possibilités quatre reviennent à moins de 130~\euro. La probabilité est donc égale à : $\dfrac{4}{6} = \dfrac{2}{3}$. 
\item %Elle s'aperçoit qu'en achetant la paire de rollers noirs et le casque à 45~\euro, elle bénéficie d'une réduction de 20\,\% sur l'ensemble.
Prix avant réduction : $99 + 45 = 144$~\euro 
	\begin{enumerate}
		\item %Calculer le prix en euros et centimes de cet ensemble après réduction.
Avoir 20\,\% de réduction c'est payer 80\,\% du prix initial soit : 

$0,80 \times 144 = 115,20$~\euro. 
		\item %Cela modifie-t-il la probabilité obtenue à la question 1 ? Justifier la réponse.
Avec cette réduction le prix passe en dessous de 130~\euro ; la probabilité est donc maintenant égale à  $\dfrac{5}{6}$.
	\end{enumerate}
\end{enumerate}
 
\bigskip

