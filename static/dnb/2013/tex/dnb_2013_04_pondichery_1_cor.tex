
\medskip

%Quatre affirmations sont données ci-dessous :
 
Affirmation 1 : $\left(\sqrt{5} - 1 \right)\left(\sqrt{5} + 1\right) = \left(\sqrt{5} \right)^2 - 1^2 = 5 - 1 = 4$. Vraie.% est un nombre entier. 

\medskip

Affirmation 2 : %4 n'admet que deux diviseurs.
4 admet 1 ; 2 ; 4 comme diviseurs. Fausse.
\medskip
 
Affirmation 3 : %Un cube, une pyramide à base carrée et un pavé droit totalisent 17 faces.
Le cube a 6 faces, la pyramide à base carrée, 5 et le pavé droit 6 : au total 17 faces. Vraie.
\medskip
 
Affirmation 4 : 

%\parbox{0.5\linewidth}{Les droites (AB) et (CD) sont parallèles.}\hfill
%\parbox{0.4\linewidth}{
%\psset{unit=1.25cm}\begin{pspicture}(4,3)
%\pspolygon(1,0.5)(3.5,0.5)(0.5,2.5)(2.7,2.5)(1,0.5)(0.5,2.5)(3.5,0.5)(2.7,2.5)%DCABDACB
%\uput[ul](0.5,2.5){A} \uput[ur](2.7,2.5){B} \uput[r](3.5,0.5){C} \uput[l](1,0.5){D} \uput[u](1.9,1.5){O}
%\rput{-34}(1.4,2.1){\footnotesize 2,8 cm} \rput{-34}(2.65,1.3){\footnotesize 5 cm}
%\rput{50}(1.4,1.3){\footnotesize 3,5 cm}\rput{50}(2.1,2.2){\footnotesize 2 cm} 
%\end{pspicture}
%}
%\medskip
%
%\emph{Pour chacune des affirmations, indiquer si elle est vraie ou fausse en argumentant la réponse.}
Si les droites sont parallèles on devrait avoir d'après le théorème de Thalès :

$\dfrac{\text{OA}}{\text{OC}} = \dfrac{\text{OB}}{\text{OD}}$ ou $\dfrac{2,8}{5} = \dfrac{2}{3,5}$ . Or $2,8 \times 3,5 = 9,8$ et $5 \times 2 = 10$. Les quotients ne sont pas égaux. Fausse.

\bigskip

