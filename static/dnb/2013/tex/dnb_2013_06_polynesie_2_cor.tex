
\medskip

\begin{enumerate}
\item %Calcule PGCD(405~;~315). Précise la méthode utilisée et indique les calculs.
On a successivement avec l'algorithme d'Euclide :

$405 = 315 \times 1 + 90$ ;

$315 = 90 \times 3 + 45$ ; 

$90 = 45 \times 2$.

On a donc  PGCD(405~;~315) = 45.
\item %Dans les bassins d'eau de mer filtrée d'une ferme aquacole de bénitiers destinés à l'aquariophilie, on compte $9$ bacs contenant chacun $35$ bénitiers de $12,5$~cm et $15$~bacs contenant chacun $27$~bénitiers de $17,5$~cm.
 
%L'exploitant souhaite répartir la totalité des bénitiers en des lots de même composition :
 
%Par lot, même nombre de bénitiers de $12,5$~cm et même nombre de bénitiers de $17,5$~cm. 
On a donc$9\times 35 = 315$ petits bénitiers et $15 \times 27 = 405$ grands bénitiers.
	\begin{enumerate}
		\item %Quel est le plus grand nombre de lots qu'il pourra réaliser ? Justifie ta réponse.
D'après la question précédente on pourra faire 45 lots . 
		\item %Quelle sera la composition de chaque lot ?
Chaque lot contient  7 petits et 9 grands 
	\end{enumerate}
\end{enumerate}

\bigskip

