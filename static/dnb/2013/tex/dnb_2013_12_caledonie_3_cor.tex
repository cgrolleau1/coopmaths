
\bigskip
 
%Un restaurant propose cinq variétés de pizzas, voici leur carte :
%
%\begin{center}
%\renewcommand\arraystretch{1.9}
%\begin{tabularx}{0.9\linewidth}{|p{3cm}X|}\hline 
%\textbf{CLASSIQUE :}& tomate, jambon, oeuf, champignons\\ \hline 
%\textbf{MONTAGNARDE :}& crème, jambon, pomme de terre, champignons\\ \hline 
%\textbf{LAGON :}& crème, crevettes, fromage\\ \hline 
%\textbf{BROUSSARDE :}& crème, chorizo, champignons, salami\\ \hline 
%\textbf{PLAGE :}& tomate, poivrons, chorizo\\ \hline
%\end{tabularx}
%\renewcommand\arraystretch{1.9}
%\end{center}
 
\begin{enumerate}
\item %Je commande une pizza au hasard, quelle est la probabilité qu'il y ait des champignons dedans ? 
Sur les cinq variétés trois contiennent des champignons ; la probabilité est donc égale à $\dfrac{3}{5} = \dfrac{6}{10} = 0,6$.
\item %J'ai commandé une pizza à la crème, quelle est la probabilité d'avoir du jambon ?
Sur les trois variété à la crème, une seule contient du jambon : la probabilité est donc égale à $\dfrac{1}{3}$. 
\item ~
%Il est possible de commander une grande pizza composée à moitié d'une variété et à moitié d'une autre. Quelle est la probabilité d'avoir des champignons sur toute la pizza ? On pourra s'aider d'un arbre des possibles.

\begin{center}
\psset{unit=0.5cm,levelsep=1cm,treesep=0.25cm}
%\begin{pspicture}(-4,0)(4,5)
\pstree[nodesep=2.5pt]{\TR{}}
{\pstree{\TR{c*}}
	{\TR{m*}
	\TR{l}
	\TR{b*}
	\TR{p}
	}
\pstree{\TR{m*}}
	{\TR{c*}
	\TR{l}
	\TR{b*}
	\TR{p}
	}
\pstree{\TR{l}}
	{\TR{c*}
	\TR{m*}
	\TR{b}
	\TR{p}
	}
\pstree{\TR{b*}}
	{\TR{c*}
	\TR{m*}
	\TR{l}
	\TR{p}
	}
\pstree{\TR{p}}
	{\TR{c*}
	\TR{m*}
	\TR{l}
	\TR{b*}
	}
}
%\end{pspicture}
\end{center}
On a marqué d'une étoile les variété qui contiennent des champignons. Sur les $5 \times 4 = 20$ choix possibles il y en a 6  qui contiennent chacune des champignons : la probabilité est donc de $\dfrac{6}{20} = \dfrac{3}{10} = 0,3$.
\item %On suppose que les pizzas sont de forme circulaire. La pizzeria propose deux tailles :

%\setlength\parindent{8mm} 
%\begin{itemize}
%\item[$\bullet~~$] moyenne : 30~cm de diamètre 
%\item[$\bullet~~$] grande  : 44~cm de diamètre.
%\end{itemize}
%\setlength\parindent{0mm} 
 
%Si je commande deux pizzas moyennes, aurai-je plus à manger que si j'en commande une grande ? Justifier la réponse.
Aire de deux moyennes : $2 \times \pi \times 15^2 = 450\pi$.

Aire d'une grande $\pi \times 22^2 = 484\pi$. La grande donne plus à manger que deux moyennes. 
\end{enumerate}

\bigskip

