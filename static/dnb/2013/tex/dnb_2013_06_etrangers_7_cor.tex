
\medskip

%On peut lire au sujet d'un médicament :
% 
%\og Chez les enfants (12 mois à 17 ans), la posologie doit être établie en fonction de la surface corporelle du patient [voir formule de Mosteller]. \fg
% 
%\og Une dose de charge unique de 70 mg par mètre carré (sans dépasser 70 mg par jour) devra être administrée \fg
%
%\medskip
% 
%Pour calculer la surface corporelle en m$^2$ on utilise la formule suivante : 
%
%Formule de Mosteller : Surface corporelle en m$^2 = \sqrt{\dfrac{\text{taille (en cm)} \times \text{masse (en kg}}{\np{3600}}}$.
%
%\bigskip
%
%On considère les informations ci-dessous :
%
%\medskip
%\begin{tabularx}{\linewidth}{|*{5}{>{\centering \arraybackslash}X|}}\hline 
%Patient &\^Age &Taille (m) &Masse (kg) &Dose administrée\\ \hline 
%Lou &5 ans &1,05 &17,5 &50 mg\\ \hline  
%Joé &15 ans &1,50 &50 &100 mg\\ \hline 
%\end{tabularx}
%\medskip 

\begin{enumerate}
\item %La posologie a-t-elle été respectée pour Joé ? Justifier la réponse. 
Pour Joé inutile d’utiliser la formule car on lui a administré 100~mg alors que le maximum journalier est de 70~mg. La posologie n’a pas  été respectée.
\item %Vérifier que la surface corporelle de Lou est environ de $0,71$ m$^2$. 
La formule donne pour Lou :

$\sqrt{\dfrac{105 \times 17,5}{\np{3600}}} \approx 0,714$~m$^2$ soit à peu près $0,71$ m$^2$.
\medskip

%\textbf{Dans cette question, toute trace de recherche, même incomplète, sera prise en compte dans l'évaluation.}
 
\item %La posologie a-t-elle été respectée pour Lou ? Justifier la réponse.
On pouvait donc administrer à Lou un maximum de $70 \times  0,71 = 49,7$~(g).

Le maximum est légèrement inférieur à la dose administrée, donc la posologie n’a pas été respectée mais le dépassement est insignifiant.
\end{enumerate}
