
Trois figures codées sont données ci-dessous. Pour chacune d'elles, déterminer la mesure de l'angle 
$\widehat{\text{ABC}}$.

\bigskip

%\begin{tabular}{|c|c|}\hline
%\multicolumn{2}{|c|}{
%\begin{tikzpicture}
%\tkzDefPoints{0/0/A,0/-3/C}
%\tkzDefPointBy[rotation= center C angle -60](A)\tkzGetPoint{B'}
%\tkzDefPointBy[homothety=center C ratio 2](B')\tkzGetPoint{B}
%\tkzDrawSegments(A,B B,C C,A)
%\tkzLabelPoint[below](C){$C$} \tkzLabelPoint[right](B){$B$} \tkzLabelPoint[left](A){$A$}
%\tkzLabelSegment[sloped](C,A){$AC=3$~cm}
%\tkzLabelSegment[sloped,below](C,B){$BC=6$~cm}
%\tkzMarkRightAngle(B,A,C)
%\tkzMarkAngle(A,B,C)\tkzLabelAngle[pos=-1.5,color=green](A,B,C){$30^{\circ}$}
%\tkzText(-4,0){Figure 1}
%\end{tikzpicture}}\\ \hline
%\begin{tikzpicture}
%\tkzDefPoints{0/0/O,2.7/0/O'}
%\tkzDefPoint(28:2.7){A} \tkzDefPoint(90:2.7){C} \tkzDefPoint(208:2.7){B}
%\tkzDrawCircle(O,O')
%\tkzDrawSegments(A,C C,B B,A O,C)
%\tkzMarkSegments[mark=||](O,A O,B O,C)
%\tkzLabelPoint[right](A){$A$} \tkzLabelPoint[below left](B){$B$}
%\tkzLabelPoint[above](C){$C$} \tkzLabelPoint[below](O){$O$}
%%\tkzMarkAngle(C,A,B)
%\tkzLabelAngle[color=blue,pos=-.6](C,A,B){$59^{\circ}$}
%%\tkzMarkAngle(O,C,A)
%\tkzLabelAngle[color=blue,pos=.6](O,C,A){$59^{\circ}$}
%\tkzLabelAngle[color=blue,pos=.6](A,O,C){$62^{\circ}$}
%\tkzLabelAngle[color=blue,pos=-.6](C,O,B){$118^{\circ}$}
%%\tkzMarkAngle(A,B,C)
%\tkzLabelAngle[color=blue,pos=1.1](C,B,A){$31^{\circ}$}
%\tkzLabelAngle[color=blue,pos=1.1](B,C,O){$31^{\circ}$}
%\tkzText(-3,2.7){Figure 2}
%\end{tikzpicture}
%&
%\begin{tikzpicture}
%\tkzDefPoints{0/0/O,0/2.7/A}
%\tkzDefPointBy[rotation=center O angle 72](A) \tkzGetPoint{B}
%\tkzDefPointBy[rotation=center O angle 72](B) \tkzGetPoint{C}
%\tkzDefPointBy[rotation=center O angle 72](C) \tkzGetPoint{D}
%\tkzDefPointBy[rotation=center O angle 72](D) \tkzGetPoint{E}
%\tkzDrawCircle(O,A)
%\tkzDrawSegments(O,A O,B O,C O,D O,E A,B B,C C,D D,E E,A)
%\tkzLabelPoint[above](A){$A$} \tkzLabelPoint[left](B){$B$}
%\tkzLabelPoint[below left](C){$C$} \tkzLabelPoint[below right](D){$D$}
%\tkzLabelPoint[right](E){$E$} \tkzLabelPoint[right](O){$O$}
%\tkzMarkSegments[mark=|](A,B B,C C,D D,E E,A)
%\tkzMarkSegments[mark=||](O,A O,B O,C O,D O,E)
%%\tkzMarkAngle(C,B,A)
%\tkzLabelAngle[color=red,pos=.8](C,B,O){$54^{\circ}$}
%\tkzLabelAngle[color=red,pos=.8](O,B,A){$54^{\circ}$}
%\tkzLabelAngle[color=red,pos=.8](B,A,O){$54^{\circ}$}
%\tkzLabelAngle[color=red,pos=.8](A,O,B){$72^{\circ}$}
%\tkzText(-3,2.7){Figure 3}
%\end{tikzpicture}\\ \hline
%\end{tabular}

\begin{tabularx}{\linewidth}{|*{2}{>{\centering \arraybackslash}X|}}\hline
\multicolumn{2}{|c|}{\psset{unit=0.6cm}
\begin{pspicture}(10,6)
\psframe(1,4.9)(1.3,5.2)\psarc(9,5.2){8mm}{-180}{-150}
\rput(-4,5.5){Figure 1}
\pspolygon(1,0.5)(1,5.2)(9,5.2)
\uput[ul](1,5.2){A} \uput[ur](9,5.2){B} \uput[dl](1,0.5){C} 
\rput{90}(0.1,2.85){AC = 3cm} 
\rput{32}(5,2.5){BC = 6cm}\rput(7.4,4.85){?}
\end{pspicture}}\\ \hline 
\psset{unit=0.6cm}
\begin{pspicture}(-5,-5)(5,5.5)
\pscircle(0,0){4.5}
\pspolygon(4.5;20)(4.5;90)(4.5;200)
\uput[ur](4.5;20){A} \uput[dl](4.5;200){B} \uput[u](4.5;90){C}
\uput[dr](0,0){O}
\psline(-1.8,-0.4)(-1.6,-0.8)\psline(-1.9,-0.4)(-1.7,-0.8)
\psline(1.8,0.4)(1.6,0.8)\psline(1.9,0.4)(1.7,0.8)
\psline(-0.2,1.9)(0.2,1.9)\psline(-0.2,2)(0.2,2)
\psline(0;0)(4.5;90)
\rput(2.1,1.75){59\degres}
\rput(-4.25,4.5){Figure 2 }
\psarc(4.5;20){8mm}{-215}{-160}
\psarc(4.5;200){8mm}{20}{54} \rput(-2.8,-0.4){?}
\rput(0,-4.8){[AB] est un diamètre du cercle de centre O.} 
\end{pspicture}&\psset{unit=0.6cm}
\begin{pspicture}(-5,-5)(5,5)
\pscircle(0,0){4}
\pspolygon(4;20)(4;92)(4;164)(4;236)(4;308)
\uput[u](4;92){A}\uput[l](4;164){B}\uput[dl](4;236){C}\uput[dr](4;308){D}
\uput[ur](4;20){E}
\psarc(4;164){8mm}{-70}{40}\rput(-2.2,1){?}
\rput(-4.25,4.5){Figure 3}
\rput{20}(2;20){\psline(0,0.2)(0,-0.2)\psline(0.1,0.2)(0.1,-0.2)}
\rput{92}(2;92){\psline(0,0.2)(0,-0.2)\psline(0.1,0.2)(0.1,-0.2)}
\rput{164}(2;164){\psline(0,0.2)(0,-0.2)\psline(0.1,0.2)(0.1,-0.2)}
\rput{236}(2;236){\psline(0,0.2)(0,-0.2)\psline(0.1,0.2)(0.1,-0.2)}
\rput{308}(2;308){\psline(0,0.2)(0,-0.2)\psline(0.1,0.2)(0.1,-0.2)}
\multido{\n=20+72,\na=19+72,\nb=21+72}{5}{\psline(0;0)(4;\n)\psline(2;\na)(2;\nb)}
\uput[r](0,-0.08){O}
\psline(3,-0.8)(3.2,-1.)
\psline(1.7,2.5)(2,2.8)
\psline(-1.9,2.45)(-2,2.8)
\psline(-3.1,-1.3)(-2.9,-1.2)
\psline(0.1,-3.4)(0.05,-3.05)
\end{pspicture}\\ \hline
\end{tabularx}

\begin{description}
\item[Figure 1] Nous sommes dans un triangle rectangle. Nous pouvons donc utiliser la trigonométrie.
\[
\sin\widehat{ABC}=\frac{AC}{BC}=\frac{3}{6}=\frac{1}{2}\Longrightarrow \widehat{ABC}\ \text{mesure}\ \textcolor{green}{30^{\circ}}
\]
\item[Figure 2] Dans tout triangle isocèle, les angles à la base sont égaux. Ici:
\[
\widehat{OAC}=\widehat{OCA}\quad\text{et}\quad\widehat{BCO}=\widehat{CBO}=\widehat{ABC}
\]
Le point $O$ est le centre du cercle, car $OA=OB=OC$. La figure laisse supposer que les points $B$, $O$ et $A$ sont alignés (diamètre). Ainsi:
\[
\widehat{BOC}+\widehat{COA}=\widehat{BOA}=\text{angle plat de mesure}\ 180^{\circ}
\]
Enfin, la somme des mesures des angles dans un triangle vaut $180^{\circ}$.

Ainsi:
\[
\widehat{OCA}=\widehat{OAC}=59^{\circ}\ ;\ \widehat{AOC}=180-2\times 59=62^{\circ}\ ;\ \widehat{BOC}=180-62=118^{\circ}\ ;\ 2\widehat{ABC}=180-118=62\Longrightarrow \widehat{ABC}=\textcolor{blue}{31^{\circ}}
\]

\textbf{Autre méthode:} le triangle $ABC$ est rectangle en $C$ puisqu'inscrit dans un demi-cercle.
\[
\widehat{OCA}=\widehat{OAC}=59^{\circ}\ ;\ \widehat{BCO}=\widehat{ABC}=90-59=\textcolor{blue}{31^{\circ}}
\]
\item[Figure 3] Le pentagone $ABCDE$ est régulier, (tous les côtés sont égaux), donc $\widehat{AOB}=\dfrac{360}{5}=72^{\circ}$.

En utilisant certaines propriétés énoncées plus haut, on obtient:
\[
\widehat{ABC}=2\widehat{ABO}=180-72=\textcolor{red}{108^{\circ}}
\]
\end{description}
