
\medskip
 
%On considère le programme de calcul suivant : 
%
%\begin{center}
%\begin{tabular}{|l|}\hline
%$\bullet~~$ Choisir un nombre\\
%$\bullet~~$ Ajouter $5$\\
%$\bullet~~$ Prendre le carré de cette somme\\ \hline
%\end{tabular}
%\end{center}

\begin{enumerate}
\item %Quel résultat obtient-on lorsqu'on choisit le nombre $3$ ? le nombre $- 7$ ?
$3 \to 3 + 5 = 8 \to 8^2 = 64$ ;

$- 7 \to -7 + 5 = - 2 \to (- 2)^2 = 4$. 
\item 
	\begin{enumerate}
		\item %Quel nombre peut-on choisir pour obtenir $25$ ?
On peut travailler à l'envers :
		
$\bullet~~$$25 \to 5 \to 5 - 5 = 0$  ou

$\bullet~~$$25 \to - 5 \to -5 - 5 = - 10$. 
		\item %Peut-on obtenir $- 25$ ? Justifier la réponse.
On ne peut pas trouver de résultat final négatif puisque celui-ci est un carré.
	\end{enumerate} 
\item %On appelle $f$ la fonction qui, au nombre choisi, associe le résultat du programme de calcul. 
	\begin{enumerate}
		\item %Parmi les fonctions suivantes, quelle est la fonction $f$ ? 

%\[\begin{array}{l l}
%	x \longmapsto x^2 + 25 &	x \longmapsto (x + 5)^2\\ 
%	x \longmapsto x^2 + 5& 	x \longmapsto 2(x + 5)
%\end{array}\]
C'est la fonction $	x \longmapsto (x + 5)^2$.	 
		\item %Est-il vrai que $- 2$ est un antécédent de $9$ ?
On a $- 2 \to - 2 + 5 = 3 \to 3^2 = 9$ : c'est faux. 
	\end{enumerate}		
\item
	\begin{enumerate}
		\item %Résoudre l'équation $(x + 5)^2 = 25$.
$(x + 5)^2 = 25$ si $(x + 5)^2 - 25 = 0$  ou $(x + 5)^2  - 5^2 = 0$ ou $(x + 5 + 5)(x + 5 - 5) = 0$ et enfin $x(x + 10) = 0$ d'où 
$\left\{\begin{array}{l c l}
x	&=&0\\
x+10&=&0\\
\end{array}\right.$

Il y a donc deux solutions $0$ et $- 10$.
		\item %En déduire tous les nombres que l'on peut choisir pour obtenir $25$ à ce programme de calcul. 
	\end{enumerate}
\end{enumerate}

\bigskip

