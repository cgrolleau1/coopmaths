
\medskip 

Le diagramme en bâtons ci-dessous nous renseigne sur le nombre de buts marqués lors de la seconde édition de la coupe de l'Outre-Mer de football en 2010. 
Nombre de buts marqués par ligue 

\begin{center} 
\psset{xunit=1cm,yunit=0.4cm}
\begin{pspicture}(-2,-6)(9,19)
\multido{\n=0+2}{9}{\psline[linewidth=0.2pt](0,\n)(9,\n)}
\psaxes[Dx=20,Dy=2](0,0)(9,16)
\psline(0,0)(0,16)
\psframe[fillstyle=solid,fillcolor=lightgray](0.75,0)(1.25,8)
\psframe[fillstyle=solid,fillcolor=lightgray](1.75,0)(2.25,9)
\psframe[fillstyle=solid,fillcolor=lightgray](2.75,0)(3.25,8)
\psframe[fillstyle=solid,fillcolor=lightgray](3.75,0)(4.25,13)
\psframe[fillstyle=solid,fillcolor=lightgray](4.75,0)(5.25,2)
\psframe[fillstyle=solid,fillcolor=lightgray](5.75,0)(6.25,14)
\psframe[fillstyle=solid,fillcolor=lightgray](7.75,0)(8.25,3)
\rput{45}(0.4,-2){\footnotesize Guadeloupe}
\rput{45}(1.4,-2){\footnotesize Guyane}
\rput{45}(2.4,-2){\footnotesize Martinique}
\rput{45}(3.4,-2){\footnotesize Mayotte}
\rput{45}(4.3,-2.2){\footnotesize Nouvelle-Calédonie}
\rput{45}(5.4,-2){\footnotesize Réunion}
\rput{45}(6.3,-2.2){\footnotesize St-Pierre et Miquelon}
\rput{45}(7.4,-2){\footnotesize Tahiti}
\rput(4,18){Nombre de buts marqués par ligue}
\rput{90}(-1,9){Nombre de buts marqués}
\rput(4,-5){Ligues de l'Outre-Mer}
\end{pspicture}
\end{center}
 
\begin{enumerate}
\item Combien de buts a marqué l'équipe de Mayotte? 
\item Quelle est l'équipe qui a marqué le plus de buts? 
\item Quelle(s) équipe(s) ont marqué strictement moins de 8 buts? 
\item Quelle(s) équipe(s) ont marqué au moins 10 buts? 
\item Quel est le nombre total de buts marqués lors de cette coupe de l'Outre-Mer 2010 ? 
\item Calculer la moyenne de buts marqués lors de cette coupe de l'Outre-Mer 2010. 
\item Compléter les cellules B2 à B10 dans le tableau ci-dessous. 

\begin{center}
\begin{tabularx}{0.9\linewidth}{|c|*{2}{>{\centering \arraybackslash}X|}}\hline
&A & B \\ \hline
1 & Ligues de l'Outre~Mer &Nombre de buts marqués\\ \hline 
2&Guadeloupe& \\ \hline
3&Guyane& \\ \hline
4&Martinique& \\ \hline
5&Mayotte& \\ \hline
6&Nouvelle-Calédonie& \\ \hline
7&Réunion& \\ \hline
8&Saint Pierre et Miquelon& \\ \hline
9&Tahiti& \\ \hline
10&TOTAL& \\ \hline
11& Moyenne& \\ \hline
\end{tabularx}
\end{center}

\item Parmi les propositions suivantes, \textbf{entourer} la formule que l'on doit écrire dans la cellule B10 du tableau pour retrouver le résultat du nombre total de buts marqués. 

\begin{center}
\begin{tabularx}{0.9\linewidth}{|*{3}{>{\centering \arraybackslash}X|}}\hline
8+9+8+13+2+14+0+3& = TOTAL(B2:B9)& =SOMME(B2:B9) \\ \hline
\end{tabularx}
\end{center}

\item Écrire dans la cellule B11 du tableau précédent une formule donnant la moyenne des buts marqués.
\end{enumerate}
 
\bigskip

