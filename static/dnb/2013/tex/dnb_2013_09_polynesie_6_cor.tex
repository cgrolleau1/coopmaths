
\medskip

%L'île d'Aratika est au Nord de l'île de Fakarava.
% 
%A l'aide des documents suivants et de l'\textbf{Annexe 1} et en considérant que tous les vols entre Tahiti et les îles des Tuamotu se font à la même vitesse moyenne, placer avec le plus de précision possible l'île d'Aratika sur l'\textbf{Annexe 1} en expliquant en détail sur ta copie ta démarche.
%
%\medskip
% 
%\textbf{Pour cette question, toute trace de recherche, même incomplète, sera prise en compte dans l'évaluation.}
%
%\medskip
%
%\begin{tabular}{|l l|}\hline
%\multicolumn{2}{|c|}{Document 1 : Temps de vol entre Tahiti et les îles des Tuamotu (Nord) :}\\ 
%Tahiti--Rangiroa : 55 min&Tahiti--Ahe : 1 h 15 min \\
%Tahiti--Apataki : 1 h 05 min&Tahiti--Aratika: 1 h 15 min\\
%Tahiti--Arutua : 1 h 05 min&\\ \hline
%\end{tabular}
%
%\bigskip 
%  
%\begin{tabular}{|l l l|}\hline
%\multicolumn{3}{|c|}{Document 2 : Distance entre les îles :}\\ 
%Tahiti--Moorea : 17 km&Apataki--Arutua : 17 km&Tahiti--Bora Bora: 268 km\\  Fakarava--Aratika : 50 km&Tahiti--Raiatea : 210 km&Fakarava--Faaite : 21 km\\ Tahiti--Rangiroa : 355 km& Faaite--Anaa : 61 km& Tahiti--Huahine : 175 km\\ \hline
%\end{tabular} 
$\bullet~~$En utilisant les temps : Rangiroa est à 55 min de Tahiti qui correspond sur la carte à une distance de 6,7~cm.

Donc la distance sur la carte entre Tahiti et Aratika est égale à $4,7 \times \dfrac{75}{55} \approx 9,1$~cm.

La croix représentant sur l'annexe l'île d'Aratika est sur le cercle de centre Tahiti et de rayon la distance 9,1~cm.

$\bullet~~$En utilisant les distances : on peut par exemple utiliser le fait que Tahiti et Hahine sont distantes de 175 km que l'on mesure sur l'annexe (sur l'annexe plus bas : 4,4~cm).

La distance sur le plan entre Takarava et Aratika est donc égale à $4,2 \times \dfrac{50}{175} = 1,2$~cm.

Aratika est sur le plan sur le cercle centré en Fakarava de rayon 1,2.

Aratika est celui des deux points communs aux deux cercles qui est au Nord de Fakarava.
\begin{center}
\textbf{\large Annexe 1 :}

\vspace{2cm}

\psset{unit=0.6cm}
\begin{pspicture}(20,10)
%\psgrid
\psline[linewidth=1.25pt]{<->}(2,8)(4,8)
\psline[linewidth=1.25pt]{<->}(3,7)(3,9)
\uput[r](4,8){Est}\uput[l](2,8){Ouest}
\uput[u](3,9){Nord}\uput[d](3,7){Sud} 
\psdots[dotstyle=+,dotangle=45,dotscale=1.5](2.5,4.3)(4.5,3.9)(8,1.2)(12.1,6.5)(15.3,8.3)(16.7,4.6)
\psdots[dotstyle=+,dotscale=1.5,linecolor=red](16,5.55)
\uput[u](2.5,4.3){BORABORA} \uput[dr](4.5,3.9){HUAHINE} \uput[ur](8,1.2){TAHITI} 
\uput[u](12.1,6.5){RANGIROA} \uput[u](15.3,8.3){AHE} \uput[ur](16.7,4.6){FAKARAVA}  
%\psarc(8,1.2){6.7}{45}{70}
\psarc(8,1.2){9.1}{0}{80}
\pscircle(16.7,4.6){1.2}
\uput[ur](16,5.55){\red Aratika}  
\rput(10,-4){RENDRE TOUT LE SUJET AVEC VOTRE COPIE} 
\end{pspicture}
\end{center}
