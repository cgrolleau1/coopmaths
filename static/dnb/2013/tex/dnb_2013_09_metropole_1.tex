
\medskip

Lorsqu'on absorbe un médicament, la quantité de principe actif de ce médicament dans le sang évolue en fonction du temps. Cette quantité se mesure en milligrammes par litre de sang.
 
Le graphique ci-dessous représente la quantité de principe actif d'un médicament dans le sang, en fonction du temps écoulé, depuis la prise de ce médicament.

\begin{center}
\psset{xunit=1.25cm,yunit=0.25cm}
\begin{pspicture*}(-0.8,-2.5)(7.5,31)
\multido{\n=0.0+0.2}{38}{\psline[linestyle=dotted,dotsep=1.5pt,linewidth=0.8pt](\n,0)(\n,30)}
\multido{\n=0+2}{16}{\psline[linestyle=dotted,dotsep=1.5pt,linewidth=0.8pt](0,\n)(7.5,\n)}
\psaxes[linewidth=1.5pt,Dy=10](0,0)(0,0)(7.5,30.2)
\psaxes[linewidth=1.5pt,Dy=10]{->}(0,0)(0,0)(7.5,30)
\psplot[plotpoints=3000,linewidth=1.25pt,linecolor=blue]{0}{7.5}{73.3936 x mul 2.71828 x exp div}
\uput[r](0,30.3){Quantité de principe actif (en mg/L)}
\uput[d](6.75,-2){Temps écoulé (en h)}
\end{pspicture*} 
 
\end{center}
 
Répondre aux questions suivantes à partir de lectures graphiques. \textbf{Aucune justification n'est demandée dans cet exercice.}

\medskip
\begin{enumerate}
\item Au bout de combien de temps la quantité de principe actif de médicament dans le sang est-elle maximale ? 
\item Quelle est la quantité de principe actif de médicament dans le sang au bout de 2~h~30~min ? 
\item Pour que le médicament soit efficace, la quantité de principe actif de médicament dans le sang doit être supérieure à 5~mg/L.
 
Pendant combien de temps le médicament est-il efficace ? 
\end{enumerate}

\bigskip 

