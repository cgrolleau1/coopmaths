
\medskip
 
%On dispose d'un carré de métal de $40$ cm de côté. Pour fabriquer une boîte parallélépipèdique, on enlève à chaque coin un carré de côté $x$ et on relève les bords par pliage.
%
%\medskip
 
\begin{enumerate}
\item %Quelles sont les valeurs possibles de x ? 
On enlève en tout $2x$ de 40, donc $0 \leqslant x \leqslant 20$.
\item %On donne $x = 5$ cm. Calculez le volume de la boîte.

On a donc un pavé de fond carré de côtés mesurant $40 - 2 \times 5 = 30$ et de hauteur 5.

Le volume du pavé est donc égal à $30 \times 30 \times 5 = 900 \times 5 = \np{4500}$~cm$^3$. 
\item %Le graphique suivant donne le volume de la boîte en fonction de la longueur $x$.

%\emph{On répondra aux questions à l'aide du graphique.} 
	\begin{enumerate}
		\item %Pour quelle valeur de $x$, le volume de la boîte est-il maximum ?
Le maximum semble atteint pour $x = 6,5$. 
		\item %On souhaite que le volume de la boîte soit \np{2000}~cm$^3$. 
		
%Quelles sont les valeurs possibles de $x$ ?
La droite d'équation $y = \np{2000}$ coupe la courbe aux points d'abscisse 1,5 et 14.
	\end{enumerate} 
\end{enumerate}

\begin{center}
%\psset{unit=0.8cm}
%\begin{pspicture}(15,6)
%\pspolygon[fillstyle=solid,fillcolor=lightgray](2,0)(5.2,0)(5.2,0.9)(6,0.9)(6,4.1)(5.2,4.1)(5.2,4.9)(2,4.9)(2,4.1)(1.2,4.1)(1.2,0.9)(2,0.9)
%\psline[arrowsize=2pt 3]{<->}(1.2,5.2)(6,5.2)
%\psline[arrowsize=2pt 3]{<->}(0.7,4.1)(0.7,4.9)
%\psline[linestyle=dashed](1.2,4)(1.2,5.7)
%\psline[linestyle=dashed](0.3,4.9)(2,4.9)
%\psline[linestyle=dashed](6,4)(6,5.4)
%\psline[linestyle=dashed](1.2,4.1)(0.3,4.1)
%\pspolygon[fillstyle=solid,fillcolor=gray](9.3,2)(11.4,3.4)(13.3,2.85)(12.05,1.85)(12.05,0.8)(10.4,1.6)
%\pspolygon[fillstyle=solid,fillcolor=lightgray](7.7,2.5)(9.3,2)(10.4,1.6)(12.05,0.8)(9.2,1.7)%petit bord
%\pspolygon[fillstyle=solid,fillcolor=lightgray](12.05,0.8)(12.05,1.95)(14.45,3.65)(14.45,2.55)%bord vertical droit
%\pspolygon[fillstyle=solid,fillcolor=lightgray](11.4,3.4)(11.4,4.65)(14.45,3.65)(13.3,2.85)%bord vertical fond
%\pspolygon[fillstyle=solid,fillcolor=lightgray](8.7,2.2)(8.2,2.75)(10.5,4.5)(11.4,3.4)(9.3,2)%bord penché gauche
%\uput[u](3.6,5.3){40}
%\uput[l](0.7,4.5){$x$}
%\end{pspicture}

\vspace{0,5cm}
\psset{xunit=0.5cm,yunit=0.001cm}
\begin{pspicture}(-1,-500)(21,5500)
\multido{\n=0.0+0.5}{43}{\psline[linewidth=0.2pt,linecolor=orange](\n,0)(\n,5500)}
\multido{\n=0+250}{23}{\psline[linewidth=0.2pt,linecolor=orange](0,\n)(21,\n)}
\psaxes[linewidth=1.5pt,Dy=6000,labelFontSize=\scriptstyle]{->}(0,0)(21,5500)
\multido{\n=0+500}{11}{\uput[l](0,\n){\footnotesize \np{\n}}}
\uput[d](21,-200){$x$}
\uput[r](0,5500){volume de la boîte}
\psplot[plotpoints=8000,linewidth=1.25pt,linecolor=blue]{0}{20}{40 x 2 mul  sub dup mul x mul}
\uput[dl](0,0){O}
\end{pspicture}
\end{center} 

\bigskip

