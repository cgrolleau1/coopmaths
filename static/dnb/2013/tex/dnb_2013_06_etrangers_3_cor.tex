
\medskip

%\parbox{0.4\linewidth}{\psset{unit=0.75cm}
%\begin{pspicture}(-3.8,-4.5)(3.8,4.5)
%\pscircle(0,0){3.5}\psline[linestyle=dashed](0.8,3.4)(-0.8,-3.4)
%\pspolygon(1.8,-3)(-3.1,-1.6)(0.8,3.4)%CBA
%\uput[ur](0.8,3.4){A} \uput[dl](-3.1,-1.6){B} \uput[dr](1.8,-3){C} \uput[ur](0,0){O}\uput[d](-0.8,-3.4){M}
%\psdots(0,0)
%\rput(-1.6,1){5} 
%\end{pspicture}}\hfill
%\parbox{0.55\linewidth}{On considère un triangle ABC isocèle en A tel que l'angle $\widehat{\text{BAC}}$ mesure 50\degres{} et AB est égal à 5~cm.
% 
%On note O le centre du cercle circonscrit au triangle ABC. La droite (OA) coupe ce cercle, noté ($C$), en un autre point M.
%
%\medskip

\begin{enumerate}
\item %Quelle est la mesure de l'angle $\widehat{\text{BAM}}$ ? Aucune justification n'est demandée. 
Le triangle est isocèle en A, donc AB = AC.

O est le centre du cercle circonscrit au triangle, donc OA = OC.

Les deux points A et O sont équidistants de A et de C, donc la droite (AO) est la médiatrice de [BC]. C’est aussi la bissectrice de $\widehat{\text{BAC}}$, donc $\widehat{\text{BAM}} = 25\degres$.
\item %Quelle est la nature du triangle BAM ? Justifier. 
A et M sont diamétralement opposés. [AM] est un diamètre, donc le triangle ABM est un triangle rectangle en B.
\item %Calculer la longueur AM et en donner un arrondi au dixième de centimètre près.
Dans le triangle ABM rectangle en M, on a $\cos   \widehat{\text{BAM}} = \dfrac{\text{AB}}{\text{AM}}$ ; donc AM $ = \dfrac{\text{AB}}{ \cos  \widehat{\text{BAM}}} = \dfrac{5}{ \cos 25} \approx 5,51$ soit environ 5,5~cm au dixième près.
\item %La droite (BO) coupe le cercle ($C$) en un autre point K. Quelle est la mesure de l'angle $\widehat{\text{BKC}}$ ?
$\widehat{\text{BAC}} = \widehat{\text{BKC}}$ car ce sont des angles inscrits qui interceptent le même arc. Donc $\widehat{\text{BKC}} = 50\degres$.

%Justifier. 
\end{enumerate}

\bigskip

