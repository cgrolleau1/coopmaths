
\medskip  

%\textbf{Dans cet exercice, si le travail n'est pas terminé, laisser tout de même une trace de la recherche. Elle sera prise en compte dans l'évaluation.}
%
%\medskip 
%
%Le même jour, à la caisse d'un cinéma, un adulte et deux enfants payent $21$~\euro, deux adultes et trois enfants payent $36$~\euro. 
%
%Trois adultes et trois enfants vont au cinéma ce jour-là. Le caissier leur réclame $43$~\euro.
%
%\og Vous vous trompez! \fg{} s'exclame un des enfants. A-t-il raison ? Pourquoi ?
Soit $a$ le prix du billet adulte et $e$ le prix du billet enfant.

On a $a + 2e = 21$, soit $a = 21 - 2e$.

On a aussi $2a + 3e = 36$ ou $2(21 - 2e) + 3e = 36$ ou $42 - 4e + 3e = 36$ soit $42 - 36 = e$. Donc $e = 6$.

Un adulte paie donc $a = 21 - 2\times 6 = 21 - 12 = 9$.

Un adulte et un enfant payent $9 + 6 = 15$, donc trois adultes et trois enfants payent trois fois plus soit $3 \times 15 = 45$~\euro. L'enfant a raison.

\emph{Remarque} : on peut également résoudre le système :

$\left\{\begin{array}{l c l}
a + 2e&=&21\\
2a + 3e&=&36
\end{array}\right.$ ou encore $\left\{\begin{array}{l c l}
2a + 4e&=&42\\
2a + 3e&=&36
\end{array}\right.$ et par différence $e = 6$, puis 

$a = 21 - 2\times 6 = 21 - 12 = 9$.

Donc $3a + 3e = 27 + 18 = 45$
