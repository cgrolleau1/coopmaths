
\medskip

Le débit d'une connexion internet varie en fonction de la distance du modem par rapport au central téléphonique le plus proche.
 
On a représenté ci-dessous la fonction qui, à la distance du modem au central téléphonique (en kilomètres), associe son débit théorique (en mégabits par seconde).

\begin{center}
\psset{xunit=1.5cm,yunit=0.15cm}
\begin{pspicture}(-0.5,-2.5)(6.75,32.5)
\multido{\n=0+.25}{28}{\psline[linewidth=0.2pt,linecolor=orange](\n,0)(\n,32.5)}
\multido{\n=0+2.5}{14}{\psline[linewidth=0.2pt,linecolor=orange](0,\n)(6.75,\n)}
\psaxes[linewidth=1.5pt,Dy=5](0,0)(0,0)(6.75,32.5)
\uput[u](5.75,0){distance (en km)}
\uput[r](0,32){débit (en Mbit/s)}
\pscurve[linewidth=1.25pt,linecolor=blue](0,26.1)(0.5,25.4)(1,23.45)(1.5,20)(2,15)(2.5,9.94)(3,6.26)(3.5,3.76)(4,2.35)(4.5,1.18)(5,0.45)(6,0)(6.75,0)
\end{pspicture}
\end{center} 

\begin{enumerate}
\item Marie habite à 2,5~km d'un central téléphonique. Quel débit de connexion obtient-elle ? 
\item Paul obtient un débit de $20$ Mbits/s. 
À quelle distance du central téléphonique habite-t-il ? 
\item Pour pouvoir recevoir la télévision par internet, le débit doit être au moins de $15$ Mbits/s.
 
À quelle distance maximum du central doit-on habiter pour pouvoir recevoir la télévision par internet ? 
\end{enumerate}

\bigskip

