
\medskip

%[AB] est un segment de milieu O tel que AB = 12~cm.
%
%Le point C appartient au cercle de centre O passant par A. De plus AC = 6~cm
%
%L'angle $\widehat{\text{ABC}}$ mesure 30\,\degres.
%
%\medskip

\begin{enumerate}
\item %Construire la figure en vraie grandeur.
On construit :

\begin{itemize}
\item le segment [AB] tel que AB = 12~cm ;
\item sa médiatrice pour trouver son milieu O ;
\item le demi-cercle de centre O et de rayon 6~cm ;
\item le cercle de centre A et de rayon 6 coupe ce demi-cercle en C ;
\item on trace [AC] et [CB].
\end{itemize}

\item %Les affirmations suivantes sont-elles vraies ou fausses ? Justifier.
	\begin{enumerate}
		\item %Le triangle ABC est rectangle.
Le triangle ABC est inscrit dans un cercle qui admet pour diamètre l'un de ses côtés [AB] ; il est donc rectangle en C.
		\item Le segment [BC] mesure 10~cm.
On peut donc appliquer le théorème de Pythagore :

$\text{AC}^2 + \text{CB}^2 = \text{AB}^2$ ou $\text{CB}^2 = \text{AB}^2 - \text{AC}^2 = 12^2 - 6^2 = 144 - 36 = 108 \ne 100$ carré de 10. Donc [CB] ne mesure pas 10~cm.
		\item %L'angle $\widehat{\text{AOC}}$ mesure 60\,\degres.
$\widehat{\text{AOC}}$ est l'angle au centre qui intercepte l'arc $\widearc{\text{AC}}$ ; sa mesure est égale au double de celle dessangle inscrit qui intercepte le même arc soit $\widehat{\text{ABC}}$, donc l''angle $\widehat{\text{AOC}}$ mesure 60\,\degres.
		\item %L'aire du triangle ABC est $18\sqrt{3}$~cm$^2$.
On a vu que $\text{CB}^2 = 108 = 9 \times 12 = 9 \times 4 \times 3 = 36 \times 3$, donc 

$\text{CB} = \sqrt{108} = \sqrt{36 \times 3} = \sqrt{36} \times \sqrt{3} = 6\sqrt{3}$.

L'aire du triangle ABC est donc égale à :

$\dfrac{\text{AC} \times \text{CB}}{2} = \dfrac{6 \times 6\sqrt{3}}{2} = 18\sqrt{3}$~cm$^2$.
		\item %L'angle $\widehat{\text{BOC}}$ mesure 31\,\degres.
		Dans BOC, on a OB = OC : le triangle est donc isocèle et on a donc 
		
		$\widehat{\text{OBC}} = \widehat{\text{OCB}} = 30$. On en déduit que $\widehat{\text{BOC}} = 180 - 30 - 30 = 120$~\,\degres.
	\end{enumerate}
\end{enumerate}

\vspace{0.5cm}

