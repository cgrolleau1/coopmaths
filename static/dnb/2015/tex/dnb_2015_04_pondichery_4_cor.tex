
\medskip

%\parbox{0.65\linewidth}{La dernière bouteille de parfum de chez Chenal a la forme
%d'une pyramide SABC à base triangulaire de hauteur [AS] telle que :
%
%$\bullet~~$ABC est un triangle rectangle et isocèle en A ;
%
%$\bullet~~$AB = 7,5~cm et AS = 15~cm.
%
%\medskip
%
%\begin{enumerate}
%\item Calculer le volume de la pyramide SABC. (On arrondira au cm$^3$ près.)
%\item Pour fabriquer son bouchon SS$'$MN, les concepteurs ont
%coupé cette pyramide par un plan P parallèle à sa base et passant par le point S$'$ tel que SS$'$ = 6~cm.
%	\begin{enumerate}
%		\item Quelle est la nature de la section plane S$'$MN obtenue ?
%		\item Calculer la longueur S$'$N.
%	\end{enumerate}
%\item Calculer le volume maximal de parfum que peut contenir cette bouteille en cm$^3$.
%\end{enumerate}} \hfill
%\parbox{0.32\linewidth}{\psset{unit=0.75cm}
%\begin{pspicture}(5.5,12)
%%\psgrid
%\pspolygon(0.4,1.4)(0.4,10.9)(3.2,0.5)%ASB
%\psline(0.4,10.9)(5.2,1.4)(3.2,0.5)%SCB
%\psline[linestyle=dashed](0.4,1.4)(5.2,1.4)%AC
%\psline(0.4,7.1)(1.5,6.7)(2.3,7.1)%S'MN
%\psline[linestyle=dashed](0.4,7.1)(2.3,7.1)%S'N
%\psframe(0.4,1.4)(0.9,1.9)
%\psline(0.9,1.4)(1.4,1.2)(1,1.2)
%\uput[u](0.4,10.9){S}\uput[l](0.4,7.1){S$'$}
%\uput[dr](1.5,6.7){M}\uput[r](2.3,7.1){N}
%\uput[l](0.4,1.4){A}\uput[d](3.2,0.5){B}\uput[r](5.2,1.4){C}
%\rput(2.6,1.4){$\circ$}\rput(2,0.9){$\circ$}
%\end{pspicture}}
\begin{enumerate}
\item La base est un triangle rectangle isocèle de côtés mesurant 7,5~cm. L'aire de cette base est donc égale à $\dfrac{7,5 \times 7,5}{2}$.

La hauteur de la pyramide est égale à 15~cm, donc le volume de la pyramide est égal à :

$V_{\text{SABC}} = \dfrac{1}{3} \dfrac{7,5 \times 7,5}{2}\times 15 = 5\times  \dfrac{7,5 \times 7,5}{2} = 140,625$~cm$^3$ soit environ 141~cm$^3$ au cm$^3$ près.
\item 
	\begin{enumerate}
		\item Le plan de coupe étant parallèle à la base de la pyramide la section S$'$MN est une réduction de la base qui est un triangle rectangle isocèle ; S$'$MN est donc lui aussi un triangle rectangle isocèle.
		\item La pyramide SS$'$MN est une réduction de la pyramide SABC et le rapport de réduction est le rapport des hauteurs soit $\dfrac{\text{SS}'}{\text{SA}} = \dfrac{6}{15} = \dfrac{2}{5}$.
		
On a donc S$'\text{N} = \dfrac{2}{5} \times \text{AC} = \dfrac{2}{5} \times 7,5 = 3$~cm.
	\end{enumerate}
\item Le volume de la petite pyramide SS$'$MN peut s'obtenir de deux façons :
\begin{itemize}
\item Avec les dimensions :

$V_{\text{SS}'\text{MN}} = \dfrac{1}{3} \dfrac{3 \times 3}{2}\times 6 = 9$~cm$^3$.
\item Soit en utilisant le rapport de réduction. Si la grande pyramide a un volume de 140,625, la petite a un volume de :

$140,625 \times \left(\dfrac{2}{5}\right)^3 = 140,625 \times \dfrac{8}{125} = 9$~cm$^3$.
\end{itemize}

Dans tous les cas il reste un volume pour le parfum de :

\[140,625 - 9 = 131,625~\text{cm}^3.\]
\end{enumerate}

\vspace{0.5cm}

