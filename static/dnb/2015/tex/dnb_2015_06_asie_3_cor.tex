
\medskip

%Un bus transporte des élèves pour une compétition multisports. Il y a là 10 joueurs de
%ping-pong, 12 coureurs de fond et 18 gymnastes. Lors d'un arrêt, ils sortent du bus en
%désordre.
%
%\medskip

\begin{enumerate}
\item %Quelle est la probabilité que le premier sportif à sortir du bus soit un joueur de ping-pong ?
$10 + 12 +18 = 40$. Dans le bus, il y a 40 élèves.

La probabilité que le premier sportif à sortir du bus soit un joueur de ping-pong est de
$\dfrac{10}{40} = \dfrac{1}{4}  = 0,25$.
\item %Quelle est la probabilité que le premier sportif à sortir du bus soit un coureur ou un
%gymnaste ?
$1 - \dfrac{1}{4}  = \dfrac{4}{4} - \dfrac{1}{4} = \dfrac{3}{4} = 0,75$.

La probabilité que le premier sportif à sortir du bus soit un coureur ou un gymnaste est de
$\dfrac{3}{4}$.
\item %Après cet arrêt, ils remontent dans le bus et ils accueillent un groupe de nageurs.

%Sachant que la probabilité que ce soit un nageur qui descende du bus en premier
%est de 1/5, déterminer le nombre de nageurs présents dans le bus.
$\dfrac{1}{5} = \dfrac{10}{50} = \dfrac{10}{10+40}$.

Si 10 nageurs sont présents dans le bus, la probabilité
que le premier sportif à sortir du bus soit un nageur est
$\dfrac{1}{5}$.

Autre méthode : soit $n$ le nombre de nageurs ; on aura à la descente :

$\dfrac{1}{5} = \dfrac{n}{n + 40}$ soit $n + 40 = 5n$ ou $r4n = 40$ et enfin $n = 10$.
\end{enumerate}

\vspace{0,5cm}

