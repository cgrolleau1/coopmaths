
\medskip

%On lance deux dés tétraédriques, équilibrés et non truqués, dont les faces sont
%numérotées de 1 à 4. On calcule la somme des nombres lus sur chacune des faces sur
%lesquelles reposent les dés.
%
%\smallskip
%
%\np{1000} lancers sont simulés avec un tableur. Le graphique suivant représente la
%fréquence d'apparition de chaque somme obtenue :
%
%\begin{center}
%\psset{xunit=1cm,yunit=0.2cm}
%\begin{pspicture}(-1,-5)(9,25)
%\psaxes[linewidth=1.25pt,Dx=10,Dy=5](0,0)(9,25)
%\multido{\n=1+1}{8}{\uput[d](\n,0){\n}}
%\psframe[fillstyle=solid,fillcolor=lightgray](1.5,0)(2.5,7)
%\psframe[fillstyle=solid,fillcolor=lightgray](2.5,0)(3.5,15)
%\psframe[fillstyle=solid,fillcolor=lightgray](3.5,0)(4.5,19)
%\psframe[fillstyle=solid,fillcolor=lightgray](4.5,0)(5.5,23)
%\psframe[fillstyle=solid,fillcolor=lightgray](5.5,0)(6.5,17)
%\psframe[fillstyle=solid,fillcolor=lightgray](6.5,0)(7.5,13)
%\psframe[fillstyle=solid,fillcolor=lightgray](7.5,0)(8.5,6)
%\rput(4.5,-4){somme des nombres inscrits sur les deux dés} \rput{90}(-1,12.5){fréquence en \,\%}
%\end{pspicture}
%\end{center}

\begin{enumerate}
\item %Par lecture graphique donner la fréquence d'apparition de la somme 3.
La fréquence d'apparition de la somme 3 est 15\,\%.
\item %Lire la fréquence d'apparition de la somme 1 ? Justifier cette fréquence.
La fréquence d'apparition de la somme 1 est 0\,\%, en effet il est impossible d'obtenir 1, la plus petite somme possible est 2 (1 sur chaque dé).
\item  
	\begin{enumerate}
		\item %Décrire les lancers de dés qui permettent d'obtenir une somme égale à 3.
Notons A et B les deux dés :

Dé A : 1 – Dé B : 2

Dé A : 2 – Dé B : 1.

Il y a deux cas qui permettent d'obtenir une somme égale à 3.
		\item %En déduire la probabilité d'obtenir la somme 3 en lançant les dés. On exprimera
%cette probabilité en pourcentage. 
		
%Expliquer pourquoi ce résultat est différent de celui obtenu à la question 1.
Il y a $4 \times 4 = 16$ cas possibles.

La probabilité d'obtenir la somme 3 est donc $\dfrac{2}{16} = \dfrac{1}{8}  = \dfrac{125}{\np{1000}} = 0,125 = 12,5\,\%.$

		Ce résultat est différent du résultat à la question 1 car seulement \np{1000} lancers ont été simulés, ce n'est pas suffisant !
	\end{enumerate} 
 \end{enumerate}
 
\vspace{0,5cm}

