
\medskip

\parbox{0.35\linewidth}{On considère la figure ci-contre dessinée à main levée.

L'unité utilisée est le centimètre.

Les points I, H et K sont alignés.}\hfill
\parbox{0.62\linewidth}{\psset{unit=1cm,varsteptol=0.2,VarStepEpsilon=10}
%\psgrid
\begin{pspicture}(8,4)
\pslineByHand(4.5,3)(0.5,0.5)(7.5,1)(4.5,3)(4.7,0.3)%IKJI
\rput(2.7,2.2){6,8}\rput(6,2.2){4}\rput(4.3,1.6){3,2}\rput(6,0.6){2,4}
\uput[ul](4.5,3){J}\uput[l](0.5,0.5){I}\uput[dl](4.6,0.8){H}\uput[d](7.5,1){K}
\end{pspicture}}

\medskip

\begin{enumerate}
\item Construire la figure ci-dessus en vraie grandeur.
\item Démontrer que les droites (IK) et (JH) sont perpendiculaires.
\item Démontrer que IH = 6 cm.
\item Calculer la mesure de l'angle $\widehat{\text{HJK}}$, arrondie au degré.
\item La parallèle à (IJ) passant par K coupe (JH) en L. Compléter la figure.
\item Expliquer pourquoi LK = $0,4 \times$ IJ.
\end{enumerate}

\vspace{0,5cm}

