
\medskip

%En 2010, l'UNESCO\footnote{\emph{UNESCO.  United Nations Educational, Scientific and Cultural Organization (en français : Organisation
%des Nations Unies pour l'Education, la Science et la Culture)}} a dressé un inventaire des langues en danger dans le monde. Il vise à susciter une prise de conscience sur la nécessité de préserver une diversité linguistique mondiale. Voici un tableau récapitulatif du nombre de langues en voie de disparition ou déjà éteintes :
%
%\begin{center}
%\begin{tabularx}{\linewidth}{|m{3cm}|*{3}{>{\centering \arraybackslash}X|}}\hline
%Niveau de vitalité& En voie de disparition &Déjà éteintes& Total\\ \hline
%Nombres de langues& ... &231 &\np{2580}\\ \hline
%\end{tabularx}
%\end{center}

\begin{enumerate}
\item %Sur \np{6000} langues répertoriées, 43\,\% sont soit en voie de disparition, soit déjà éteintes.

%Montrer, par un calcul, que cela représente un total de \np{2580} langues.
On a $\np{6000} \times 0,43 = \np{2580}$~(langues).
\item %En déduire le nombre de langues qui sont en voie de disparition.
Il reste $\np{2580} - 231 = \np{2349}$ langues  en voie de disparition.
\item %Calculer le pourcentage de langues qui sont déjà éteintes sur les \np{6000} langues répertoriées dans le monde.
$\dfrac{231}{\np{6000}} = \dfrac{77}{\np{2000}} = \np{0,0385}$ soit 3,85\,\% pourcentage de langues éteintes.
\end{enumerate} 

\vspace{0,5cm}

