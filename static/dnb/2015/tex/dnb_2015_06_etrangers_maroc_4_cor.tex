
\medskip

Des ingénieurs de l'Office National des Forêts font le marquage d'un lot de pins
destinés à la vente.

\medskip

\begin{enumerate}
\item %Dans un premier temps, ils estiment la hauteur des arbres de ce lot, en plaçant
%leur œil au point O.

%\parbox{0.55\linewidth}{\psset{unit=1cm}
%\begin{pspicture}(7.3,5)
%%\psgrid
%\psline{<->}(7.2,0.3)(7.2,4.5)\uput[l](7.2,2.4){$h$}
%\psline(0,0.3)(7.3,0.3)
%\psline[linestyle=dotted](5.9,1.1)(0.6,1.1)(5.9,4.5)%AOS
%\psline[linestyle=dotted](0.6,1.1)(5.9,0.3)(5.9,4.5)%OPS
%\psarc(0.6,1.1){0.8}{0}{35}
%%(5.6,1.1)(5.6,1.4)(5.9,1.4)
%\uput[ul](5.9,1.1){A}\uput[u](5.9,4.5){S}
%\uput[d](5.9,0.3){P}\uput[d](0.6,1.1){O}
%\pscurve(5.8,0.3)(5.85,1.2)(5.8,3)(5.6,3.3)(5,3.4)(4.7,4)(4.8,4.2)(5,4.3)(5.9,4.5)(6.3,4.4)(6.7,4.2)(6.9,4)(6.5,3.6)(6.2,3.3)(6.12,2.8)(6.11,1.5)(6.14,0.3)
%\psarc(0.6,1.1){0.5}{-11}{0}
%\rput(1.65,1.3){\footnotesize 45\degres}\rput(2.2,1){\footnotesize 25\degres}
%%\psline
%\end{pspicture}}\hfill
%\parbox{0.32\linewidth}{Ils ont relevé les données suivantes :
%
%OA = 15 m
%
%$\widehat{\text{SOA}} = 45\degres$ et $\widehat{\text{AOP}} = 25\degres$}
%
%Calculer la hauteur $h$ de l'arbre arrondie au mètre.
Dans le triangle rectangle en A, OAS, on a : $\tan \widehat{\text{AOS}}  = \dfrac{\text{AS}}{\text{OA}}$ soit $\tan 45 = \dfrac{\text{AS}}{15}$, d'où $\text{AS} = 15\tan 45 $ ;

Dans le triangle rectangle en A, OAP, on a : $\tan \widehat{\text{AOP}}  = \dfrac{\text{AP}}{\text{OA}}$ soit $\tan 25 = \dfrac{\text{AP}}{15}$, d'où $\text{AP} = 15\tan 25$.

La hauteur de l'arbre est :

$h = \text{AS} + \text{AP} = 15\tan 45 + 15\tan 25 = 15 (\tan 45 + \tan 25) \approx 21,99$ soit 22~m au mètre près.
\item  %Dans un second temps, ils effectuent une mesure de diamètre sur chaque arbre et
%répertorient toutes les données dans la feuille de calculs suivante :

%\begin{center}
%\begin{tabularx}{\linewidth}{|c|m{1.75cm}|*{12}{>{\centering \arraybackslash}X|}}\hline
%	&A &B &C &D &E &F &G &H &I &J &K &L &M\\ \hline
%1	& Diamètre (cm)	&30 &35 &40 &45 &50 &55 &60 &65 &70 &75 &80&\\ \hline
%2 	&Effectif 		&2 &4 &8 &9 &10 &12 &14 &15 &11 &4 	&3	&\\ \hline
%\end{tabularx}
%\end{center}

	\begin{enumerate}
		\item %Quelle formule doit-on saisir dans la cellule M2 pour obtenir le nombre total
%d'arbres ?
Il faut inscrire en M2 : $=\text{SOMME}(\text{B}2 : \text{L}2)$.
		\item %Calculer, en centimètres, le diamètre moyen de ce lot. On arrondira le résultat à
%l'unité.
Si $d$ est le diamètre moyen, alors :

$d = \dfrac{30 \times 2 + 35  \times  4 + 40  \times  8 + \cdots + 80  \times  3}{2 + 4 + 8 + \cdots + 3} = \dfrac{\np{5210}}{92} \approx 57$~cm au centimètre près.
	\end{enumerate}
\item %Pour calculer le volume commercial d'un pin en mètres cubes, on utilise la formule
%suivante :

%\[V = \dfrac{10}{24} \times D^2 \times h\]
%
%où $D$ est le diamètre moyen d'un pin en mètres et $h$ la hauteur en mètres.
%
%Le lot est composé de 92 arbres de même hauteur 22 m dont le diamètre moyen
%est 57 cm.
%
%Sachant~ qu'un mètre cube de pin rapporte 70~\euro, combien la vente de ce lot
%rapporte+elle ? On arrondira à l'euro.
Le volume des 92 arbres est égal à :

$92 \times \dfrac{10}{24} \times 0,57^2 \times 22$.

Chaque mètre cube rapportant 70~\euro{}, la vente rapportera :

$70 \times 92 \times \dfrac{10}{24} \times 0,57^2 \times 22 = \np{19179,93} \approx \np{19180}$~\euro.
\end{enumerate}

\vspace{0,5cm}

