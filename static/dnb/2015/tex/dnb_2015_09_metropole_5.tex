
\medskip 

On considère le programme de calcul ci-dessous: 

\medskip
\begin{center}
\begin{tabularx}{0.71\linewidth}{|lX|}\hline 
$\bullet~~$&Choisir un nombre.\\ \hline 
$\bullet~~$&Soustraire 6. \\ \hline 
$\bullet~~$&Multiplier le résultat obtenu par le nombre choisi.\\ \hline 
$\bullet~~$&Ajouter 9. \\ \hline 
\end{tabularx}
\end{center}

\begin{enumerate}
\item Vérifier que lorsque le nombre choisi est 11, le résultat du programme est 64. 
\item Lorsque le nombre choisi est $- 4$, quel est le résultat du programme ? 
\item Théo affirme que, quel que soit le nombre choisi au départ, le résultat du programme est toujours un nombre positif. A-t-il raison ? 
\end{enumerate}

\vspace{0.5cm}

