
\medskip
 
%Chez le fleuriste un bouquet composé de 5 tulipes et 2 roses   coûte 13,70 euros.
% 
%Une tulipe et une rose valent ensemble 4,30 euros.
% 
%Calculer le prix d'une tulipe et le prix d'une rose.
%
%\medskip
% 
%$\left.\begin{tabular}{l}
%\text{T T T T T}\\
%\text{R R}
%\end{tabular}\right\}$ \quad 13,70~\euro
%
%\medskip
%
%$\left.\begin{tabular}{l}\text{T}\\
%\text{R}
%\end{tabular}\right\}$ \quad4,30~\euro

%\bigskip
%
%T $\rightarrow$ \ldots~\euro
%
%R $\rightarrow$ \ldots~\euro
Si $t$ est le prix d’une tulipe et $r$ le prix d’une rose, on a donc :

$\left\{\begin{array}{l c l}
5t + 2r&=&13,70\\
\phantom{5}t + \phantom{5}r&=&4,30
\end{array}\right.$

On peut en déduire en multipliant chaque membre de la deuxième équation par 5 :

$\left\{\begin{array}{l c l}
5t + 2r&=&13,70\\
5t + 5r&=&21,50
\end{array}\right.$ soit par différence :

$3r = 7,80$ et finalement $r = 2,60$~\euro. Par complément à 4,30, on obtient 

$t = 4,30 - 2,60 = 1,70$~(\euro).
\vspace{0,5cm}

