
\medskip

%Peio, un jeune Basque décide de vendre des glaces du 1\up{er} juin au 31 août inclus à
%Hendaye.
%
%Pour vendre ses glaces, Peio hésite entre deux emplacements :
%
%\setlength\parindent{6mm}
%\begin{itemize}
%\item une paillotte sur la plage
%\item une boutique au centre-ville.
%\end{itemize}
%\setlength\parindent{0mm}
%
%En utilisant les informations ci-dessous, aidez Peio à choisir l'emplacement le
%plus rentable.
%
%\medskip
%
%\begin{tabularx}{\linewidth}{|X c|}\hline
%\textbf{Information 1} : les loyers des deux emplacements proposés :&\\
%$\bullet~~$la paillotte sur la plage : \np{2500}~\euro{} par mois.&\\
%$\bullet~~$la boutique au centre-ville : $60$~\euro{} par jour.&\\ \hline
%\end{tabularx}
%
%\medskip
%
%\begin{tabularx}{\linewidth}{|X|}\hline
%\textbf{Information 2} : la météo à Hendaye\\
%Du 1\up{er} juin au 31 août inclus :\\
%$\bullet~~$Le soleil brille 75\,\% du temps\\
%$\bullet~~$Le reste du temps, le temps est nuageux ou pluvieux.\\ \hline
%\end{tabularx}
%
%\medskip
%
%\textbf{Information 3} : prévisions des ventes par jour selon la météo :
%
%\medskip
%
%\begin{tabularx}{\linewidth}{|l|*{2}{>{\centering \arraybackslash}X|}}\cline{2-3}
%\multicolumn{1}{c|}{~}&Soleil & Nuageux - pluvieux\\ \hline
%La paillotte& 500~\euro& 50~\euro\\ \hline
%La boutique& 350~\euro& 300~\euro\\ \hline
%\end{tabularx}
%
%\medskip
%
%On rappelle que le mois de juin comporte 30 jours et les mois de juillet et août
%comportent 31 jours.
%
%\medskip
%
%\textbf{Toute piste de recherche même non aboutie, sera prise en compte dans
%l'évaluation.}
$\bullet~~$\textbf{Sur la plage} :

Peio paiera 3 mois à \np{2500} soit $3 \times \np{2500} = \np{7500}$~\euro{} de location de paillote.

Il encaissera les trois quarts du temps soit $0,75 \times 92$~jours 500~\euro{} par jour et 

le reste du temps soit $0,25 \times 92$~jours 50~\euro{} par jour.

Ses recettes pour tout l'été s'élèveront donc à :

\begin{center}$0,75 \times 92 \times 500 + 0,25 \times 92 \times 50 = \np{34500} + \np{1150} = \np{35650}$~\euro.\end{center}

Il gagnera donc sur la plage :

\[\np{35650} - \np{7500} = \np{28150}~\text{\euro}.\]

$\bullet~~$\textbf{En ville}

Peio paiera 92 jours  à 60 soit $92 \times 60   = \np{5520}$~\euro{} de location.

Il encaissera les trois quarts du temps soit $0,75 \times 92$~jours 350~\euro{} par jour et 

le reste du temps soit $92 \times 0,25$~jours 300~\euro{} par jour.

Ses recettes pour tout l'été s'élèveront donc à :

\[0,75 \times 92 \times 350 + 0,25 \times 92 \times 300 = \np{24150} + \np{6900} = \np{31050}~\text{\euro}.\]

Il gagnera donc en ville :

\[ \np{31050} - \np{5520} = \np{25530}~\text{\euro}.\]

$\bullet~~$\textbf{Conclusion} : Peio gagnera gagnera plus sur la plage.
 

