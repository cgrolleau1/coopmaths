
\medskip
 

On donne le programme suivant qui permet de tracer plusieurs triangles équilatéraux de tailles différentes.
 
Ce programme comporte une variable nommée \og \textbf{côté} \fg. Les longueurs sont données en pixels.
 
On rappelle que l'instruction \raisebox{-3.5mm}{\begin{scratch}\blockmove{s'orienter à  \ovalnum{90\selectarrownum}}\end{scratch}} signifie que l'on se dirige vers la droite.
 

\begin{center}
\begin{tabular}[t]{|c|c|c|}\hline
\begin{tabular}[t]{c} Numéros\\ d'instruction\\\end{tabular}&Script&Le bloc \textbf{triangle}\\ 

\raisebox{-\height}{
\renewcommand{\arraystretch}{1.82}\begin{tabular}{c}
\\[-.5cm]
1\\
2\\
3\\
4\\
5\\
6\\
7\\
8\\
9\\ 
\end{tabular}
}
& \raisebox{-\height}{
\begin{scratch}
\blockinit{Quand \greenflag est cliqué}
\blockpen{effacer tout}
\blockmove{aller à x:\ovalnum{$-200$} y: \ovalnum{$-100$}}
\blockmove{s'orienter à  \ovalnum{90\selectarrownum}}
\blockvariable{Mettre \selectmenu{côté} à \txtbox{100}}
\blockrepeat{répéter \ovalnum{$5$} fois}
{
\blockmoreblocks{triangle}
\blockmove{avancer de \ovalvariable{côté}  }
\blockvariable{Ajouter à  \selectmenu{côté} \ovalnum{$-20$}}
}
\end{scratch}
}&
\raisebox{-\height}{\begin{scratch}
\initmoreblocks{définir \namemoreblocks{triangle}}
\blockpen{stylo en position écriture}
\blockrepeat{répéter \ovalnum{3} fois}
{
\blockmove{avancer de \ovalvariable{côté} }
\blockmove{tourner \turnleft{} de \ovalnum{120} degrés}
}
\blockpen{relever le stylo}
\end{scratch}}\\\hline
\end{tabular}
\end{center}

\parbox{0.55\linewidth}{\begin{enumerate}
\item Quelles sont les coordonnées du point de départ du tracé ?
\item Combien de triangles sont dessinés par le script ?
\item 
	\begin{enumerate}
		\item Quelle est la longueur (en pixels) du côté du deuxième triangle tracé?
		\item Tracer à main levée l'allure de la figure obtenue quand on exécute ce script.
	\end{enumerate}
\item  On modifie le script initial pour obtenir la figure ci-contre.
 
Indiquer le numéro d'une instruction du script \textbf{après laquelle} on peut placer
l'instruction \raisebox{-3.5mm}{\begin{scratch}\blockmove{tourner \turnleft{} de \ovalnum{60} degrés}\end{scratch}} pour obtenir cette nouvelle figure.
\end{enumerate}}\hfill
\parbox{0.45\linewidth}{\psset{unit=1cm,linecolor=blue}
\begin{pspicture}(5,5)
%\psgrid
\def\tria{\pspolygon(2;-30)(2;90)(2;210)}
\def\trib{\pspolygon(1.6;-30)(1.6;90)(1.6;210)}
\def\tric{\pspolygon(1.2;-30)(1.22;90)(1.22;210)}
\def\trid{\pspolygon(0.8;-30)(0.8;90)(0.8;210)}
\def\trie{\pspolygon(0.4;-30)(0.4;90)(0.4;210)}
\rput(2,1.5){\tria}\rput(3.75,2.1){\rotatedown{\trib}}
\rput(4.1,3.5){\tric}\rput(3.4,4.3){\rotatedown{\trid}}
\rput(2.7,4.3){\trie}
\end{pspicture}}

\vspace{0,5cm}

