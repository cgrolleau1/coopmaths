
\medskip

\begin{enumerate}
\item $\dfrac{18}{15} = \dfrac{x}{60}$. Sa fréquence cardiaque est donc  $\dfrac{18 \times 60}{15} = 72$ pulsations par minute.

Ou en supposant les pulsations régulières sur 60 secondes :

18 en 15~(s) donnent 36 en 30~(s) et 72 en 60~(s).
\item Il y a $\dfrac{60}{0,8} = \frac{600}{8} = \dfrac{8 \times 75}{8 \times 1} =  75$ intervalles donc $76$ pulsations/min.
\item 
	\begin{enumerate}
		\item L'étendue est la différence entre la plus haute et la plus basse fréquence : E $= 182 - 65 = 117$ pulsations /min.
		\item On divise le nombre total de pulsation par la fréquence moyenne, d'où
		
$\dfrac{\np{3640}}{130} = 28$ minutes.
		
L'entrainement a duré environ 28 minutes.
	\end{enumerate}		
\item
	\begin{enumerate} 
		\item Denis a 32 ans, donc sa FCMC est $f(32) = 220 - 32 = 188$ pulsations/minute.
		\item Pour une personne de 15 ans, la FCMC est $f(15) = 220 - 15 = 205$ pulsations/minute.

La FCMC de Denis est inférieure à la FCMC d'une personne de 15 ans.
	\end{enumerate}
\item $=191,5 - 0,007*\text{A}2*\text{A}2$.
\end{enumerate}

\vspace{0,5cm}

