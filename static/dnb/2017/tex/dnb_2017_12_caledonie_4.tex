
\medskip

\emph{Dans cet exercice, toute trace de recherche, même incomplète ou non fructueuse, sera prise en compte dans l'évaluation. Les questions sont indépendantes.}

\bigskip

\og --- Salut Antoine, bonne idée d'aller à la pêche aux coquillages ce matin !

--- Salut Aurel! Oui à la pêche aux coquillages et aux poissons!

--- AUREL: Où va-t-on ?

--- ANTOINE : Ici, la croix sur la carte, c'est à 5~km.

--- AUREL : Super ton bateau ! A-t-on assez d'essence ?

--- ANTOINE : Oui sans problème ! Le réservoir est plein, j'ai 12 L d'essence.

--- AUREL : On navigue à quelle vitesse ?

--- ANTOINE : Dans la mangrove, en moyenne, 8 noeuds.

--- AUREL : Avec cette pêche, le bateau sera plus lourd.

--- ANTOINE : Oui, on devrait consommer 1 L d'essence de plus qu'à l'aller.\fg

\medskip

\begin{enumerate}
\item En prenant $1$ noeud $= 1,852$ km/h, combien de temps faut-il à Antoine et Aurel pour atteindre leur lieu de pêche ? 

Exprimer le résultat en minutes
(arrondi à l'unité).
\item Les deux amis ont consommé, à l'aller, un quart du réservoir. Comme le bateau sera plus
lourd au retour, quel volume d'essence restera-t-il dans le réservoir à leur arrivée ?
\end{enumerate}

\vspace{0,5cm}

