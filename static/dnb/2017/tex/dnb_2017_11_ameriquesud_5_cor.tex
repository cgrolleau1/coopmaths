 

Le tableau ci-dessous donne la répartition (exprimée en pourcentages) de la consommation des différents types d’énergie entre 1973 et 2014. 

\begin{center}
\begin{tabularx}{\linewidth}{|l|*{5}{>{\centering \arraybackslash}X|}}\hline
								&1973   &1980   &1990   &2002 &2014 \\ \hline   
Électricité   					&4,3   	&11,7   &36,4   &41,7 &45,4 \\ \hline     
Pétrole   						&67,6  	&56,4   &38,7   &34,6 &30,2 \\ \hline     
Gaz  							&7,4   	&11,1  	&11,5   &14,7 &14,0 \\ \hline     
Énergies renouvelables 			&5,2   	&4,4   	&5,0   	&4,3  &7,0 \\ \hline     
Charbon   						&15,5  	&16,4   &8,4   	&4,7  &3,4 \\ \hline
\multicolumn{6}{r}{\emph{Sources : INSEE}}\\
\end{tabularx}
\end{center} 

\begin{minipage}{8cm}
\begin{enumerate}
\item En 1980, le pétrole représente 56,4~\%\ de la consommation d’énergie 
\item À partir du tableau précédent, on a créé, pour unedes années, un diagramme représentant la répar-tition des différents types d’énergie.

Le diagramme montre que la part du pétrole est du même ordre de grandeur que la part de l'électricité, donc on élimine les années 1973 et 1980.

Le diagramme montre que la part du gaz est plus de 3 fois plus grande que la part du charbon, donc on élimine les années 2002 et 2014.

Il s'agit donc de l'année 1990. 
\end{enumerate}

\vspace{1cm}

\end{minipage}
\hspace{1cm}\begin{minipage}{10.5cm}
%\begin{center}
\psset{unit=0.9cm}
\begin{pspicture}(-3.5,-3.5)(3.5,3.5)
\pscircle(0,0){3}
\newgray{gris5}{0.55}\pswedge[fillstyle=solid,fillcolor=gris5](0,0){3}{-41}{90}
%\pstextpath[c](0,0.5){\psarcn(0,0){3}{90}{-41}}{ÉlectricitéŽ}
\pstextpath[c](0,0.5){\psarcn(0,0){3}{90}{-41}}{Électricité}
\newgray{gris6}{0.65}\pswedge[fillstyle=solid,fillcolor=gris6](0,0){3}{90}{120.24}
\pstextpath[c](0,0.5){\psarcn(0,0){3}{120.24}{90}}{Charbon}
\newgray{gris7}{0.75}\pswedge[fillstyle=solid,fillcolor=gris7](0,0){3}{120.24}{138.24}
\pstextpath[c](0,1){\psarcn(0,0){3}{138.24}{120.24}}{\small renouve-}
\pstextpath[c](0,0.5){\psarcn(0,0){3}{138.24}{120.24}}{\small lables}
\newgray{gris8}{0.85}\pswedge[fillstyle=solid,fillcolor=gris8](0,0){3}{138.24}{179.64}
\pstextpath[c](0,0.5){\psarcn(0,0){3}{179.64}{138.24}}{Gaz}
\newgray{gris9}{0.95}\pswedge[fillstyle=solid,fillcolor=gris9](0,0){3}{179.64}{-41}
\pstextpath[c](0,-0.5){\pswedge[fillstyle=solid,fillcolor=gris9](0,0){3}{179.64}{-41}}{Pétrole}
\end{pspicture}
%\end{center}
\end{minipage}
%\vspace{-1cm}

\begin{enumerate}
\item[\textbf{3.}] On peut observer l’évolution de la part du pétrole au fil des années à  partir d'une représentation graphique comme celle proposée ci-dessous.

Les pointillés indiquent que l'on suppose que la baisse de la part du pétrole va se poursuivre sur le rythme observé depuis 2002. \textcolor{red}{\textbf{On peut donc prolonger les pointillés}}.

\psset{xunit=1cm,yunit=0.75cm}
\begin{pspicture}(-1,-1)(11.5,10)
%\psgrid
\psgrid[gridlabels=0pt,subgriddiv=1,gridwidth=0.2pt](0,0)(11,8)
\psaxes[linewidth=1.25pt,Dx=13,Dy=10]{->}(0,0)(0,0)(11.5,8.25)
\uput[d](0,0){1973} \uput[d](1,0){1980} \uput[d](2,0){1990} 
\uput[d](3,0){2002} \uput[d](4,0){2014} \uput[d](5,0){2026} 
\uput[d](6,0){2038} \uput[d](7,0){2050} \uput[d](8,0){2062}
\uput[d](8,0){2062} \uput[d](9,0){2074}  \uput[d](10,0){2086}
\uput[l](0,8.4){Pourcentage}
\uput[l](0,0){0\,\%} \uput[l](0,1){10\,\%}\uput[l](0,2){20\,\%}
\uput[l](0,3){30\,\%}\uput[l](0,4){40\,\%}\uput[l](0,5){50\,\%}
\uput[l](0,6){60\,\%}\uput[l](0,7){70\,\%}\uput[l](0,8){80\,\%}
\uput[u](11.5,0){Année} 
%\rput(3.5,9){Part du péŽtrole}
\rput(3.5,9){Part du pétrole}
\rput(3.5,8.5){(en pourcentage des énergies consommées)}
\psline(0,6.8)(1,5.6)(2,3.95)(3,3.5)(4,3)
\psline[linestyle=dashed](4,3)(6,2.1) 
\psline[linestyle=dashed,linecolor=red](6,2.1)(10.6,0)(12,-0.6)
\psframe(0.2,8.2)(6.8,9.3)
\end{pspicture}  
  
En suivant cette supposition, on peut modéliser la part du pétrole (exprimée en pourcentage) en fonction de l'année $a$ par la fonction $P$, définie ainsi: 

\[P(a) = \dfrac{- 17}{48}a + 743,5.\] 

	\begin{enumerate}
		\item $P(1990)=\dfrac{- 17}{48}\times 1990 + 743,5 \approx 38,7$. 
		\item $\bullet$ \textbf{Par essais successifs}, on effectue plusieurs calculs : \quad 

$P(2090) = \dfrac{- 17}{48}\times 2090 + 743,5 \approx 3,3$

$P(2099) = \dfrac{- 17}{48}\times 2099 + 743,5 \approx 0,1$

$P(2100)=\dfrac{- 17}{48}\times 2100 + 743,5  \approx - 0,25$

\vspace{0.2cm}
		$\bullet$ \textbf{Par mise en équation}, la part du pétrole est nulle se traduit par : \quad $\dfrac{- 17}{48}\times a + 743,5=0$
$\dfrac{- 17}{48}\times a =- 743,5$
$a = - 743,5 \div \dfrac{- 17}{48}$
$a = - 743,5 \times \dfrac{48}{- 17}$. Finalement  
$a \approx 2099,3$	\end{enumerate}
\end{enumerate}

\medskip

