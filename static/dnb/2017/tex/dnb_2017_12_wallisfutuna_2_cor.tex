
\vspace{0.2cm}
Voici les tailles, en cm, de 29 jeunes plants de blé 10 jours après la mise en germination.

\vspace{0.2cm}

\begin{tabularx}{\linewidth}{|l|*{9}{>{\centering \arraybackslash}X|}}\hline
Taille (en cm) &0 &10 &15 &17 &18 &19 &20 &21 &22\\ \hline
Effectif &1 &4 &6 &2 &3 &3 &4 &4 &2\\ \hline
Effectif cumulé croissant&1 &5 &11 &13 &16 &19 &23 &27 &29\\ \hline
\end{tabularx}

\vspace{0.2cm}

\begin{enumerate}
\item $\dfrac{1\times 0 + 4\times10  + 6\times15  + 2\times17  + 3\times18  + 3\times 19 + 4\times20  +4\times21  +2\times 22}{29}=\dfrac{483}{29} \approx16,7$

La taille moyenne d'un jeune plant de blé est \textcolor{red}{\textbf{d’environ}} 16,7~cm 10 jours après la mise en germination.
\item 
	\begin{enumerate}
		\item L’effectif total est égal à 29. $29\div2=14,5$. La médiane est la 15\ieme\ donnée de la série de données ordonnée dans l’ordre croissant. La médiane de cette série est égale à 18~cm.
		\item Dire que la médiane de cette série est égale à 18~cm signifie qu’au moins la moitié des plants de blé mesurent 18~cm ou moins de 18~cm, 10 jours après la mise en germination.
	\end{enumerate}
\end{enumerate}


