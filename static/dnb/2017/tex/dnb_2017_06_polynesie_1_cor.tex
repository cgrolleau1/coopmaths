
\medskip

%Cet exercice est un QCM (questionnaire à choix multiple).
%
%Pour chaque ligne du tableau, une seule affirmation est juste.
%
%\medskip
%\end{center}
%\textbf{Indiquer le numéro de la question et recopier l'affirmation juste sur votre copie.\\
%Aucune justification n'est attendue. Aucun point n'est retiré en cas de mauvaise
%réponse.}
%
%\begin{center}
%\begin{tabular}{|m{5.75cm}|c|c|c|}\hline
%%\begin{tabularx}{\linewidth}{|m{5.75cm}|*{3}{>{\centering arraybackslash}X|}}\hline
%\textbf{Questions}&\multicolumn{3}{|c|}{\textbf{Affirmations}}\\ \hline
%			&A 		&B 		&C\\ \hline
%\textbf{1.} Combien faut-il environ de CD de $700$ Mégaoctets pour stocker autant de données qu'une clé de 32 Gigaoctets ?&46 	&\np{4600} 	&\np{4600000}\\ \hline 
%\textbf{2.} La diagonale d'un rectangle de 10 cm par 20 cm est d'environ:	&15 cm &22 cm &30 cm\\ \hline
%\textbf{3.}   Une solution de l'équation 
%
%$2x + 3 = 7x - 4$ est: &$\dfrac{5}{7}$&1,4 &$- 0,7$\\ \hline
%\textbf{4.}  La fraction irréductible de la fraction $\dfrac{882}{\np{1134}}$ est : \rule[-3mm]{0mm}{8mm}&$\dfrac{14}{9}$& $\dfrac{63}{81}$& $\dfrac{7}{9}$\\ \hline
%\textbf{5.}   On considère la fonction
% 
%$f \: : x \longmapsto  3x + 4$.
%
%Quelle formule doit-on entrer en B2 puis recopier vers la droite afin de calculer les
%images des nombres de la ligne 1 par la fonction $f$ ?
%
%\begin{tabularx}{\linewidth}{|c|*{4}{>{\centering \arraybackslash}X|}}\hline
%B2	&		&$f_x$	&	& \\ \hline
%	&A		&B		&C	&D\\ \hline
%1	&$x$	&5		&6	&7\\ \hline
%2	&$f(x)$	&		&	&\\ \hline
%3	&		&		&	&\\ \hline
%\end{tabularx}
%&$=3\star \text{A}1 + 4$& $= 3 \star 5 + 4$ & $=3\star \text{B}1 + 4$\\ \hline
%\end{tabular}
%\end{center}
\begin{enumerate}
\item Un gigaoctets vaut \np{1024} mégaoctets, donc 32 Go $=  32 \times \np{1024}$~Mo.

Il faut donc $\dfrac{32 \times \np{1024}}{700}\approx 46,8$, donc 47 CD de 700 Mo.
\item D'après le théorème de Pythagore on a :

$d^2 = 10^2 + 20^2  = 100 + 400 = 500$ ; donc $d = \sqrt{500} \approx 22,3$, soit environ 22~cm à l'unité près.
\item Si $2x + 3 = 7x - 4$ alors $3 + 4 = 7x - 2x$ ou $7 = 5x$ ; donc $x = \dfrac{7}{5} = \dfrac{14}{10} = 1,4$.
\item $882 = 2 \times 441 = 2 \times (21)^2 = 2\times (3 \times 7)^2 = 2 \times 3^2 \times 7^2$ ;

$\np{1134} = 2 \times 567  = 2 \times 7 \times 81 = 2 \times 7 \times 9^2 = 2 \times 7 \times 3^4$. Donc 

$\dfrac{882}{\np{1134}} = \dfrac{2 \times 3^2 \times 7^2}{2 \times 7 \times 3^4} = \dfrac{7}{3^2} = \dfrac{7}{9}$.
\item $=3\star \text{B}1 + 4$.
\end{enumerate}

\bigskip

