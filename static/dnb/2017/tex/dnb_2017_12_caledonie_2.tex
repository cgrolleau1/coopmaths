
\medskip

\parbox{0.3\linewidth}{Voici un programme de calcul :}\hfill
\parbox{0.68\linewidth}{
\begin{tabular}{|l|}\hline
$\bullet~~$Choisir un nombre\\
$\bullet~~$Ajouter 1 à ce nombre\\
$\bullet~~$Calculer le carré du résultat\\
$\bullet~~$Soustraire le carré du nombre de départ au résultat précédent.\\
$\bullet~~$Écrire le résultat.\\ \hline
\end{tabular}
}

\medskip

\begin{enumerate}
\item On choisit 4 comme nombre de départ. Prouver par le calcul que le résultat obtenu avec le programme
est 9.
\item On note $x$ le nombre choisi.
	\begin{enumerate}
		\item Exprimer le résultat du programme en fonction de $x$.
		\item Prouver que ce résultat est égal à $2x + 1$.
	\end{enumerate}
\item Soit $f$ la fonction définie par $f(x) = 2x + 1$.
	\begin{enumerate}
		\item Calculer l'image de 0 par $f$.
		\item Déterminer par le calcul l'antécédent de $5$ par $f$.
		\item En annexe 1, tracer la droite représentative de la fonction $f$.
		\item Par lecture graphique, déterminer le résultat obtenu en choisissant $- 3$ comme nombre de départ dans le programme de calcul. Sur l'annexe, laisser les traits de construction apparents.
	\end{enumerate}	
\end{enumerate}
\begin{center}
\textbf{\large À RENDRE AVEC LA COPIE}

\bigskip

\textbf{\large ANNEXE 1}

\bigskip

\psset{unit=0.5cm}
\begin{pspicture}(-10,-9)(10,9)
\psgrid[gridlabels=0pt,subgriddiv=1,griddots=10]
\psaxes[linewidth=1.25pt,Dx=2,Dy=2]{->}(0,0)(-10,-9)(10,9)
\uput[u](9.5,0){$x$}\uput[r](0,9.5){$y$}
\end{pspicture}

\end{center}
\vspace{0,5cm}

