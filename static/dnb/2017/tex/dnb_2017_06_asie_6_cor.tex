
\medskip

%Gaspard réalise des motifs avec des carreaux de mosaïque blancs et gris de la façon
%suivante :
%
%\begin{center}
%\parbox{0.70\linewidth}{
%\psset{unit=0.6cm}
%\begin{pspicture}(14,7)
%\psframe[fillstyle=solid,fillcolor=lightgray](1,1)(2,2)
%\psframe[fillstyle=solid,fillcolor=lightgray](5,1)(7,3)
%\psframe[fillstyle=solid,fillcolor=lightgray](10,1)(13,4)
%\multido{\n=0+1}{4}{\psline(\n,0)(\n,3)}
%\multido{\n=0+1}{4}{\psline(0,\n)(3,\n)}
%\multido{\n=4+1}{5}{\psline(\n,0)(\n,4)}
%\multido{\n=0+1}{5}{\psline(4,\n)(8,\n)}
%\multido{\n=9+1}{6}{\psline(\n,0)(\n,5)}
%\multido{\n=0+1}{6}{\psline(9,\n)(14,\n)}
%\rput(1.5,6){Motif 1} \rput(6.5,6){Motif 2}\rput(11.5,6){ Motif 3}
%\end{pspicture}
%}\hfill
%\parbox{0.28\linewidth}{
%
%Gaspard forme un carré avec des carreaux gris puis le borde avec des carreaux blancs.
%}
%\end{center}
%\medskip

\begin{enumerate}
\item %Combien de carreaux blancs Gaspard va-t-il utiliser pour border le carré gris du
%motif 4 (un carré ayant 4 carreaux gris de côté) ?
Le nombre de carrés blancs est successivement :

$3^2 - 1^2 = 8~;~4^2 - 2^2 = 12~;~5^2 - 3^2 = 16$ et donc dans le motif 4 : $6^2 - 4^2 = 36 - 16 = 20$.
\item  
	\begin{enumerate}
		\item %Justifier que Gaspard peut réaliser un motif de ce type en utilisant exactement $144$ carreaux gris.
Le nombre de carrés gris est successivement :
		
$1^2 = 1~;~2^2 = 4~;~3^2 = 9~;~\ldots 11^2 = 121~;~12^2 = 144$.
		\item Combien de carreaux blancs utilisera-t-il alors pour border le carré gris obtenu ?
		Le nombre de carreaux blancs sera alors de : $14^2 - 12^2 = 196 - 144 = 52$.
	\end{enumerate}
\item  On appelle \og  motif $n$ \fg{} le motif pour lequel on borde un carré de $n$ carreaux gris de
côté.
	
%Trois élèves ont proposé chacun une expression pour calculer le nombre de
%carreaux blancs nécessaires pour réaliser le \og motif $n$ \fg{} :
%\setlength\parindent{2cm}
%\begin{itemize}
%\item[$\bullet~~$] Expression \no 1 : $2 \times n + 2 \times (n + 2)$
%\item[$\bullet~~$] Expression \no 2 : $4 \times (n + 2)$
%\item[$\bullet~~$] Expression \no 3 : $4 \times (n + 2) - 4$
%\end{itemize}
%\setlength\parindent{0cm}
%
%Une seule de ces trois expressions ne convient pas. Laquelle ?
C'est l'expression \no 2 : avec elle pour $n = 1$, le nombre de carreaux blancs serait $4\times (1 + 2) = 4 \times 3 = 12$ ; or on a vu que dans ce motif le carré gris est entouré de 8 carreaux blancs.
\end{enumerate}

\vspace{0,5cm}

