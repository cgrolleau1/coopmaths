
\medskip

$\bullet~~$Route descendant du château des  Adhémar, à Montélimar. La pente est égale à 24\,\%.

$\bullet~~$ Tronçon d'une route descendant du col du Grand Colombier (Ain) :
Le triangle est rectangle. 

On appelle $d$ le déplacement horizontal.

D'après l'égalité de Pythagore, on a : $d^2 = \np{1500}^2 - 280^2 = \np{2171600}$.

$ d = \sqrt{\np{2171600}} \approx  \np{1474}$~m.

Donc la pente est égale à  $\dfrac{280}{\np{1474}} \approx 18,9\,\%$.

$\bullet~~$Tronçon d'une route descendant de l'Alto de l'Angliru (région des Asturies, Espagne) :
le triangle est rectangle,

donc $\tan 12,4 = \dfrac{\text{dénivelé}}{146}$, d'où dénivelé $ = 146 \times \tan 12,4 \approx 32,10$~(m).

La pente est égale à $\dfrac{32,10}{146} \approx 21,98\,\%$ soit environ 22\,\%.

$\bullet~~$On pouvait aussi simplement dire que $\tan 12,4 = \dfrac{\text{côté opposé}}{\text{côté adjacent}} =$

$ \dfrac{\text{dénivelé}}{\text{déplacement horizontal}} \approx 0,22 = 22$\,\%. 

$\bullet~~$ Classement : 

\textbf{1.} Route descendant du château des Adhémar, à Montélimar

\textbf{2.} Tronçon d'une route descendant de l'Alto de l'Angliru (région des Asturies, Espagne)

\textbf{3.} Tronçon d'une route descendant du col du Grand Colombier (Ain)

\vspace{0,5cm}

