
\medskip

Pour mesurer les précipitations, Météo France utilise deux sortes
de pluviomètres:

\begin{itemize}
\item des pluviomètres à lecture directe ;
\item des pluviomètres électroniques.
 \end{itemize}
 
La mesure des précipitations s'exprime en millimètre. On donne ainsi la hauteur d'eau $H$
qui est tombée en utilisant la formule :
\begin{center}
\begin{tabularx}{\linewidth}{m{2.5cm}X }
$H = \dfrac{V}{S}$&
où $V$ est le volume d'eau tombée sur une surface $S$.

Pour $H$ exprimée en mm, $V$ est exprimé en mm$^3$ et $S$ en mm$^2$.
\end{tabularx}
\end{center}

\medskip

\textbf{Partie I : Pluviomètres à lecture directe}

\medskip

Ces pluviomètres sont composés d'un cylindre de réception et d'un réservoir conique
gradué.

\medskip

\begin{enumerate}
\item Vérifier à l'aide de la formule que lorsqu'il est tombé $1$~mm de pluie, cela correspond
à 1~L d'eau tombée sur une surface de 1 m$^2$.
\item Un pluviomètre indique $10$~mm de pluie. La surface qui reçoit la pluie est de $0,01$~m$^2$.

Quel est le volume d'eau dans ce pluviomètre ?
\end{enumerate}

\medskip

\textbf{Partie II : Pluviomètres électroniques}

\medskip

Durant un épisode pluvieux, on a obtenu le graphique suivant grâce à un pluviomètre
électronique :

\begin{center}
\psset{xunit=0.0042cm,yunit=1.8cm,comma=true}
\begin{pspicture}(-100,-0.5)(2600,4)
\multido{\n=0+250}{11}{\psline[linewidth=0.2pt](\n,0)(\n,3.5)}
\multido{\n=0+0.5}{8}{\psline[linewidth=0.2pt](0,\n)(2500,\n)}
\psaxes[linewidth=1.25pt,Dx=250,Dy=0.5]{->}(0,0)(0,0)(2600,3.5)
\uput[r](0,3.8){Hauteur d'eau mesurée en mm}
\rput(1250,4){Hauteur d'eau en fonction du temps écoulé}
\uput[u](2200,0){Temps écoulé (en s)}
\pscurve[linewidth=1.25pt,linecolor=blue](0,0)(250,0.3)(300,0.45)(375,1.3)(500,1.5)(750,1.7)(875,1.8)(925,2)(980,2.5)(1000,2.65)(1250,2.82)(1500,2.9)(2000,2.98)(2500,3)
\end{pspicture}
\end{center}

\begin{enumerate}
\item L'épisode pluvieux a commencé à 17~h~15.

Vers quelle heure la pluie s'est-elle arrêtée ?
\item On qualifie les différents épisodes pluvieux de la façon suivante :

\begin{center}
\begin{tabularx}{\linewidth}{|*{2}{>{\centering \arraybackslash}X|}}\hline
\textbf{Types de pluie}	& \textbf{Vitesse d'accumulation}\\ \hline
Pluie faible 			&Jusqu'à 2,5~mm/h\\ \hline
Pluie modérée 			&Entre 2,6 à 7,5~mm/h\\ \hline
Pluie forte 			&Supérieure à 7,5~mm/h\\ \hline
\end{tabularx}
\end{center}

À l'aide des informations données par le graphique et le tableau ci-dessus, cette pluie
serait-elle qualifiée de faible, modérée ou forte ?
\end{enumerate}

\vspace{0,5cm}

