
\medskip

%\parbox{0.6\linewidth}{Aux États-Unis, la température se mesure en
%degré Fahrenheit (en \degres F).
%En France, elle se mesure en degré Celsius (en \degres C).
%Pour faire les conversions d'une unité à l'autre, on a utilisé un tableur.
%
%Voici une copie de l'écran obtenu ci-contre.}
%\hfill 
%\parbox{0.39\linewidth}{\begin{tabularx}{\linewidth}{|c|*{2}{>{\centering \arraybackslash}X|}}\hline
%		&A 						&B\\ \hline
%1		&\multicolumn{2}{|c|}{\textbf{Conversions}}\\ \hline
%		&\small Températures 	&\small  Températures\\ \hline
%2		&en \degres C			&en \degres F\\ \hline
%3		&$- 5$					&23\\ \hline
%4		&0						&32\\ \hline
%5		&5						&41\\ \hline
%6		&10						&50\\ \hline
%7		&15 					&59\\ \hline
%8		&20 					&68\\ \hline
%9		&25 					&77\\ \hline
%\end{tabularx}}
%
%\medskip

\begin{enumerate}
\item %Quelle température en \degres F correspond à une température de 20 \degres C ?
On lit sur la tableau : 20~\degres C = 68~\degres F.
\item %Quelle température en \degres C correspond à une température de 41 \degres F ?
De même 41~\degres F = 5~\degres C.
\item %Pour convertir la température de \degres C en \degres F, il faut multiplier la température en \degres C par $1,8$ puis ajouter $32$.

%On a écrit une formule en B3 puis on l'a recopiée vers le bas.

%Quelle formule a-t-on pu saisir dans la cellule B3 ?
=A3*1,8+32
\end{enumerate}

\vspace{0,5cm}

