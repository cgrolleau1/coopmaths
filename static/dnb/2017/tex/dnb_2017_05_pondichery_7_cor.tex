
\medskip

\begin{enumerate}
\item Si le tarif était proportionnel à la masse, la lettre de $100 = 5 \times 20$~(g) devrait être affranchie $5 \times 0,80 = 4$~\euro. Non, le tarif n'est pas proportionnel à la masse.
\item Il lui faut $1$ enveloppe et $4$ pages.

$\bullet~~$Une enveloppe a un poids de $\dfrac{175}{50} = \dfrac{350}{100} = 3,5$~g.

$\bullet~~$Une feuille a une aire de :

$0,21 \times 0,297 = \np{0,06237}$~m$^2$ et donc un poids de :

$\np{0,06237} \times 80 = \np{4,9896}$.

4 feuilles ont donc un poids de $4 \times \np{4,9896}  = \np{19,9584}$

Masse totale d'un courrier (sans compter sur le poids du timbre !) : 

3,5 + \np{19,9584} = \np{23,4584}~g. Il dépasse 20~g.

Il doit donc payer 1,60~\euro.
\end{enumerate}
