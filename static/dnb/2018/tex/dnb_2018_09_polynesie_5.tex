
\medskip

Un collégien français et son correspondant anglais ont de nombreux
centres d'intérêt communs comme le basket qu'ils pratiquent tous les deux.

Le tableau ci-dessous donne quelques informations sur leurs ballons.

\begin{center}
\begin{tabularx}{\linewidth}{|X|X|}\hline
\multicolumn{1}{|c|}{Ballon du collégien français}& \multicolumn{1}{|c|}{Ballon du correspondant anglais}\\ \hline
\multicolumn{1}{|c|}{$A \approx \np{1950}$ cm"}& \multicolumn{1}{|c|}{$D \approx 9,5$ inch}\\ \hline
&$D$ désigne le diamètre du ballon.\\
$A$ désigne l'aire de la surface du ballon et $r$ son rayon. On a $A = 4 \times \pi \times r^2$.& L'inch est une unité de longueur anglo-saxonne.
On a 1 inch $= 2,54$ cm.\\ \hline
\end{tabularx}
\end{center}

Pour qu'un ballon soit utilisé dans un match officiel, son diamètre doit être compris
entre $23,8$ cm et $24,8$ cm.

\medskip

\begin{enumerate}
\item Le ballon du collégien français respecte-t-il cette norme ?
\item  Le ballon du collégien anglais respecte-t-il cette norme ?
\end{enumerate}

\bigskip

