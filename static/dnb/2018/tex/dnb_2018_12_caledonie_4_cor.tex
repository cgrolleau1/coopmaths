
\medskip

%On étudie les performances de deux nageurs (nageur 1 et nageur 2).
%
%La distance parcourue par le nageur 1 en fonction du temps est donnée par le graphique ci-dessous.
%
%\medskip

\begin{figure}
\begin{center}
\psset{xunit=0.25cm,yunit=0.0035cm,arrowsize=2pt 4}
\begin{pspicture}(-1,-50)(50,2100)
\multido{\n=0+5}{10}{\psline[linestyle=dashed,linewidth=0.25pt](\n,0)(\n,2100)}
\multido{\n=0+200}{11}{\psline[linestyle=dashed,linewidth=0.2pt](0,\n)(50,\n)}
\psaxes[linewidth=1.25pt,Dx=5,Dy=200](0,0)(0,0)(50,2100)
\uput[r](0,2100){Distance parcourue (en mètres)}
\uput[u](44,0){Temps (en minutes)}
\psline[linewidth=1.2pt](0,0)(10,400)(30,1600)(45,2000)
\psline[ArrowInside=->]{->}(45,0)(45,2000)(0,2000)
\psline[ArrowInside=->]{->}(0,200)(5,200)(5,0)
\end{pspicture}
\end{center}
\end{figure}

\smallskip

\begin{enumerate}
\item Répondre aux questions suivantes par lecture graphique. Aucune justification n'est demandée.
	\begin{enumerate}
		\item %Quelle est la distance totale parcourue lors de cette course par le nageur 1 ?
Le point d'abscisse 45 a pour ordonnée \np{2000}. Le nageur 1 a parcouru \np{2000}~m.
		\item %En combien de temps le nageur 1 a-t-il parcouru les $200$ premiers mètres ?
Le point d'ordonnée 200 a pour antécédent 5. Le nageur 1 a parcouru les $200$ premiers mètres en 5 minutes.
	\end{enumerate}
%Y a-t-il proportionnalité entre la distance parcourue et le temps sur l'ensemble de la course ? 
	
%Justifier.
\item La distance parcourue n'est pas une application linéaire du temps. Dans ce cas tous les points devraient être alignés sur une droite contenant l'origine.
\item %Montrer que la vitesse moyenne du nageur 1 sur l'ensemble de la course est d'environ $44$ m/min.
Le nageur a parcouru \np{2000}~m en 45~min ; sa vitesse moyenne est donc égale à $\dfrac{2000}{45}\approx 44,444$, soit à l'unité près environ 44~m/min.
\item %On suppose maintenant que le nageur 2 progresse à vitesse constante.
%La fonction $f$ définie par $f(x) = 50x$ représente la distance qu'il parcourt en fonction du temps $x$.
	\begin{enumerate}
		\item %Calculer l'image de $10$ par $f$.
On a $f(10) = 50 \times 10 = 500$~(m).
		\item %Calculer $f(30)$.
$f(30) = 50 \times 30 = \np{1500}$~(m).
	\end{enumerate}
\item %Les nageurs 1 et 2 sont partis en même temps,
	\begin{enumerate}
		\item %Lequel est en tête au bout de $10$~min ? Justifier.
Au bout de 10~min, le nageur 1 a parcouru 400~m et le nageur 2, $f(10) = 500~$m : le nageur 2 est en tête.
		\item %Lequel est en tête au bout de $30$min ? Justifier.
Au bout de 30~min, le nageur 1 a parcouru \np{1600}~m et le nageur 2, $f(30) = \np{1500}~$m : le nageur 1 est en tête.
	\end{enumerate}
\end{enumerate}

\vspace{0,5cm}

