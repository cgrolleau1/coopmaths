
\medskip

\begin{enumerate}
\item On considère la fonction $g$ représentée dans le repère en \textbf{annexe 1}.
	\begin{enumerate}
		\item Donner l'antécédent de $4$ par la fonction $g$.
		\item Dans l'\textbf{annexe 1}, compléter le tableau de valeurs de la fonction $g$.
 	\end{enumerate}
\item  La fonction $f$ est donnée par $f(x) = 2x$.
	\begin{enumerate}
		\item Quelle est l'image de $- 2$ par la fonction $f$ ?
		\item Calculer $f(3)$.
		\item Dans l'\textbf{annexe 1}, tracer la représentation graphique de la fonction $f$.
 	\end{enumerate}
\item  Déterminer graphiquement l'abscisse du point d'intersection S des deux représentations graphiques.
	
Faire apparaître en pointillés la lecture sur le graphique de l'\textbf{annexe 1}.
\item L'expression de la fonction $g$ est $g(x) = - 2x + 8$.
	\begin{enumerate}
		\item Résoudre l'équation $2x = -2x + 8$
		\item Que représente graphiquement le résultat précédent ?
	\end{enumerate}
\end{enumerate}

\begin{center}
\textbf{ANNEXES à rendre avec la copie}

\medskip

\textbf{Annexe 1}

\medskip

\parbox{0.5\linewidth}{\psset{unit=0.5cm}
\begin{pspicture}(-3,-5)(7,13)
\psgrid[gridlabels=0pt,subgriddiv=1,linestyle=dashed]
\psaxes[linewidth=1.25pt]{->}(0,0)(-3,-5)(7,13)
\psaxes[linewidth=1.25pt](0,0)(1,1)
\psplot[plotpoints=2000,linewidth=1.25pt,linecolor=blue]{-2.5}{6.5}{8 2 x mul sub}
\rput(2,-5.5){Représentation graphique de la fonction}
\end{pspicture}}
\hfill
\parbox{0.48\linewidth}{
\begin{tabularx}{\linewidth}{|*{5}{>{\centering \arraybackslash}X|}}\hline
$x$		&$- 2$	&	&4	&\\ \hline
$g(x)$	&		&8	&	&$- 4$\\  \hline
\end{tabularx}}
\end{center}


\vspace{0,5cm}

