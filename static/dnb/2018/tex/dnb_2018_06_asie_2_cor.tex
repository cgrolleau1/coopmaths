
\medskip

\begin{enumerate}
\item Aire de la base de la yourte : $\pi \times 3,5^2 \approx 38,48$~m$^2$ soit plus de 35.
\item Le volume de la yourte est la somme du volume du cylindre et de de celui du cône :

$V_{\text{yourte}} = \pi \times 3,5^2 \times 2,5 + \dfrac{1}{3} \times \pi \times 3,5^2 \times 2 = \pi \times 3,5^2\left(2,5 + \dfrac{2}{3}\right) \approx 121,868$~m$^3$ soit environ 122~m$^3$ au m$^3$ près.
\item Les dimensions sont divisées par 25 : la hauteur de la maquette sera donc de $\dfrac{4,5}{25} = \dfrac{18}{100} = 0,18$~(m) soit 18~cm.
\end{enumerate}

\bigskip

