
\medskip

Bien que l'exercice ne demande pas de justification, on en donnera une rapide, dans ce corrigé.

\begin{enumerate}
	\item \textbf{Réponse C}
	
	On peut procéder par décompositions successives : 
	
$\np{1600} = \np{100}\times 16 = 10^2\times 4^2 = (2\times 5)^2 \times (2^2)^2= 2^2\times 5^2\times 2^2\times 2^2 = 2^{2 + 2 + 2}\times 5^2 = 2^6 \times 5^2$.
	
On peut aussi procéder par élimination :  la proposition A donne \np{1600} mais ni 4 ni 10 ne sont des nombres premiers. La proposition B ne donne pas \np{1600}, en effet, $2^8\times 5^2 = \np{6400}$.
	
	\item  \textbf{Réponse B}
	\begin{itemize}
		\item Les points E, A et M sont alignés, dans cet ordre;
		\item Les points F, A et N sont alignés, dans le même ordre
	\end{itemize}
On a donc une configuration de Thalès.
	
Comme les droites (EF) et (MN) sont parallèles, on peut appliquer le théorème de Thalès, et donc, on en déduit :
	\quad
	$\dfrac{\text{AE}}{\text{AM}} = \dfrac{\text{AF}}{\text{AN}} = \dfrac{\text{EF}}{\text{MN}}$.
	
Notamment : $\dfrac{\text{AE}}{\text{AM}} = \dfrac{\text{EF}}{\text{MN}}$, soit, avec les distances données : $\dfrac{\text{2}}{\text{5}} = \dfrac{\text{4}}{\text{MN}}$.
	
Et donc, grâce à un produit en croix, $\text{MN} = \dfrac{4\times 5}{2} =\np[cm]{10}$ 
	
	\item \textbf{Réponse A}
	
On développe : $6x(3x-5) + 7x = 6x\times 3x - 6x\times 5 + 7x = 18x^2 - 30x + 7x = 18x^2 - 23x$.
	
On remarque que la proposition B est notre avant dernière étape. Ce n'est pas la bonne réponse car l'expression n'est pas réduite, avec deux termes en $x$.
	
\end{enumerate}

\vspace{0,5cm}

