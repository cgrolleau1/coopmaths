
\medskip

\begin{enumerate}
\item Le responsable du plus grand club omnisport de la région a constaté qu'entre le 1\up{er} janvier 2010 et le 31 décembre 2012 le nombre total de ses adhérents a augmenté de
10\,\% puis celui-ci a de nouveau augmenté de 5\,\% entre le 1\up{er} janvier 2013 et le 31
décembre 2015. 

Le nombre total d'adhérents en 2010 était de \np{1000}.
	\begin{enumerate}
		\item Calculer, en justifiant, le nombre total d'adhérents au 31 décembre 2012.
		\item Calculer, en justifiant, le nombre total d'adhérents au 31 décembre 2015.
		\item Martine pense qu'au 31 décembre 2015, il devrait y avoir \np{1150} adhérents car elle affirme : \og \emph{une augmentation de $10$\,\% puis une autre de $5$\,\%, cela fait une
augmentation de} 15\,\% \fg. 
		
Qu'en pensez-vous? Expliquez votre réponse.
	\end{enumerate}
\item  Au 1\up{er} janvier 2017, les effectifs étaient de \np{1260}~adhérents.
	
Voici le tableau de répartition des adhérents en 2017 en fonction de leur sport de
prédilection.
	
\begin{center}
\begin{tabularx}{\linewidth}{|c|*{3}{>{\footnotesize \centering \arraybackslash}X|}}\hline
				&Effectif en 2017&Angle en degrés correspondant 
(pour construire le diagramme circulaire)&Fréquence en \%\\ \hline
Planche à voile &392		&			&\\ \hline
Beach volley 	&224		&			&\\ \hline
Surf			&644		&			&\\ \hline
Total 			&\np{1260}	&360\degres	&100\,\%\\ \hline
\end{tabularx}
\end{center}

	\begin{enumerate}
		\item Compléter sur l'annexe, à la fin, la colonne intitulée \og Angle en degrés correspondant \fg.
		
(\emph{Pour expliquer votre démarche, vous ferez figurer sur votre copie les calculs
correspondants.})
		\item Pour représenter la situation, construire un diagramme circulaire de rayon 4~cm.
		\item Compléter sur l'annexe la colonne \og Fréquence en \% \fg. (\emph{Pour expliquer votre démarche, vous ferez figurer sur votre copie les calculs correspondants. Vous donnerez le résultat arrondi au centième près.})
	\end{enumerate}
\end{enumerate}

\begin{center}
	\textbf{\large ANNEXE}
	
	\vspace{2cm}
	
	\textbf{À DÉTACHER DU SUJET ET À JOINDRE AVEC LA COPIE}
	
	
	\vspace{3cm}
	
	\begin{flushleft}	
	\textbf{question 2 :} IMAGE MANQUANTE !!!
	%/Users/denis/Library/Containers/com.apple.mail/Data/Library/Mail Downloads/372C85B8-DEC4-41AD-AC51-6A5284517250/6302318471001.jpg
	\end{flushleft}
	
	\vspace{1cm}
	
	\begin{center}
	\begin{tabularx}{\linewidth}{|c|*{3}{>{\centering \arraybackslash}X|}}\hline
					&\textbf{Effectif en 2017}&\textbf{Angle en degrés correspondant} 
	&\textbf{Fréquence en \%}\\ \hline
	Planche à voile &392		&			&\\ \hline
	Beach volley 	&224		&			&\\ \hline
	Surf			&644		&			&\\ \hline
	\textbf{Total} 			&\np{1260}	&360\degres	&100\,\%\\ \hline
	\end{tabularx}
	\end{center}
	\end{center}

\bigskip

