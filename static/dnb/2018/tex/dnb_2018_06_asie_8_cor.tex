
\medskip

\begin{enumerate}
\item 1,5 L d'eau donne 1,62 L de glace, donc 1 L d'eau donne $\dfrac{1,62}{1,5} = \dfrac{3 \times 0,54}{3 \times 0,5} = \dfrac{2 \times 0,5}{2 \times 0,5} = 1,08$ L de glace.
\item D'après la question précédente, on passe de C1 à C2 en multipliant par 1,08.

La formule est donc =B1 *1,08
\item La fonction permettant de passer du volume d'eau au volume de glace est l'application affine $x \longmapsto 1,08x$. On sait que la représentation de cette fonction est une droite (graphique \no 1 exclu) contenant l'origine (graphique \no 3 exclu).

Le graphique \no 2 est donc la représentation graphique.
\end{enumerate}
