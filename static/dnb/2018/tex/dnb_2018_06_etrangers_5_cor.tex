
\medskip

\begin{enumerate}
\item Tarif A : $202,43 + \np{0,0609} \times \np{17500} = \np{1268,18}$ €. La famille est abonnée au tarif A.
\item 
	\begin{enumerate}
		\item Nombre de kWh consommés en 2017 : $\np{17500} \times  \dfrac{80}{100} = \np{14000}$.
		\item Montant à payer en 2017 : $202,43 + \np{0,0609} \times \np{14000} = \np{1055,03}$ (\euro).
		
Montant des économies réalisées par la famille de Romane entre 2016 et 2017 : 
		
		$\np{1268,18} - \np{1055,03} = 213,15$~(\euro).
	\end{enumerate}
\item  On souhaite déterminer la consommation maximale assurant que le tarif A est le plus avantageux.
	
Pour cela :

\setlength\parindent{1cm}
\begin{itemize}
\item[$\bullet~~$] on note $x$ le nombre de kWh consommés sur l'année.
\item[$\bullet~~$] on modélise les tarifs A et B respectivement par les fonctions $f$ et $g$ :

\[f(x) = \np{0,0609}x + 202,43\quad  \text{et}\quad  g(x) = \np{0,0574}x + 258,39.\]

\end{itemize}
\setlength\parindent{0cm}
	\begin{enumerate}
		\item Ce sont des fonctions affines, leurs représentations graphiques sont des droites.
		\item $\np{0,0609}x + 202,43 < \np{0,0574}x + 258,39$
		
$\np{0,0609}x - 0,0574x < 258,39 - 202,43$
		
$\np{0,0035}x < 55,96$

$x < \dfrac{55,96}{\np{0,0035}}$. Or $\dfrac{55,96}{\np{0,0035}} \approx \np{15988,6}$.
		\item Le tarif A est le plus avantageux jusqu'à une consommation maximale d'environ \np{15989}kWh.
	\end{enumerate}
\end{enumerate}

\vspace{0,5cm}

