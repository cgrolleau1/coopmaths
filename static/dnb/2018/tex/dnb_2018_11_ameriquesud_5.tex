
\medskip
 
Dans tout l'exercice, on étudie les performances réalisées par les athlètes qui ont participé
aux finales du $100$~m masculin des Jeux Olympiques de 2016 et de 2012.

On donne ci-dessous des informations sur les temps mis par les athlètes pour parcourir
$100$~m.

\medskip

\textbf{Finale du \boldmath $100$\unboldmath~m aux Jeux Olympiques de 2016 :}

Temps réalisés par tous les finalistes :

\begin{center}
\begin{tabularx}{\linewidth}{|*{8}{>{\centering \arraybackslash}X|}}\hline
10,04 s&9,96 s&9,81 s&9,91 s&10,06 s &9,89 s&9,93 s&9,94 s\\ \hline
\end{tabularx}
\end{center}

\textbf{Finale du 100 m aux Jeux Olympiques de 2012 :}

\begin{center}
\begin{tabularx}{0.7\linewidth}{|r X X r|}\hline
$\bullet~~~~$& nombre de finalistes&\dotfill&8\\
$\bullet~~$& temps le plus long&\dotfill&11,99 s\\
$\bullet~~$& étendue des temps&\dotfill&2,36 s\\
$\bullet~~$& moyenne des temps&\dotfill&10,01 s\\
$\bullet~~$&médiane des temps&\dotfill&9,84 s\\ \hline
\end{tabularx}
\end{center}

\begin{enumerate}
\item Quel est le temps du vainqueur de la finale en 2016 ?
\item Lors de quelle finale la moyenne des temps pour effectuer $100$~m est-elle la plus petite ?
\item Lors de quelle finale le meilleur temps a-t-il été réalisé ?
\item L'affirmation suivante est-elle vraie ou fausse ?

\textbf{Affirmation :} \og Seulement trois athlètes ont mis moins de $10$~s à parcourir les $100$~m de
la finale de 2012 \fg.
\item C'est lors de la finale de 2012 qu'il y a eu le plus d'athlètes ayant réussi à parcourir le
$100$~m en moins de $10$~s.

Combien d'athlètes ont-ils réalisé un temps inférieur à $10$~s lors de cette finale de 2012 ?
\end{enumerate}

\medskip

