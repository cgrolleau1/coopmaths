
\medskip

%\textit{Dans cet exercice, aucune justification n'est attendue.}
%
%Simon travaille sur un programme. Voici des copies de son écran :
%
%\smallskip
%
%\renewcommand*{\arraystretch}{1.5}
%
%\begin{tabularx}{\linewidth}{|*{2}{>{\centering \arraybackslash} X|}} \hline
%	\textbf{Script principal }&\textbf{Bloc Carré}\\
%	\multirow{3}{*}{\begin{scratch}
%		\blockinit{quand ~\greenflag ~est cliqué}
%		\blockmove{aller à x : \ovalnum{--200} y : \ovalnum{0}}
%		\blockmove{s'orienter à \ovalnum{90 ~\selectarrownum}}
%		\blockpen{effacer tout}
%		\blockpen{mettre la taille du stylo à \ovalnum{1}}
%		\blockvariable{mettre \selectmenu{côté} à \ovalnum{40}}
%		\blockrepeat{répéter \ovalnum{4} fois}
%		{
%		\blockmoreblocks{carré}
%		\blockmove{avancer de \ovalvariable{côté}}
%		\blockvariable{ajouter à \selectmenu{côté} \ovalnum{20}}
%		}
%	\end{scratch}}
%	&
%	\begin{scratch}
%		\initmoreblocks{définir \namemoreblocks{carré}}
%		\blockpen{stylo en position d'écriture}
%		\blockrepeat{répéter \ovalnum4 fois}
%		{\blockmove{avancer de \ovalvariable{coté}}
%			\blockmove{tourner \turnleft{} de \ovalnum{90} degrés}
%		}
%		\blockpen{relever le stylo}
%	\end{scratch}
%\vspace{7mm}
%	\\ \cline{2-2}
%	&\textbf{Information}\\
%	&L'instruction \begin{scratch}
%		\blockmove{s'orienter à \ovalnum{90 ~\selectarrownum}}
%	\end{scratch} \linebreak signifie qu'on se dirige vers la droite.  
% 	\vspace{7mm}\\ \hline
%\end{tabularx}
%
%\begin{minipage}[t]{0.55\linewidth}
\begin{enumerate}
		\item %Il obtient le dessin ci-contre.
		\begin{enumerate}
			\item %D'après le script principal, quelle est la longueur du côté du plus petit carré dessiné ?
Au départ côté est mis à 40 ; le premier carré a ses côtés de longueur 40.
			\item %D'après le script principal, quelle est la longueur du côté du plus grand carré dessiné ?
À chaque fois côté est augmenté de 20, donc le dernier carré a pour longueur de ses côtés  : $40 + 20 + 20 +20 = 100$.
		\end{enumerate}
	\vspace{5mm}
	\item %Dans le script principal, où peut-on insérer l'instruction \begin{scratch}
%\blockpen{ajouter \ovalnum{2} à la taille du stylo} de façon à obtenir le dessin ci-contre ?
%\end{scratch} de façon à obtenir le dessin ci-contre ?
Il faut augmenter la taille du stylo à la fin de chaque tracé de carré, donc après l'instruction : ajouter à côté 20.

	\vspace{5mm}
	
	\item %On modifie maintenant le script principal pour obtenir celui qui est présenté ci-contre :
	
%Parmi les dessins ci-dessous, lequel obtient-on ?
%	
%\begin{tabularx}{\linewidth}{|X|} \hline 
%\textbf{Dessin 1}\\
%\begin{tikzpicture}[x = 0.017cm, y = 0.017cm]
%\foreach \c in {0,...,3} {
%\draw[shift = {(60*\c+10*\c*\c,-10*\c)}] (0,0) -- (40+20*\c,0) -- (40+20*\c,40+20*\c) -- (0 , 40+20*\c) -- cycle;  }
%\end{tikzpicture} \\ \hline
%\textbf{Dessin 2}\\
%\begin{tikzpicture}[x = 0.017cm, y = 0.017cm]
%\foreach \c in {0,...,3} {
%\draw[shift = {(60*\c+10*\c*\c,0)}] (0,0) -- (40+20*\c,0) -- (40+20*\c,40+20*\c) -- (0 , 40+20*\c) -- cycle;  }
%\draw (0,0)--(400,0);
%\end{tikzpicture} \\ \hline
%\textbf{Dessin 3}\\
%\begin{tikzpicture}[x = 0.017cm, y = 0.017cm]
%\foreach \c in {0,...,3} {
%\draw[shift = {(60*\c+10*\c*\c,0)}] (0,0) -- (40+20*\c,0) -- (40+20*\c,40+20*\c) -- (0 , 40+20*\c) -- cycle;  }
%\end{tikzpicture} \\ \hline
%\end{tabularx}
%\end{enumerate}
%\end{minipage} \hfill 
%\begin{minipage}[t]{0.42\linewidth}
%	\begin{center}
%		\begin{tikzpicture}[baseline = {(0,1.8)},x = 0.017cm, y = 0.017cm]
%		\foreach \c in {0,...,3} {
%		\draw[shift = {(30*\c+10*\c*\c,0)}] (0,0) -- (40+20*\c,0) -- (40+20*\c,40+20*\c) -- (0 , 40+20*\c) -- cycle;  }
%	\end{tikzpicture}
%	
%	\vspace{ 10mm}
%		\begin{tikzpicture}[x = 0.017cm, y = 0.017cm]
%	\foreach \c in {0,...,3} {
%		\draw[line width = \c pt,shift = {(30*\c+10*\c*\c,0)}] (0,0) -- (40+20*\c,0) -- (40+20*\c,40+20*\c) -- (0 , 40+20*\c) -- cycle;  }
%	\end{tikzpicture}
%	
%	\vspace{8mm}
%	
%	\begin{scratch}
%		\blockinit{quand ~\greenflag ~est cliqué}
%		\blockmove{aller à x : \ovalnum{--200} y : \ovalnum{0}}
%		\blockmove{s'orienter à \ovalnum{90 ~\selectarrownum}}
%		\blockpen{effacer tout}
%		\blockpen{mettre la taille du stylo à \ovalnum{1}}
%		\blockvariable{mettre \selectmenu{côté} à \ovalnum{40}}
%		\blockrepeat{répéter \ovalnum{4} fois}
%		{
%			\blockmoreblocks{carré}
%			\blockmove{avancer de \ovaloperator{\ovalvariable{côté} + \ovalnum{30}}}
%			\blockvariable{ajouter à \selectmenu{côté} \ovalnum{20}}
%		}
%	\end{scratch}
On obtient le dessin \no 3.
\end{enumerate}

%\textbf{Pour rappel : le bloc carré}
%\begin{scratch}
%	\initmoreblocks{définir \namemoreblocks{carré}}
%	\blockpen{stylo en position d'écriture}
%	\blockrepeat{répéter \ovalnum4 fois}
%	{\blockmove{avancer de \ovalvariable{coté}}
%		\blockmove{tourner \turnleft{} de \ovalnum{90} degrés}
%	}
%	\blockpen{relever le stylo}
%\end{scratch}
%	\end{center}
%\end{minipage}

\vspace{0,5cm}

