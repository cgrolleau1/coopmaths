
\medskip

%\parbox{0.7\linewidth}{Sam a écrit le programme ci-dessous qui permet de tracer un rectangle comme
%ci-contre.
%
%Ce programme comporte deux variables (Longueur) et (Largeur) qui représentent les dimensions du rectangle.
%
%On rappelle que l'instruction \begin{scratch} \blockmove{s'orienter à \ovalnum{90\selectarrownum} degrés} \end{scratch} signifie que l'on s'oriente vers la droite.
%}\hfill
%\parbox{0.25\linewidth}{\psset{unit=0.6cm}
%\begin{pspicture}(5.6,3.3)
%\psframe(0.5,0.3)(5.6,3.3)
%\rput(0.5,0.1){\footnotesize Départ}
%\end{pspicture}
%}
%
%\bigskip

%\begin{tabularx}{\linewidth}{|X|X|}\hline
%Script&bloc rectangle\\ \hline
%\begin{scratch}
%\blockinit{Quand \greenflag est cliqué}
%\blockpen{effacer tout}
%\blockvariable{mettre \selectmenu{Longueur} à \txtbox{50}}
%\blockvariable{mettre \selectmenu{Largeur} à \txtbox{30}}
%\blockmove{aller à x: \ovalnum0 y: \ovalnum0}
%\blockmove{s'orienter à \ovalnum{90\selectarrownum}}
%\blockmoreblocks{rectangle}
%\end{scratch}&\begin{scratch} 
%\initmoreblocks{définir \namemoreblocks{rectangle}}
%\blockpen{stylo en position d'écriture}
%\blockrepeat{répéter \ovalnum{\ldots} fois}
%{
%\blockmove{avancer de \ovalnum{\ldots\ldots}}
%\blockmove{tourner \turnleft{} de \ldots degrés}
%\blockmove{avancer de \ovalnum{\ldots\ldots}}
%\blockmove{tourner \turnleft{} de \ldots degrés}
%}
%\end{scratch}\\ \hline
%\end{tabularx}

\bigskip

\begin{enumerate}
\item ~%Compléter le bloc rectangle ci-dessus avec des nombres et des variables pour que le script fonctionne.
\begin{scratch} 
%\initmoreblocks{définir \namemoreblocks{rectangle}}
%\blockpen{stylo en position d'écriture}
\blockrepeat{répéter \ovalnum{2} fois}
{
\blockmove{avancer de \ovalnum{Longueur}}
\blockmove{tourner \turnleft{} de 90 degrés}
\blockmove{avancer de \ovalnum{Largeur}}
\blockmove{tourner \turnleft{} de 90 degrés}
}
\end{scratch}
%On recopiera et on complétera uniquement la boucle répéter sur sa copie.
\item %Lorsque l'on exécute le programme, quelles sont les coordonnées du point d'arrivée et dans quelle direction est-on orienté?
Les coordonnées sont celles du point de départ et l'orientation à 90\degres.
\item %Sam a modifié son script pour tracer également l'image du rectangle par l'homothétie de centre le point de coordonnées (0~;~0) et de rapport 1,3.

%\parbox{0.45\linewidth}{
	\begin{enumerate}
		\item ~%Compléter le nouveau script de Sam donné ci-contre afin d'obtenir la figure ci-dessous. On recopiera et on complètera sur sa copie les lignes 9 et 10 ainsi que l'instruction manquante en ligne 11.

%\psset{unit=0.5cm}
%\begin{pspicture}(-0.5,-0.5)(6.5,4)
%\psframe(6.5,3.9)
%\psframe(5,3)
%\uput[dl](0,0){\small Départ}
%\end{pspicture}
%\end{enumerate}
%}\hfill
%\parbox{0.52\linewidth}{\begin{scratch}[num blocks]
%\blockinit{Quand \greenflag est cliqué}
%\blockpen{effacer tout}
%\blockvariable{mettre \selectmenu{Longueur} à \txtbox{50}}
%\blockvariable{mettre \selectmenu{Largeur} à \txtbox{30}}
%\blockmove{aller à x: \ovalnum0 y: \ovalnum0}
%\blockmove{s'orienter à \ovalnum{90\selectarrownum}}
%\blockmoreblocks{rectangle}
%\blockcontrol{attendre \ovalnum{3} secondes}
%\blockvariable{mettre \selectmenu{Longueur} à \ovaloperator{\ovalvariable{Longueur} x \ovalnum{\ldots}}}
%\blockvariable{mettre \selectmenu{Largeur} à \ovaloperator{\ovalvariable{\ldots } x \ovalnum{\ldots}}}
%\setscratch{print=true}
%\blockmoreblocks{~}
%\end{scratch}
%}
\begin{scratch}
\blockvariable{mettre \selectmenu{Longueur} à \ovaloperator{\ovalvariable{Longueur} x \ovalnum{1,3}}}
\blockvariable{mettre \selectmenu{Largeur} à \ovaloperator{\ovalvariable{Largeur} x \ovalnum{1,3}}}
%\setscratch{print=true}
\blockmoreblocks{rectangle}
\end{scratch}

\medskip

%\begin{enumerate}[resume]
		\item%[\textbf{b.}] %Sam exécute son script. Quelles sont les nouvelles valeurs des variables Longueur et Largeur à la fin de l'exécution du script?
À la fin de l'exécution du programme la longueur est de $50 \times 1,3 = 65$ et la largeur à 

$30 \times 1,3 = 39$.
	\end{enumerate}
\end{enumerate}

\vspace{0,5cm}

