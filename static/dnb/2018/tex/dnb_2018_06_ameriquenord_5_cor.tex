
\medskip

%Gaspard travaille avec un logiciel de géométrie dynamique pour construire une frise.
%
%Il a construit un triangle ABC isocèle en C (motif 1) puis il a obtenu le losange ACBD (motif 2).
%
%Voici les captures d'écran de son travail.
%
%\smallskip
%
%\begin{tabularx}{\linewidth}{|*{2}{>{\centering \arraybackslash} X|}} \hline
%	\textbf{Motif 1} & \textbf{Motif 2}\\ \hline
%	\begin{tikzpicture}
%		\draw (0,0) -- (2.5,.8) -- (2.5,-0.8) -- cycle;
%		\node at (0,0) [left = 1mm]{C};
%		\node at (2.5,0.8) [above = 1mm]{A};
%		\node at (2.5,-0.8) [below = 1mm]{B};
%	\end{tikzpicture}	
%	& 	\begin{tikzpicture}
%	\draw (0,0) -- (2.5,.8) -- (2.5,-0.8) -- cycle (2.5,0.8) --(5,0)--(2.5,-0.8);
%	\node at (0,0) [left = 1mm]{C};
%	\node at (2.5,0.8) [above = 1mm]{A};
%	\node at (2.5,-0.8) [below = 1mm]{B};
%	\node at (5,0) [right = 1mm]{D};
%	\end{tikzpicture}	 \\ \hline
%\end{tabularx}

\begin{enumerate}
	\item %Préciser une transformation permettant de compléter le motif 1 pour obtenir le motif 2.
Le motif 2 est obtenu à partir du motif 1, soit par symétrie orthogonale par rapport à la droite (AB), soit par symétrie centrale autour du milieu de [AB].
	
	\item %Une fois le motif 2 construit, Gaspard a appliqué à plusieurs reprises une translation. 
	
%Il obtient ainsi la frise ci-dessous.
	
%Préciser de quelle translation il s'agit.
	
%	\begin{center}
%			\begin{tikzpicture}
%		\foreach \r in {0,...,3}
%		{\draw[shift = {(2.5*\r,-0.8*\r)}] (0,0) -- (2.5,.8) -- (2.5,-0.8) -- cycle (2.5,0.8) --(5,0)--(2.5,-0.8);}
%		\node at (0,0) [left = 1mm]{C};
%		\node at (2.5,0.8) [above = 1mm]{A};
%		\node at (2.5,-0.8) [below = 1mm]{B};
%		\node at (5,0) [above right = 0.7mm]{D};
%		\end{tikzpicture}
%	\end{center}
La translation répétée trois fois est la translation qui transforme C en B ou qui transforme A en D.
\end{enumerate}

\vspace{5mm}

