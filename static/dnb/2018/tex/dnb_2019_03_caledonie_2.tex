
\medskip


Lors d'un voyage à Osaka, Jade a mangé des TAKOYAKI (gâteaux japonais)
qu'elle veut refaire chez elle. 

Pour cela, elle dispose d'une plaque de cuisson
comportant plusieurs moules à gâteaux. Tous les moules sont identiques.

Chaque moule a la forme d'une demi-sphère de rayon 3 cm.

\textbf{Rappels} : 1 L = 1 dm$^3$

\hspace{1.5cm}Volume d'une boule de rayon $r$ : $V = \dfrac{4}{3} \times \pi \times r^3$

\medskip

\begin{enumerate}
\item Calculer le volume d'un moule $\left(\text{en cm}^3\right)$, arrondir le résultat au dixième.
\item Dans cette question, on considère que le volume d'un moule est de 57 cm$^3$.

Jade a préparé 1 L de pâte. Elle doit remplir chaque moule aux $\dfrac{3}{4}$ de son volume.

Combien de TAKOYAKI peut-elle faire ? Justifier la réponse.
\end{enumerate}

\vspace{0,5cm}

