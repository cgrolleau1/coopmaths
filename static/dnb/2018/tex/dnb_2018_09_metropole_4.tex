
\medskip

Les abeilles ouvrières font des allers-retours entre les fleurs et la ruche pour transporter le nectar et le pollen des fleurs qu'elles stockent dans la ruche.

\medskip

\begin{enumerate}
\item Une abeille a une masse moyenne de $100$ mg et rapporte en moyenne $80$ mg de charge (nectar, pollen) à chaque voyage.

Un homme a une masse de $75$ kg. S'il se chargeait proportionnellement à sa masse, comme une abeille, quelle masse cet homme transporterait-il ?
\item Quand elles rentrent à la ruche, les abeilles déposent le nectar récolté dans des alvéoles.

On considère que ces alvéoles ont la forme d'un prisme de $1,15$~cm de hauteur et dont la base est un hexagone d'aire $23$ mm$^2$ environ, voir la figure ci-dessous.
	\begin{enumerate}
		\item Vérifier que le volume d'une alvéole de ruche est égal à $264,5$ mm$^3$.
		
\begin{center}\psset{unit=0.8cm}
\begin{pspicture}(14,5)
\pspolygon(3,2.8)(4.8,2.8)(5.8,3.9)(4.85,5.1)(3.05,5.1)(2.1,3.9)
\psline(2.1,1.3)(3,0)(4.8,0)(5.8,1.3)
\psline(2.1,3.9)(2.1,1.3)
\psline(3,2.8)(3,0)
\psline(4.8,2.8)(4.8,0)
\psline(5.8,3.9)(5.8,1.3)
\psline{<->}(1.9,1.3)(1.9,3.9)
\uput[l](1.9,2.6){\footnotesize 1,15 cm}
\psline[linestyle=dashed](2.1,1.3)(3.05,2.3)(4.85,2.3)(5.8,1.3)
\psline[linestyle=dashed](3.05,2.3)(3.05,5.1)
\psline[linestyle=dashed](4.85,2.3)(4.85,5.1)
\pspolygon(10.2,1.5)(12,1.5)(12.9,3)(12,4.5)(10.2,4.5)(9.4,3)
\uput[u](11.1,4.5){Aire$_{\text{Base}} = 23$ mm$^2$}
\rput(11.1,0.3){Base hexagonale}
\end{pspicture}

\emph{Le volume d'un prisme est donné par la formule : V$_{\text{prisme}} =$ Aire$_{\text{Base}} \times $ Hauteur}
\end{center}

		\item L'abeille stocke le nectar dans son jabot. Le jabot est une petite poche sous l'abdomen d'un volume de $6 \times  10^{-5}$ litre. Combien de sorties au minimum l'abeille doit-elle faire pour remplir une alvéole ?

(rappel: 1 dm$^3$ = 1 litre)
	\end{enumerate}
\item Le graphique ci-dessous présente la production française de miel en 2015 et 2016.

\begin{center}
\psset{xunit=2.5cm,yunit=0.0005cm}
\begin{pspicture}(-0.25,-3000)(4,12000)
\rput(2.1,11500){\large \textbf{Production française de miel en 2015 et 2016 (en tonnes)}}
\psaxes[linewidth=1.25pt,Dx=10,Dy=2000](0,0)(0,0)(4,12000)
\psframe[fillstyle=solid,fillcolor=orange](0.2,0)(0.5,6057)\rput(0.35,3000){\small \np{6057}}
\psframe[fillstyle=solid,fillcolor=orange](1.2,0)(1.5,2587)\rput(1.35,1300){\small \np{2587}}
\psframe[fillstyle=solid,fillcolor=orange](2.2,0)(2.5,6379)\rput(2.35,3200){\small \np{6379}}
\psframe[fillstyle=solid,fillcolor=orange](3.2,0)(3.5,9201)\rput(3.35,4600){\small \np{9201}}
\psframe[fillstyle=solid,fillcolor=lightgray](0.5,0)(0.8,3965)\rput(0.65,2000){\small \np{3965}}
\psframe[fillstyle=solid,fillcolor=lightgray](1.5,0)(1.8,1869)\rput(1.65,900){\small \np{1869}}
\psframe[fillstyle=solid,fillcolor=lightgray](2.5,0)(2.8,4556)\rput(2.65,2300){\small \np{4556}}
\psframe[fillstyle=solid,fillcolor=lightgray](3.5,0)(3.8,5709)\rput(3.65,2850){\small \np{5709}}
\rput(1.5,10000){\psframe[fillstyle=solid,fillcolor=orange](-0.1,-200)(0.3,700) 2015}
\rput(2.5,10000){\psframe[fillstyle=solid,fillcolor=lightgray](-0.1,-200)(0.3,700) 2016}
\rput(0.5,-500){possédant}\rput(0.5,-1100){au plus 49}\rput(0.5,-1600){ruches}
\rput(1.5,-500){possédant}\rput(1.5,-1100){de 50 à 149}\rput(1.5,-1600){ruches}
\rput(2.5,-500){possédant}\rput(2.5,-1100){150 à 399}\rput(2.5,-1600){ruches}
\rput(3.5,-500){possédant}\rput(3.5,-1100){au moins}\rput(3.5,-1600){400 ruches}
\rput(1.5,-2400){\emph{Source : Observatoire de la production de miel et gelée royale
FranceAgriMer 2017}}
\end{pspicture}
\end{center}

	\begin{enumerate}
		\item Calculer la quantité totale de miel (en tonnes) récoltée en 2016.
		\item Sachant que la quantité totale de miel récoltée en 2015 est de \np{24224} tonnes, calculer le pourcentage de baisse de la récolte de miel entre 2015 et 2016.
	\end{enumerate}
\end{enumerate}

\vspace{0,5cm}

