
\medskip

Indiquer si les affirmations suivantes sont vraies ou fausses. Justifier vos réponses.

\medskip

\textbf{Affirmation 1}

\smallskip

On lance un dé équilibré à six faces numérotées de 1 à 6.

Un élève affirme qu'il a deux chances sur trois d'obtenir un diviseur de 6.

A-t-il raison ?

\medskip

\textbf{Affirmation 2}

\smallskip

On considère le nombre $a = 3^4 \times 7$.

Un élève affirme que le nombre $b = 2 \times 3^5 \times 7^2$ est un multiple du nombre $a$.

A-t-il raison ?

\medskip

\textbf{Affirmation 3}

\smallskip

En 2016, le football féminin comptait en France \np{98800} licenciées alors qu'il y en avait \np{76000} en 2014.

Un journaliste affirme que le nombre de licenciées a augmenté de $30$\,\% de 2014 à
2016.

A-t-il raison ?

\medskip

\textbf{Affirmation 4}

\smallskip

Une personne A a acheté un pull et un pantalon de jogging dans un magasin.

Le pantalon de jogging coûtait 54~\euro. Dans ce magasin, une personne B a acheté le
même pull en trois exemplaires; elle a dépensé plus d'argent que la personne A.

La personne B affirme qu'un pull coûte $25$~\euro.

A-t-elle raison ?

\bigskip

