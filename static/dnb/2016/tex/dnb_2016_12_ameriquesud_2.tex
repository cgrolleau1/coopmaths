
\medskip

Un avion de ligne transportant des passagers atterrit à l'aéroport international Galeao à Rio de Janeiro.

On étudie la distance de freinage de l'appareil en fonction de sa vitesse au moment de
l'atterrissage.

Le pilote peut décider d'un freinage \og rapide \fg{} s'il souhaite raccourcir la distance de freinage, ou d'un freinage \og confort\fg{} plus modéré et donc plus confortable pour les passagers.

Les courbes suivantes donnent la distance de freinage d'un avion en fonction de sa vitesse au
moment de l'atterrissage selon le mode freinage choisi (confort ou rapide).

\begin{center}


\psset{xunit=0.02cm,yunit=0.0023cm}
\begin{pspicture}(-15,-120)(520,4400)
\multido{\n=0+10}{53}{\psline[linestyle=dotted,linewidth=0.4pt](\n,0)(\n,4200)}
\multido{\n=0+100}{43}{\psline[linestyle=dotted,linewidth=0.4pt](0,\n)(520,\n)}
\psaxes[linewidth=1.25pt,Dx=20,Dy=500,labelFontSize=\scriptscriptstyle]{->}(0,0)(0,0)(520,4200)
\psplot[plotpoints=3000,linewidth=1.25pt,linecolor=blue]{0}{400}{x dup mul 0.011719 mul}
\psplot[plotpoints=3000,linewidth=1.25pt,linecolor=green]{0}{400}{x dup mul 0.023669 mul}
\uput[u](380,0){Vitesse d'atterrissage $\left(\text{en km.h}^{-1}\right)$}
\uput[r](0,4100){Distance de freinage en mètres}
\uput[r](400,3700){freinage \og confort\fg}
\uput[r](400,1900){freinage \og rapide\fg}
\rput(260,4300){\textbf{Distance de freinage de l'avion en fonction de la vitesse d'atterrissage}}
\end{pspicture}
\end{center}


\begin{enumerate}
\item Donner par lecture graphique, sans justification:
	\begin{enumerate}
		\item Une valeur approchée de la distance de freinage \og confort\fg{} de l'appareil si l'avion arrive à une vitesse de 320 km.h$^{-1}$.
		\item Une valeur approchée de la vitesse d'atterrissage d'un avion dont la distance de freinage \og rapide \fg{} est de \np{1500}~m.
	\end{enumerate}
\item Pour regagner la zone de débarquement des passagers, l'avion doit emprunter une des
quatre sorties précisées dans les documents ci-dessous :

\begin{center}
Distances des sorties au point d'atterrissage
\begin{tabularx}{0.9\linewidth}{|c|*{4}{>{\centering \arraybackslash}X|}}\hline
Numéro de sortie& 1 &2 &3 &4\\ \hline
Distance (en mètres)& 900 &\np{1450} &\np{2050} &\np{2950}\\ \hline
\end{tabularx}

\bigskip

\psset{unit=0.55cm,arrowsize=2pt 4}
\begin{pspicture}(20,10.5)
%\psgrid
\psframe(20,10.5)
\psline(0.8,9.3)(17.3,0.)(17.9,1.2)
\psline(1,9.6)(5.6,7)(7,7)\rput(6.3,7.3){Sortie 1}
\psline(7,6.7)(6,6.7)(8.8,5.2)(10,5.2)\rput(9.4,5.6){Sortie 2}
\psline(10.4,4.9)(9.3,4.9)(12.1,3.3)(13.3,3.3)\rput(12.6,3.8){Sortie 3}
\psline(13.5,3)(12.4,3)(12.8,2.9)(17.1,0.4)(17.6,1.3)\rput(17.8,1.6){Sortie 4}
\rput(2.6,10){Point d'atterrissage}
\psline{->}(2.2,9.7)(1,9.4)
\rput(15,9.5){Aéroport international Galeao}
\rput(15,9){Rio de Janeiro}
\psline[linestyle=dashed]{->}(1,9.4)(4,7.7)
\psline{->}(4.4,7.4)(5.6,6.8)(6.4,6.9)
\psline{->}(8,5.4)(8.8,5.05)(10.4,5.05)
\psline{->}(11,3.7)(12,3.15)(13.2,3.15)
\psline{->}(16,0.85)(17.2,0.2)(17.6,1)
\end{pspicture}
\end{center}


	\begin{enumerate}
		\item L'avion atterrit à 260 km.h$^{-1}$. Le pilote décide un freinage \og confort\fg. Avec la distance de freinage correspondante, quelle est ou quelles sont les sorties qu'il va dépasser ?
		\item Seule la sortie 1 étant disponible, le pilote envisage un freinage \og rapide \fg.
		
Déterminer avec la précision du graphique, la vitesse maximale avec laquelle il peut
atterrir pour pouvoir emprunter cette sortie.
	\end{enumerate}
\end{enumerate}

\vspace{0,5cm}

