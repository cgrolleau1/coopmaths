
\medskip

\begin{enumerate}
\item Soit $x$ le prix en euros d'une pizza ronde.

Le prix d'une pizza carrée est donc $x + 1$ 

Les deux pizzas coûtent : $x + x + 1 = 14,20$ soit 

$2x + 1 = 14,20$ ou 

$2x = 13,20$ soit 

$x = \dfrac{13,2}{2}  = 6,60$.

La pizza ronde coûte 6,60~\euro{} et la pizza carrée coûte 7,60~\euro.
\item $\bullet~~$ Pizza ronde :

Rayon de la pizza : $\dfrac{34}{2} = 17$~cm.

Aire de la pizza : $\pi \times 17^2 = 289\pi~\left(\text{cm}^2\right)$.

L'aire d'une part est donc :  $\dfrac{289\pi}{8} \approx  113,5~\left(\text{cm}^2\right)$.

$\bullet~~$ Pizza carrée :

Aire  de la pizza : $34^2 = \np{1156}~\left(\text{cm}^2\right)$.

L'aire d'une part est donc : $\dfrac{\np{1156}}{9} \approx  128,4~\left(\text{cm}^2\right)$.

$\bullet~~$ C'est donc  la pizza carrée qui donne  les parts les plus grandes.
\end{enumerate}
