
\medskip

\begin{enumerate}
\item La température du four n'est pas proportionnelle au temps car la courbe n'est pas
une droite.
\item Au bout de 3 minutes, la température est de 70~\degres C.
\item À la deuxième minute, la température est de 50~\degres C et à la septième minute, la
température est de 140~\degres C. Entre la deuxième et la septième minute, la température a
donc augmenté de 90~\degres C.
\item La température de 150~\degres C nécessaire à la cuisson des macarons est atteinte au bout
de 8 minutes.
\item Passé 8 minutes, la température continue à augmenter, puis fluctue autour de 150~\degres C. Le responsable ne peut pas être satisfait car la température ne reste pas constante
à 150~\degres C.
\end{enumerate}

\vspace{0,5cm}

