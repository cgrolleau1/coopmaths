
\medskip

Le viaduc de Millau est un pont franchissant la vallée du Tarn, dans le département
de l'Aveyron, en France. Il est constitué de $7$ pylônes verticaux équipés chacun de 22
câbles appelés haubans.

Le schéma ci-dessous, qui n'est pas à l'échelle, représente un pylône et deux de ses
haubans.

\begin{center}
\psset{unit=0.6cm}
\begin{pspicture}(19,9)
\psline(0,0.5)(19,0.5)%AD
\psline(2.3,0.5)(2.3,8.2)%AB
\psline(13.1,0.5)(2.3,5.4)%FE
\psline(14.9,0.5)(2.3,6.3)%DC
\uput[d](2.3,0.5){A} \uput[l](2.3,8.2){B} \uput[l](2.3,6.3){C}
\uput[d](14.9,0.5){D} \uput[l](2.3,5.4){E} \uput[d](13.1,0.5){F}
\rput{90}(2,2.5){Pylône}\rput{-26}(7.5,4.4){Haubans}
\uput[u](16.5,0.5){Chaussée}
\end{pspicture}
\end{center}

On dispose des informations suivantes :

AB = 89 m ; AC = 76 m ; AD = 154 m ; FD = 12 m et EC = 5 m.

\medskip

\begin{enumerate}
\item Calculer la longueur du hauban [CD]. Arrondir au mètre près.
\item Calculer la mesure de l'angle $\widehat{\text{CDA}}$ formé par le hauban [CD] et la chaussée.

Arrondir au degré près.
\item Les haubans [CD] et [EF] sont-ils parallèles ?
\end{enumerate}

\bigskip

