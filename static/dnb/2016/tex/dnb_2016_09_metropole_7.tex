
\medskip

Une pizzeria fabrique des pizzas rondes  de 34~cm  de diamètre et des pizzas carrées de 34~cm de côté.

\begin{center}
%\begin{tabularx}{\linewidth}{*{2}{\centering \arraybackslash}X}
\psset{unit=1cm}
\begin{pspicture}(-2.5,-2.5)(2.5,2.5)
\pscircle(0,0){2.25}
\multido{\n=0+45}{8}{\psline(2.25;\n)}
\end{pspicture}
%&
\psset{unit=1cm}
\begin{pspicture}(-2.5,-2.5)(2.5,2.5)
\psframe(-2.25,-2.25)(2.25,2.25)
\multido{\n=-2.25+1.50}{3}{\psline(-2.25,\n)(2.25,\n)}
\multido{\n=-2.25+1.50}{3}{\psline(\n,-2.25)(\n,2.25)}
\end{pspicture}
%\end{tabularx}
\end{center}

Toutes les pizzas

\begin{itemize}
\item[$\bullet~~$]ont la même épaisseur ;
\item[$\bullet~~$]sont livrées dans des boîtes identiques.
\end{itemize}
 
\medskip
 
Les  pizzas carrées coûtent 1~\euro{} de plus que les pizzas rondes.
 
 \medskip
 
\begin{enumerate}
\item Pierre achète deux pizzas : une ronde et une carrée. Il paye 14,20~\euro. Quel est le prix de chaque pizza ?
\item Les pizzas rondes sont découpées en huit parts de même taille et les pizzas carrées en neuf parts de même taille.

Dans quelle pizza trouve-t-on les parts les plus grandes ? 
\end{enumerate}



