
\medskip

\textbf{Figure 1}

On a BC = CJ = JA = 6~cm. Donc CA = 12~cm.

On applique le théorème de Pythagore au triangle ABC rectangle en B :

$\text{AC}^2 = \text{CB}^2 + \text{BA}^2$ d'où $\text{BA}^2 = \text{AC}^2 - \text{CB}^2 = 12^2 - 6^2  = 144 - 36 = 108 = 9 \times 12 = 9 \times 4 \times 3 = 36 \times 3$.

Donc $\text{AB} = \sqrt{108} = 6\sqrt{3} \approx 10,39\approx 10,4$~(cm).

\textbf{Figure 2}

Par définition $\sin \widehat{\text{C}} = \dfrac{\text{AB}}{\text{BC}}$, soit $\sin 53= \dfrac{\text{AB}}{36}$, d'où $\text{AB} = 36 \sin 53 \approx 28,750 \approx 28,8$~(cm).

\textbf{Figure 3}

On sait que la longueur du cercle est égale à $\text{AB} \times \pi$, d'où l'équation :

$\text{AB} \times \pi = 154$ et par conséquent $\text{AB} = \dfrac{154}{\pi} \approx 49,02 \approx 49$~cm.

\bigskip

