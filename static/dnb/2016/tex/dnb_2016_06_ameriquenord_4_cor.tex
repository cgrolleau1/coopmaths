
\medskip

\begin{enumerate}
\item Le télésiège est ouvert de 9 h à 16 h, soit une durée de 7 h.

Le télésiège peut transporter \np{3000} skieurs par heure, je calcule donc :
$7 \times \np{3000} = \np{21000}$.
d  m "'" s 
\item  $t = \dfrac{d}{v} = \dfrac{\np{1453}}{5,5} \approx  264$~(s), 
et $264  = 4 \times 60 + 24$,donc  le temps est égal à  4 min 24 s.

La durée du trajet d'un skieur est d'environ 4 min 24 s.

Ou vous pouvez également utiliser un tableau de proportionnalité, puis déterminer la
durée en secondes à l'aide des produits en croix.
\begin{center}
\begin{tabularx}{0.5\linewidth}{|l|*{2}{>{\centering \arraybackslash}X|}}\hline
Temps (en s)	& 1 	& \\ \hline
Distance (en m)	& 5,5	& \np{1453}\\ \hline
\end{tabularx}
\end{center}
\item  ~

\parbox{0.45\linewidth}{On peut schématiser la situation de la façon suivante :
ABC est un triangle rectangle en B.

AC = \np{1453} (m)

BC $= \np{2261} - \np{1839} 
= 422$~(m)

On calcule la mesure de $\widehat{\text{BAC}}$.}
\hfill \parbox{0.5\linewidth}{
\psset{unit=1cm}
\begin{pspicture}(6,4)
\pspolygon(0.5,0.5)(5.5,0.5)(5.5,3.5)
\psframe(5.5,0.5)(5.2,0.8)
\uput[l](0.5,0.5){A} \uput[dr](5.5,0.5){B} \uput[ur](5.5,3.5){C}
\uput[r](5.5,2){422 m} \uput[ul](3,2){\np{1453} m} 
\end{pspicture}
}

$\sin \left(\widehat{\text{BAC}}\right) = \dfrac{422}{\np{1453}}$, d'où $\widehat{\text{BAC}} \approx 17$~\degres (à la calculatrice).

L'angle formé avec l'horizontale par le câble de ce télésiège est d'environ 17~\degres.
\end{enumerate}

\vspace{0,5cm}

