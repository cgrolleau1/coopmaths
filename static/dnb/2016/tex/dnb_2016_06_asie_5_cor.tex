
\medskip

%On considère les fonctions $f$ et $g$ définies par:
%
%\[f(x) = 2x + 1\quad  \text{et}\quad  g(x) = x^2 + 4x - 5.\]
%
%Léa souhaite étudier les fonctions $f$ et $g$ à l'aide d'un tableur. Elle a donc rempli les
%formules qu'elle a ensuite étirées pour obtenir le calcul de toutes les valeurs.
%
%Voici une capture d'écran de son travail :
%
\begin{center}
\begin{tabularx}{\linewidth}{|c|*{8}{>{\centering \arraybackslash}X|}}\hline
\multicolumn{2}{|c|}{B3}&\multicolumn{7}{|l|}{=B1*B1+4*B1$-$5}\\ \hline
	&A 		&B 		&C 		&D 		&E 		&F 	&G &H\\ \hline
1 	&$x$ 	&$-3$ 	&$-2$ 	&$-1$ 	&0 		&1 	&2 &3\\ \hline
2 	&$f(x)$ &$-5$ 	&$-3$ 	&$-1$ 	&1 		&3 	&5 &7\\ \hline
3 	&$g(x)$ &$-8$	& 		&$-8$ 	&$-5$	&0 	&7 &16\\ \hline
4	&		&		&		&		&		&	&	&\\ \hline
\end{tabularx}
\end{center}

\begin{enumerate}
\item %Quelle est l'image de 3 par la fonction $f$ ?
$3$ a pour image $f(3) = 2\times 3 + 1 = 6 + 1 = 7$.
\item %Calculer le nombre qui doit apparaître dans la cellule C3.
En C3 il doit apparaître $g(- 2) = (- 2)^2  + 4 \times (- 2) - 5 = 4 - 8 - 5 = 4 - 13 = - 9$.
\item %Quelle formule Léa a-t-elle saisie dans la cellule B2 ?
En B2 Léa a inscrit =2*B1+1.
\item %À l'aide de la copie d'écran et sans justifier, donner une solution de
%l'inéquation $2x + 1 < x^2 + 4x - 5$.
On voit que pour $x = 2$, $f(2) < g(2)$. Une solution de l'inéquation est donc 2.
\item %Déterminer un antécédent de $1$ par la fonction $f$.
D'après le tableau un antécédent de 1 est 0. 
\end{enumerate}

\bigskip

